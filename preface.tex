%!TEX root = /Users/markelikalderon/Documents/Git/formwithoutmatter/aristotle.tex
\chapter*{Preface} % (fold)
\label{cha:preface}


This is an essay in the philosophy of perception written in the medium of historiagraphy. 

My motives for writing the present essay, for pursuing the philosophy of perception through its history of philosophy, derive from a number of sources. Let me take this opportunity to describe them.

For a number of years, over a series of papers, I have defended an anti-modern conception of color and color perception. According to that conception, colors are mind-independent qualities of material surfaces, transparent volumes, and radiant light sources, and color perception is the presentation of particular instances of these qualities in the visual awareness afforded by the subject's perceptual experience. This is how, I argue, our color experience presents itself as being, and I undertook to defend this conception from challenges that arise from the problem of conflicting appearances---what Hume regarded as the slightest bit of philosophy required to demonstrate the impossibility of any such conception of color and color perception. This conception is an anti-modern conception of color in that, in the face of a consensus that persisted for four centuries---remarkable indeed in philosophy---colors are not conceived to be secondary qualities. Indeed, they are on a par with shape. And this conception is an anti-modern conception of color perception in that color perception is not merely a mental reaction, a conscious modification of the perceiving subject, but the presentation of a mind-independent quality of a spatiotemporal particular located at a distance from the perceiving subject. 

Defending an anti-modern conception of color and color perception prompted me to investigate pre-modern conceptions of color and color perception. I had, at any rate, been drawing on ancient sources in thinking about the problem of conflicting appearances as it arises in its contemporary guise. However, just because the conception defended was anti-modern, it did not make it pre-modern. I turned to pre-modern sources for at least two reasons. 

First, to further loosen the grip of the modern conception of color and color perception. Though thoroughly convinced of its error, it had been my experience that the effects of this dominant paradigm lingered in my thinking and I engaged with pre-modern thinkers to counteract this. 

Second, not only did I seek alternative perspectives in order to see the subject matter with fresh eyes---to think outside the modern paradigm, but I was also prepared to learn from my predecessors. Both about color and color perception, but also about problems or challenges that had been obscured by the modern paradigm. 

Aristotle was a natural figure for me to focus on. He was the great defender of the manifest image in the ancient world. 

The problematic relationship between contemporary philosophy of perception and its history provided an additional motive. Approximately forty years ago philosophy of perception remained in the core of curriculum. For various reasons philosophy of perception ceased to be an area of active concern and was no longer regularly taught as part of the core curriculum. There was thus a period of forgetting. 

I have mixed feelings about this. The present renaissance in the philosophy of perception perhaps owes its renewed energy precisely to this forgetting. On the other hand, I was frustrated that hard won insights had been lost. To address this, at least for myself, I undertook to study its history. The present essay grew out of that study.

There were more specific motives as well. For example, I have long been puzzled by the prevalence of a primordial family of tactile metaphors for visual awareness. Thus the objects of perception are said to be ``grasped'' or ``apprehended''. Visual experience puts us into perceptual ``contact'' with its objects. Thus Broad speaks of visual awareness as a mode of prehension. What unites these metaphors is that they are all a mode of assimilation and ingestion is a natural variant. I wanted to understand what would make these metaphors apt. Most philosophers deploy these metaphors with little self-consciousness. Looking at earlier occurrences of these metaphors, when they were more vivid and powerful, 

As Sorabji warns, interpretation requires creativity and this invites invention. I have done my best not to contribute to the long history of disortions. Two features of the present work are potential cause for concern, a reliance on a metaphysics of fire of Heraclitean provenance, and a portrait of Aristotle's philosophical concerns that perhaps only an Oxford realist could love. 

A Heraclitean fire burns throughout this book. Reflection on Heraclitean fire reveals explanatory resources available to Aristotle insofar as the presence and activity of the fiery substance is meant to be the determinant of light and visibility. Here, I can only say that I have made my case in what follows and that we should be mindful of the explanatory fruits of the attribution. 

The second feature of the present essay that might invite the charge of invention is the similarity between Aristotle's perceptual realism, as I portray it, and the Oxford realism inaugurated by John Cook Wilson and extended and elaborated by a variety of thinkers, most recently, Charles Travis and John McDowell. Here too the threat of invention arises in the form of potential anachronism. Here too I have made my case in what follows, but this is not all that I can say. It ought not to be surprising that there are similarities between Aristotle and Oxford realists. 

% chapter preface (end)