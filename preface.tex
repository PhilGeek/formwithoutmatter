%!TEX root = /Users/markelikalderon/Documents/Git/formwithoutmatter/aristotle.tex
\chapter*{Preface} % (fold)
\label{cha:preface}


This is an essay in the philosophy of perception written in the medium of historiography. 

My motives for writing the present essay, for pursuing the philosophy of perception through its history, derive from a number of sources. Let me take this opportunity to describe some of them.

For a number of years, over a series of papers \citep{Kalderon:2006tg,Kalderon:2008fk,Kalderon:2010fj,Kalderon:2007mr,Kalderon:2011fk} that followed on from an earlier collaboration \citep{Hilbert:2000on}, I have defended an anti-modern conception of color and color perception. According to that conception, colors are mind-independent qualities of material surfaces, transparent volumes, and radiant light sources, and color perception is the presentation of particular instances of these qualities in the visual awareness afforded by the subject's perceptual experience. This is how, I argue, our color experience presents itself as being, and I undertook to defend this conception from challenges that arise from the problem of conflicting appearances---what \citet{Hume:1748zr} regarded as the ``slightest bit of philosophy'' required to reject any such conception of color and color perception. This conception is an anti-modern conception of color in that, in the face of a consensus that persisted for four centuries---remarkable indeed in philosophy---colors are not conceived to be secondary qualities. Indeed, they are on a par with shape. And this conception is an anti-modern conception of color perception in that color perception is not merely a mental reaction, a conscious modification of the perceiving subject, but the presentation of mind-independent qualities of spatiotemporal particulars located at a distance from the perceiving subject. 

Defending an anti-modern conception of color and color perception prompted me to investigate premodern conceptions of color and color perception. I had, at any rate, been drawing on ancient sources in thinking about the problem of conflicting appearances as it arises in its contemporary guise. However, just because the conception defended was anti-modern, did not make it premodern. I turned to premodern sources for at least two reasons. 

First, to further loosen the grip of the modern conception of color and color perception. Though thoroughly convinced of its error, it had been my experience that the effects of this dominant paradigm lingered in my thinking, and I engaged with premodern thinkers to counteract this. In this regard, I was self-consciously following the example of Deleuze whose early historical studies were an attempt to break free from what was, within his tradition, the then dominant structuralist paradigm.

Second, not only did I seek alternative perspectives in order to see the subject matter with fresh eyes---to think outside the modern paradigm, but I was also prepared to learn from my predecessors. Both about color and color perception, but also about problems or challenges that had been obscured by the modern paradigm. Aristotle was a natural figure for me to focus on. He was the great defender of the manifest image in the classical world. And like Aristotle, I was engaged in just such a reconciliationist project.

The problematic relationship between contemporary philosophy of perception and its history provided an additional motive. Until the Second World War, philosophy of perception remained a central and relatively active area of concern. For various reasons, philosophy of perception ceased to be an area of active concern in the postwar period and was no longer regularly taught as part of the core curriculum, at least in the Anglophone world. There was thus a period of forgetting. 

I have mixed feelings about this. The present renaissance in the philosophy of perception perhaps owes its renewed energy precisely to this forgetting. On the other hand, I was frustrated that hard won insights had been lost. To address this, at least for myself, I undertook to study its history. The present essay grew out of that study.

There were more specific motives as well. For example, I have long been puzzled by the prevalence of a primordial family of tactile metaphors for visual awareness. Thus the objects of perception are said to be ``grasped'' or ``apprehended''. Visual experience puts us into perceptual ``contact'' with its objects. \citet{Broad:1965dq}, for example, speaks of visual awareness as a mode of ``prehension''. What unites these metaphors is that they are all a mode of assimilation and ingestion is a natural variant. I wanted to understand what would make these metaphors apt. Most contemporary philosophers deploy these metaphors with little self-consciousness. They are lifeless in their hands. Looking at earlier occurrences of these metaphors, when they were more vivid and powerful, could, however, provide guidance. As Nietzsche insightfully observed:
\begin{quote}
The relief-like, incomplete presentation of an idea, of a whole philosophy, is sometimes more effective than its exhaustive realization: more is left for the beholder to do, he is more impelled to continue working on that which appears before him so strongly etched in light and shadow, to think it through to the end, and to overcome even that constraint which has hitherto prevented it from stepping forth fully formed. \citep[§178 92]{Nietzsche:1996na}
\end{quote}
With that in mind, I undertook what might be described as a conceptual genealogy. That is to say, I looked at early historical occurrences of these metaphors, when they remained strongly etched in light and shadow, in order to interpret their significance for us. Unsurprisingly, in their very earliest occurrences, they do not occur as metaphors at all. Thus according to Empedocles, color perception is literally a form of ingestion. Colors are understood to be effluences, fine material bodies, that must be taken within the interior of the eye in order to be presented to the organ of sight and so be seen. For early thinkers, deploying tactile descriptions for visual awareness is not unselfconscious. Rather it is the means of conceptual innovation, that is to say, the means of self-consciously reconceptualizing perceptual experience and its relation to its object. It is easier to see what made tactile metaphors for visual perception apt when self-consciously deployed by these early thinkers, in a way that it is difficult, if not indeed impossible, to do merely by reflecting upon the unselfconscious, lifeless, and yet widely prevalent use of these metaphors by contemporary philosophers of perception.

The present essay makes historical and philosophical claims. The way I approach Aristotle's texts illustrates how historical and philosophical claims combine in the present work. Beginning with the phenomenon under investigation, as we presently understood it to be, I would ask how it might appear so that it would be apt, or at least intelligible, to describe the phenomena the way Aristotle describes it. This prompted me to closely examine and elaborate Aristotle's visual examples. On this basis, I make some important claims of interpretation, for example, about the nature of transparency as conceived by Aristotle. To read Aristotle's texts in this way, is not merely the exercise of interpretive charity, though that it may be. It is also to use Aristotle's text as a means of attending to the phenomena under investigation. On this basis, I make some important philosophical claims, for example, about the puzzles that arise about sensory presentation and what the nature of sensory presentation must be like if these puzzles admit of genuine resolution.

The present approach can seem starkly opposed to the approach associated with Burnyeat and Williams. On this alternative, mindful of lines of possibility forever cut off, since the historical conditions that made them possible may no longer obtain, there is a readiness to encounter the alien, what is not intelligible in terms that we presently understand. Thus \citet{Burnyeat:1992fk} notoriously claims that Aristotle's philosophy of mind is simply no longer credible. It is easy to exaggerate the difference in approach. I agree that part of the point and interest of the history of philosophy is encountering perspectives other than our own. Indeed, that is what prompted me to look to the ancients. But incommensurability is a hard claim. It is only after we have done our best to understand the thought of another and failed, should we consider whether we have encountered something genuinely alien. The remaining difference between Burnyeat and myself is neither a matter of methodology, nor temperament, but concerns a larger philosophical background. There is, perhaps, a sense in which Burnyeat is right: Aristotle's philosophy of mind is no longer credible within the modern paradigm that Burnyeat endorses. But I am skeptical about that paradigm. I believe that we presently have resources to think of perceptual experience in other terms. And so I also believe that we presently have resources in terms of which Aristotle's philosophy of perception may be understood. Many of Burnyeat's criticisms of Aristotle are the result of cleaving too closely to the modern paradigm. Nor is he alone in this. It is arguable that the criticisms of \citet{Broadie:1993fk} and \citet{Sorabji:1971fr} (to cite but two prominent examples) are as well.

In pursuing the philosophy of perception through its history, and, in particular, pursuing it through a close examination of Aristotle's psychological writings, I am following the path laid out before me by the ancient commentators. Though not a linear commentary, and not following the methodologies of the commentators, let alone the Neoplatonist background assumptions of many of them, like the work of the ancient commentators, the present essay is meant to be a contribution to the philosophy of perception and not, or not merely, to the history of ideas. Given the professionalization of philosophy, with its seemingly incumbent specialization, the present essay thus has a divided readership. I mean to address both contemporary philosophers of perception and historians interested in Aristotle's psychological works. I fear that I am bound to disappoint both. 

The philosophers of perception, even if they are willing to follow me in the examination of the history of their subject matter, may be disappointed since little in what follows may be aptly described as introductory. Major themes of \emph{De Anima} are simply ignored if they are not directly relevant to the narrative. However, introductions to Aristotle's thought have been written by others better suited to the task than I. And the project of bringing out what I take to be what is of continuing interest in Aristotle, his contributions to the metaphysics of color and the metaphysics of sensory presentation, dictated the present approach. The philosophers of perception may also be disappointed that I do not do more to integrate contemporary philosophical concerns into the discussion. Here, I can only say that I wanted to broaden, not only these concerns, but our approach to them.

Historians, I suspect, will be disappointed for other reasons. They will miss, perhaps, familiar scholarly apparatuses, and may complain of the lack of systematic engagement with alternative interpretations, especially in light of the explosion of interest in Aristotle's psychological writings over the last four decades. In writing the present essay, I experimented with engaging more systematically with the (vast) secondary literature. This had two drawbacks. First, the book would have tripled in size without a proportionate gain in substance. But, second, and more importantly, I found that the narrative thread was quickly lost, and that I was pulled into controversies that were not my main preoccupation. Historians might also be put off by the extensive use of quotation. Here, I was motivated to put before the eyes of those unfamiliar with ancient philosophy crucial aspects of the texts under discussion and so share my enthusiasm with this literature. This also explains the choice of translations. I wanted to draw on readily available translations so that the uninitiated may more easily consult them in assessing the present work.

Despite my concerns about a divided readership, I remained undeterred. Moreover, I remained undeterred for historical and philosophical reasons. Reading \emph{De Anima} and \emph{De Sensu} through the lens of Empedoclean puzzlement about the sensory presentation of remote objects had the effect that certain passages came powerfully to life in a way they had not for me before. Moreover, few, if any, were seeing what I was seeing. So despite my lack of expertise, or perhaps because of it, I felt I could make some small contribution to our collective understanding. Moreover, reading Aristotle through Empedocles held out the promise of making progress in topics central to my thinking. And thus I persisted in the present literary high wire act. Whether it was advisable for me to do so without a net is for the reader to decide.

Concerning \emph{De Anima} and \emph{Parva Naturalia}, \citet[vii]{Hammond:1902kx} wryly remarks of Aristotle's ``breviloquence''. Given the brevity of their description, Aristotle's visual examples require careful elaboration in offering an interpretation of them. In so doing, I have been mindful of Sorabji's \citeyearpar[225]{Sorabji:2003fk} warning that interpretation requires creativity and that this invites invention. I have done my best not to contribute to the long history of ``distortions'', fruitful though that history may be. Two features of the present work may nevertheless invite the charge of invention: (1) a reliance on a metaphysics of fire of Heraclitean provenance, and (2) a portrait of Aristotle's philosophical concerns that perhaps only an Oxford realist could love. 

A Heraclitean fire burns throughout this book. Reflection on Heraclitean fire reveals explanatory resources available to Aristotle insofar as the presence and activity of the fiery substance is meant to be the determinant of light and visibility. The threat of invention arises in the form of potential anachronism. Here, I can only say that I have made my case in what follows and that one should be mindful of the explanatory fruits of the attribution. 

The second feature of the present essay that might invite the charge of invention is the similarity between Aristotle's perceptual realism, as I portray it, and the Oxford realism inaugurated by \citet{Cook-Wilson:1926sf} and extended and elaborated by a variety of thinkers, including, \citet{Prichard:1909yg,Prichard:1950kx}, \citet{Ryle:1949qr}, \citet{Austin:1961bs,Austin:1962lr}, \citet{Hinton:1973js}, and more recently, \citet{McDowell:1994am}, \citet{Travis:2008la}, and \citet{Williamson:2000lr}. Here, too, the threat of invention arises in the form of potential anachronism. Here, too, I have made my case in what follows, but this is not all that I can say. 

It ought not to be surprising that there are similarities between Aristotle and the Oxford realists, since the former influenced the latter. Cook Wilson, in revolting against idealism, drew upon Aristotle's realism as a model. And Aristotle's philosophy was drawn upon, in different ways, in the development of Oxford realism by subsequent thinkers in this tradition. Cook Wilson explicitly worked on Aristotle, and many of the early thinkers in this tradition took Greats. Ryle wrote on Plato and developed, in his own way, in \emph{The Concept of Mind}, ideas derived from Aristotle's \emph{De Anima}. Austin edited a book series on Aristotle and borrows a title of Aristotle's for his lectures on Ayer and perception. Non-accidentally, it turns out, as Austin borrows, as well, some philosophical doctrines and examples. (For a revealing account of the connection between Aristotle and Oxford philosophy, especially ordinary language philosophy, see \citealt[Introduction]{Ackrill:1997tg}.) It ought not to be surprising, then, that Aristotle and the Oxford realists should share a family resemblance. For the early Oxford realists, Aristotle was a distant if revered ancestor from whom they drew strength and sustenance in their defense of perceptual realism against the idealism that the moderns had bequeathed to them.

% chapter preface (end)