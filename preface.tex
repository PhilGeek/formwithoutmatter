%!TEX root = /Users/markelikalderon/Documents/Git/formwithoutmatter/aristotle.tex
\chapter*{Preface} % (fold)
\label{cha:preface}


This is an essay in the philosophy of perception written in the medium of historiography. 

My motives for writing the present essay, for pursuing the philosophy of perception through its history, derive from a number of sources. Let me take this opportunity to describe some of them.

For a number of years, over a series of papers \citep{Kalderon:2006tg,Kalderon:2008fk,Kalderon:2010fj,Kalderon:2007mr,Kalderon:2011fk}, I have defended an anti-modern conception of color and color perception. According to that conception, colors are mind-independent qualities of material surfaces, transparent volumes, and radiant light sources, and color perception is the presentation of particular instances of these qualities in the visual awareness afforded by the subject's perceptual experience. This is how, I argue, our color experience presents itself as being, and I undertook to defend this conception from challenges that arise from the problem of conflicting appearances---what \citet{Hume:1748zr} regarded as the ``slightest bit of philosophy'' required to reject any such conception of color and color perception. This conception is an anti-modern conception of color in that, in the face of a consensus that persisted for four centuries---remarkable indeed in philosophy---colors are not conceived to be secondary qualities. Indeed, they are on a par with shape. And this conception is an anti-modern conception of color perception in that color perception is not merely a mental reaction, a conscious modification of the perceiving subject, but the presentation of a mind-independent quality of a spatiotemporal particular located at a distance from the perceiving subject. 

Defending an anti-modern conception of color and color perception prompted me to investigate pre-modern conceptions of color and color perception. I had, at any rate, been drawing on ancient sources in thinking about the problem of conflicting appearances as it arises in its contemporary guise. However, just because the conception defended was anti-modern, it did not make it pre-modern. I turned to pre-modern sources for at least two reasons. 

First, to further loosen the grip of the modern conception of color and color perception. Though thoroughly convinced of its error, it had been my experience that the effects of this dominant paradigm lingered in my thinking and I engaged with pre-modern thinkers to counteract this. 

Second, not only did I seek alternative perspectives in order to see the subject matter with fresh eyes---to think outside the modern paradigm, but I was also prepared to learn from my predecessors. Both about color and color perception, but also about problems or challenges that had been obscured by the modern paradigm. Aristotle was a natural figure for me to focus on. He was the great defender of the manifest image in the classical world. And like Aristotle, I was engaged in just such a reconciliationist project.

The problematic relationship between contemporary philosophy of perception and its history provided an additional motive. Approximately forty years ago philosophy of perception remained a central and relatively active area of concern. For various reasons, philosophy of perception ceased to be an area of active concern and was no longer regularly taught as part of the core curriculum. There was thus a period of forgetting. 

I have mixed feelings about this. The present renaissance in the philosophy of perception perhaps owes its renewed energy precisely to this forgetting. On the other hand, I was frustrated that hard won insights had been lost. To address this, at least for myself, I undertook to study its history. The present essay grew out of that study.

There were more specific motives as well. For example, I have long been puzzled by the prevalence of a primordial family of tactile metaphors for visual awareness. Thus the objects of perception are said to be ``grasped'' or ``apprehended''. Visual experience puts us into perceptual ``contact'' with its objects. Broad, for example, speaks of visual awareness as a mode of ``prehension''. What unites these metaphors is that they are all a mode of assimilation and ingestion is a natural variant. I wanted to understand what would make these metaphors apt. Most contemporary philosophers deploy these metaphors with little self-consciousness. They are lifeless in their hands. Looking at earlier occurrences of these metaphors, when they were more vivid and powerful, could, however, provide guidance. As Nietzsche insightfully observed:
\begin{quote}
The relief-like, incomplete presentation of an idea, of a whole philosophy, is sometimes more effective than its exhaustive realization: more is left for the beholder to do, he is more impelled to continue working on that which appears before him so strongly etched in light and shadow, to think it through to the end, and to overcome even that constraint which has hitherto prevented it from stepping forth fully formed. \citep[§178 92]{Nietzsche:1996na}
\end{quote}
With that in mind, I undertook what might be described as a conceptual genealogy. That is to say, I looked at early historical occurrences of these metaphors, when they remained strongly etched in light and shadow, in order to interpret their significance for us. Unsurprisingly, in their very earliest occurrences, they do not occur as metaphors at all. Thus according to Empedocles, color perception is literally a form of ingestion. Colors are understood to be effluences that must be taken within the interior of the eye in order to be presented to the organ of sight and so be seen. For early thinkers, deploying tactile metaphors for visual awareness is not unselfconscious. Rather it is the means of conceptual innovation, that is to say, the means of self-consciously reconceptualizing perceptual experience and its relation to its object. It is easier to see what made tactile metaphors for visual perception apt when self-consciously deployed by these early thinkers, in a way that it is difficult, if not indeed impossible, to do merely by reflecting upon the unselfconscious, lifeless, and yet widely prevalent use of these metaphors by contemporary philosophers of perception.

The present essay makes historical and philosophical claims. The way I approached Aristotle's texts illustrates the way that historical and philosophical claims interact in the present work. In general, beginning with the phenomenon under investigation, as I presently understood it to be, I would ask how might it appear so that it is apt, or at least intelligible, to describe the phenomena the way Aristotle describes it. This prompted me to closely examine Aristotle's visual examples. On this basis, I make some important claims of interpretation, for example, about the nature of transparency as conceived by Aristotle. To read Aristotle's texts in this way, is not merely the exercise of interpretive charity. It is also to use Aristotle's text as a means of attending to the phenomena under investigation. On this basis, I make some important philosophical claims, about the puzzles that arise about sensory presentation and what the nature of sensory presentation must be like if these puzzles admit of genuine resolution.

Concerning \emph{De Anima} and \emph{Parva Naruralis}, Hammond wryly remarks of Aristotle's ``breveloquence''. Given the brevity of their description, Aristotle's examples require careful elaboration in offering an interpretation of them. In so doing, I have been mindful of Sorabji's \citeyearpar[225]{Sorabji:2003fk} warning that interpretation requires creativity and that this invites invention. I have done my best not to contribute to the long history of ``distortions'', fruitful though that history of commentary may be. Two features of the present work may nevertheless be potential cause for concern: (1) a reliance on a metaphysics of fire of Heraclitean provenance, and (2) a portrait of Aristotle's philosophical concerns that perhaps only an Oxford realist could love. 

A Heraclitean fire burns throughout this book. Reflection on Heraclitean fire reveals explanatory resources available to Aristotle insofar as the presence and activity of the fiery substance is meant to be the determinant of light and visibility. The threat of invention arises in the form of potential anachronism. Here, I can only say that I have made my case in what follows and that one should be mindful of the explanatory fruits of the attribution. 

The second feature of the present essay that might invite the charge of invention is the similarity between Aristotle's perceptual realism, as I portray it, and the Oxford realism inaugurated by \citet{Cook-Wilson:1926sf} and extended and elaborated by a variety of thinkers, including, \citet{Prichard:1909yg,Prichard:1950kx}, \citet{Ryle:1949qr}, \citet{Austin:1961bs,Austin:1962lr}, \citet{Hinton:1973js}, and more recently, \citet{McDowell:1994am}, \citet{Travis:2008la}, and \citet{Williamson:2000lr}. Here too the threat of invention arises in the form of potential anachronism. Here too I have made my case in what follows, but this is not all that I can say. 

It ought not to be surprising that there are similarities between Aristotle and the Oxford realists, since the former influenced the latter. Cook Wilson explicitly worked on Aristotle, and many of the early thinkers in this tradition took Greats. Ryle wrote on Plato and developed, in his own way, in \emph{The Concept of Mind}, ideas derived from Aristotle's \emph{De Anima}. Austin edited a book series on Aristotle and borrows a title of Aristotle's for his lectures on Ayer and perception. Non-accidentally, it turns out, as Austin borrows, as well, some philosophical doctrines and examples. (For a revealing account of the connection between Aristotle and Oxford philosophy, especially ordinary language philosophy, see \citealt[Introduction]{Ackrill:1997tg}.) Cook Wilson, in revolting against idealism, drew upon Aristotle's realism as a model. And Aristotle's philosophy was drawn upon, in different ways, in the development of Oxford realism by subsequent thinkers in this tradition. It ought not to be surprising, then, that Aristotle and the Oxford realists should share a family resemblance. For the early Oxford realists, Aristotle was a distant if revered ancestor from whom they drew strength and sustenance in their defense of perceptual realism against the idealism that the moderns had bequeathed them.

% chapter preface (end)