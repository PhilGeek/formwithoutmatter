%!TEX root = /Users/markelikalderon/Documents/Git/formwithoutmatter/aristotle.tex
\chapter{Form Without Matter} % (fold)
\label{cha:form_without_matter}

\section{The Capacity to Assimilate} % (fold)
\label{sec:the_capacity_to_assimilate}

Color is the power to affect light. This power grounds the derived power to mediately affect sense organs sensitive to such alterations. The eyes in seeing are filled with light. Extending the external light within triggers the reactive capacity for sight, the form and substance of the eye. In seeing the colors of external particulars arrayed in the natural environment the animal exercises their natural capacity to perceive and so realizes their nature as a perceiver. This is no alteration. The perceiver both sees and has seen, perceptual activity being complete at every instant. In seeing the color of an external particular the perceiver, or perhaps their experience, becomes like the object of their visual experience actually is. Perception is a mode of assimilation, though not as Empedocles conceived of it. On the ingestion model, assimilation is understood as a material mode of assimilation, the immediate object of sense, the emitted effluence, being literally taken within the sense organ so that it may be in contact with the perceptive part of the soul. Perceptual activity, for Aristotle, may not be an instance of qualitative alteration. Nevertheless, as in the case of alteration, the perceiver, or perhaps their experience, becomes like the way the perceived object was prior to perception. The perceiver, or their experience, assimilates the sensible object in a manner necessarily distinct from the material assimilation of that object, since to be palpable is to be imperceptible. What is this non-material mode of assimilation? And how does it help resolve the Empedoclean puzzlement with which we began? These questions structure the present and final chapter. 

% section the_capacity_to_assimilate (end)

\section{The Wax Analogy} % (fold)
\label{sec:the_wax_analogy}

The assimilation of sensible form without matter is explained in terms of an analogy with wax receiving the impression of a signet ring:
\begin{quote}
	Generally, about all perception, we can say that a sense is what has the power of receiving into itself the sensible forms of things without the matter, in the way in which a piece of wax takes on the impress of a signet-ring without the iron or gold; what produces the impression is a signet of bronze or gold, but not qua bronze or gold: in a similar way the sense is affected by what is coloured or flavoured or sounding not insofar as each is what it is, but insofar as it is of such and such a sort and according to its form. (Aristotle, \emph{De Anima} \textsc{ii} 12 424\( ^{a} \)18--23; Smith in \citealt[42--43]{Barnes:1984uq})
\end{quote}
Aristotle is explicit that the analogy is meant to hold of perception generally, as opposed to just vision or audition, say. A sense is what has the power to assimilate the sensible form without the matter of a remote external particular. What does Aristotle mean by sense here? Does he mean the sense organ or the perceptual capacity the sense organ possesses, or at least endows the perceiver with? Given that Aristotle claims that a sense is what possesses this power, it is natural to understand him as having the sense organ in mind. So, for example, it is the eye that possesses the capacity to see, or at least endows their possessor with that capacity. In possessing or endowing the capacity to see, the eye potentially assimilates the color of remote external particulars. 

Is the assimilation of sensible form a material change that the eye undergoes in seeing, or is it a psychological change, the exercise of sight in seeing? Though Aristotle rejects the answer in the style of Gorgias, an echo of the ingestion model remains in his optical anatomy. The eye in seeing is like a lantern, not because it emits fire from its interior, but because its interior is illuminated. The eye, being transparent, admits the external light within, in the sense in which it can, light being a state, the state of illumination, and not a material body. While a materially sufficient condition for the perceptual availability of a colored particular, it is merely a necessary condition for its perception. The wounded soldier's eyes are filled with light though he does not see since the passages leading within have been severed. While we have identified a material change that the eye undergoes in seeing---its interior is illuminated---it is merely a necessary condition for sight's realization in seeing. The assimilation of sensible form by the eye is the exercise of the capacity that it endows its possessor with. So understood, it is the psychological change that the eye makes possible, the perceiver undergoing an episode of seeing, that the assimilation of sensible form is meant to characterize.

While denying that perception involves the assimilation of material effluences, Aristotle retains Empedocles' conception of sensory awareness as a mode of assimilation, it is just that we assimilate form without matter. Indeed, this pattern of dialectical refinement continues in the very next line where Aristotle uses Plato's metaphor of wax receiving an impression, not to characterize judgment as Plato does in the \emph{Theaetetus} 194\( ^{c} \)--195\( ^{a} \), but to characterize the assimilation of sensible form in perception. The assimilation of sensible form is compared with the wax's reception of the impression sealed by a signet ring. The wax receives the form imposed upon it by the signet ring, but it does not receive any of the matter that composes the ring, be it bronze or gold, say. So, by analogy, when an animal perceives the white of the sun, they assimilate the chromatic form of the sun but none of its matter. Moreover, just as it is the form of the ring, and not its gold or bronze, that produces the sealed impression, its distinctive shape, it is the whiteness of the sun, and not its matter, that produces the sensory impression, the perceptual experience of the white of the sun. We shall consider these points separately in the following subsections.

\subsection{Assimilation of Form} % (fold)
\label{sub:assimilation_of_form}

Aristotle's analogy is, significantly, of Platonic provenance. In the \emph{Theaetetus}, Plato writes:
\begin{quotation}
	\textsc{socrates}: You will think better of it when you hear the rest. To judge truly is a fine thing and there is something discreditable in error.
	
	\textsc{theaetetus}: Of course.
	
	\textsc{socrates}: Well, they say the differences arise in this way. When a man has in his mind a good thick slab of wax, smooth and knealed to the right consistency, and the impressions that come through the senses are stamped on these tables of the `heart'---Homer's words hints at the mind's likeness to wax---then the imprints are clear and deep enough to last a long time. Such people are quick to learn and also have good memories, and besides they do not interchange the imprints of their perceptions but think truly. These imprints being distinct and well spaced are quickly assigned to their several stamps---the `real things' as they are called---and such men are said to be clever. Do you agree?
	
	\textsc{theaetetus}: Most emphatically.
	
	\textsc{socrates}: When a person has what the poet's wisdom commends as a `shaggy heart', or when the block is muddy or made of impure wax, or oversoft or hard, the people with soft wax are quick to learn, but forgetful, those with hard wax, the reverse. Where it is shaggy or rough, a gritty kind of stuff containing a lot of earth or dirt, the impressions obtained are indistinct; so are they too when the stuff is hard, for they have no depth. Impressions in soft wax also are indistinct, because they melt together and soon become blurred. And if besides this, they overlap through being crowded together into some wretched little narrow mind, they are still indistinct. All these types are likely to judge falsely. When they see or hear or think of something, they cannot quickly assign things to their several imprints. Because they are so slow and sort things into their wrong places, they constantly see and hear and think amiss, and say they are mistaken about things and stupid. (Plato, \emph{Theaetetus} 194\( ^{c} \)--195\( ^{a} \); Cornford in \citealt{Hamilton:1961fk})
\end{quotation}
Aristotle differs from Plato in:
\begin{enumerate}[(1)]
	\item what he uses the analogy for---to explain perception rather than judgment,
	\item the details of the analogy---in particular, the nature of the agent acting upon the wax, a signet ring as opposed to a stylus, and
	\item how to understand the analogy---the way in which perceptions are meant to be like sealed impressions is different from the way in which judgments are like impressions made by styli.
\end{enumerate}
We shall consider these in turn. As will emerge, Aristotle's varying the agent acting upon the wax, his substitution of a signet ring for a stylus, importantly bears on how he understands that analogy.

Whereas Plato deploys the wax analogy to explain judgment, Aristotle does so to explain perception. This difference is most likely intentional and pointed. First, Aristotle is a keen student of the \emph{Theaetetus} and discusses many of its arguments in a number of works. The variation is thus most likely intentional. But to what end? A second observation not only supports the first but sheds light on the significance of this variation. As we discussed in chapters~\ref{sec:definition} and \ref{sec:the_objects_of_perception}, and as \citet{Sorabji:1971fr,Sorabji:2003fk} emphasizes, Aristotle is extending the domain of perception as Plato conceives of it. Not only are the objects of perception no longer confined to the primary objects---we perceive common and incidental sensibles as well---but Aristotle also maintains that we can discriminate among sensory objects and that this is the exercise of our perceptual capacities. Plato, in contrast, maintained that what is ``common'' to the objects of sense---that they are each the same and different from the others---is determined by cognitive, not perceptual capacities. Aristotle underscores this extension of our perceptual capacities by his use of the Platonic analogy. Aristotle emphasizes the fact that he is assigning to perception some of the functions that Plato assigns to judgment by using the analogy that Plato used to explain judgment to explain perception instead.

Another, perhaps less salient, difference between the Platonic and Aristotelian analogies concerns the nature of the agent acting upon the wax. Where\-as what makes an impression for Plato is a stylus, what makes an impression for Aristotle is a signet ring. Plato has in mind a wax tablet, used for writing, upon which characters are impressed with a stylus. Plato thus belongs to the Western tradition, and perhaps inaugurates it, of using the then current writing technology as a model for the mind. Think of Locke's blank slate, or the functionalist slogan that the mind is the software of the brain, itself coinciding with the emergence of text-editing and word-processing. That thought has the grammatical structure of written language, developed in different ways by Ockham and Fodor, is a close variation. As is Lacan's claim that the unconscious is structured like a language, at least if we regard analysis as a discursive technology. Perhaps Nietzsche was right that our writing tools act upon our thoughts. They at least have a tendency to influence our philosophy of mind. Aristotle varies this aspect of the Platonic analogy. It is not a stylus on a wax tablet that creates the impression, it is a signet ring. Why the substitution? I believe the variation is intentional. Both a stylus and a signet ring are involved in the production of writing. The difference concerns their distinctive discursive roles. The significance of this discursive difference will emerge in sequel. 

Not only does Aristotle deploy the Platonic analogy, varied in this way, to explain his conception of perception, but, importantly, he also transforms how the analogy is understood. Plato's explanation of the reliability of memory and judgment crucially relies on causal features of the situation. An objects' impression is the effect it has on the mind's wax. Importantly, however, Aristotle has in mind a non-causal sense of impression. As we shall see, the distinctive discursive role of signet rings will bear on the sense in which a ring's seal is an impression.

To get a sense of this contrast, first consider how Hume himself appropriates the Platonic analogy:
\begin{quote}
	All the perceptions of the human mind revolves themselves into two distinct kinds, which I shall call \textsc{impressions} and \textsc{ideas}. The difference betwixt these consists in the degree of force or liveliness, with which they strike upon the mind, and make their way into thought and consciousness. (\citealt{Hume:1739kx}, \emph{Treatise} \textsc{i} 1 1 1)
\end{quote}
Hume begs his reader's indulgence in taking liberty with the use of the terms ``impression'' and ``idea'' (\citealt{Hume:1739kx}, \emph{Treatise} \textsc{i} 1 1 n2) though he claims that it is at least a virtue of his regimentation that it ``restore[s] the word, \emph{idea}, to its original sense, from which Mr. \emph{Locke} had perverted it, in making it stand for all our perceptions.'' Hume is presumably taking liberty with his use of the term ``perception'', as well. Locke perverts the use of ``idea'' by making it stand for all perceptions. But perception, here, is not a sensory experience, nor is it confined to the objects of sensory experience. By ``perception'' Hume means whatever is or could be the object of the mind (in, admittedly, a post-Cartesian conception of mind unavailable to the ancients). The perceptions present to the human mind are either impressions or ideas, depending upon the force or liveliness with which they strike the mind. Impressions are the objects that are presented with ``the most force and violence'' (\citealt{Hume:1739kx}, \emph{Treatise} \textsc{i} 1 1 1).  Concerning impressions, Hume further explains:
\begin{quote}
	By the term \emph{impression} I wou'd not be understood to express the manner, in which our lively perceptions are produc'd in the soul, but merely the perceptions themselves; for which there is no particular name either in \emph{English} or any other language, that I know of. (\citealt{Hume:1739kx}, \emph{Treatise}, \textsc{i} 1 1 n2)
\end{quote}
Consistent with this qualification, Hume is, nevertheless, operating with a causal notion of impression. The lively perceptions presented before the mind in viewing the streets of Edinburgh are themselves the effects of external causes. Inquiring after the manner in which such lively perceptions are produced in the soul does not, however, fall within the purview of Hume's new science of human nature. It belongs, rather, to a particular branch of natural philosophy, speculative anatomy. The qualification is not the denial that impressions are effects. It rather signals Hume's intention to confine himself to what might be described in a later terminology as the phenomenology, the intentional objects of consciousness, in particular, to the few regular principles that govern the persistent change among the perceptions present before the mind.

Like Aristotle, Hume is departing from Plato in using the analogy of impressions on wax to describe perceptual experience (and more besides, passions are impressions as well). But Hume retains a key feature of Plato's original use of the analogy. Hume is still thinking of sensory impressions as the effects produced in the perceiver by external objects acting upon them. Is there an intelligible alternative? How else might talk of impressions be understood? 

Consider the closely related metaphor of shaping. There is clearly a causal sense of shaping. When the stylus shapes the wax tablet it causes the wax to be modified in a certain way. The wax takes on the shape imposed upon it by the stylus. Similarly, Nazi bombing shaped the London skyline. It caused that skyline to be configured in a certain way, the way imposed upon it by the bombing. Importantly, however, there is another sense of shaping, not a causal sense, but a constitutive sense. Whereas Nazi bombing shaped the London skyline merely in a causal sense, St Paul's constitutively shapes that skyline by being a contour of it. This is dramatically demonstrated in Herbert Mason's iconic photograph (see figure~\ref{fig:stpauls}). St Paul's defiantly shapes the London skyline by being a part of it, despite the causal impact of Nazi bombing.

\begin{figure}[htbp]
	\centering
		\includegraphics[scale=0.6]{graphics/stpauls.jpg}
	\caption{St Paul's 29 December 1940}
	\label{fig:stpauls}
\end{figure}

Humean sensory impressions are shaped by the environment merely in a causal sense. This is central to Hume's use of the Platonic analogy. Just as a stylus impinging upon the wax causes an impression, the environment impinging upon a perceiver with the appropriate sensory capacities causes a sensory impression. How exactly such sensory impressions are produced is a matter for the speculative a\-na\-to\-mist. Hume's new science of human nature confines itself to sensory impressions and the regularities that can be discerned in the flux of sensory experience. But sensory impressions, episodes in the sensory flux, remain effects, nonetheless. But perhaps perceptual sensitivity is more than the environment impinging upon the state of a conscious subject. Perhaps there is more to perception than objects eliciting a conscious modification of the perceiving subject. Perhaps the environment can shape sensory consciousness in a constitutive, rather than merely a causal, sense.

Before exploring this idea further, let us consider another important, and importantly related, aspect of Aristotle's use of the Platonic analogy. Earlier we observed that Aristotle varies the Platonic analogy by substituting a signet ring for Plato's stylus. What is the significance of this variation? Both are involved in the production of writing. The difference lies in their distinctive discursive roles. Caston observes that the impression produced by a signet ring is linked to that particular ring and, hence, metonymically at least, to the legitimate possessor of that ring:
\begin{quote}
	A signet produces a sealing, an impression that establishes the identity of its owner and consequently his authority, rights, and prerogatives. When a sealing is placed on a document, especially for legal or official use, it authorizes the claims, obligations, promises, or orders made therein. A sealing thus differs from other impressions in that it \emph{purports to originate from a particular signet}. \citep[302]{Caston:2005cr}
\end{quote}
The impression of a signet ring thus plays a similar role to signatures. Just as a signature is linked to the particular person whose signature it is, the impression sealed upon the wax by a signet ring is linked to the legitimate possessor of that ring. Moreover, signatures, like sealed impressions, carry a certain authority, the authority endowed by their legitimate possessors. Of course, signatures can be forged, as can signet rings, which can also be stolen, but these practices gain there point precisely by the link between a signature and sealed impression, on the one hand, and their legitimate possessors, on the other. Signet rings and styli thus have distinctive discursive roles. The impression made by a stylus is not linked to its legitimate possessor---one scribe may borrow another's stylus---the way an impression sealed by a signet ring is.

Taking this feature of the analogy seriously has an important consequence for how sensory impressions are individuated. Just as a forged signature is not my signature, an impression sealed by a forged ring, or by a stolen ring, is not the seal of the ring's legitimate possessor. Impressions are individuated by their legitimate sources. If this feature of the analogy carries over, then perceptions, conceived on the model of sealed impressions, are individuated by their objects which are their source. A perception of Castor and a perception of Pollux are different perceptions, no matter how closely the twins may resemble one another. Just as a forged seal is not my seal, a perception of Castor is not a perception of Pollux. A forged seal may be a perfect duplicate of a genuine seal but it is not the seal of the ring's legitimate possessor. Castor may be a perfect duplicate of Pollux, but my visual impression of Castor is not an impression of Pollux. 

Notice that a causal understanding of sensory impressions, as merely the effects of causal shaping, does not have this consequence. If, as Hume maintained, cause and effect are contingently connected, the same effect, the same impression, could have been produced by a different cause. Sensory impressions, understood as the effects of causal shaping, are not individuated by their causes. If sensory impressions are individuated by their objects which are their sources, they cannot be understood as merely the effects of causal shaping. How else might they be understood?

If sensory impressions are individuated by their objects, perhaps these objects shape sensory consciousness not causally, or at least not merely. Perhaps in being individuated by their objects, these objects constitutively shape our sensory impressions of them (for contemporary discussion of this suggestion see \citealt{McDowell:1998vn,Martin:2004fj,Fish:2009fk,Kalderon:2011fk}). Looking up, you see the brilliance of the late morning sun burning white. The whiteness of the sun is a constituent of your experience. Sensory experience is an encounter with, at least, its primary objects. In sensory consciousness, we simply confront the primary object of the given modality. We cannot be confronted truly or falsely, correctly or incorrectly. We simply encounter what is presented to us in sensory consciousness. The whiteness of the sun is a constituent of your experience insofar as that experience involves the presentation of that whiteness in the visual awareness afforded you by your experience of the sun. And since your experience is constitutively linked to the whiteness of the sun, the sun's whiteness, whose brilliance can inspire both glory and terror, shapes the contours of your visual consciousness by being present in that consciousness. The whiteness of the sun shapes the contours of your visual experience in the way that St Paul's defiantly shapes the London skyline, the Shard notwithstanding, simply by being present. The whiteness of the sun is present in the awareness that sight affords you of the scene. That experience has a certain character. The character of that experience depends upon a derives from the character of the presented whiteness. Your experience, in this sense, becomes like the way the sun actually is, brilliant white. Just as what the London skyline is like depends, in part, upon what St Paul's is like, since the London skyline involves the presence of St Paul's as a part, what your experience of the sun is like depends upon what the sun's whiteness is like, since your experience involves the presentation in sight of that whiteness.

What the London skyline is like depends, in part, upon what St Paul's is like since the London skyline involves the presence of St Paul's as a part. But the way in which the brilliant white of the sun is present in your experience of it is not the way St Paul's is present in the London skyline. The brilliant white of the sun is not a part of your experience, though it may be a constituent. The white of the sun and St Paul's are present in different ways and this makes for a corresponding difference in the sense of likeness that their presence grounds. Thus the London skyline is actually like St Paul's, in sharing a common contour, say. But neither the perceiver nor their perceptual experience becomes actually white in assimilating the chromatic form of the sun. The perceiver, or the perceptual experience, may be like the way the sun actually is, brilliant white, but this does not require that they or their experience be actually white. Echoing \citet{Austin:1962lr} we might say, just because the perceiver, or their perceptual experience, becomes like the perceived object actually is dos not mean that they become exactly like the perceived object. In seeing the white of the sun, the character of your perceptual experience depends upon and derives from the character of the object of visual consciousness as presented to your point of view. But this does not require that perceiver or their perceptual experience actually take on the color of the perceived particular.

Consider, again, assimilation as it figures, not in perception, but in the production of artifacts. In the perceptual case, the perceiver assimilates the form of the perceived object. In the production of artifacts, the direction of assimilation is reversed. It is not a person assimilating the form of an external object, but an external object assimilating a form that is in some sense in the person. The iron in being forged into a sword takes on or assimilates the form contained in the soul of the person possessing the art.  But there is no actual sword in the smith’s soul. So, in seeing the brilliant white of the sun, the perceiver may become like the sun actually is, but there need not be within them anything brilliant white.

In perceiving the color of a remote external particular, the perceiver assimilates that particular's chromatic form. The proposal is that we understand the assimilation of chromatic form as the presented color's constitutively shaping visual experience. In being present in the awareness afforded by visual experience, the color constitutively shapes that experience. The character of that experience depends upon and derives from the character of the presented color. In this sense does color experience become like the presented color. And in this sense is the perceiver, or at least their perceptual experience, potentially like the colored particular actually is prior to perception.

% subsection assimilation_of_form (end)

\subsection{Without Matter} % (fold)
\label{sub:without_matter}

In perception, not only do we assimilate the sensible form of the object of perception, but we do so without assimilating any of its matter. This is a direct consequence of Aristotle's case against the Empedoclean principle, to be perceptible is to be palpable to sense (discussed in chapter~\ref{sec:against_the_empedoclean_principle}). If contact with the sense organ precludes sensation, then the assimilation of anything material by the organ of sensation would be incompatible with the exercise of our perceptual capacities. Concerning this, Aristotle uses the analogy to make two remarks:
\begin{enumerate}[(1)]
	\item that the wax takes on the impression of the signet ring without the bronze or gold, and
	\item that the bronze or gold signet ring produces the impression, but not \emph{qua} bronze or gold.
\end{enumerate}

First, in sealing, the signet ring makes an impression upon the wax. The wax takes on the seal, the form imposed upon it by the signet ring. However, the wax does not take on any of the matter that composes the ring.  The contribution of the ring to the sealed impression is exhausted by the form it imposes. The ring makes no further material contribution to the sealed impression. It leaves no deposits of gold. This feature of the analogy is the summation of Aristotle's case against the Empedoclean principle. Perception may be a mode of assimilation, but it is not the assimilation of anything material. In the case of qualitative alteration, as Aristotle understands it, nothing material is assimilated. What is assimilated is not a body, but a quality or state. And as I have observed in chapter~\ref{sec:transparency_in_de_anima}, whereas bodies have locations, qualities and states do not. Perceptual activity may not be an instance of qualitative alteration, nevertheless, as in the case of alteration, the perceiver, or perhaps their experience, becomes like the way the perceived object actually is. The perceiver, or their experience, assimilates the sensible form of the perceived particular in a manner necessarily distinct from the material assimilation of that particular, since to be palpable is to be imperceptible.

Second, the bronze or gold signet ring produces the impression but not \emph{qua} bronze or gold. Aristotle is characteristically brief in his statement of this claim, which requires some explanation. After all, it is not as if being made of bronze or gold has nothing to do with the power to produce impressions upon wax. As Plato's elaboration of the wax analogy makes clear, to impress a form upon the wax requires that the agent of this change be sufficiently hard and rigid. Styli and signet rings could not be composed of water, say, because water lacks fixed boundaries and so would lack the requisite rigidity. Let the capacity to impose a sealed impression upon wax be the form of a signet ring. There may be material constraints on sustaining that form. Perhaps only matter with certain elemental compositions could be signet rings. Consistent with this, Aristotle is making the further claim that when the bronze or gold signet ring acts upon the wax to produce the sealed impression, the ring is acting but not qua bronze or gold. Aristotle's idea seems to be this. Being made of bronze or gold may contribute to the requisite hardness and rigidity of a signet ring. But when that ring seals an impression upon the wax, the wax takes on the form it does, less because of the golden composition of the ring, say, than because of its shape. To sustain that shape sufficient to impress a recognizable form upon wax may place material constraints on what the ring could be made of. A ring with the same shape could be made of iron or bronze, say, but not of water. But what explains the form that the wax takes on is primarily the shape of the ring that imposes that form.

How would this feature of the analogy carry over to the case of perception?

Vision is an encounter with the visible. Looking up you are momentarily dazzled by the brilliance of the late morning sun. In seeing the sun, you encounter the sun's brilliant whiteness. The sun's whiteness is the primary object of your visual experience. And it is for the sake of presenting the primary objects of vision---the visible, colors in light and the luminous in dark---that you possess eyes to see with. The sun's brilliant whiteness is the power of the sun to act on what is actually transparent. Although, in the case of the sun, given the strength of its illumination, the sun is also the cause of daylight. The sun does not merely have the power to alter what is actually transparent, but it also has the power to make what is potentially transparent actually transparent. The whiteness of the sun is the sun's power to affect light in a certain way, to both produce it and endow it with a certain character. In possessing the power to affect light in this way, the sun also possesses the power to mediately affect organs of sensation sensitive to such alterations in illuminated media. The whiteness of the late morning sun, in all its brilliance, is the cause of its presentation in your visual experience. What explains your taking in the sun's color is the sun's brilliant whiteness being open to your view. Just as the signet ring must have the shape it has prior to imposing its form upon the wax, a particular must have the color that it has prior to its chromatic form being assimilated in visual perception. And just as it is the shape of the signet ring, and not its matter, that primarily explains the form that it imposes upon the wax, it is the color of the particular and not its matter that primarily explains the chromatic form assimilated, the color's presentation in visual consciousness. (For contemporary defense of the claim that colors are explanatorily indispensable in the production of color experience see \citealt{Campbell:1997dq,Broackes:1997pa}; for related discussion in Aristotle scholarship see \citealt{Broadie:1993fk,Broackes:1999uq})

Previously, we distinguished causal and constitutive senses of shaping. Something may causally shape a thing by causing it to have a certain shape. But, importantly, something may constitutively shape a thing, by being a part, or contour, of that thing's shape. I suggested that we might understand the doctrine of the assimilation of sensible form in terms of the primary objects constitutively shaping sensory consciousness. In experiencing the sun's brilliant whiteness, the sun's chromatic form constitutively shapes your visual experience by being present in the awareness that it affords. How does this square with Aristotle's insistence that the assimilation of sensible form is the product of the sensible form assimilated?

In claiming that the sun's brilliant whiteness constitutively shapes your visual experience, the contrast is with the sun's whiteness merely causally shaping that experience. The sun's brilliant whiteness consist in its power to affect light in a certain way, to both produce it and endow it with a certain character. The sun possesses this power prior to being perceived. Indeed it must possess this power prior to perception since it is the cause of that perception, and so must be actually like the perceiver's experience potentially is. The sun's brilliant whiteness is presented in the awareness that your visual experience affords you of the sun. The sun's whiteness shapes the contours of your visual experience by being present in the awareness that it affords. Consistent with its constitutively shaping visual consciousness, the sun's brilliant whiteness brings about its presentation in your consciousness, by altering your eye's interior illumination and so triggering your reactive capacity for sight. The first actuality of color is the cause, in appropriate circumstances, of the second actuality of color, its presentation in visual consciousness. So the primary object of sight, color, causes a suitably placed, awake, and attentive perceiver to stand in a certain relation to it, that of visual awareness. And in standing in that relation, in being presented in visual awareness, the color constitutively shapes the perceiver's visual experience. The object of perception is a cause that brings it about that a perceiver is perceptually related to it, like a wind causing a fire to burn in its direction.

Colors may be causes, but that does not make the shaping of visual experience by color merely causal. The sun's whiteness may elicit a perception of it in suitably placed, awake, and attentive viewers, but what that perceptual experience is like depends upon and derives from what the sun is actually like prior to perception, namely, brilliant white. If color shaped visual experience in a merely casual sense this latter conjunct would not be true. Consider Descartes' \citeyearpar{Descartes:1637uq} striking and paradoxical comparison, in the \emph{Optics}, of color vision with a blind person's use of a stick in navigation (see figure~\ref{fig:blind}). Part of the point of the comparison is that there need be nothing in the objects which resembles the ideas or sensations that we have of them (\citealt{Descartes:1637uq}, \emph{Optics}, First Discourse, 85). Descartes is here supposing that colors are the ways that bodies receive light and reflect it against our eyes (\citealt{Descartes:1637uq}, \emph{Optics}, First Discourse, 85). Nevertheless, he maintains that our sensations of color do not resemble the colors that elicits them. Colors, according to Descartes, shape sensory consciousness merely causally since there is nothing in the colored particular that resembles our visual impression of it. Descartes' claim, here, is representative of a dominant theme in early modern thinking about sensory experience that coincides with an emerging consensus that colors are, in some suitable sense, secondary qualities.

\begin{figure}[htbp]
	\centering
		\includegraphics[scale=2]{graphics/blind.jpeg}
	\caption{\citealt{Descartes:1637uq}}
	\label{fig:blind}
\end{figure}

This is not without ancient precedent. Consider Democritus, at least as presented by Sextus \citeauthor{Empiricus:1997kx} (\emph{Against the Logicians}, \emph{adv. math.} \textsc{vii} 135--136). Linguistic conventions may license, in certain circumstances, our predicating ``white'' of the sun given the character of the sensory experience that it elicits, but there is nothing corresponding to this predication over and above our sensory reaction to atomic stimuli. What is novel in early modern thinking about sensory experience is not the idea that the causes of color perception do not resemble the colors that perception purports to present. That idea is present in the ancient atomists and arguably has Parmenidean roots. What is novel about the early modern period is the enthusiasm with which this doctrine was met, and the widespread consensus that emerged, a consensus that remarkably persisted for four centuries. It is only against the background of such a consensus that \citet[]{Hume:1748zr} can claim that it takes but the slightest bit of philosophy to show that there are no colors that inhere in bodies that resemble our impressions of them. While there may be ancient precedent for this doctrine, there is nothing like the modern consensus in the ancient world. There is no ancient correlate of the modern paradigm that would render intelligible Hume's dismissiveness. Aristotle himself is an important and notable dissenter. As the great defender of the manifest image in the classical world, Aristotle defends a pre-Parmenidean perceptual realism by post-Parmenidean means. (For a recent different account of these matters see \citealt{Lee:2011ys}.)

Colors that constitutively shape visual experience may be causes of such experiences, but the character of our visual experience depends upon and derives from the character of the presented color. In this sense does the perceptual experience come to resemble its object in assimilating its sensible form. Should color experience not resemble, in the relevant sense, the colors that are their objects, then colors would shape our visual experience merely causally, as Democritus and Descartes maintained. According to Aristotle's perceptual realism, colors are causes of color perceptions that resemble them, by perceptions being the presentation of the colors.

We have been discussing Aristotle's two remarks about the material aspect of the wax analogy: (1) that the wax takes on the impression of the signet ring without the bronze or the gold, and (2) that the bronze or gold signet ring produces the impression but not qua bronze or gold. The first remark is the summation of Aristotle's case against the Empedoclean principle. Color perception may be the assimilation of chromatic form, but nothing material is assimilated. The second remark is the expression of a color realism that was rejected by early modern thinkers---not only by Descartes, but by a diverse group of thinkers that includes Galileo, Locke, Boyle, and others as well. According to this premodern realism, it is the color of the particular and not its matter that primarily explains the perception of its color. Colors are causes of color perceptions that resemble them, in the sense in which they can, by color perception being the presentation of the colors. Color's presence in visual awareness shapes the experience that affords that awareness in that its character depends upon and derives from the character of the presented color. Nevertheless, consistent with this, color is the cause of its presentation in perception. Color, being the power to affect light, has the derived power to mediately affect the organ of sight by affecting the intervening medium. In altering the character of the illumination in the eye's interior, color triggers the reactive capacity for sight whose exercise is the presentation in visual awareness of that color. Like a wind causing a fire to burn in its direction, color causes a perceiver to stand in a perceptual relation to it, that of visual awareness. And in being so presented in visual awareness color constitutively shapes color experience.

% subsection without_matter (end)

\subsection{Sensory Presentation as a Mean} % (fold)
\label{sub:sensory_presentation_as_a_mean}

% subsection sensory_presentation_as_a_mean (end)

% section the_wax_analogy (end)

\section{The Resolution of Empedoclean Puzzlement} % (fold)
\label{sec:the_resolution_of_empedoclean_puzzlement}

Empedoclean puzzlement about the sensory presentation of the colors of distal particulars was due, in part, to Empedocles' adherence to a general conception of sensory awareness for which ingestion provides the model. Central to that model was the Empedoclean principle, to be perceptible is to be palpable to sense. Emedoclean puzzlement about the sensory presentation of color consists in the apparent tension between two claims:
\begin{enumerate}[(1)]
    \item The objects of color perception are qualities of external particulars located at a distance from the perceiver.
    \item \emph{The Empedoclean principle}: To be perceptible is to be palpable to sense---in order for something to be the object of perception it must be in contact with the relevant sense organ.
\end{enumerate}
Effluences in Empedocles' theory of vision are meant to resolve this puzzle by explaining how the colors of remote external particulars may be palpable to the organ of sight. Distant objects may be sensed by sensing the material effluences they emit. If the color of an object is the material effluence that it emits, then the color of a remote object can be assimilated and so be palpable to sight. Not only does Aristotle reject the theory of effluences, and with it Empedocles' own resolution of the puzzle, but he also rejects the principle that generated the puzzle that effluences were meant to resolve. Specifically, he rejects the Empedoclean principle, to be perceptible is to be palpable to sense. Far from being a material precondition for the sensing of color, contact precludes sensation. A colored particular's contact with the eye, the organ of sight, blinds the perceiver to that particular and its color. To be palpable is to be imperceptible. Aristotle's rejection of the Empedoclean principle is a resolution of Empedoclean puzzlement, at least in its original form, precisely by that puzzlement being generated by the tension between the Empedoclean principle and the claim that the objects of color perception are qualities of external particulars located at a distance from the perceiver.

Aristotle, nevertheless, retains a conception of perception as a mode of assimilation even as he rejects the ingestion model. The naturalness of thinking of seeing as taking in the external scene before one persists even after rejecting the Empedoclean principle. This natural thought gives rise to a residual puzzlement. How can one take in what remains external? And if one can, what could taking in mean, here, such that one could? Empedoclean puzzlement, in its most general form, consists in the persistence of this latter question. 

How can one take in what remains external? If the generalized form of the Empedoclean puzzlement consists in the persistence of this question, then Aristotle's definition of perception can be understood to address this puzzlement precisely by offering an answer. A perceiver takes in what remains external by assimilating the chromatic form of the remote external particular while leaving its matter in place. Aristotle's definition of perception as the assimilation of sensible form without matter is meant to address the generalized form of Empedoclean puzzlement. It is meant to be the sense in which the perceiver takes in the scene before one. More specifically, it is meant to be the sense in which the perceiver takes in what remains external. 

In seeing, we take in the scene before us by assimilating the chromatic form of the particulars arrayed in that scene, by our experience being constitutively shaped by their color. In chapter~\ref{sec:the_general_characterization_of_perception}, the epistemological significance of the doctrine of the assimilation of sensible form was stressed. If the perceiver becomes like the way the perceived object actually is, in the sense that their perceptual experience is constitutively shaped by that object, then it is impossible for their experience to be as it is and that object be some way other than it actually is at least in its sensible respects. The assimilation of sensible form thus underwrites the objectivity of perceptual content. Moreover, this objectivity is achieved quite independently of any spatial contact with the object of perception. In Aristotle's definition, the qualification, without matter, emphasizes just this point. Only matter is spatially located, and thus only matter may be in contact with the organ of sensation. 


The Giants were rightly impressed by the reality of what can be handled and offers resistance to touch, though they went too far in insisting that only the palpable was real. Virtues and capacities are real but are not palpable. And Empedocles was wrong in thinking that it is only by being in contact with the sense organ may an object be sensed. To be palpable is to be imperceptible. Grasping may be a phenomenologically vivid and primitively compelling mode of sensory awareness, but to model all sensory awareness on our capacity to grasp material bodies is to overgeneralize what is by Aristotle's lights a misconception of a paradigm case of sensory presentation. In 

% section the_resolution_of_empedoclean_puzzlement (end)

\section{Presence and Modernity} % (fold)
\label{sec:presence_and_modernity}

There is a form of skepticism about presence found within the phenomenological tradition. According to that skepticism, modern philosophy is ultimately unsustainable because of its adherence to the metaphysics of presence. Consider, however, not presence more generally, but sensory presentation more specifically. Even so restricted, phenomenological skepticism about sensory presentation is a large topic. Detailing the variety of forms such skepticism takes would require a monograph of its own. However, I would like to end with an observation. 

If the metaphysics of presence cannot be coherently sustained within modern philosophy, this is not because of any contradiction or incoherence within the metaphysics of presence, but because modern philosophy abandoned presence from the beginning. 

% section presence_and_modernity (end)

% chapter form_without_matter (end)
