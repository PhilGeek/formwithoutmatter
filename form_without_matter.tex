%!TEX root = /Users/markelikalderon/Documents/Git/formwithoutmatter/aristotle.tex
\chapter{Form Without Matter} % (fold)
\label{cha:form_without_matter}

\section{The Generalized Form of Empedoclean Puzzlement} % (fold)
\label{sec:the_generalized_form_of_empedoclean_puzzlement}

% section the_generalized_form_of_empedoclean_puzzlement (end)

\section{The Wax Analogy} % (fold)
\label{sec:the_wax_analogy}

The assimilation of sensible form without matter is explained in terms of an analogy with wax receiving the impression of a signet ring:
\begin{quote}
	Generally, about all perception, we can say that a sense is what has the power of receiving into itself the sensible forms of things without the matter, in the way in which a piece of wax takes on the impress of a signet-ring without the iron or gold; what produces the impression is a signet of bronze or gold, but not qua bronze or gold: in a similar way the sense is affected by what is coloured or flavoured or sounding not insofar as each is what it is, but insofar as it is of such and such a sort and according to its form. (\emph{De Anima} \textsc{ii}.12 424\( ^{a} \)18--23; Smith in \citealt[42--43]{Barnes:1984uq})
\end{quote}
While denying that sight involves the assimilation of material effluences, Aristotle retains Empedocles' conception of sensory awareness as a mode of assimilation, it is just that we assimilate form without matter. Indeed, this pattern of dialectical refinement continues in the very next line where Aristotle uses Plato's metaphor of wax receiving an impression, not to characterize judgment as Plato does in the \emph{Theaetetus} 194\( ^{c} \)--195\( ^{a} \), but to characterize the assimilation of sensible form in perception. The analogy is making two points. First the assimilation of sensible form is compared with the wax's reception of the impression sealed by the signet ring. Second, Aristotle emphasizes what the wax receives, namely, the form of the seal, it's shape. The wax receives the form imposed upon it by the signet ring, but if does not receive any of the matter that composes the ring, be it bronze or gold, say. So, by analogy, when an animal perceives the white of the sun, they assimilate the chromatic form of the sun but none of its matter. Moreover, just as it is the form of the ring, and not its gold or bronze, that produces the sealed impression, its distinctive shape, it is the whiteness of the sun, and not its matter, that produces the sensory impression, the perceptual experience of the white of the sun. We shall consider these points separately in the following subsections.

\subsection{Assimilation of Form} % (fold)
\label{sub:assimilation_of_form}

Aristotle's analogy is, as we observed, of Platonic provenance. In the \emph{Theaetetus}, Plato writes:
\begin{quotation}
	\textsc{socrates}: You will think better of it when you hear the rest. To judge truly is a fine thing and there is something discreditable in error.
	
	\textsc{theaetetus}: Of course.
	
	\textsc{socrates}: Well, they say the differences arise in this way. When a man has in his mind a good thick slab of wax, smooth and knealed to the right consistency, and the impressions that come through the senses are stamped on these tables of the `heart'---Homer's words hints at the mind's likeness to wax---then the imprints are clear and deep enough to last a long time. Such people are quick to learn and also have good memories, and besides they do not interchange the imprints of their perceptions but think truly. These imprints being distinct and well spaced are quickly assigned to their several stamps---the `real things' as they are called---and such men are said to be clever. Do you agree?
	
	\textsc{theaetetus}: Most emphatically.
	
	\textsc{socrates}: When a person has what the poet's wisdom commends as a `shaggy heart', or when the block is muddy or made of impure wax, or oversoft or hard, the people with soft wax are quick to learn, but forgetful, those with hard wax, the reverse. Where it is shaggy or rough, a gritty kind of stuff containing a lot of earth or dirt, the impressions obtained are indistinct; so are they too when the stuff is hard, for they have no depth. Impressions in soft wax also are indistinct, because they melt together and soon become blurred. And if besides this, they overlap through being crowded together into some wretched little narrow mind, they are still indistinct. All these types are likely to judge falsely. When they see or hear or think of something, they cannot quickly assign things to their several imprints. Because they are so slow and sort things into their wrong places, they constantly see and hear and think amiss, and say they are mistaken about things and stupid. (Plato, \emph{Theaetetus} 194\( ^{c} \)--195\( ^{a} \); Cornford in \citealt{Hamilton:1961fk})
\end{quotation}

One salient difference between the Platonic and Aristotelian analogies is that whereas Plato deploys the wax analogy to explain judgment, Aristotle does so to explain perception. This difference is most likely intentional and pointed. As we discussed in chapters~\ref{sec:definition} and \ref{sec:the_objects_of_perception}, and as \citet{Sorabji:1971fr,Sorabji:2003fk} emphasizes, Aristotle is extending the domain of perception as Plato conceives of it. Not only are the objects of perception no longer confined to the primary objects---we perceive common sensibles as well---but Aristotle also maintains that we can discriminate among sensory objects and that this is the exercise of our perceptual capacities. Plato, in contrast, maintained that what is ``common'' to the objects of sense---that they are each the same and different from the others---is determined by cognitive, not perceptual capacities. Aristotle underscores this extension of our perceptual capacities by his use of the Platonic analogy. Aristotle emphasizes the fact that he is assigning to perception some of the functions that Plato assigns to judgment by using the analogy that Plato used to explain judgment to explain perception instead.

Another, perhaps less salient, difference between the Platonic and Aristotelian analogies concerns the nature of the agent acting upon the wax. Where\-as what makes an impression for Plato is a stylus, what makes an impression for Aristotle is a signet ring. Plato has in mind a wax tablet, used for writing, upon which characters are impressed with a stylus. Plato thus belongs to the Western tradition, and perhaps inaugurates it, of using the then current writing technology as a model for the mind. Think of Locke's blank slate, or the functionalist slogan that the mind is the software of the brain, itself coinciding with the emergence of text-editing and word-processing. That thought has the grammatical structure of written language, developed in different ways by Ockham and Fodor, is a close variation. As is Lacan's claim that the unconscious is structured like a language, at least if we regard analysis as a discursive technology. Perhaps Nietzsche was right that our writing tools act upon our thoughts. They at least have a tendency to influence our philosophy of mind. Aristotle varies this aspect of the Platonic analogy. It is not a stylus on a wax tablet that creates the impression, it is a signet ring. Both a stylus and a signet ring are involved in the production of writing. The difference concerns their distinctive roles in writing production. Why the substitution? I believe the variation is intentional. It's significance will emerge in sequel. 

Not only does Aristotle deploy the Platonic analogy to explain his conception of perception, but, importantly, he also transforms how the analogy is understood. Plato's explanation of the reliability of memory and judgment crucially relies on causal features of the situation. An objects' impression is the effect it has on the mind's wax. Importantly, however, Aristotle has in mind a non-causal sense of impression.

To get a sense of this, first consider how Hume himself appropriates the Platonic analogy:
\begin{quote}
	All the perceptions of the human mind revolves themselves into two distinct kinds, which I shall call \textsc{impressions} and \textsc{ideas}. The difference betwixt these consists in the degree of force or liveliness, with which they strike upon the mind, and make their way into thought and consciousness. (Hume, \emph{Treatise} \textsc{i}.1.1.1)
\end{quote}
Hume begs his reader's indulgence in taking liberty with the use of the terms ``impression'' and ``idea'' (\emph{Treatise} \textsc{i}.1.1 n2) though he proclaims that it is at least a virtue of his regimentation that it ``restore[s] the word, \emph{idea}, to its original sense, from which Mr. \emph{Locke} had perverted it, in making it stand for all our perceptions.'' Hume is presumably taking liberty with his use of the term ``perception'', as well. Locke perverts the use of ``idea'' by making it stand for all perceptions. But perception, here, is not a sensory experience, nor is it confined to the objects of sensory experience. By ``perception'' Hume means whatever is or could be the object of the mind (in, admittedly, a post-Cartesian conception of mind unavailable to the ancients). The perceptions present to the human mind are either impressions or ideas, depending upon the force or liveliness with which they strike the mind. Impressions are the objects that are presented with ``the most force and violence'' (\emph{Treatise} \textsc{i}.1.1.1).  Concerning impressions, Hume further explains:
\begin{quote}
	By the term \emph{impression} I wou'd not be understood to express the manner, in which our lively perceptions are produc'd in the soul, but merely the perceptions themselves; for which there is no particular name either in \emph{English} or any other language, that I know of. (\emph{Treatise}, \textsc{i}.1.1 n2)
\end{quote}
Consistent with this qualification, Hume is, nevertheless, operating with a causal notion of impression. The lively perceptions presented before the mind in viewing the streets of Edinburgh are themselves the effect of external causes. Inquiring after the manner in which such lively perceptions are produced in the soul does not, however, fall within the purview of Hume's new science of human nature. It belongs, rather, to a particular branch of natural philosophy, speculative anatomy. The qualification is not the denial that impressions are effects. It rather signals Hume's intention to confine himself to, what might be described in a later terminology as, the phenomenology, the intentional objects of consciousness, in particular, to the few regular principles that govern the persistent change among the perceptions presented to the mind.

Like Aristotle, Hume is departing from Plato in using the analogy of impressions on wax to describe perceptual experience (and more besides, passions are impressions as well). But Hume retains a key feature of Plato's original use of the analogy. Hume is still thinking of sensory impressions as the effects produced in the perceiver by external objects acting upon them. Is there an intelligible alternative? How else might talk of impressions be understood? 

Consider the closely related metaphor of shaping. There is clearly a causal sense of shaping. When the stylus shapes the wax tablet it causes the wax to be modified in a certain way. The wax takes on the shape imposed upon it by the stylus. Similarly, Nazi bombing shaped the London skyline. It caused that skyline to be configured in a certain way, the way imposed upon it by the bombing. Importantly, however, there is another sense of shaping, not a causal sense, but a constitutive sense. Whereas Nazi bombing shaped the London skyline merely in a causal sense, St Paul's constitutively shapes that skyline by being a contour of it. This is dramatically demonstrated in Herbert Mason's iconic photograph (see figure~\ref{fig:stpauls}). St Paul's defiantly shapes the London skyline by being a part of it, despite the causal impact of Nazi bombing.

\begin{figure}[htbp]
	\centering
		\includegraphics[scale=0.6]{graphics/stpauls.jpg}
	\caption{St Paul's 29 December 1940}
	\label{fig:stpauls}
\end{figure}

Humean sensory impressions are shaped by the environment merely in a causal sense. This is central to Hume's use of the Platonic analogy. Just as a stylus impinging upon the wax causes an impression, the environment impinging upon a perceiver with the appropriate sensory capacities causes a sensory impression. How exactly such sensory impressions are produced is a matter for the speculative anatomist. Hume's new science of human nature confines itself to sensory impressions and the regularities that can be discerned in the flux of sensory experience. But sensory impressions, episodes in the sensory flux, remain effects, nonetheless. But perhaps perceptual sensitivity is more than the environment impinging upon the state of a conscious subject. Perhaps there is more to perception than objects eliciting a conscious modification of the perceiving subject. Perhaps the environment can shape sensory consciousness in a constitutive, rather than merely a causal, sense.

Before exploring this idea further, let us first consider another important aspect of Aristotle's use of the Platonic analogy. Earlier we observed that Aristotle varies the Platonic analogy by substituting a signet ring for Plato's stylus. What is the significance of this variation? Caston observes that the impression produced by a signet ring is linked to that particular ring and, hence, metonymically at least, to the legitimate possessor of that ring:
\begin{quote}
	A signet produces a sealing, an impression that establishes the identity of its owner and consequently his authority, rights, and prerogatives. When a sealing is placed on a document, especially for legal or official use, it authorizes the claims, obligations, promises, or orders made therein. A sealing thus differs from other impressions in that it \emph{purports to originate from a particular signet}. \citep[302]{Caston:2005cr}
\end{quote}
The impression of a signet ring thus plays a similar role to signatures. Just as a signature is linked to the particular person whose signature it is, the impression sealed upon the wax by a signet ring is linked to the legitimate possessor of the ring. Moreover, signatures, like sealed impressions, carry a certain authority, the authority endowed by their legitimate possessors. Of course, signatures can be forged, as can signet rings, which can also be stolen, but these practices gain there point precisely by the link between a signature and sealed impression, on the one hand, and their legitimate possessors, on the other. 

Taking this feature of the the analogy seriously has an important consequence for how sensory impressions are individuated. Just as a forged signature is not my signature, an impression sealed by a forged ring, or by a stolen ring, is not the seal of the ring's legitimate possessor. Impressions are individuated by their legitimate sources. If this feature of the analogy carries over, then perceptions, conceived on the model of sealed impressions, are individuated by their objects. A perception of Castor and a perception of Pollux are different perceptions, no matter how closely the twins may resemble one another. Just as a forged seal is not my seal, a perception of Castor is not a perception of Pollux. A forged seal may be a perfect duplicate of a genuine seal but it is not the seal of the ring's legitimate possessor. Castor may be a perfect duplicate of Pollux, but my visual impression of Castor is not an impression of Pollux. 

If we take this feature of the analogy seriously, it gives expression in \emph{De Anima} to a doctrine of the \emph{Metaphysica}. Sense and sensibilia, perception and the perceptible, are correlatives. In part, they are correlatives because sense perception is relative to its object. But if sense perception is relative to its object, in the sense in which it is, then a difference in object will result in a difference in perception. And it is precisely this that the signet ring of the analogy gives expression to. Perception is like an impression sealed by a signet ring in the manner of its individuation. Just as a sealed impression is individuated by the ring's legitimate possessor, perception is individuated by its object.

Notice that a causal understanding of sensory impressions, as merely the effects of causal shaping, does not have this consequence. If, as Hume maintained, cause and effect are contingently connected, the same effect, the same impression, could have been produced by a different cause. Sensory impressions, understood as the effects of causal shaping, are not individuated by their causes. If sensory impressions are individuated by their objects, they cannot be understood as merely the effects of causal shaping. How else might they be understood?

If sensory impressions are individuated by their objects, perhaps these objects shape sensory consciousness not causally, or at least not merely. Perhaps sensory impressions are individuated by their objects because their objects constitutively shape sensory consciousness (for contemporary discussion of this suggestion see \citealt{McDowell:1998vn,Martin:2004fj,Fish:2009fk,Kalderon:2011fk}). Looking up, you see the sun burning white. The whiteness of the sun is a constituent of your experience. Sensory experience is an encounter with, at least, its primary objects. In sensory consciousness, we simply confront the primary object of the given modality. We cannot be confronted truly or falsely, correctly or incorrectly. We simply encounter what is presented to us in sensory consciousness. The whiteness of the sun is a constituent of your experience insofar as that experience involves the presentation of that whiteness in the visual awareness afforded you by your experience of the sun. And since your experience is constitutively linked to the whiteness of the sun, the sun's whiteness, whose dazzling brightness can inspire both glory and terror, shapes the contours of your visual consciousness by being present in that consciousness. The whiteness of the sun shapes the contours of your visual experience in the way that St Paul's defiantly shapes the London skyline, the Shard notwithstanding, simply by being present. The whiteness of the sun is present in the awareness that sight affords you of the scene. That experience has a certain character. The character of that experience depends upon a derives from the character of the presented whiteness. Your experience, in this sense, becomes like the way the sun actually is, burning white. What your experience of the sun's whiteness is like depends upon what the sun's whiteness is actually like since your experience involves the presentation in sight of that whiteness.



% subsection assimilation_of_form (end)

\subsection{Without Matter} % (fold)
\label{sub:without_matter}

In perception, not only do we assimilate the sensible form of the object of perception, but we do so without assimilating its matter. This is an direct consequence of Aristotle's argument against Empedocles. If contact with the sense organ precludes sensation, if to be palpable to is to be imperceptible, then the assimilation of anything material would be incompatible with the exercise of our perceptual capacities. Concerning this, Aristotle uses the analogy to make two remarks. First, that the wax takes on the impression of the signet ring without the bronze or gold. Second, that the bronze or gold signet ring produces the impression, but not qua bronze or gold.

In sealing, the signet ring makes an impression upon the wax. The wax takes on the seal, the form imposed upon it by the signet ring. However, the wax does not take on any of the matter that composes the ring.  The contribution of the ring to the sealed impression is exhausted by the form it imposes. The ring makes no further material contribution to the sealed impression. This feature of the analogy is the summation of Aristotle's critique of Empedocles. Perception may be a mode of assimilation, but it is not the assimilation of anything material.

The bronze or gold signet ring produces the impression but not qua bronze or gold. Aristotle is characteristically brief in his statement of this claim, which requires some explanation. First, it is not as if being made of bronze or gold has nothing to do with the power to produce impressions upon wax. As Plato's elaboration of the wax analogy makes clear, to impress a form upon the wax requires that the agent of this change be sufficiently hard and rigid. Styli and signet rings could not be composed of water, say, because water lacks fixed boundaries and so would lack the requisite rigidity. Let the capacity to impose a sealed impression upon wax be the form of a signet ring. There may be material constraints on sustaining that form. Perhaps only matter with elemental compositions of certain kinds could be signet rings. Consistent with this, Aristotle is making the further claim that when the bronze or gold signet ring acts upon the wax to produce the sealed impression, the ring is acting but not qua bronze or gold. What is the content of this further claim?

Aristotle's idea seems to be this. Being made of bronze or gold may contribute to the requisite hardness and rigidity of a signet ring. But when that ring seals an impression upon the wax, the wax takes on the form it does, less because of the golden composition of the ring, say, than because of its shape. To sustain that shape sufficient to impress a recognizable form upon wax may place material constraints on what the ring could be made of. A ring with the same shape could be made of iron or bronze, say, but not of water. But what explains the form that the wax takes on is primarily the shape of the ring that imposes that form.

How would this feature of the analogy carry over to the case of perception?

Looking up you are momentarily dazzled by the glare of the late morning sun. In seeing the sun, as you did, you encountered the sun's brilliant whiteness. The sun's brilliant whiteness is the primary object of your visual experience. And it is for the sake of presenting the primary objects of vision, the visible, colors in light and the luminous in dark, that you possess eyes to see with. Vision is an encounter with the visible. The sun's brilliant whiteness is the power of the sun to act on what is actually transparent. Although in the case of the sun, given the strength of its illumination, the sun is the cause of daylight. The sun does not merely have the power to alter what is actually transparent, but it also has the power to make what is potentially transparent actually transparent. The whiteness of the sun is the sun's power to alter the illuminated media (even if the sun is itself the source of that illumination). In possessing the power to affect light in a certain way, the sun also possesses the power to mediately affect organs of sensation sensitive to such alterations in illuminated media. The whiteness of the sun, in all its brilliance, is the cause of its presentation in your visual experience. The presentation of the sun's brilliant whiteness is, according to \emph{De Anima}, the assimilation of sensible form. What explains the sensible form that is assimilated, in the sense in which it is, is  primarily the sensible form of the remote external particular possessed prior to its perception. What explains your taking in the sun's brilliant whiteness is the sun's brilliant whiteness.

In the previous subsection, we distinguished causal and constitutive senses of shaping. Something may causally shape a thing by causing it to have a certain shape. But, importantly, something may constitutively shape a thing, by being a part, or contour, of that thing's shape. I suggested that we might understand the doctrine of the assimilation of sensible form in terms of the primary objects constitutively shaping sensory consciousness. In experiencing the sun's brilliant whiteness, the sun's chromatic form shapes your visual experience. How does this square with Aristotle's insistence that the assimilation of sensible form is the product of the sensible form assimilated?

In claiming that the sun's brilliant whiteness constitutively shapes your visual experience, the contrast is with the sun's brilliant whiteness merely causally shaping your visual experience. The sun's brilliant whiteness consist in its power to affect light in a certain way, to both produce it and endow it with a certain character. The sun possesses this power prior to being perceived. Indeed it must possess this power prior to perception since it is the cause of that perception, and so must be actually like the perceiver's experience potentially is. The sun's brilliant whiteness is presented in the awareness that your visual experience affords you. The sun's whiteness shapes the contours of your visual experience by being present in that experience. Consistent with its constitutively shaping visual consciousness, the sun's brilliant whiteness brings about its presentation in that consciousness, by altering the eye's interior illumination and so triggering the reactive capacity for sight. 

Color may be a cause, but that does not make color's shaping of visual experience merely causal. The sun's brilliant whiteness may elicit a perception of it in suitably placed, awake, and attentive viewers, but what that perceptual experience is like depends upon and derives from what the sun is actually like prior to perception, namely, brilliant white. If color shaped visual experience in a merely casual sense this latter conjunct would not be true. Consider Descartes' striking and paradoxical comparison, in the \emph{Optics}, of color vision with a blind person's use of a stick in navigation. Part of the point of the comparison is that there need be nothing in the objects which resembles the ideas or sensations that we have of them (\emph{Optics}, First Discourse, 85). Descartes is here supposing that colors are the ways that bodies receive light and reflect it against our eyes (\emph{Optics}, First Discourse, 85). Thus our sensations of color do not resemble the colors that elicits them. Colors, according to Descartes, shape sensory consciousness merely causally since there is nothing in the colored particular that resembles our visual impression of it. 

% Sealing an impression does not involve the ring contributing not only the form of the seal but some of its matter, the gold that composes it, as well, perhaps as a fine deposit.

% subsection without_matter (end)

% section the_wax_analogy (end)

\section{The Resolution of Empedoclean Puzzlement} % (fold)
\label{sec:the_resolution_of_empedoclean_puzzlement}



% section the_resolution_of_empedoclean_puzzlement (end)

\section{Presence and Modernity} % (fold)
\label{sec:presence_and_modernity}

If the metaphysics of presence cannot be coherently sustained within modern philosophy, this is not because of any contradiction or incoherence within the metaphysics of presence, but because modern philosophy abandoned presence from the beginning. 

% section presence_and_modernity (end)

% chapter form_without_matter (end)
