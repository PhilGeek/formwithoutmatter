%!TEX root = /Users/markelikalderon/Documents/Git/formwithoutmatter/aristotle.tex
\chapter{Perception at a Distance} % (fold)
\label{cha:perception_at_a_distance}

In its original form, Empedoclean puzzlement about the sensory presentation of remote objects consists in the apparent tension between two claims:
\begin{enumerate}[(1)]
    \item The objects of color perception are qualities of external particulars located at a distance from the perceiver.
    \item \emph{The Empedoclean principle}: To be perceptible is to be palpable to sense---in order for something to be the object of perception it must be in contact with the relevant sense organ.
\end{enumerate}
Short of embracing the theory of effluences, how might one respond to Empedoclean puzzlement as it arises in its original form? Either (1) or (2) may be rejected. 

In case it is unobvious that anyone would reject (1), Parmenides counts among its deniers. Specifically, Parmenides (DK 28\textsc{b}8.41) claims that it merely appears that things alter their color. The Way of Mortal Opinion, in maintaining otherwise, conflates appearance with reality. Strictly speaking, this claim is consistent with remote external particulars having unchanging colors. However, if we bear in mind the broader philosophical context, it is plausible that Parmenides meant to deny not just that things alter their color as they appear to do, but that things are colored at all. Colors are qualities that appear in sensory experience and are in this sense part of the sensible world. The sensible world is associated with becoming as opposed to being. (Perhaps, the underlying thought is that it is characteristic of sensory experience that it presents us with a flux of sensible qualities.) Sensible qualities, qualities that appear in sensory experience, are subject to change. Since change is impossible, things merely appear to have sensible qualities. The attributes of the one being are intelligible as opposed to sensible. There are thus no qualities of external particulars that are the objects of color perception. 

Despite his shift to a pluralist metaphysics, Democritus retains Parmenides' color irrealism. Concerning Democritus, Sextus Empiricus reports:
\begin{quote}
	And Democritus in some places abolishes the things that appear to the senses and asserts that none of them appears in truth but only in opinion, the true fact in things existent being the existence of atoms and void; for ``By convention,'' he says, ``is sweet, by convention bitter, by convention hot, by convention cold, by convention color; but by verity atoms and void.'' (This means: Sensible objects are conventionally assumed and opined to exist, but they do not truly exist, but only the atoms and the void.) (Sextus \citeauthor{Empiricus:1997kx}, \emph{Against the Logicians}, \emph{adv. math.}, vii, 135--136) 
\end{quote}
Linguistic conventions may license, in certain circumstances, our predicating ``white'' of the sun given the character of the sensory experience it elicits, but there is nothing corresponding to this predication over and above this reaction to atomic stimuli. There are thus no qualities of external particulars that are the objects of color perception. 

Parmenides and Democritus deny (1) by denying the existence of the colors. A more modern denial of (1) retains the existence of the colors but denies that they are qualities of external particulars. Thus while \citet{Berkeley:1734fk} maintains that colors are the objects of sight, he denies that they are qualities inherent in external bodies. 

In addition, instead of rejecting either (1) or (2), one may propose an alternative reconciliation, one not involving the theory of effluences. 

Aristotle clearly accepts (1)---that the objects of color perception are qualities of external particulars located at a distance from the perceiver. Importantly, this is the result, in part, of some fundamental commitments that Aristotle undertakes with respect to the nature of our perceptual capacities and their objects. Moreover, when combined with the Empedoclean principle, to be perceptible is to be palpable to sense, it generates precisely the puzzlement that the theory of effluences is designed to resolve. However, Aristotle rejects Empedocles' theory of effluences. And with no obvious alternative reconciliation to hand, Aristotle is committed to rejecting (2). 

The nature of Aristotle's case is our present topic. First, we will discuss the fundamental commitments about our perceptual capacities and their objects that underly Aristotle's acceptance of (1)---that the objects of color perception are qualities of external particulars located at a distance from the perceiver. Second, we will discuss Aristotle's argument against (2)---the Empedoclean principle that to be perceptible is to be palpable to sense. As we shall see, according to Aristotle, not only is the Empedoclean principle an overgeneralization from a paradigm case, but a misconception of it as well. Third, Empedoclean puzzlement, in its most general form, survives the rejection of the principle that to be perceptible is to be palpable to sense. There thus remain unanswered questions in rejecting (2). Moreover, these questions partly set the agenda in Aristotle's discussion of perception (the present essay, as a whole, constitutes an argument for this latter claim). 

\section{The Sensible Qualities of Remote External Particulars} % (fold)
\label{sec:sensible_qualities_of_remote_external_particulars}

Color perception involves the visual presentation of qualities of remote external particulars. The particulars are \emph{external} in that they exist and have their natures and powers independently of the perceiver. The relevant sense of independence may guarantee that the perceiver and the particular are spatially non-coincident (since two bodies cannot occupy the same space at the same time), but it does not guarantee that they are non-contiguous. Spatially non-coincident bodies may be in contact with one another. Thus talk of \emph{remote} external particulars is not redundant. Their remoteness consists in the non-contiguity of the perceiver and external particular.

In this section we shall discuss the general reasons that underly Aristotle's commitment to three claims:
\begin{enumerate}
    \item The objects of perception are external.
    \item The objects of perception are particulars and their qualities.
    \item The objects of perception can be remote from the perceiver.
\end{enumerate}
Taken together they explain Aristotle's commitment the objects of color perception being qualities of external particulars located at a distance from the perceiver. It is the conjunction of this commitment and the Empedoclean principle that generates the puzzlement that Empedocles' theory of effluences was designed to resolve.

\subsection{External} % (fold)
\label{sub:external}

Sensible qualities inhere in external particulars. They are external in that they exist and have the natures and powers they do independently of being perceived. The relevant sense of independence is usefully highlighted by contrasting Aristotle's position with Protagoras'.

Protagoras notoriously claims that man is the measure of all things. According to Plato's \emph{Theaetetus} and Aristotle's \emph{Metaphysics} \( \Gamma \), Protagoras' measure doctrine is supported by an account of perception where the relation between perception and the perceptible is symmetrical. Neither the perception nor the object of perception takes precedence over one another. Perception and the object of perception, so understood, are correlatives. Moreover, this account of the relation of perception and its object is meant to have the startling consequence that nothing we perceive exists prior to our perception. Perhaps surprisingly, Aristotle accepts that perception and its object are correlatives. Nevertheless, on his own account of relatives, the startling Protagorean consequence is avoided. Indeed, consistent, with perception and the perceptible being correlatives, Aristotle maintains that the perceptible is prior to perception.

In \emph{Categories} 7, Aristotle defines \emph{relatives} as follows:
\begin{quote}
    We call \emph{relatives} all such things as are said to be just what they are, \emph{of} or \emph{than} things, or in some other way \emph{in relation to} something else. (\emph{Categories} 7 6\( ^{a} \)36; \citealt{Ackrill:1963fk})
\end{quote}
So understood, relatives are not relations, but relational properties---properties that obtain because some relation obtains. Aristotle further holds that ``All relatives are spoken of in relation to correlatives that reciprocate'' (\emph{Categories} 7 6\( ^{b} \)36) provided that they are properly given (\emph{eanper oikeiōs apodidotai}) (\emph{Categories} 7 7\( ^{a} \)22--23). Thus the larger is larger than the smaller and the smaller is smaller than the larger. Aristotle explains the constraint that correlatives that reciprocate be properly given as follows:
\begin{quote}
    For example, if a wing is given as \emph{of a bird}, \emph{bird of a wing} does not reciprocate; for it has not been given properly in the first place as wing of a bird. For it is not as being a bird that a wing is said to be of it, but as being a winged, since many things that are not birds have wings. Thus if it given properly there is reciprocation; for example, a wing is wing of a winged and a winged is winged with a wing. (\emph{Categories} 7 6\( ^{b} \)38--7\( ^{a} \)5; \citealt{Ackrill:1963fk})
\end{quote}

According to Aristotle, perception (\emph{aisthēsis}) and the perceptible (to \emph{aisthēton}) are correlatives, as are knowledge and the knowable. Perceptual and epistemic correlatives importantly differ from other correlatives, however. First, Aristotle makes the following linguistic observation: ``Sometimes, however, there will be a verbal difference, of ending. Thus knowledge is called knowledge \emph{of} what is knowable, and what is knowable knowable \emph{by} knowledge; perception perception \emph{of} the perceptible, and the perceptible perceptible \emph{by} perception'' (\emph{Categories} 7 6\( ^{b} \)28--36; \citealt{Ackrill:1963fk}). Second, Aristotle argues that perception and the perceptible are not simultaneous by nature, but that the perceptible is prior to perception:
\begin{quote}
    The perceptible seems to be prior to perception. For the destruction of the perceptible carries perception to destruction, but perception does not carry the perceptible to destruction. For perceptions are to do with body and in body, and if the perceptible is destroyed, body too is destroyed (since body is itself a perceptible), and if there is not body, perception too is destroyed; hence the perceptible carries perception to destruction. But perception does not carry the perceptible. For if animal is destroyed perception is destroyed, but there will be something perceptible, such as body, hot, sweet bitter, and all the other perceptibles. Moreover, perception comes into existence at the same time as what is capable of perceiving---an animal and perception come into existence at the same time---but the perceptible exists before perception exists; fire and water and so on, of which an animal is itself made up, exist even before there exists an animal at all, or perception. Hence the perceptible would seem to be prior to perception. (\emph{Categories} 7 7\( ^{b} \)37--8\( ^{a} \)12; \citealt{Ackrill:1963fk})
\end{quote}
Perception and the perceptible may be correlatives, but that is consistent with the following asymmetry---the perceptible can exist prior to perception and so does not depend on perception the way perception depends on its object.

The sense of independence is brought out clearly in one of a battery of arguments that Aristotle brings to bear against Protagoras in \emph{Metaphysics} \( \Gamma \):
\begin{quote}
	And, in general, if only the sensible exists, there would be nothing if animate things were not; for there would be no faculty of sense. Now the view that neither the sensible qualities nor the sensations would exist is doubtless true (for they are affections of the perceiver), but that the substrata which cause the sensation should not exist even apart from sensations is impossible. For sensation is surely not the sensation of itself, but there is something beyond the sensation, which must be prior to the sensation; for that which moves is prior in nature to that which is moved, and if they are correlative terms, this is no less the case. (\emph{Metaphysics} \( \Gamma \) 1010\( ^{b} \)30--1011\( ^{a} \)2)
\end{quote}

If only what can be perceived exists, then if no animate things existed, nothing would be perceived; from which it is meant to follow that nothing could exist. Whether the conclusion follows depends on the sense of the thesis assumed for the sake of \emph{reductio}. The conclusion would follow from the principle that:
\begin{quote}
	If something exists, then it is perceived
\end{quote}
since a world without animate things is a world without perception. Will the following weaker principle suffice?:
\begin{quote}
	If something exists, then it is possible that it is perceived 
\end{quote}
Whether it does depends on the sense of possibility involved. In a world without perceivers, given how things are in such a world, it is not possible for anything to be perceived. In order for something to be perceived, there must be perceivers, but there are none. Nevertheless, in such a world, there is another reasonable sense in which it \emph{is} possible for something to be perceived. Even if no perceivers existed for the rest of eternity, the particulars in the natural environment could be of such a nature that had there been perceivers, they could have been perceived. This latter thought is no aid to the Protagorean, however. Aristotle thinks that the envisioned possibility is only intelligible against the background of a realist metaphysics. Aristotle concedes to Protagoras that neither sensible qualities nor perceptions would exist if no perceivers exist. There certainly would be no sensible qualities as Protagoras conceives of them, at least if he is a perceptual relativist. But even if no perceivers exist, there would remain a ``substrata'' which retains the power to cause perceptions in animals with suitable sensory capacities. If perception is a mode of sensitivity, as it must if it is to be perception at all, the objects of perception must exist independently of perception and be the potential cause of perception. Not only does Aristotle provide metaphysical grounds for the existence of substrata, arguably at least, he provides phenomenological grounds as well, ``For sensation is surely not the sensation of itself, but there is something beyond the sensation, which must be prior to the sensation'' (\emph{Metaphysics} \( \Gamma \) 1011\( a \)1). Given the phenomenology of our perceptual experience, perception seems to present us with objects that exists independently of perception.

It is unclear what Aristotle means by ``substrata''. Does he mean persisting particulars like artifacts and natural objects? Or does he mean something broader in this context, broad enough to include not only external particulars but their qualities as well? Just because particulars must be independent causes of perception, it does not follow that they possess their sensible qualities independently of perception, or that their qualities can themselves be the cause of perception. If, on the other hand, in the argument against Protagoras, ``substrata'' is meant to include not only external particulars but their sensible qualities, then external particulars possess their sensible qualities independently of being perceived. 

Can sensible qualities cause perception? If so, they are among the substrata that exist independently of perception. If colors exist independently of being perceived, this is strong prima facie evidence that Aristotle is committed to the kind of color objectivism that the early moderns, such as Descartes, Boyle, and Locke, rejected. (There are other passages, however, that can seem to conflict with this interpretation. Whether Aristotle is, in fact, a color objectivist, and in what sense, will be determined later, in chapter~\ref{cha:color}.) 

The discussion in \emph{Metaphysics} \( \Delta \) suggests this broader interpretation.
\begin{quote}
	Similarly, sight is the sight of something, not `of that of which it is the sight' (though of course it is true to say this); in fact it is relative to color or to something else of the sort. But according to the other way of speaking the same thing would be said twice---`the sight is of that of which it is'. (\emph{Metaphysics} \( \Delta \) 1021\( ^{a} \)34--1021\( ^{b} \)3)
\end{quote}
Perception is relative to its object, in the case of sight, it is relative to the perceived color. In order for sight to be relative to color, in the sense in which it is, color must exist and have a nature independently of being perceived. If color was not independent of being perceived, then the claim that sight is relative to its object would amount to the triviality: the object of perception is what is perceived. But the Protagorean claim that color is relative to perception is meant to be a substantive thesis. Aristotle's underlying suggestion is that Protagorean relativism is not an intelligible alternative. If perception is relative to its object, its object must exist and have a nature independently of being perceived.

% (For discussion see \citealt{Gottlieb:1990kx})

% subsection external (end)

\subsection{Particular} % (fold)
\label{sub:particular}
The objects of perception are external particulars and their sensible qualities:
\begin{quote}
	Again, actual sensation corresponds to the exercise of knowledge; with this difference, that the objects of sight and hearing (and similarly for those of the other senses), which produce the actuality of sensation are external. This is because actual sensation is of particulars, whereas knowledge is of universals; these in a sense exist in the soul itself. So it lies in man's power to use his mind whenever he chooses, but it is not in his power to experience sensation; for the presence of the sensible object is essential. (\emph{De Anima} \textsc{ii}.5 417\( ^{b} \)?--?)
\end{quote}
This passage is part of an extended comparison of perception and knowledge. It makes two related points. Not only does perception and knowledge differ in object, but they are the exercise of different kinds of capacities as well. Moreover, this latter difference is partly explained in terms of the former. The overall lesson will be: Perceptual capacities would not be the kind of capacities that they are---they would not be a mode of sensitivity---unless perception takes as its object an external particular.

Let us begin with the difference in object. Whereas as the objects of perception are particulars, the objects of knowledge are universal. So whereas one may see the sun as it is at the moment of perception, burning white say, what one sees is a particular. But particulars, according to Aristotle, are not known. The objects of knowledge are universal in a way that precludes their being particulars. It is not just knowledge whose objects are universals, this is a general feature of our cognitive capacities. When one thinks that the sun is burning white, on thinks that thought not with the whiteness with which the sun actually burns but with a whiteness that the sun may share with the son of Diares, at least when viewed from a distance. Aristotle's claim that perception and knowledge can be distinguished, in this way, by the nature of their objects is echoed Prichard:
% \begin{quote}
% 	A thought always contains something reaching out beyond the particular case so that this is presented to us as falling under something general. \citep[4]{Frege:1882uq}
% 	
% 	\noindent But don't we see that the sun has set? And don’t we also thereby see that this is true? That the sun has set is no object which emits rays which arrive in our eyes, is no visible thing like the sun itself. That the sun has set is recognized as true on the basis of sensory input. \citep[64]{Frege:1918lq}
% \end{quote}
\begin{quote}
	There seems to be no way of distinguishing perception and conception as the apprehension of different realities except as the apprehension of the individual and the universal respectively. Distinguished in this way, the faculty of perception is that in virtue of which we apprehend the individual, and the faculty of conception is that power of reflection in virtue of which a universal is made the explicit object of thought. \citep[]{Prichard:1909yg}
\end{quote}
(For contemporary discussion of particularity and the content of perception see \citealt{Brewer:2008fk}, \citealt{Martin:2002jb}, \citealt{Soteriou:2000iz,Soteriou:2005fk}, and \citealt{Travis:2005ys}.)

The claim that the objects of perception are particulars and their sensible qualities is tolerably clear. The claim that the objects of knowledge and thought are universals is less so. What does Aristotle mean by universal here? To ask, in this context, what Aristotle means by universal is to ask for a substantive characterization of universal. But any substantive characterization would be controversial. Suppose, for example, that by ``universal'', Aristotle meant a generality that holds universally. Universal validity is a hard standard. Moreover, it is plausibly more than is required to distinguish the objects of knowledge from particulars. Perhaps what's known involves a kind of generality that precludes it from being a particular. However, having a generality that precludes being a particular may fall short of strict universality. At the very least, then, on any substantive characterization, it would remain an open question whether the characterization is a necessary condition for being non-particular. For this reason, I suspect that ``particular'' wears the trousers in the distinction here. If so, then by ``universal'' Aristotle merely means non-particular or, at most, having a generality that goes beyond the particular case. If that is right, then he is committed to no substantive characterization of universals.

Not only does perception and knowledge differ in object, they are the exercise of distinct kinds of capacities. Perception may be the exercise of the perceiver's sensory capacities in a way that corresponds to the exercise of knowledge, but sensory capacities are capacities of a distinctive kind. In Nietzsche's \citeyearpar{Nietzsche1887On-the-Genealog} terminology, they are \emph{reactive} capacities. Sensory capacities only act by reacting to the presence of the sensible particular. Aristotle made this point earlier by means of an analogy with combustion:
\begin{quote}
	The question arises as to why we have no sensation of the senses themselves; that is, why they give no sensation apart from external objects, although they contain fire and earth and the other elements which (either in themselves, or by their attributes) excite sensation. It is clear from this that the faculty of sensation has no actual but only potential existence. So it is like the case of fuel, which does not burn by itself without something to set fire to it; for otherwise it would burn itself, and would not need any fire actually at work. (\emph{De Anima} \textsc{ii}.5 417\( ^{a} \)3--10)
\end{quote}
The presence of the sensible particular ignites sensory consciousness. Perception is essentially a reactive capacity, otherwise it would not be a mode of sensitivity to external particulars and their qualities.

Perception and knowledge are the exercise of different kinds of capacities. Our epistemic capacities, and cognitive capacities more generally, are not reactive capacities like our sensory capacities. Their exercise does not require the presence of any particular. One can think of the sun burning white even when the sun is absent and night has fallen. Our epistemic and cognitive capacities are thus not modes of sensitivity to external particulars and their sensible qualities. Our epistemic and cognitive capacities do not act by reacting. They are \emph{active}, not reactive. Whereas we can \emph{choose} to exercise our knowledge in a given circumstance, we are \emph{subject} to what we perceive. Kantians would describe this difference between perception and knowledge as a difference between the exercise of receptivity and spontaneity.

According to Aristotle, the difference in object between perception and know\-ledge explains why perception and knowledge are the exercise of different kinds of capacities. Since the objects of perception are external particulars, our perceptual capacities are only ever exercised in the presence of such particulars. In this way, perception is a mode of sensitivity to particulars in the natural environment that not only exist independently of being perceived but whose natures and powers obtain independently of being perceived as well. But since the objects of knowledge, and cognition more generally, are universal, the exercise of our epistemic and cognitive capacities need not be constrained in this way by the particular case.

% subsection particular (end)

\subsection{Remote} % (fold)
\label{sub:remote}



% subsection remote (end)

% section sensible_qualities_of_remote_external_particulars (end)

\section{Against the Empedoclean Principle} % (fold)
\label{sec:against_the_empedoclean_principle}

In its original form, Empedoclean puzzlement about the sensory presentation of remote objects is generated by a general conception of sensory awareness---the ingestion model. Specifically, given the ingestion model, a question arises about how to coherently combine the distal character of the objects of sight with a key feature of that model, the principle that to be perceivable is to be palpable to sense. A cogent argument against that principle would undermine whatever puzzlement that it generates. 

Aristotle believes that a simple empirical observation constitutes such an argument:
\begin{quote}
	If one puts that which has color right up to the eye, it will not be visible. (\emph{De Anima} \textsc{ii}.7 419\( ^{a} \)13--14)
\end{quote}
If a colored particular's being in contact with the eye blinds the perceiver to its color, then the colored particular must be at a distance from the perceiver if its color is to be seen. And if the colored particular is remote from the perceiver, an intervening medium is necessary in order for the the organ of sight to be acted upon, as it must be if it is to be a mode of sensitivity:
\begin{quote}
	Color moves the transparent medium, such as the air, and this, being continuous, acts upon the sense organ. Democritus is mistaken in thinking that if the intervening space were empty, even an ant in the sky would be clearly visible; for this is impossible. For vision occurs when the sensitive faculty is acted upon; and it cannot be acted upon by the actual color which is seen, there only remains the medium to act on it, so that some medium must exist. In fact, if the intervening space were void, not merely would accurate vision be impossible, but nothing would be seen at all. (\emph{De Anima} \textsc{ii}.7 418\( ^{b} \)13--22)
\end{quote}

While Empedocles is not mentioned in this passage, Democritus instead being singled out for criticism, when this issue is raised again in \emph{De Sensu}, the connection with Empedocles is made explicit:
\begin{quote}
	But to say, as the old philosophers did, that colors are effluences from objects and visible on this account, is unreasonable; for in any case they would have to explain sensation by contact, so that it would be better to say at once that sensation is caused because the sensible object sets in motion the medium of sensation, that is by contact not effluence. (\emph{De Sensu} \textsc{iii} 440\( ^{a} \)16--21)
\end{quote}

Aristotle is making negative and positive claims in these passages. The negative claim is that an object's contact with the eye is incompatible with its being seen. The positive claim is that the eye is acted upon not by the object seen but by the intervening medium. By Aristotle's lights, Democritus and Empedocles make distinct, if related mistakes. Each fail to appreciate the neccessity of a medium acting upon the organ of sight, but the do so for different reasons. Whereas Democritus is committed to the denial of the postive claim, Empedocles is committed to the denial of the negative claim.

% The capacity for sight is a reactive capacity, it is a mode of sensitivity. As such, it must be acted upon before it can be exercised. Since, the object of perception cannot directly act upon the eye, the organ of sight must be acted upon by the intervening medium. Strictly speaking all that follows is that the organ of sight must be acted upon by \emph{something} other than the object of perception. However, the intervening medium is plausibly held to be the best explanatory candidate. Perhaps, Aristotle's reasoning is non-demonstrative here; perhaps, it is an instance of inference to the best explanation.

First, consider the positive claim that the existence of a suitable medium is necessary for sight. This is meant to follow from the conjunction of the negative claim and a thesis about the nature of perceptual capacities. Specifically, perceptual capacities are a mode of sensitivity. They are reactive capacities. As such, they are only ever exercised when acted upon by something external. Since contact with a colored particular precludes perception of the particular and its color, the eye cannot be acted upon by the colored particular. But the eye must be acted upon if the colored particular is to be seen. Only the intervening medium could act upon the organ of sight in the requisite manner. In the absence of an intervening medium, nothing would act upon the eye, and nothing would be seen.

Democritus is thus insensitive to the way in which sight is a reactive capacity. Since sight is a reactive capacity, the organ of sight must be acted upon if the subject's potential for sight is to be actualized. But the void that Democritus postulates precludes there being anything that could act upon the eye, the organ of sight. Democritus is thus committed to denying the positive claim that in seeing a colored particular the intervening medium acts upon the organ of sight.

Second, consider Aristotle's negative claim that a particular's contact with the eye is incompatible with seeing its color.  We can distinguish specific and more general versions of the negative claim. Whereas the specific claim is about color, the more general claim is about the objects of sense more generally, and thus holds of sound and smell as well:
\begin{enumerate}[(1)]
	\item A colored particular is imperceptible if it is in contact with the organ of sight.
	\item A sensible particular is imperceptible if it is in contact with the relevant sense organ.
\end{enumerate}

Consider first the specific claim about color. Here the thought is that in order to have a colored particular in view the perceiver must have a view on that colored particular. A colored particular's contact with the eye, the organ of sight, would preclude a point of view on that particular and its color. It is a necessary condition for a perceiver to have a point of view on a particular and its color that the particular be at a distance from the perceiver. To have a point of view on something is for that thing to be remote from one. 

The specific claim about color is echoed in Aristotle's criticism of the likeness theory. According to the likeness theory, perception is to be explained in terms of the similarity of the elements with which the sense organ and the object of sense are composed. The likeness theory is subject to a range of criticisms especially in the first book of \emph{De Anima}; however, in \emph{De Sensu}, Aristotle writes:
\begin{quote}
	For it is not true that the one sees, and the other is seen, just because the two are in a certain relation, \emph{e.g.}, that of equality; for in that case there would be no need for each of them to be in some particular place; for when things are equal it makes no difference whether they are near to or far from something. (\emph{De Sensu} \textsc{iii} 446\( ^{b} \)?--?)
\end{quote} 
Aristotle's complaint, here, is that equality, understood as complete compositional similarity of the sense organ and the object of sense, does not afford the perceiver with a point of view. The perceiver's point of view on a particular depends on that particular being at some distance from the perceiver. Moreover, that point of view varies as the object of sense is near or far. However, compositional similarity does not determine that the object of sense is any particular distance from the perceiver and hence fails to determine a point of view on that particular.

At least with respect to color vision, then, Aristotle's rejection of the Empedoclean principle, to be perceivable is to be palpable to sense, is unequivocal. Far from being a necessary condition on sight, contact with a colored particular blinds us to that particular and its color. Consistent with that denial, the Empedoclean principle may nevertheless be true of other objects of sense, such as taste and touch. A more thoroughgoing rejection of the principle, then, would regard the specific claim about color as an instance of the more general claim about the objects of sense. An object being in contact with the relevant sense organ, far from being a necessary condition for sensing that object, precludes it from being the object of sensation. The claim here is general, applicable to all objects of sense---contact precludes sensation, to be palpable is to be imperceptible.

While Aristotle at least makes the specific claim about color, his complete case against the Empedoclean principle, to be perceivable is to be palpable to sense, may involve the more general claim. The distinction between the specific and more general claim is relevant not only to the depth of Aristotle's case against the Empedoclean principle, but the distinction is relevant as well as to the relative plausibility of these claims. Even if the more general claim should prove to be false---of taste or touch, say---the specific claim about color may yet be true. It could turn out that vision is distinctive in being a sensory mode of presentation of the qualities of remote objects. Thus, for example, \citet[]{Broad:1952kx} claims that a comparative phenomenology of our sensory capacities supports this view (even if he thinks that our phenomenology is misleading in this regard, and that the distinctive phenomenological character of vision is ultimately undermined by the common causal mechanisms underlying all of our sensory capacities).

Aristotle's discussion of the special senses makes plain that he endorses the more general claim that contact precludes perception, that to be palpable is to be imperceptible:
\begin{quote}
	The same theory applies also to sound and smell; no sound or smell provokes sensation because it touches the sense organ, but movement is produced in the medium by smell and sound, and in the appropriate sense organ by the medium; but when one puts the sounding or smelling object in contact with the sense organ, no sensation is produced. The same thing is true of touch and taste, although it is not apparent \ldots\ (\emph{De Anima} \textsc{ii}.7 419\( ^{a} \)26--34)
\end{quote}
And later, in a discussion of why humans can only smell when they inhale, the general denial of the Empedoclean principle is invoked as a constraint on an adequate explanation:
\begin{quote}
	That what is placed on the sense organ should be imperceptible is common to all senses. (\emph{De Anima} \textsc{ii}.9 421\( ^{b} \)16--18)
\end{quote}

Indeed, Aristotle's conviction that to be palpable is to be imperceptible drives him to deny that flesh is the organ of touch:
\begin{quote}
    In a general sense we may say that as air and water are related to vision, hearing, and smell, so is the relation of the flesh and the tongue to the sense organ in the case of touch. In neither class of case mentioned would sensation result from touching the sense organ; for instance, if one were to put a white body on the surface of the eye. From this it is clear that that which is perception of what it touched is within. (\emph{De Anima \textsc{ii} 423\( ^{b} \)}18--23)
\end{quote}
This is a surprising claim. One may be forgiven for thinking that Aristotle has taken his opposition to the Empedoclean principle too far. But let's see what can be said on behalf of it.

First, consider the following examples of Dennett's:
\begin{quote}
    Blindfold yourself and take a stick (or a pen or pencil) in your hand. Touch various things around you with this wand, and notice that you can tell their textures effortlessly---as if your nervous system had sensor out at the tip of the wand. \ldots\ For an even more indirect case, think of how you can feel the slipperiness of an oil spot on the highway under the wheels of your car as you turn a corner. The phenomenological focal point of contact is the point where the rubber meets the road, not any point on your innervated body, seated, clothed, on the car seat, or on your gloved hands on the steering wheel. \citep[47]{Dennett:1993ce}
\end{quote}
These are nice examples of artificially extending tactile consciousness beyond the limits of the perceiver's innervated body. We feel the texture at the end of the pen, but not by feeling the pen in our hand. If we accept Dennett's description of these cases, then contact with flesh is not necessary for something to be the object of tactile awareness. While a good objection to the conjunction of Empedoclean principle and the claim that flesh is the organ of touch, this is too weak to establish Aristotle's counter-principle, to be palpable is to be imperceptible. Contact with the organ of touch may not be necessary for something to be the object of tactile awareness, but that is consistent with contact being sufficient for tactile awareness. Nevertheless, Dennett's examples remove an important obstacle to the acceptance of Aristotle's position. They make vivid the possibility of tactile awareness through a medium of objects remote from the organ of touch.

Aristotle's bold thought is that we are always already in the position described by Dennett. When we feel the texture of a body with our fingertips, the phenomenological point of contact is remote from the organ of touch, no less than when we feel the texture of that same body with a pen. Like the pen, the flesh of our fingertips is not the organ of touch but the medium through which the texture of the body is felt. Aristotle's conception of touch is an internalization of the model provided by Dennett's examples. Just as the phenomenological point of contact can be extended from the sense organ by means of an external medium, the phenomenological point of contact is always already extended from the sense organ by means of an internal medium. The perceiver's flesh is an internal medium, the organ of touch residing within. Indeed, Aristotle appeals to the internalization of an external medium to motivate the claim that flesh is not the organ of touch but its medium:
\begin{quote}
    Whether the sense organ is within, or whether the flesh feels directly, is not decided by the fact that sensation occurs instantly upon contact. For even as it is, if the flesh is surrounded with a closely fitting fabric, as soon as this is touched sensation is registered as before; and yet it is quite clear that the sense organ is not in the fabric. And if the fabric actually grew on the flesh, the sensation would traverse it even more quickly. (\emph{De Anima} \textsc{ii}.11 422\( ^{b} \)34--423\( ^{a} \)6)
\end{quote}

If Aristotle's counter-principle, to be palpable is to be imperceptible, can be sustained in its fully generality, then the Empedoclean principle, to be perceivable is to be palpable to sense, not only involves an overgneralization from a paradigm case but a misconception of it as well. 

% section against_the_empedoclean_principle (end)

\section{The Generalized Form of Empedoclean Puzzlement} % (fold)
\label{sec:the_generalized_form_of_empedoclean_puzzlement}

% section the_generalized_form_of_empedoclean_puzzlement (end)


% chapter perception_at_a_distance (end)
