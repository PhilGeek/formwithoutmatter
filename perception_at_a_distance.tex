%!TEX root = /Users/markelikalderon/Documents/Git/formwithoutmatter/formwithoutmatter.tex
\chapter{Perception at a Distance} % (fold)
\label{cha:perception_at_a_distance}

In its original form, Empedoclean puzzlement about the sensory presentation of remote objects consists in the apparent tension between two claims:
\begin{enumerate}[(1)]
    \item The objects of color perception are qualities of external particulars located at a distance from the perceiver.
    \item \emph{The Empedoclean principle}: To be perceptible is to be palpable to sense---in order for something to be the object of perception it must be in contact with the relevant sense organ.
\end{enumerate}
How might one respond to Empedoclean puzzlement in its original form? Either (1) or (2) may be rejected. In case it is unobvious that anyone would reject (1), Democritus and \citet{Berkeley:1734fk} counts among its deniers. Or one may propose an alternative reconciliation of (1) and (2), one not involving the theory of effluences, though it is admittedly hard to imagine what such an alternative would be.

Aristotle clearly accepts (1)---that the objects of color perception are qualities of external particulars located at a distance from the perceiver. Importantly, this is the result, in part, of some fundamental commitments that Aristotle undertakes with respect to the nature of our perceptual capacities and the objects of perception. Moreover, when combined with the Empedoclean principle, to be perceptible is to be palpable to sense, it generates precisely the puzzlement that the theory of effluences is designed to resolve. However, Aristotle rejects Empedocles' theory of effluences. And with no obvious alternative reconciliation to hand, Aristotle is committed to rejecting (2). 

This chapter addresses three topics. First, we will discuss the fundamental commitments about our perceptual capacities and the objects of perception that underly Aristotle's acceptance of (1)---that the objects of color perception are qualities of external particulars located at a distance from the perceiver. Second, we will discuss Aristotle's argument against (2)---the Empedoclean principle that to be perceptible is to be palpable to sense. As we shall see, according to Aristotle, not only is the Empedoclean principle an overgeneralization from a paradigm case, but a misconception of it as well. Third, Empedoclean puzzlement, in its most general form, survives the rejection of the principle that to be perceptible is to be palpable to sense. There thus remain unanswered questions in rejecting (2). Moreover, these questions partly set the agenda in Aristotle's discussion of perception (the present essay, as a whole, constitutes an argument for this latter claim). 

\section{The Sensible Qualities of Remote External Particulars} % (fold)
\label{sec:sensible_qualities_of_remote_external_particulars}

\begin{quote}
	And, in general, if only the sensible exists, there would be nothing if animate things were not; for there would be no faculty of sense. Now the view that neither the sensible qualities nor the sensations would exist is doubtless true (for they are affections of the perceiver), but that the substrata which cause the sensation should not exist even apart from sensations is impossible. For sensation is surely not the sensation of itself, but there is something beyond the sensation, which must be prior to the sensation; for that which moves is prior in nature to that which is moved, and if they are correlative terms, this is no less the case. (\emph{Metaphysica} \( \Gamma \) 1010\( ^{b} \)30--1011\( ^{a} \)2)
\end{quote}

\begin{quote}
	Similarly, sight is the sight of something, not `of that of which it is the sight' (though of course it is true to say this); in fact it is relative to color or to something else of the sort. But according to the other way of speaking the same thing would be said twice---`the sight is of that of which it is'. (\emph{Metaphysica} \( \Delta \) 1021\( ^{a} \)34--1021\( ^{b} \)3)
\end{quote}
(For discussion see \citealt{Gottlieb:1990kx})

\begin{quote}
	Again, actual sensation corresponds to the exercise of knowledge; with this difference, that the objects of sight and hearing (and similarly for those of the other senses), which produce the actuality of sensation are external. This is because actual sensation is of particulars, whereas knowledge is of universals; these in a sense exist in the soul itself. So it lies in man's power to use his mind whenever he chooses, but it is not in his power to experience sensation; for the presence of the sensible object is essential. (\emph{De Anima} \textsc{ii}.5 417\( ^{b} \)?--?)
\end{quote}
This passage is part of an extended comparison of perception and knowledge. It makes two related points. Not only does perception and knowledge differ in object, but they are the exercise of different kinds of capacities as well. Moreover, this latter difference is partly explained in terms of the former. The overall lesson will be: Perceptual capacities would not be the kind of capacities that they are---they would not be a mode of sensitivity---unless perception takes as its object an external particular.

Let us begin with the difference in object. Whereas as the objects of perception are particulars, the objects of knowledge are universal. So whereas one may see the sun as it is at the moment of perception, burning white say, what one sees is a particular. But particulars, according to Aristotle, are not known. The objects of knowledge are universal in a way that precludes their being particulars. It is not just knowledge whose objects are universals, this is a general feature of our cognitive capacities. When one thinks that the sun is burning white, on thinks that thought not with the whiteness with which the sun actually burns but with a whiteness that the sun may share with the son of Diares, at least when viewed from a distance. 

The claim that the objects of perception are particulars and their manifest qualities is tolerably clear. The claim that the objects of knowledge and thought are universals is less so. Aristotle does not here mean something wholly present wherever instantiated like redness; though one can know things about redness, redness is not an object of knowledge---it could not be what's known. I suspect that ``particular'' wears the trousers in the distinction here. If so, then by ``universal'' Aristotle merely means non-particular or, at most, having a generality that goes beyond the particular case. If that is right, then he is committed to no substantive positive characterization of universals. Moreover, perhaps that is a good thing. Any substantive positive characterization would be controversial. Suppose, for example, that by ``universal'', Aristotle meant a generality that holds universally. Universal validity is a hard standard. Moreover, it is plausibly more than is required to distinguish the objects of knowledge from particulars. Perhaps what's known involves a kind of generality that precludes it from being a particular. However, having a generality that precludes being a particular may fall short of strict universality.


Aristotle's claim that perception and knowledge can be distinguished by the nature of their objects is echoed, in their own ways, by Frege and Prichard:
\begin{quote}
	A thought always contains something reaching out beyond the particular case so that this is presented to us as falling under something general. \citep[4]{Frege:1882uq}
	
	\noindent But don't we see that the sun has set? And don’t we also thereby see that this is true? That the sun has set is no object which emits rays which arrive in our eyes, is no visible thing like the sun itself. That the sun has set is recognized as true on the basis of sensory input. \citep[64]{Frege:1918lq}
\end{quote}
\begin{quote}
	There seems to be no way of distinguishing perception and conception as the apprehension of different realities except as the apprehension of the individual and the universal respectively. Distinguished in this way, the faculty of perception is that in virtue of which we apprehend the individual, and the faculty of conception is that power of reflection in virtue of which a universal is made the explicit object of thought. \citep[]{Prichard:1909yg}
\end{quote}
(For contemporary discussion of particularity and the content of perception see \citealt{Brewer:2008fk}, \citealt{Martin:2002jb}, \citealt{Soteriou:2000iz,Soteriou:2005fk}, and \citealt{Travis:2005ys}.)

Not only does perception and knowledge differ in object, they are the exercise of distinct kinds of capacities. Perception may be the exercise of the perceiver's sensory capacities, but sensory capacities are capacities of a distinctive kind. In Nietzsche's \citeyearpar{Nietzsche1887On-the-Genealog} terminology, they are \emph{reactive} capacities. Sensory capacities only act by reacting to the presence of the sensible particular. Reactive capacities are only exercised by reacting to the presence of something. Aristotle made this point earlier by means of an analogy with combustion:
\begin{quote}
	The question arises as to why we have no sensation of the senses themselves; that is, why they give no sensation apart from external objects, although they contain fire and earth and the other elements which (either in themselves, or by their attributes) excite sensation. It is clear from this that the faculty of sensation has no actual but only potential existence. So it is like the case of fuel, which does not burn by itself without something to set fire to it; for otherwise it would burn itself, and would not need any fire actually at work. (\emph{De Anima} \textsc{ii}.5 417\( ^{a} \)3--10)
\end{quote}
The presence of the sensible particular ignites sensory consciousness. Perception is essentially a reactive capacity, otherwise it would not be a mode of sensitivity to external particulars and their qualities.

Perception and knowledge are the exercise of different kinds of capacities. Our epistemic capacities, and cognitive capacities more generally, are not reactive capacities like our sensory capacities. Their exercise does not require the presence of any particular. One can think of the sun burning white even when the sun is absent and night has fallen. Our epistemic and cognitive capacities are thus not modes of sensitivity to external particulars and their manifest qualities. Our epistemic and cognitive capacities do not act by reacting. They are \emph{active}, not reactive. Kantians would describe the difference between perception and knowledge as a difference between the exercise of receptivity and spontaneity.

% section sensible_qualities_of_remote_external_particulars (end)

\section{Against the Empedoclean Principle} % (fold)
\label{sec:against_the_empedoclean_principle}

In its original form, Empedoclean puzzlement about the sensory presentation of remote objects is generated by a general conception of sensory awareness---the ingestion model. Specifically, given the ingestion model, a question arises about how to coherently combine the distal character of the objects of sight with a key feature of that model, the principle that to be perceivable is to be palpable to sense. A cogent argument against that principle would undermine whatever puzzlement that it generates. Aristotle believes that a simple empirical observation constitutes such an argument:
\begin{quote}
	If one puts that which has color right up to the eye, it will not be visible. (\emph{De Anima} \textsc{ii}.7 419\( ^{a} \)13--14)
\end{quote}
If a colored particular's being in contact with the eye blinds the perceiver to its color, then the colored particular must be at a distance from the perceiver if its color is to be seen. And if the colored particular is remote from the perceiver, an intervening medium is necessary in order for the the organ of sight to be acted upon, as it must be if it is to be a mode of sensitivity:
\begin{quote}
	Color moves the transparent medium, such as the air, and this, being continuous, acts upon the sense organ. Democritus is mistaken in thinking that if the intervening space were empty, even an ant in the sky would be clearly visible; for this is impossible. For vision occurs when the sensitive faculty is acted upon; and it cannot be acted upon by the actual color which is seen, there only remains the medium to act on it, so that some medium must exist. In fact, if the intervening space were void, not merely would accurate vision be impossible, but nothing would be seen at all. (\emph{De Anima} \textsc{ii}.7 418\( ^{b} \)13--22)
\end{quote}
While Empedocles is not mentioned in this passage, Democritus instead being singled out for criticism, when this issue is raised again in \emph{De Sensu}, the connection with Empedocles is made explicit:
\begin{quote}
	But to say, as the old philosophers did, that colors are effluences from objects and visible on this account, is unreasonable; for in any case they would have to explain sensation by contact, so that it would be better to say at once that sensation is caused because the sensible object sets in motion the medium of sensation, that is by contact not effluence. (\emph{De Sensu} \textsc{iii} 440\( ^{a} \)16--21)
\end{quote}

We can distinguish a specific claim about color and a more general claim about the objects of sense, one true of sound and smell as well, say:
\begin{enumerate}[(1)]
	\item A colored particular is imperceptible if it is in contact with the organ of sight.
	\item A sensible object is imperceptible if it is in contact with the relevant sense organ.
\end{enumerate}

Let us begin with the specific claim about color. Here the thought is that in order to have a colored particular in view the perceiver must have a view on that colored particular. A colored particular's contact with the eye, the organ of sight, would preclude a point of view on that particular and its color. It is a necessary condition for a perceiver to have a point of view on a particular and its color that the particular be at a distance from the perceiver. To have a point of view on something is for that thing to be remote from one. 

The specific claim about color is echoed in Aristotle's criticism of the likeness theory. According to the likeness theory, perception is to be explained in terms of the similarity of the elements with which the sense organ and the object of sense are composed. The likeness theory is subject to a range of criticisms especially in the first book of \emph{De Anima}. However, in \emph{De Sensu}, Aristotle writes:
\begin{quote}
	For it is not true that the one sees, and the other is seen, just because the two are in a certain relation, \emph{e.g.}, that of equality; for in that case there would be no need for each of them to be in some particular place; for when things are equal it makes no difference whether they are near to or far from something. (\emph{De Sensu} \textsc{iii} 446\( ^{b} \)?--?)
\end{quote} 
Aristotle's complaint here is that equality, understood as complete compositional similarity of the sense organ and the object of sense, does not afford the perceiver with a point of view. The perceiver's point of view on a particular depends on the particular being at some distance from the perceiver. Moreover, that point of view varies as the object of sense is near or far. However, compositional similarity does not determine that the object of sense is any particular distance from the perceiver and hence fails to determine a point of view on that particular.

At least with respect to color vision, then, Aristotle's rejection of the Empedoclean principle, to be perceivable is to be palpable to sense, is unequivocal. Far from even being a necessary condition on sight, contact with a colored particular blinds us to that particular and its color. Consistent with that denial, the Empedoclean principle may nevertheless be true of other objects of sense, such as taste and touch. A more thoroughgoing rejection of the principle, then, would regard the specific claim about color as an instance of the more general claim about the objects of sense. An object being in contact with the relevant sense organ, far from being a necessary condition for sensing that object, precludes it from being the object of sensation. The claim here is general, applicable to all objects of sense---contact precludes sensation, to be palpable is to be imperceptible.

While Aristotle at least makes the specific claim about color, his complete case against the Empedoclean principle, to be perceivable is to be palpable to sense, may involve the more general claim. The distinction between the specific and more general claim is relevant not only to the depth of Aristotle's case against the Empedoclean principle---if the general claim is true, then not only is the Empedoclean principle false of color vision, it is false of every mode of sensory awareness---but the distinction is relevant as well as to the relative plausibility of these claims. Even if the more general claim should prove to be false---of taste or touch, say---the specific claim about color may yet be true. It could turn out that vision is distinctive in being a sensory mode of presentation of the qualities of remote objects. Thus, for example, \citet[]{Broad:1952kx} claims that a comparative phenomenology of our sensory capacities supports this view (even if he thinks that our phenomenology is misleading in this regard, and that the distinctive phenomenological character of vision is ultimately undermined by the common causal mechanisms underlying all of our sensory capacities).

As Aristotle's discussion of the special senses continues, however, it becomes clear that he endorses the more general claim that contact precludes perception, that to be palpable is to be imperceptible:
\begin{quote}
	The same theory applies also to sound and smell; no sound or smell provokes sensation because it touches the sense organ, but movement is produced in the medium by smell and sound, and in the appropriate sense organ by the medium; but when one puts the sounding or smelling object in contact with the sense organ, no sensation is produced. The same thing is true of touch and taste, although it is not apparent \ldots\ (\emph{De Anima} \textsc{ii}.7 419\( ^{a} \)26--34)
\end{quote}
And later, in a discussion of why humans can only smell when they inhale, the general denial of the Empedoclean principle is invoked as a constraint on an adequate explanation:
\begin{quote}
	That what is placed on the sense organ should be imperceptible is common to all senses. (\emph{De Anima} \textsc{ii}.9 421\( ^{b} \)16--18)
\end{quote}

% section against_the_empedoclean_principle (end)

\section{The Generalized Form of Empedoclean Puzzlement} % (fold)
\label{sec:the_generalized_form_of_empedoclean_puzzlement}

% section the_generalized_form_of_empedoclean_puzzlement (end)


% chapter perception_at_a_distance (end)
