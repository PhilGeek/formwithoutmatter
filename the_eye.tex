%!TEX root = /Users/markelikalderon/Documents/Git/formwithoutmatter/aristotle.tex
\chapter{The Eye} % (fold)
\label{cha:the_eye}

\section{The Soul of the Eye} % (fold)
\label{sec:the_soul_of_the_eye}

Sight, Aristotle tells us, is the soul of the eye, or would be if it were an animal. This claim is made in the context of explaining what the soul of an animal is, and Aristotle proceeds by analogy with artifacts and parts of animals:
\begin{quote}
	Suppose that a tool, e.g. an axe, were a natural body, then being an axe would have been its essence, and so its soul; if this disappeared from it, it would have ceased to be an axe, except in name. As it is, it is an axe; for it is not of a body of that sort that what it is to be, i.e. its account, is a soul, but of a natural body of a particular kind, viz. one having in itself the power of setting itself in movement and arresting itself. Next, apply this doctrine in the case of the parts of the living body. Suppose that the eye were an animal—sight would have been its soul, for sight is the substance of the eye which corresponds to the account, the eye being merely the matter of seeing; when seeing is removed the eye is no longer an eye, except in name—no more than the eye of a statue or of a painted figure. (\emph{De Anima} \textsc{ii}.1 412\( ^{b} \)12--22)
\end{quote}

If an axe were a natural body, then what it is to be an axe would be both essential to the axe and its soul. If what it is to be an axe were somehow removed from a thing, then it would cease to be an axe. For a thing to have what it takes to be an axe is for that thing to possess a capacity for motion and rest characteristic of axes. Specifically, the thing must possess the capacity to cut in the manner of axes. If the thing had some other capacity, to join, for example, or if it cut in some other manner, then it would not be an axe, but a vice or a knife, say. Thus should a thing lose its capacity to cut, or to cut in that manner, it would cease to be an axe. It would be an axe in name only. The capacity to cut in the manner of axes is the essential form and substance of an axe. If the capacity to cut in that manner is the form of an axe, then the material parts of the axe---the wooden shaft, the bronze head---constitute its matter. 

In the quoted passage, Aristotle notes an important limitation of the analogy. ``As it is, it is an axe.'' That is to say, an axe is not, in fact, a natural body but is, rather, an artifact. Unlike a natural body, such as a living being, an axe does not contain within itself the power of motion and rest. An axe has the capacity to cut because of the use we make of it. A living being is a natural body with a soul. But an axe, being an artifact, does not contain within itself the power of motion and rest and so is not a natural body. Since an axe is not a natural body, neither is it a living being, and so lacks a soul. Thus the essential form and substance of an axe is not its soul, at least not literally.

Aristotle's account of the soul of an axe conceived as a natural body is the model for his account of the soul of the eye conceived as an animal. If an eye were an animal, then what it is to be an eye would be both essential to the eye and its soul. If what it is to be an eye were somehow removed from a thing, then it would cease to be an eye. For a thing to have what it takes to be an eye is for that thing to possess the capacity for sight. Thus should a thing lose this capacity, it would cease to be an eye. It would be an eye in name only. A thing that lacks the capacity to see is like the eye of a statue, such as the drooping eye of King Seuthes \textsc{iii} in the 4th century \textsc{bc} Thracian bronze portrait, the trace, perhaps of an old battle wound (see figure~\ref{fig:seuthesiii}). The portrait is remarkably naturalistic, and the King's gaze is arresting. The artist used glass paste of different colors to distinguish the white of the eye, the pupil, and the iris and used thin copper wire for the eyelashes. Despite the striking naturalism and the intensity of the King's gaze, the drooping eye is, nonetheless, no real eye. Despite the naturalism and psychological expression achieved by the Thracian masterpiece, the colored glass paste, being opaque, lacks the capacity to see. The capacity to see is the essential form and substance of an eye. If the capacity to see is the form of the eye, then the material parts of the eye---the membrane, the interior water---constitute its matter.

\begin{figure}[htbp]
	\centering
		\includegraphics[scale=1]{graphics/seuthesiii.jpg}
	\caption{Detail of 4th century \textsc{bc} Thracian bronze portrait of King Seuthes \textsc{iii}}
	\label{fig:seuthesiii}
\end{figure}

If we suppose the eye to be an animal, perhaps it makes sense to think of the hypothetical creature as possessing the capacity to see. But if we relaxe that supposition, and consider an eye as it naturally occurs as part of an animal, then it would be wrong to think that an eye possesses the capacity to see. His eyes may have endowed King Seuthes III with the capacity to see (at least when he was alive, awake, and attentive), but they did not themselves possess this capacity. Similarly amputated eyes, eyes separated from the animal in which they naturally occur as parts, neither posesses the capacity to see nor endow anything else with that capacity. 

There are thus grounds for criticizing a commitment of Empedocles that arises in a passage that, as \citet[211]{Wright:1981zr} observes, Aristotle finds sufficiently interesting to quote three times (in \emph{De Anima} \textsc{iii}.6 430\( ^{a} \)27, \emph{De Caelo} \textsc{iii}.2 300\( ^{b} \)25, and \emph{De Generatione Animalium} \textsc{i}.18 722\( ^{b} \)17):
\begin{verse}
	as many heads without necks sprouted up\\
	and arms wandered naked, bereft of shoulders,\\
	and eyes roamed alone, impoverished of foreheads\\
	(DK \textsc{b}57; \citealt[64 245]{Inwood:2001ve})
\end{verse}
At a certain stage of the cosmic cycle, where Strife still dominates but whose influence is on the wane as Love grows stronger, the parts of animals arise spontaneously and in a disordered state. These combine to give rise to fantastical animals, some with clear mythological precedent:
\begin{verse}
	Many with two faces and two chests grew,\\
	oxlike with men's faces, and again their came up\\
	androids with ox-heads, mixed in one way from men\\
	and in another way in female form, outfitted with shadowy limbs.\\
	(DK \textsc{b}61; \citealt[66 247]{Inwood:2001ve})
\end{verse}
These animals tended not to survive. It is only when animal parts are combined in harmony, due to the increased influence of Aphrodite's Love, do the animals that we presently recognize emerge. Due to the harmony among their parts which fits them to a life in their natural environment, present animals not only survive but have the means of reproduction. Now consider the eyes roaming alone, impoverished of foreheads. Like amputated eyes, they are separated from animals in which they would harmoniously occur as parts. And like amputated eyes, they neither possess the capacity to see nor endow anything else with that capacity. Though eye-like in structure and composition, these are eyes in name only, at least by Aristotle's lights.

The capacity to cut in the manner of axes is a power and potentiality. A thing may possess this power, and so retain the potential to cut in that manner, even when at rest, when it is not actually cutting anything. Similarly, the capacity to see is a power and potentiality. A thing may possess or endow this power, and so retain the potential to see, even when at rest, when it is not actually seeing anything, because of darkness or sleep, say. Aristotle also claims that, in general, matter is potentiality and that form is actuality. There is no inconsistency here as the actual and the potential are said of in many ways. A thing is actually an axe if it possesses what it takes to be an axe, the capacity to cut in the manner of axes, the form and substance of an axe. Moreover the material parts of the thing, the matter of the axe---the wooden shaft, the bronze head---are potentially an axe since they are capable of taking on the form of an axe. When the bronze is suitably fashioned, and honed, and securely fixed to the wooden shaft, the matter, in taking on the form that it does, in so acquiring the capacity to cut in the manner of axes, realizes this potentiality. But what it is to be an actual axe is itself a power and potentiality, the capacity to cut in the manner of axes, a potentiality actualized in so cutting. Similarly, a thing is actually an eye if it possesses what it takes to be an eye, the capacity to see, the form and substance of an eye. Moreover the material parts of the thing, the matter of the eye---the membrane, the interior water---are potentially an eye since they are capable of taking on the form of an eye. When interior water is bound by the membrane and the other parts of the eye are suitably arranged, the matter in taking on the form that it does, in so acquiring the capacity for sight, actualizes this potentiality. What it is to be an actual eye is itself to possess or endow a power and potentiality, the capacity to see, a potentiality actualized in seeing.

The actual and the potential are said of in many ways. In his discussion of these analogies, Aristotle distinguishes two senses of the actual/potential distinction. There is an intial potentiality had by some matter. This potentiality is realized by that matter taking on a form. The taking on of the relevant form is the first actuality. However, in the cases at hand, the relevant form is undertood as the possession, or perhaps the endowment of, a capacity. So the first actuality is itself a potentiality. The realization of this potentiality, the exercise of the capacity which is the form of the matter, is the second actuality. These distinctions are schematically represented in table~\ref{tab:potential}

\begin{table}[htbp]
	\centering
		\begin{tabular}{cccc}
			& \emph{An Axe} & \emph{An Eye}\\
			\hline
			\emph{Potenitality} & the matter of an axe & the matter of an eye\\
			\hline
			\emph{First Actuality} & the capacity to cut & the capacity to see\\
			\hline
			\emph{Second Actuality} & cutting & seeing\\
			\hline
		\end{tabular}
	\caption{Two senses of the actual/potential distinction}
	\label{tab:potential}
\end{table}

Eyes endow animals that possess them with the capacity to see. This is a capacity that animals enjoy even when alseep or inattentive. Endowing the perceiver with the capacity for sight is what the organ of sight is for. As \citet{Johansen:1997zr} argues, this further telological claim motivates a certain explanatory strategy with respect to the anatomical structure of the eye, what he describes as a ``top-down explanation''. Begin with what the organ of sight is for, to endow the possessor of such organs with the capacity to see. If the perceiver is endowed with this capacity, then the objects of sight, the colors of remote external particulars, potentially appear in the perceiver's experience of them. In order for the colors of remote external particulars to appear in the perceiver's experience of them, the organ of sight must be transparent, it must be perceptually penetrable at least to some degree. The eye has the structure and composition that it does to sustain the transparency necessary to endow the perceiver with the capacity for sight. So our initial discussion of transparency (chapter~\ref{cha:transparency}) was necessary not only to understand Aristotle's definition of color (chapter~\ref{cha:color}) but also to understand Aristotle's explanatory strategy with respect to the anatomical structure of the eye. And understanding Aristotle's explanatory strategy, here, will yield insight into the significance of its limitations.



% section the_soul_of_the_eye (end)

\section{Transparency and the Anatomy of the Eye} % (fold)
\label{sec:transparency_and_the_anatomy_of_the_eye}

% section transparency_and_the_anatomy_of_the_eye (end)

% chapter the_eye (end)
