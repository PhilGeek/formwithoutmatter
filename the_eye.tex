%!TEX root = /Users/markelikalderon/Documents/Git/formwithoutmatter/aristotle.tex
\chapter{The Eye} % (fold)
\label{cha:the_eye}

\section{The Soul of the Eye} % (fold)
\label{sec:the_soul_of_the_eye}

Sight, Aristotle tells us, is the soul of the eye, or would be if it were an animal. The claim is made in the context of explaining what the soul of an animal is, and Aristotle proceeds by analogy with artifacts and parts of animals:
\begin{quote}
	Suppose that a tool, e.g. an axe, were a natural body, then being an axe would have been its essence, and so its soul; if this disappeared from it, it would have ceased to be an axe, except in name. As it is, it is an axe; for it is not of a body of that sort that what it is to be, i.e. its account, is a soul, but of a natural body of a particular kind, viz. one having in itself the power of setting itself in movement and arresting itself. Next, apply this doctrine in the case of the parts of the living body. Suppose that the eye were an animal—sight would have been its soul, for sight is the substance of the eye which corresponds to the account, the eye being merely the matter of seeing; when seeing is removed the eye is no longer an eye, except in name—no more than the eye of a statue or of a painted figure. (\emph{De Anima} \textsc{ii}.1 412\( ^{b} \)12--22)
\end{quote}

% section the_soul_of_the_eye (end)

\section{Transparency and the Anatomy of the Eye} % (fold)
\label{sec:transparency_and_the_anatomy_of_the_eye}

% section transparency_and_the_anatomy_of_the_eye (end)

% chapter the_eye (end)
