%!TEX root = /Users/markelikalderon/Documents/Git/formwithoutmatter/aristotle.tex
\chapter{The Eye} % (fold)
\label{cha:the_eye}

\section{The Soul of the Eye} % (fold)
\label{sec:the_soul_of_the_eye}

Sight, Aristotle tells us, is the soul of the eye, or would be if it were an animal. This claim is made in the context of explaining what the soul of an animal is, and Aristotle proceeds by analogy with artifacts and parts of animals:
\begin{quote}
	Suppose that a tool, e.g. an axe, were a natural body, then being an axe would have been its essence, and so its soul; if this disappeared from it, it would have ceased to be an axe, except in name. As it is, it is an axe; for it is not of a body of that sort that what it is to be, i.e. its account, is a soul, but of a natural body of a particular kind, viz. one having in itself the power of setting itself in movement and arresting itself. Next, apply this doctrine in the case of the parts of the living body. Suppose that the eye were an animal—sight would have been its soul, for sight is the substance of the eye which corresponds to the account, the eye being merely the matter of seeing; when seeing is removed the eye is no longer an eye, except in name—no more than the eye of a statue or of a painted figure. (\emph{De Anima} \textsc{ii}.1 412\( ^{b} \)12--22)
\end{quote}

If an axe were a natural body, then what it is to be an axe would be both essential to the axe and its soul. If what it is to be an axe were somehow removed from a thing, then it would cease to be an axe. For a thing to have what it takes to be an axe is for that thing to possess a capacity for motion and rest characteristic of axes. Specifically, the thing must possess the capacity to cut in the manner of axes. If the thing had some other capacity, to join, for example, or if it cut in some other manner, then it would not be an axe, but a vice or a knife, say. Thus should a thing lose its capacity to cut, or to cut in that manner, it would cease to be an axe. It would be an axe in name only. The capacity to cut in the manner of axes is the essential form and substance of an axe. If the capacity to cut in that manner is the form of an axe, then the material parts of the axe---the wooden shaft, the bronze head---constitute its matter. 

In the quoted passage, Aristotle notes an important limitation of the analogy. ``As it is, it is an axe.'' That is to say, an axe is not, in fact, a natural body but is, rather, an artifact. Unlike a natural body, such as a living being, an axe does not contain within itself the power of motion and rest. An axe has the capacity to cut because of the use we make of it. A living being is a natural body with a soul. But an axe, being an artifact, does not contain within itself the power of motion and rest and so is not a natural body. Since an axe is not a natural body, neither is it a living being, and so lacks a soul. Thus the essential form and substance of an axe is not its soul, at least not literally.

Aristotle's account of the soul of an axe conceived as a natural body is the model for his account of the soul of the eye conceived as an animal. If an eye were an animal, then what it is to be an eye would be both essential to the eye and its soul. If what it is to be an eye were somehow removed from a thing, then it would cease to be an eye. For a thing to have what it takes to be an eye is for that thing to possess the capacity for sight. Thus should a thing lose this capacity, it would cease to be an eye. It would be an eye in name only. A thing that lacks the capacity to see is like the eye of a statue, such as the drooping eye of King Seuthes \textsc{iii} in the 4th century \textsc{bc} Thracian bronze portrait, the trace, perhaps of an old battle wound (see figure~\ref{fig:seuthesiii}). The portrait is remarkably naturalistic, and the King's gaze is arresting. The artist used glass paste of different colors to distinguish the white of the eye, the pupil, and the iris and used thin copper wire for the eyelashes. Despite the striking naturalism and the intensity of the King's gaze, the drooping eye is, nonetheless, no real eye. Despite the naturalism and psychological expression achieved by the Thracian masterpiece, the colored glass paste, being opaque, lacks the capacity to see. The capacity to see is the essential form and substance of an eye. If the capacity to see is the form of the eye, then the material parts of the eye---the membrane, the interior water---constitute its matter.

\begin{figure}[htbp]
	\centering
		\includegraphics[scale=1]{graphics/seuthesiii.jpg}
	\caption{Detail of 4th century \textsc{bc} Thracian bronze portrait of King Seuthes \textsc{iii}}
	\label{fig:seuthesiii}
\end{figure}

If we suppose the eye to be an animal, perhaps it makes sense to think of the hypothetical creature as possessing the capacity to see. But if we relaxe that supposition, and consider an eye as it naturally occurs as part of an animal, then it would be wrong to think that an eye possesses the capacity to see. His eyes may have endowed King Seuthes III with the capacity to see (at least when he was alive, awake, and attentive), but they did not themselves possess this capacity. Similarly amputated eyes, eyes separated from the animal in which they naturally occur as parts, neither posesses the capacity to see nor endow anything else with that capacity. 

There are thus grounds for criticizing a commitment of Empedocles that arises in a passage that, as \citet[211]{Wright:1981zr} observes, Aristotle finds sufficiently interesting to quote three times (in \emph{De Anima} \textsc{iii}.6 430\( ^{a} \)27, \emph{De Caelo} \textsc{iii}.2 300\( ^{b} \)25, and \emph{De Generatione Animalium} \textsc{i}.18 722\( ^{b} \)17):
\begin{verse}
	as many heads without necks sprouted up\\
	and arms wandered naked, bereft of shoulders,\\
	and eyes roamed alone, impoverished of foreheads\\
	(DK \textsc{b}57; \citealt[64 245]{Inwood:2001ve})
\end{verse}
At a certain stage of the cosmic cycle, where Strife still dominates but whose influence is on the wane as Love grows stronger, the parts of animals arise spontaneously and in a disordered state. These combine to give rise to fantastical animals, some with clear mythological precedent:
\begin{verse}
	Many with two faces and two chests grew,\\
	oxlike with men's faces, and again their came up\\
	androids with ox-heads, mixed in one way from men\\
	and in another way in female form, outfitted with shadowy limbs.\\
	(DK \textsc{b}61; \citealt[66 247]{Inwood:2001ve})
\end{verse}
These animals tended not to survive. It is only when animal parts are combined in harmony, due to the increased influence of Aphrodite's Love, do the animals that we presently recognize emerge. Due to the harmony among their parts which fits them to a life in their natural environment, present animals not only survive but have the means of reproduction. Now consider the eyes roaming alone, impoverished of foreheads. Like amputated eyes, they are separated from animals in which they would harmoniously occur as parts. And like amputated eyes, they neither possess the capacity to see nor endow anything else with that capacity. Though eye-like in structure and composition, these are eyes in name only, at least by Aristotle's lights:
\begin{quote}
	What a thing is is always determined by its function: a thing really is itself when it can perform its function; an eye, for instance, when it can see. When a thing cannot do so it is that thing only in name, like a dead eye or one made of stone, just as a wooden saw is no more a saw than one in a picture. (\emph{Meterologica} \textsc{iv}.12 390\( ^{a} \)10--14; Webster in \citealt[86]{Barnes:1984uq})
\end{quote}

The capacity to cut in the manner of axes is a power and potentiality. A thing may possess this power, and so retain the potential to cut in that manner, even when at rest, when it is not actually cutting anything. Similarly, the capacity to see is a power and potentiality. A thing may possess or endow this power, and so retain the potential to see, even when at rest, when it is not actually seeing anything, because of darkness or sleep, say. Aristotle also claims that, in general, matter is potentiality and that form is actuality. There is no inconsistency here as the actual and the potential are said of in many ways. A thing is actually an axe if it possesses what it takes to be an axe, the capacity to cut in the manner of axes, the form and substance of an axe. Moreover the material parts of the thing, the matter of the axe---the wooden shaft, the bronze head---are potentially an axe since they are capable of taking on the form of an axe. When the bronze is suitably fashioned, and honed, and securely fixed to the wooden shaft, the matter, in taking on the form that it does, in so acquiring the capacity to cut in the manner of axes, realizes this potentiality. But what it is to be an actual axe is itself a power and potentiality, the capacity to cut in the manner of axes, a potentiality actualized in so cutting. Similarly, a thing is actually an eye if it possesses what it takes to be an eye, the capacity to see, the form and substance of an eye. Moreover the material parts of the thing, the matter of the eye---the membrane, the interior water---are potentially an eye since they are capable of taking on the form of an eye. When interior water is bound by the membrane and the other parts of the eye are suitably arranged, the matter in taking on the form that it does, in so acquiring the capacity for sight, actualizes this potentiality. What it is to be an actual eye is itself to possess or endow a power and potentiality, the capacity to see, a potentiality actualized in seeing.

The actual and the potential are said of in many ways. In his discussion of these analogies, Aristotle distinguishes two senses of the actual/potential distinction. There is an intial potentiality had by some matter. This potentiality is realized by that matter taking on a form. The taking on of the relevant form is the first actuality. However, in the cases at hand, the relevant form is undertood as the possession, or perhaps the endowment of, a capacity. So the first actuality is itself a potentiality. The realization of this potentiality, the exercise of the capacity which is the form of the matter, is the second actuality. These distinctions are schematically represented in table~\ref{tab:potential}

\begin{table}[htbp]
	\centering
		\begin{tabular}{cccc}
			& \emph{An Axe} & \emph{An Eye}\\
			\hline
			\emph{Potenitality} & the matter of an axe & the matter of an eye\\
			\hline
			\emph{First Actuality} & the capacity to cut & the capacity to see\\
			\hline
			\emph{Second Actuality} & cutting & seeing\\
			\hline
		\end{tabular}
	\caption{Two senses of the actual/potential distinction}
	\label{tab:potential}
\end{table}

Eyes endow animals that possess them with the capacity to see. This is a capacity that animals enjoy even when alseep or inattentive. Endowing the perceiver with the capacity for sight is what the organ of sight is for. As \citet{Johansen:1997zr} argues, this further telological claim motivates a certain explanatory strategy with respect to the anatomical structure of the eye, what he describes as a ``top-down explanation''. Begin with what the organ of sight is for, to endow its possessor with the capacity to see. If the perceiver is endowed with this capacity, then the primary objects of sight, the colors of remote external particulars, potentially appear in the perceiver's experience of them. In order for the colors of remote external particulars to appear in the perceiver's experience of them, the organ of sight must be transparent, it must be perceptually penetrable at least to some degree. The eye has the structure and composition that it does to sustain the transparency necessary to endow the perceiver with the capacity for sight. So our initial discussion of transparency (chapter~\ref{cha:transparency}) was necessary not only to understand Aristotle's definition of color (chapter~\ref{cha:color}) but also to understand Aristotle's explanatory strategy with respect to the anatomical structure of the eye. And understanding Aristotle's explanatory strategy, here, will yield insight into the significance of its limitations.



% section the_soul_of_the_eye (end)

\section{Transparency and the Anatomy of the Eye} % (fold)
\label{sec:transparency_and_the_anatomy_of_the_eye}

Why must the organ of sight be transparent?

Sight is a reactive capacity. It is a mode of sensitivity to the colors of remote external particulars. It only acts by reacting to the presence of a particular's color. Since sight is a reactive capacity, it must be acted upon in order for it to be exercised. (That its exercise, an episode of seeing, just is being acted upon in this way is a further claim, that Aristotle denies, \emph{De Anima} \textsc{ii}.5. For a contemporary defence of this denial see \citealt{Travis:2009fk}) Color could not immediately act upon the eye, since contact would blind the perceiver to the particular and its color. Nevertheless, the color of a remote external particular can act upon the eye mediately, by acting upon the intervening medium. Thus was the moral of Aristotle's criticism of Democritus and Empedocles (chapter~\ref{sec:against_the_empedoclean_principle}). The eye is itself affected by the color's effect on light. Since color is the power to alter what is actually transparent, the eye is affected by color's effect on illuminated media by itself being transparent, at least in part. They eye is transparent, at least in part, so that the colors of remote external particulars may mediately act upon it. Which they must do, since sight is a reactive capacity, a mode of sensitvity to the colors of remote external particulars.

Suppose that is right. Suppose that sight could only be the reactive capacity that it is, a chromatic sensitivity, if the organ of sight were transparent, at least in part. This would constrain its elemental composition. Whereas some elements are receptive to the presence and activity of the fiery substance, such as air and water, other elements, such as earth, exclude the fiery substance or at least retard its activity. Transparency is present only in matter with a certain elemental composition, an elemental composition that allows for the presence and activity of the fiery substance.

That sight is a reactive capacity, a chromatic sensitivity, not only constrains the elemental composition of the eye, the organ of sight, but it also constrains its structure. Transparency is a nature or power common to different substances such as water and air. So the requirement that the internal medium of the eye be transparent does not by itself determine whether an eye must have either elemental composition. Further material considerations determine that the internal medium of the eye be water:
\begin{quote}
	True, then, the visual organ proper is composed of water, yet vision appertains to it not because it is water, but because it is transparent---a property common alike to water and to air. But water is more easily confined and more easily condensed than air; it is that the pupil, i.e. the eye proper, consists of water. (\emph{De Sensu} \textsc{ii} 438\( ^{a} \)13--18)
\end{quote}
Air and water are liquids and so lack fixed boundaries. If the organ of sight is thus composed, at least in part, of the transparent, this liquid must somehow be confined to the organ of sight, by a membrane composed of earth, say. That this is more easily done with water than air favors the conclusion that the eye is composed, at least in part, of water. However, what is presently important is that reflection on the material constraints of sustaining an internal transparent medium not only determines the elemental compsoition of the internal medium but also significant anatomical structure, the existence of a membrane that confines and condenses the internal water.


% Democritus was right at least about this even if he was wrong in thinking tht the eye was water to sustain reflection rather than transparency. 

\begin{quote}
	Now, as vision outwardly is impossible without light, so also it is impossible inwardly. There must, therefore, be some transparent medium within the eye, and, as this is not air, it must be water. The soul or its perceptive part is not situated at the external surface of the eye, but obviously somewhere within: whence the necessity of the interior of the eye being transparent, i.e. capable of admitting light. And that it is so is plain from actual occurrences. It is matter of experience that soldiers wounded in battle by a sword slash on the temple, so inflicted as to sever the passages of the eye, feel a sudden onset of darkness, as if a lamp had gone out; because what is called the pupil, i.e. the transparent, which is a sort of lamp, is then cut off. (\emph{De Sensu} 438b7-438b15)
\end{quote}
This passage is dialectically complex. Democritus and Empedocles are both subject to criticism, here. Importantly, however, Aristotle retains elements of each of their rejected theories. So the passage involves a dialectical refinement of the \emph{endoxa}. 

According to Aristotle, Democritus maintains that the eye sees because of its reflective surface, itself due to the smoothness of the eye and the presence of lachrmal fluid. Democritus is wrong in thinking that this was due to water's capacity for reflection, and consequently wrong in taking the locus of sight to be on the surface of the eye. Nevertheless, Democritus was right to link the eye's capacity to see with the presence of water. It is the transparency of the eye's water that is required to endow its possesor with the capacity to see. 

If Democritus was right in thinking that the eye is composed of water, then Empedocles was wrong in thinking that the eye contained fire. This should not obscure for us the way in which Aristotle retains important elements of the Empedoclean theory of vision.

First, Aristotle's discussion of Democritus parallels the way that Empedocles accomodates the dominant medical opinion of his time (discussed in chapter~\ref{sec:empedocles_theory_of_vision}). On the ingestion model, chromatic influences must first be assimilated through ocular passages to the eye's interior and so that they may be palpable to sense. On the ingestion model, sight is located within. By conceiving of receptors on the surface of the eye as passages to its interior, Empedocles reconciles the ingestion model with the dominant medical opinion of his time. Aristotle is similarly here asserting that ``the soul or its perceptive part is \ldots\ obviously within''. And like Empedocles, Aristotle retains something of Democritus opinion. But not by finding a role for the surface of the eye in seeing, but by claiming that the eye is partly composed of water. 

This also marks a disagreement with Empedocles---the eye is not composed of fire. However, there is another way to undesrtand Empedocles talk of interior fire. If the eye's interior is composed of confined and condensed water in order to sustain its transparency, then interior water is receptive to the presence and activity of the fiery substance. The external medium must be illuminated if the color of a remote external particular is to be visible in it. Similarly, for that color to be visible, the external light must be extended within. The transparent water of the eye's interior must itself be illuminated. And the internal medium is only illuminated by to the presence and activity of the fiery substance. While seeing the white of the sun may not involve the assimilation of fiery effluences as Empedocles maintained; nevertheless, if the white of the sun is visible to the perceiver this is due, in part, to the presence and activity of the fiery substance in the eye's interior. This is why Aristotle makes the deliberate allusion to Empeocles lantern analogy (DK 84) (quoted in full just prior to this passage, \emph{De Sensu} \textsc{ii} 437\( ^{b} \)27--438\( ^{a} \)3). Like Empedocles, Aristotle compares the eye to a lantern. But not because the eye emits fire the way a lantern does. But because the interior of the eye must be illuminated, the way the interior of a lamp is illuminated, if the external scene is to be visible. 

So not only does Aristotle retain the phenomenological insight of Empedocles, that seeing is a mode of assimilation, ``the soul or its perceptive part'' within. But Aristotle also retains the Empedoclean doctrine that the exercise of the capacity for sight involves fire in the eye's interior, understood as the fiery substance illuminating the internal medium. Let me make some observations about the Aristotelian reinterpretation of the Empoclean lantern analogy. 

First, according to Empeocles, what is assimilated is a material particular, a chromatic effluence conceived as a fine body composed of fire with a distinctive magnitude. On Aristotle's reinterpretation of the lantern analogy, what is assimilated is not a material particular, but the state of the external medium, a state sustained by the illuminating presence and activity of the fiery substance. The state of the internal medium need not be same as the state of the external medium. Internal and external media may differ in their degree of transparency, and so dark eyes may be illuminated to a lesser degree than the surrounding air. Nevertheless, the state of illumination of the interior water is determined by the amount of fiery substance encountered in the external medium and the degree of transparency of the internal medium, the degree to which it is susceptible to the illuminating presence and activity of the fiery substance. 

Second, the transparency of the interior water of the eye has, for Aristotle, a function that occular passages play in Empeocles' theory of vision. Chromatic effluences, because of their distinctive magnitudes, are commensurate with ocular passages. As such, they can travel through such passages and so be palpable to the organ of sight. While a state lacks space occupancy, and so cannot travel, the external light is extended within, in the sense in which it can, so that the color of remote external particulars may be present in visual consciousness. It is thanks to the illuminated state of the water in the eye's interior that the perceiver is able to assimilate the chromatic form of dital particulars in the natural environment.

The present understanding of Aristotle's deployment of the lantern analogy is further confirmed by the empirical evidence that Aristotle marshals at the end of this passage. That the eye must be ``capable of admitting light \ldots\ is plain from actual occurences'', such as the blindness resulting from a wound sustained by a slash on the temple. The wound severs passages leading from the eye and so impairs the eye's canapacity to endow its possessor with sight. We have here another significant part of the eye's anatomy, the passages from the eye leading within. One should not anachronistically assume that Aristotle has in mind the optic nerve. Given that the seat of sensation is at or near the heart, a better hypothesis would be that they are vessels leading to the heart.

% section transparency_and_the_anatomy_of_the_eye (end)

% chapter the_eye (end)
