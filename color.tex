%!TEX root = /Users/markelikalderon/Documents/Git/formwithoutmatter/aristotle.tex
\chapter{Color} % (fold)
\label{cha:color}

Aristotle defines color as the power to move what is actually transparent (\emph{De Anima} \textsc{ii}.7 418\( ^{a} \)31--418\( ^{b} \)33; \textsc{ii}.7 419\( ^{a} \)10--12).

First, as \citet[367]{Hicks:1907uq} observes, by motion Aristotle does not mean locomotion or change in position. Rather, in \emph{De Anima}, \emph{kinēsis} is Aristotle's general term for change of any kind. Thus \emph{kinētikon} in Aristotle's definition means productive of change rather than productive of spatial movement, more narrowly.

Second, and frustratingly, Aristotle does not directly specify the nature of the change color induces in the transparent medium it acts upon. Indeed, in \emph{De Anima} \textsc{ii}.7 the only effect of color discussed is the effect in terms of which the transparent is defined---the transparent is not visible in itself, but \emph{owning its visibility to the color of another thing}. Is this change, the rendering visible of the transparent, sufficient to understand Aristotle's definition? If it is, this would explain Aristotle's apparent silence about the nature of the change induced in the transparent by color---he merely says nothing \emph{further}, having \emph{already} specified the nature of the change in his definition of the transparent. Given the paucity of textual evidence, a conservative strategy would begin with this hypothesis and only abandon it in favor of speculation should it prove to be an insufficient basis for understanding Aristotle's definition.

Third, a doubt may be registered about the occurrence of transparency in Aristotle's definition. Color is a proper object of sight, and, as such, partly defines the nature of sight. It might reasonably be thought that color could only play a role in defining sight if it had a nature independent of sight. But defining color in terms of the power to move what is actually transparent potentially threatens this order of explanation given the definitional connection between transparency and visibility. It is on these grounds that \citet{Zabarella:1605kx} rejects Aristotle's definition \citep[see][for discussion]{Broackes:1999uq}.

\section{The Objects of Perception} % (fold)
\label{sec:the_objects_of_perception}

Color is an object of sight. What is an object of sight or perception, more generally? Among the objects of perception, Aristotle distinguishes three kinds (\emph{De Anima} \textsc{ii}.6):
\begin{enumerate}[(1)]
	\item Primary objects of sense
	\item Common sensibles
	\item Incidental sensibles
\end{enumerate}

The primary objects of sense must meet two conditions: 
\begin{enumerate}[(1)]
	\item A primary object of a sense must only be percetible by that sense alone;
	\item No error is possible about a primary object of sense.
\end{enumerate}
Both conditions, but especially the second, require elaboration.

A primary object of sense must only be perceptible by that sense alone. There are two elements at work in this condition: (1) the primary objects of sense are \emph{perceptible} to that sense; (2) they are perceptible to that sense \emph{alone}---they are available to no other sensory modality. Thus Aristotle claims that colors can only be seen, sounds can only be heard, and flavors can only be tasted. Common and incidental sensibles differ from the primary objects of sense. Each fails one of the elements of this condition on being a primary object. Incidental objects of perception are not perceptible in themselves, but are perceptible only in the sense that they are incidentally related to something which is perceptible. Thus one can see the son of Diares by seeing a white speck in the distance, but being the son of Diares is not sensible the way that whiteness is. Since incidental sensibles are not percetible in themselves, at least in the present circumstances of perception, they fail the first element of the condition on being a primary object of perception. Common sensibles, unlike incidental sensibles, are perceptible in themselves. They thus satisfy the first element of the condition on being a primary object. However, they fail to satisfy the second. We can see a shape, and feel that shape. Common sensibles are common precisely in being perceptible to more than one sense.

No error is possible about a primary object of sense. More specifically, no error is possible about being presented with the primary object of a sense. No error is involved in a color being present in sight, though one may be mistaken about the location of the presented color. One striking thing about this second condition is its negative characterization. This is potentially philosophically significant since there are two ways to understand this denial. No error may be possible either in the sense that:
\begin{enumerate}[(1)]
	\item Perceptions of primary objects are always true or correct; or
	\item Perceptions of primary objects are not the kind of thing that can be true or false, correct or incorrect.
\end{enumerate}
If the perception of primary objects were always true or correct, then no error would be possible, at least about their presence. If, however, the perception of primary objects were not the kind of thing that so much as could be true or false, correct or incorrect, no error would be possible, but in a difference sense. The sensing of primary objects would be impervious to error not because of some guarantee that the primary objects of a sense falls within its ken but because the sensing of a primary object fails to be evaluable as correct or incorrect.

While Aristotle's usual formulation is that no error is possible about the presence of primary objects, he does sometimes say that the perception of primary objects is always true. This provides \emph{prima facie} support for the first interpretation. Against this suggestion, an advocate of the second interpreation might claim that, by itself, this leaves unexplained what needs explaining, Aristotle's apparent preference for the negative characterization. Aristotle's preference for the negative characterization is well explained by the second interpretation. On that understanding, the denial of the possibility of error is not consistent with perceptions being always true, and so the condition could only be expressed by the negative characterization. The problem for the second interpretation is the potential embarassment of explaining away the occasional claim that the perception of primary objects is always true as merely loose talk, if not indeed a slip on Aristotle's part.

We can decide between these rival interpretations by considering Aristotle's theory of error. According to Aristotle, error requires a certain kind of complexity, a complexity that the sensory presentation of the primary objects lacks. Specifically, only with the combination is error possible. The simple presentation of the white of the sun, when not combined with other sensible elements of the scene, is not in error. But not because of any guarantee that color perception is always true. Rather, it is only when sensible objects are combined that the senses may be misled. We cannot be mistaken about the presence of the sun's whiteness upon seeing it, but we can be mistaken about the location of the whiteness, when we combine whiteness, a primary object, with other sensibles, such as location, a common sensible. Since the sensory presentation of primary objects does not involve combination, and combination is necessary for error, then no error is possible about the presence of sensory objects in the strong sense that their perception is not the kind of thing that so much as could be evalubale as true or false, correct or incorrect.


% section the_objects_of_perception (end)

% chapter color (end)
