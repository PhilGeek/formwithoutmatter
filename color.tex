%!TEX root = /Users/markelikalderon/Documents/Git/formwithoutmatter/formwithoutmatter.tex
\chapter{Color} % (fold)
\label{cha:color}

Aristotle defines color as the power to move what is actually transparent (\emph{De Anima} \textsc{ii}.7 418\( ^{a} \)31--418\( ^{b} \)33; \textsc{ii}.7 419\( ^{a} \)10--12).

First, as \citet[367]{Hicks:1907uq} observes, by motion Aristotle does not mean locomotion or change in position. Rather, in \emph{De Anima}, \emph{kinēsis} is Aristotle's general term for change of any kind. Thus \emph{kinētikon} in Aristotle's definition means productive of change rather than productive of spatial movement, more narrowly.

Second, and frustratingly, Aristotle does not directly specify the nature of the change color induces in the transparent medium it acts upon. Indeed, in \emph{De Anima} \textsc{ii}.7 the only effect of color discussed is the effect in terms of which the transparent is defined---the transparent is not visible in itself, but \emph{owning its visibility to the color of another thing}. Is this change, the rendering visible of the transparent, sufficient to understand Aristotle's definition? If it is, this would explain Aristotle's apparent silence about the nature of the change induced in the transparent by color---he merely says nothing \emph{further}, having \emph{already} specified the nature of the change in his definition of the transparent. Given the paucity of textual evidence, a conservative strategy would begin with this hypothesis and only abandon it in favor of speculation should it prove to be an insufficient basis for understanding Aristotle's definition.

Third, a doubt may be registered about the occurrence of transparency in Aristotle's definition. Color is a proper object of sight, and, as such, partly defines the nature of sight. It might reasonably be thought that color could only play a role in defining sight if it had a nature independent of sight. But defining color in terms of the power to move what is actually transparent potentially threatens this order of explanation given the definitional connection between transparency and visibility. It is on these grounds that \citet{Zabarella:1605kx} rejects Aristotle's definition \citep[see][for discussion]{Broackes:1999uq}.

\section{The Generation of the Hues} % (fold)
\label{sec:the_generation_of_the_hues}

Aristotle's account of the generation of the hues may be surprising, but it would be wrong to prematurely dismiss it. It has ancient precedent. Thus when Theophrastus discusses Democritus's view that there are four primary colors, he contrasts it with the then dominant view that black and white are the two primary colors:
\begin{quote}
    But first of all, his increase of the number of primaries presents a difficulty; for the other investigators propose white and black as the only simple colours. (\emph{De Sensibus} 79; 68 A 135 DK; \citealt{Stratton:1917vn})
\end{quote}
Importantly, Empedocles is among the other investigators.

Not only does the view have ancient precedent, but it can be supported, to some degree, by empirical observation. 

In 1894, an English toy maker, Charles Benham, devised a top adorned with a black and white pattern (see Figure~\ref{fig:benham}). Sold through Messrs. Newton and Co., an announcement of the ``Artificial Spectrum Top'' was published in \emph{Nature}:
	\begin{quote}
		The top consists of a disc, one half of which is black, while the other half has twelve arcs of concentric circles drawn upon it. Each arc subtends an angle of forty-five degrees. In the first quadrant there are three such concentric arcs, in the next three more, and so on; the only difference being that the arcs are parts of circles of which the radii increase in arithmetical progression. Each quadrant thus contains a group of arcs differing in length from those of the other quadrants. The curious point is that when this disc is revolved, the impression of concentric circles of different colors is produced upon the retina. If the direction of rotation is reversed, the order of these tints is also reversed. \citep{Benham:1894kx}
	\end{quote}
Specifically, if rotated clockwise, the innermost arcs form reddish rings, the next greenish rings, the next light blue rings, and the outermost arcs form violet rings. If rotated counterclockwise, the pattern is reversed with the innermost arcs now forming violet rings and the outermost reddish rings. The apparent colors of Benham's spinning disk are the ``subjective colors'' first described by \citep{Fechner:1838vn} and, hence, are also sometimes described as ``Fechner-Benham colors''.

\begin{figure}[ht]
    \begin{center}
        \begin{tikzpicture}
			
		\end{tikzpicture}
    \end{center}
    \caption{}
    \label{fig:2}
\end{figure}

Consider a puzzling aspect of the subjective colors of the Benham disk. Each of the spinning arcs reflect light with the same spectral content and with equal average luminance. In advance of observing the spinning disk, one might reasonably expect the spinning arcs to appear as gray rings of equal brightness. Why, then, do the rings appear reddish, greenish, light blue, and violet? The subjective colors of the Benham disk are not completely understood \citep[for a review of some of the color science see][]{Campenhausen:1995yq}. However, this much is clear: The innermost ring appearing reddish is the result of the visual system visually integrating temporal inhomogeneities presented by the spinning disk. 

% section the_generation_of_the_hues (end)


% chapter color (end)
