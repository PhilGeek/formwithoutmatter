%!TEX root = /Users/markelikalderon/Documents/Git/formwithoutmatter/aristotle.tex
\chapter{Color} % (fold)
\label{cha:color}

\section{Aristotle's Explanatory Strategy} % (fold)
\label{sec:aristotle_s_explanatory_strategy}

At the opening of \emph{De Anima} \textsc{ii}.4, Aristotle describes an explanatory strategy to be pursued in his subsequent discussion of the special senses including color vision:
\begin{quote}
	It is necessary for the student of these forms of soul first to find a definition of each, expressive of what it is, and then to investigate its derivative properties, \&c. But if we are to express what each is, viz. what the thinking power is, or the perceptive, or the nutriative, we must go farther back and first give an account of thinking or perceiving; for activities and actions are prior in definition to potentialities. If so, and if, still prior to them, we should have reflected on their correlative objects, then for the same reason we must first determine about them, i.e. about food and the objects of perception and thought. (\emph{De Anima} \textsc{ii}.4 415\( ^{a} \)14--22; Smith in \citealt[26]{Barnes:1984uq})
\end{quote}
Perceptual capacties are to be understood in terms of perceptual activities that are their exercise and what they are the potential for. Thus sight is the perceiver's potential for seeing. In seeing the perceiver exercises their capacity for sight. Moreover, seeing is what sight is the potential for. Sight is for the sake of seeing. Aristotle's thought here is that potentialities are individuated by what they are the potential for, that for the sake of which the perceiver has the relevant capacity. Just as perceptual capacities are understood in terms of perceptual activities, perceptual activities are themselves to be understood in terms of their object. More specifically, perceptual activities are to be understood in terms of their primary object. The primary object of a given sensory modality must be perceptible to that sense alone and about which no error is possible, at least about its presence. The presentation of the primary object in sense perception is what a sensory capacity is the potential for. Sight is, by its very nature, the potential to bring the colors of remote external particulars into view when it is light and to bring ``fiery or shining'' (\emph{De Anima} \textsc{ii}.7 419\( ^{a} \)1--2) things into view when it is dark. Not every object of perception is a primary object. We can see motion and feel motion. But Aristotle maintains that we see motion only incidentally. His thought seems to be that sight is the potential for seeing colors in light and the fiery or shining in dark. This capacity enables us to see a variety of other objects as well. But sight is not for the sake of seeing motion or magnitude. Sight is for the sake of seeing color in light and the fiery or shining in dark. The non-primary objects of sight are incidental in the sense of being incidental to the nature of sight. The nature of sight as a potentiality is wholly determined by what it is a potential for, to present in visual consciousness colors in light, and the fiery or shining in dark.

% section aristotle_s_explanatory_strategy (end)

\section{The Objects of Perception} % (fold)
\label{sec:the_objects_of_perception}

Among the objects of perception, Aristotle distinguishes three kinds (\emph{De Anima} \textsc{ii}.6):
\begin{enumerate}[(1)]
	\item Primary objects of sense
	\item Common sensibles
	\item Incidental sensibles
\end{enumerate}
Not only is color an object of sight but it is a primary object of sight as well.

The primary objects of sense must meet two conditions: 
\begin{enumerate}[(1)]
	\item A primary object of a sense must be percetible to that sense alone;
	\item No error is possible about a primary object of sense.
\end{enumerate}
Both conditions, but especially the second, require elaboration.

First, a primary object of sense must be perceptible to that sense alone. This condition has two parts: (1) the primary objects of sense are \emph{perceptible} to that sense; (2) they are perceptible to that sense \emph{alone}---they are available to no other sensory modality. Thus Aristotle claims that colors can only be seen, sounds can only be heard, and flavors can only be tasted. Common and incidental sensibles differ from the primary objects of sense---each fails one part of this condition on being a primary object. Incidental objects of perception are not perceptible in themselves, at least in the circumstances of perception, but are perceptible only in the sense that they are incidentally related to something that is perceptible. Thus one can see the son of Diares by seeing a white speck in the distance, but being the son of Diares is not sensible in these circumstances the way that whiteness is. Since incidental sensibles are not percetible in themselves, at least in the given circumstances of perception, they fail the first part of the condition on being a primary object of perception. Common sensibles, unlike incidental sensibles, are perceptible in themselves. They thus satisfy the first part of the condition on being a primary object. However, they fail to satisfy the second. We can see the motion of a sensible particular and feel that motion. Common sensibles are common precisely in being perceptible to more than one sense.

Aristotle inherits this first condition on being a primary object---that it be perceptible to one sense alone---from Plato's discussion of perception in the \emph{Theaetetus}:
\begin{quotation}
	\textsc{socrates}: And are you also willing to admt that what you perceive through one power, you can't perceive through another? For instance, what you perceive through hearing, you couldn't perceive through sight, and similarly what you perceive through sight you couldn't perceive through hearing?
	
	\textsc{theaetetus}: I could hardly refuse to grant that. (\emph{Theaetetus} 184\( ^{e} \)8--185\( ^{a} \)3)
\end{quotation}
Notice that Plato links objects being perceptible to one sense alone to a conception of the senses as powers or capacities. Two thoughts seem to be at work here: (1) that powers or capacities are individuated by their proper exercise, and (2) that the proper exercise of sensory capacities is the presentation of its proper object in sensory awareness. These two claims in conjunction with specific claims about the proper objects of vision and audition imply that sight just is the capacity to see color, and audition just is the capacity to hear sound. If that is right, then, at least in broad outline, the avowed explanatory strategy of \emph{De Anima} \textsc{ii}.4 has its roots in Aristotle's reading of the \emph{Theaetetus}. Of course, Aristotle departs from Plato in crucial ways. Plato seems to limit sense perception to the presentation of the primary objects in sensory awareness. However, Aristotle allows the senses to present objects common to other sensory modalities, though this is incidental to their nature, a nature that wholly consists in their potential to present their proper objects. Plato, in contrast, maintains that what is ``common'' to the objects of the senses---that they are each the same and different from the others---is determined by cognitive, not pereptual, capacities (\emph{Theaetetus} 184). This is the basis for a further difference. Aristotle maintains that we can discriminate among the presented objects of sense and that this is the exercise of perceptual, not cognitive, capacities (for discussion of just how far Aristotle extends the perceptual domain as Plato conceives of it see \citealt{Sorabji:2003fk} as well as his earlier discussion \citealt{Sorabji:1971fr}).

% (And arguably the \emph{Sophist}, especially the Eleatic Visitor's ammendation to the cosmology of the Giants: 
% \begin{quotation}
% 	\textsc{visitor}: I'm saying that a thing really is if it has any capacity at all, either by nature to do something to something else or to have even the smallest thing done to it by evven the most trivial thing, even if it only happens once. I'll take it as a definition that \emph{those which are} amount to nothing other than \emph{capacity}. (\emph{Sophist} 247\( ^{d} \)8--247\( ^{e} \)4; White in \citealt[269]{Cooper:1997fk})
% \end{quotation}
% For a suggestive discussion of how Aristotle may be defending and elaborating the Elaetic Visitor's suggestion see \citealt[chapter 1]{Beere:2009vn}.)

Second, not only must proper objects be perceptible to one sense alone, but about their presence no error is possible. No error may be involved in a color being present in sight, though one may be mistaken about the location of the presented color. One striking thing about this second condition is its negative characterization. This is potentially philosophically significant since there are two ways to understand this denial. No error may be possible either in the sense that:
\begin{enumerate}[(1)]
	\item Perceptions of primary objects are always true or correct; or
	\item Perceptions of primary objects are not the kind of thing that can be true or false, correct or incorrect.
\end{enumerate}
If the perception of primary objects were always true or correct, then no error would be possible, at least about their presence. If, however, the perception of primary objects were not the kind of thing that so much as could be true or false, correct or incorrect, no error would be possible, but in a difference sense. The sensing of primary objects would be impervious to error not because of some guarantee that the primary objects of a sense falls within its ken but because the sensing of a primary object fails to be evaluable as correct or incorrect. 

This latter thesis is the one that Travis attributes to Descartes:
\begin{quotation}
	\noindent Over a wider area which includes perception, Descartes said this:
	\begin{quote}
		By the mere intellect I do no more than perceive the ideas that are matters for judgment; and precisely so regarded the intellect contains, properly speaking, no error. (fourth meditation)
		
		Whence, then, do my errors originate? Surely just from this: my will extends more widely than my understanding, and yet I do not restrain it within the same bounds, but apply it to what I do not understand. (fourth meditation)
	\end{quote}
	Sensory experience is, for Descartes, one more case where I am simply confronted with `ideas'. I cannot be confronted correctly or incorrectly, veridically or deceptively. I simply confront what is there. Perception leads me astray only where I judge erroneously, failing to make out what I confront for what it is. \citep[64--65]{Travis:2004kx}
\end{quotation}

While Aristotle's usual formulation in Book \textsc{ii} of \emph{De Anima} is that no error is possible about the presence of primary objects, he does sometimes say, especially in Book \textsc{iii} that the perception of primary objects is always true. This provides \emph{prima facie} support for the first interpretation. On this interpretation, sense perception has something like an intentional or representational content; it is at least evalubale as true or false, correct or incorrect. Against this suggestion, an advocate of the second interpreation might claim that, by itself, this leaves unexplained what needs explaining---\-Aristotle's apparent preference for the negative characterization. Aristotle's preference for the negative characterization is well explained by the second interpretation. On that interpretation, the denial of the possibility of error is not consistent with perceptions being always true, and so the condition could only be expressed by the negative characterization. The problem for the second interpretation is the potential embarassment of explaining away the claim that the perception of primary objects is always true as merely loose talk if not indeed a slip on Aristotle's part.

We can decide between these rival interpretations by considering Aristotle's account of error (\emph{De Anima} \textsc{iii}.3 428\( ^{b} \)17-26, 430\( ^{a} \)27--430\( ^{b} \)5). According to Aristotle, error requires a certain kind of complexity, a complexity that the sensory presentation of the primary objects lacks. Specifically, only with the combination of what is presented in sensation is error possible. Combination is necessary for error:
\begin{quote}
	\ldots\ where the alternative of true or false applies, there we always find a sort of combining of objects of thought in a quasi-unity. As Empedocles said that ``where heads of many a creature sprouted without necks'' they afterwards by Love’s power were combined, so here too objects of thought which were separate are combined \ldots (\emph{De Anima} \textsc{iii}.6 430\( ^{a} \)27--32; Smith in \citealt[54]{Barnes:1984uq})
	
	For falsehood always involves a combining; for even if you assert that what is white is not white you have combined not-white. (\emph{De Anima} \textsc{iii}.6 430\( ^{b} \)1--3; Smith in \citealt[54]{Barnes:1984uq})
\end{quote}
The simple presentation of the white of the sun, when not combined with other sensible elements of the scene, is not in error. But not because of any guarantee that color perception is always true. Rather, it is only when sensible objects are combined that the senses may mislead. We cannot be mistaken about the presence of the sun's whiteness upon seeing it, but we can be mistaken about the location of the whiteness, when we combine whiteness, a primary object, with other sensibles, such as location, in this case, a common sensible. Since the sensory presentation of primary objects does not involve combination, and combination is necessary for error, then no error is possible about the presence of these sensory objects in the strong sense that their perception is not the kind of thing that so much as could be evalubale as true or false, correct or incorrect. In sensory consciousness we simply confront the primary object of the given modality. We cannot be confronted truly or falsely, correctly or incorrectly. We simply confront what is presented to us in sensory consciousness. 

This is the basis for the second contrast that Aristotle draws between perception and understanding in the following passage from Book \textsc{iii} of \emph{De Anima}:
\begin{quote}
	That perceiving and understanding are not identical is therefore obvious; for the former is universal in the animal world, the latter is found in only a small division of it. Further, thinking is also distinct from perceiving---I mean that in which we find rightness and wrongness---right\-ness in understanding, knowledge, true opinion, wrongness in their opposites; for perception of the special objects of sense is always free from error, and is found in all animals, while it is possible to think falsely as well as truly, and thought is found only where there is discourse of reason. (\emph{De Anima} \textsc{iii}.3 427\( ^{b} \)7--15; Smith in \citealt[49]{Barnes:1984uq})
\end{quote}
All animals perceive, but not all animals are rational. Rational activity, such as thinking, is evaluable as correct or incorrect. But the perception of primary objects, being the simple presentation of these sensory objects, are insusceptible to error in this way. The line of reasoning behind this way of contrasting perception and understanding can be found in the \emph{Theaetetus}, on at least some interpretations (see \citealt{Cooper:1970kx} and \citealt{Burnyeat:1990dp}). So it is possible that the second condition on being a primary object itself derives from Aristotle's reading of the \emph{Theaetetus} as well.

The primary object of sight is the visible. What is visible is either color or ``a certain kind of object which can be described in words but which has no single name'' (\emph{De Anima} \textsc{ii}.7 418\( ^{b} \)4). So color is \emph{a} primary object of sight not \emph{the} primary object of sight. That a sense can have a plurality of primary objects is consistent with Aristotle's two defining conditions on being a primary object---that it be perceptible to one sense alone and about whose presence no error is possible. Distinct kinds of objects can each satisfy these conditions. So it does not follow from Aristotle's definition of primary objects that for each sense there is exactly one primary object. If there is a problem, especially if, as in the case of touch, there are too many primary objects, this must be due not soley to the definition of primary object but must involve as well further explanatory assumptions. As we shall see, any difficulty posed by a plurality of primary objects is due less to the definition of primary object than to the explantory role they play in Aristotle's avowed strategy of explaining perceptual capacities in terms of perceptual activities and explaining perceptual activities in terms of their primary objects.

Colors depend upon light for their visibility. But not everything depends upon light for their visibility. That which has no name does not so depend:
\begin{quote}
	Some objects of sight which in light are invisible, in darkness stimulate the sense; that is, things that appear fiery or shining. This class of objects has no simple common name, but instances of it are fungi, horns, heads, scales, and eyes of fish. In none of these is what is seen their own proper colour. Why we see these at all is another question. (\emph{De Anima} \textsc{ii}.7 419\( ^{a} \)xx--xx)
\end{quote}
Philoponus, in his commentary on \emph{De Anima}, reports a slightly different list of examples:
\begin{quote}
	\ldots\ glow worms, heads of fish, fish scales, eyes of hedgehogs, shells of sea-creatures, which things are seen not in light but in dark. (\emph{On \emph{De Anima}} 319 25--27; \citealt[3]{Charlton:2005fk})
\end{quote}
That which has no name possess qualities visible in the dark and not the light and differ from the proper colors of these same things which are visible in the light and not the dark. This is most likely the source of Austin's example from \emph{Sense and Sensibilia} (an appropriately Aristotelian title, at least in the present context):
\begin{quote}
	Suppose \ldots\ that there is a species of fish which looks vividly multi-coloured, slightly glowing perhaps, at a depth of a thousand feet. I ask you what its real colour is. So you catch a specimen and lay it out on deck, making sure the condition of the light is just about normal, and you find that it looks a muddy sort of greyish white. Well, is \emph{that} its real colour? \citep[lecture \textsc{vii}, 65--66]{Austin:1962lr}
\end{quote}
In the darkness, at the depth of a thousand feet, the fish may look vividly multi-colored and slightly glowing, but on the sun drenched deck they look a muddy sort of greyish white. Aristotle would contend that only the latter is the creature's proper color, the former being an instance of that which has no name. Austin is, of course, making a different point with the Aristotelian example, that the `real' color of a thing may depend on the practical point of attributing color to it in the circumstances of saying.

There is a question about how broadly the domain of that which has no name extends. Some commentators have suggested that shining be interepreted as reflective highlights. So fish scales, having a highly reflective surface, can produce highlights discernable even in conditions of very low illumination resulting in a shimmering effect set amidst the surrounding darkness. The trouble with this interpretation is that it does not fit all of Aristotle's examples---fungi lack smooth, reflective surfaces and so give rise to no reflective highlights. One plausible thought, supported by Philoponus' additional example of glow worms, and exploited by Austin in his appropriation, is that these are examples of bioluminescence. One minor problem with this interpretation is that the eyes of hedgehogs glowing in a dark field---assuming, for the moment, that Philoponus' example is of genuine Aristotelian province---are not radiant light sources the way that they appear to be and the way that cases of  bioluminescence genuinely are. Rather, they are reflecting ambient light in circumstances of low illumination, say, from a lantern of shaved horn held by an ancient spectator traversing the field at night. All of Aristotle's examples are biological, but the claim that that which has no name is visible in the absence of light suggests a generalization. After all, there are mineral deposits, of phosphorus, say, whose glow can only be seen in the absence of competing illumination. So perhaps that which has no name includes not only the bioluminescent, but the luminous more generally. Philoponus suggests this broader interpretation and provides the nice example of starlight, visible only in the absence of the sun's light (\emph{On \emph{De Anima}} 347 11). 

It may be tempting to think that the difference between color and the luminous as Aristotle conceives of it is the difference between, as we we put it, surface color and the color of radiant light sources. But this would be a mistake. Among the examples that Aristotle gives of colored particulars are radiant light sources, prominently, the sun. Moreover, if the defining feature of the luminous is its visibility in darkness, then, as our discussion has made clear, the luminous should include not only radiant light sources but reflections as well. The contrast between color and the luminous is not the contrast between surface color and radiant color, but the contrast between light and dark as conditions on the visibility of distinct kinds of objects.

Suppose then, that the two species of the visible are to be understood in this way. Whereas color is visible in the light, the luminous is visible in the dark. One problem with Aristotle's view, so interpreted, is that the contrast does not consitute a partition and so could not demarcate distinct species of visibilia. Some things are visible in the light \emph{and} the dark, as Philoponus observes:
\begin{quote}
	For some of them are super-shining, some are dimly shining, and some middling. Those that are dimly shining are seen only at night, such as glow-worms and fish scales and the like; for their shining does not appear by day, being overcome by a greater. And also a majority of stars. Those that are middling are seen both at night and by day, such as the moon and some of the stars, for instane the Morning Star when the sun is near the horizon and the Morning Star itself is near its perigee. Fire also. For this perfects air so as to show also the colours that are in it, but in the rest it shows itself, indeed, but does not bring transparency to that part to actuality. Hence we see itself when we are a long way off in the dark, but none of the colours around us. By day, again, fire appears as something shining, but not as doing anything to the air because that is already affected by a greater shining, and then it appears in a way like the other colours, but more in the way of the shining of the moon also, because it is not too dim, appears by day; so also with the shining of fire, when it is shown not far away and it is light. But the super-shining are seen only by day, viz. the sun, since indeed it is the cause of day and of light. (\emph{On \emph{De Anima}} 347 7--24; \citealt[32]{Charlton:2005fk})
\end{quote}

It is hard to know how Aristotle would address this difficulty since he gives no explanation for how the luminous may be visible in the dark. Recall, against Democritus, Aristotle argues, on general grounds (\emph{De Anima} \textsc{ii}.7 418\( ^{b} \)13--22), that remote objects of perception require a medium if they are to act upon the perceiver's sense organ, which they must do since sensation is a mode of sensitivity, a reactive capacity. There is no action at a distance. Action, or at any rate immediate action, requires contact. But the remote objects, being remote, are not in contact with the perceiver's sense organ. But this is consistent with them acting upon the perceiver's sense organ mediately, by acting upon something else which is in contact, that is to say, by acting upon an intervening medium. The reasoning here is sufficiently general to hold true, as well, of seeing the luminous set admidst the surrounding darkness. Moreover, just as in the case of color vision, the medium is transparent, the difference being that the medium, be it air or water, is potentially if not actually transparent. Exactly how, though, the luminous acts upon the potentially if not actually transparent, Aristotle declines to say, simply dropping the matter. Indeed, he proceeds to speak as if color were the sole primary object of sight. But exactly how the luminous acts upon the potentially if not actually transparent medium is an essential part of the explanation for how the luminous may be visible in the dark. In the absence of such an explanation, there is no saying what Aristotle might say about the formal difficulty raised by Philoponus' observation. Aristotle must have been apprised of this difficulty since he makes the crucial observation himself:
\begin{quote}
	Fire on the other hand is seen both in darkness and in light; this double possibility follows necessarily from our theory, for it is just fire that makes what is potentially transparent actually transparent. (\emph{De Anima} \textsc{ii}.7 419\( ^{a} \)23--24; Smith in \citealt[34]{Barnes:1984uq})
\end{quote}

Earlier I observed that a plurality of primary objects is consistent with Aristotle's two defining conditions on being a primary object of sense. If a difficulty is posed by a plurality of primary objects, this must be due to further explanatory assumptions, indeed assumptions linked to Aristotle's avowed explanatory strategy (\emph{De Anima} \textsc{ii}.4). However, understood in this light, there is less a difficulty about the plurality of primary objects, nor even about their being large in number, than about their diversity. It is a lack of unity among a plurality of primary objects that is potentially puzzling in the context of Aristotle's explanatory strategy. 

This is what underlies Aristotle's contrasting attitudes towards vision and touch. The proper object of vision is the visible, and there are two species of visibilia. About this Aristotle is apparently sanguine, if not terribly forthcoming. But when it comes to the plurality of primary objects of touch, Aristotle remarks that this is a source of puzzlement (\emph{aporia}). The difference concerns the unity displayed by the visible and the diversity displayed by the tangible. 

Color is visible in the light and the luminous is visible in the dark. Light is a fundamental condition for the visibility of color, just as dark is a fundamental condition for the visibility of the luminous. So understood light and dark are states of potentially transparent media such as air and water, light being the state of the medium when it is actually transparent and dark being the state of the medium when it is not actually but only potentially transparent. There is a related sense in which light and dark are visible qualities of these states. Air and water, when actually transparent, is light or bright, just as it is dark when it is not actually but only potentially transparent. Light or brightness is the color of the actually transparent (\emph{De Anima} \textsc{ii}.7 418\( ^{b} \)11-12; \emph{De Sensu} \textsc{iii} 439\( ^{b} \)1--2); and so dark is the color of what is not actually transparent. Moreover, so understood, light and dark are themselves united as opposing contraries. (Contrast contraries from different opposing pairs of sensible qualities, light and loud, say.) Furthermore, the opposing pair of contraries, light and dark, have a common determinant, the presence or absence, respectively, of the fiery substance in a potentially transparent medium. 

The tangible, in contrast, is remarkable and potentially puzzling because of the diversity of its objects. ``Touch, indeed discriminates more than one set of different qualities'' (\emph{De Anima} \textsc{ii}.6 418\( ^{a} \)14; Smith in \citealt[32]{Barnes:1984uq}). The tangible comprises a variety of opposing pairs of sensible qualities, including hot and cold, dry and wet, hard and soft, and so on. And it is the apparent lack of unity among these opposing pairs of contraries that is the source of puzzlement. Aristotle illustrates this apparent lack of unity by contrasting the tangible with the audible. There are a plurality of opposing pairs of contraries that qualify the audible---sharp and flat, loud and soft, smooth and rough, and so on---but these are all intelligibly qualities of a single \emph{substrata}, sound. ``Nevertheless we are unable clearly to detect in the case of touch what the single subject is which corresponds to sound in the case of hearing'' (\emph{De Anima} \textsc{ii}.11 422\( ^{b} \)31--32; Smith in \citealt[40]{Barnes:1984uq}). We need not suppose that the requisite unity necessarily takes the form it does in the case of audibilia, that the plurality of opposing pairs of contraries are united by a common substrata. After all, the unity of the visible was not achieved in this way. All that is required is that the plurality of primary objects be united in a manner sufficient to discharge their explanatory role. So what is puzzling is not the plurality of primary objects of touch, nor their being large in number, but the apparent lack of unity displayed among the primary objects of touch.

The diversity of the tangible is puzzling when set in the context of Aristotle's avowed explanatory strategy. Aristotle is seeking definitions of each of the soul's capacities, such as an animal's cognitive, perceptual, and nutriative capacities, that are ``expressive of what it is''. Capacities are a species of potentiality. What a capacity is depends upon what it is a potential for. What makes sight sight is that it is a potential for seeing. Given the special nature of perceptual and epsitemic activities---they each take an object---a definition of perceptual activity expressive of what it is would make essential reference to its object. Seeing is sensory awareness of color just as hearing is sensory awareness of sound. And it is their primary objects that distinguish these perceptual activities. If the primary objects of sense are to play this explantory role, if perceptual activities are to be understood in terms of the presentation of these objects, and perceptual capacities are to be understood in terms of the perceptual activities that are their proper exercise, then primary objects of a given sense, if they constitute a plurality, must display sufficient unity to play this explanatory role. Diversity among them would, within this framework, naturally lead one to wonder whether there really was one sensory modality at work here rather than many. Why should there be a single sense that affords awareness of hot and cold, wet and dry, hard and soft; rather than there being seperate senses of temperature, moisture, and texture? This is precisely the puzzlement that inagurates Aristotle's discussion of touch in \emph{De Anima} \textsc{ii}.11.

The unity of the visible bears on the significance of Aristotle's declining to explain the visibility of the luminous. In a cynical mood, it is tempting to understand Aristotle's begging off as the expression of embarassment at not having an adequate explanation. On this understanding, it is at least an open possibility that no adequate explanation is to be had. However, the unity of the visible makes it possible to understand Aristotle's declining to explain the visibility of the luminous in a different way. Rather than an expression of embarassment, Aristotle's begging off is the expression of confidence that an explanation of the visibility of the luminous, consistent with the general principles at work in his explantion of the visibility of color, is possible. And it is the unity of the visible, that which unites color with the luminous, being qualities whose visibility is determined by the presence or absence of the fiery substance in potentially transparent media, that is the grounds of such confidence. Aristotle's habit of speaking as if color is the sole primary object of sight can be understood as the expression of this confidence grounded in the unity displayed by the plurality of the visible. Without pausing to consider further whether this confidence is misplaced, we shall follow Aristotle in proceeding to speak as if color were the sole primary object of sight.

% section the_objects_of_perception (end)

\section{The Definition of Color} % (fold)
\label{sec:the_definition_of_color}

Aristotle's explanatory strategy also importantly constrains the interpretation of his definition of color. Color is the primary object of sight. Given, Aristotle's avowed explanatory strategy, color, so defined, must be understood independently of color perception.

Aristotle defines color as the power to move what is actually transparent (\emph{De Anima} \textsc{ii}.7 418\( ^{a} \)31--418\( ^{b} \)33; \textsc{ii}.7 419\( ^{a} \)10--12).

First, color is a \emph{power} of external particulars, a power that is exercised in moving what is actually transparent. A particular may possess this power even when it is not exercised---when the medium is potentially but not actually transparent, say. It is thus a kind of, or at least grounds a, potentiality. Aristotle stresses the importance of distinguishing the potential and actual colors of things:
\begin{quote}
	
\end{quote}

First, as \citet[367]{Hicks:1907uq} observes, by motion Aristotle does not mean locomotion or change in position. Rather, in \emph{De Anima}, \emph{kinēsis} is Aristotle's general term for change of any kind. Thus \emph{kinētikon} in Aristotle's definition means productive of change rather than productive of spatial movement, more narrowly.

Second, and frustratingly, Aristotle does not directly specify the nature of the change color induces in the transparent medium it acts upon. Indeed, in \emph{De Anima} \textsc{ii}.7 the only effect of color discussed is the effect in terms of which the transparent is defined---the transparent is not visible in itself, but \emph{owning its visibility to the color of another thing}. Is this change, the rendering visible of the transparent, sufficient to understand Aristotle's definition? If it is, this would explain Aristotle's apparent silence about the nature of the change induced in the transparent by color---he merely says nothing \emph{further}, having \emph{already} specified the nature of the change in his definition of the transparent. Given the paucity of textual evidence, a conservative strategy would begin with this hypothesis and only abandon it in favor of speculation should it prove to be an insufficient basis for understanding Aristotle's definition.

Third, a doubt may be registered about the occurrence of transparency in Aristotle's definition. Color is a proper object of sight, and, as such, partly defines the nature of sight. It might reasonably be thought that color could only play a role in defining sight if it had a nature independent of sight. But defining color in terms of the power to move what is actually transparent potentially threatens this order of explanation given the definitional connection between transparency and visibility. It is on these grounds that \citet{Zabarella:1605kx} rejects Aristotle's definition \citep[see][for discussion]{Broackes:1999uq}.

% section the_definition_of_color (end)

\section{Color and Transparency} % (fold)
\label{sec:color_and_transparency}

One of the novel claims of \emph{De Sensu} is that color resides in the proportion of transparent that exists in all bodies.

% section color_and_transparency (end)

% chapter color (end)
