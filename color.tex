%!TEX root = /Users/markelikalderon/Documents/Git/formwithoutmatter/formwithoutmatter.tex
\chapter{Color} % (fold)
\label{cha:color}

Aristotle defines color as the power to move what is actually transparent (\emph{De Anima} \textsc{ii}.7 418\( ^{a} \)31--418\( ^{b} \)33; \textsc{ii}.7 419\( ^{a} \)10--12).

First, as \citet[367]{Hicks:1907uq} observes, by motion Aristotle does not mean locomotion or change in position. Rather, in \emph{De Anima}, \emph{kinēsis} is Aristotle's general term for change of any kind. Thus \emph{kinētikon} in Aristotle's definition means productive of change rather than productive of spatial movement, more narrowly.

Second, and frustratingly, Aristotle does not directly specify the nature of the change color induces in the transparent medium it acts upon. Indeed, in \emph{De Anima} \textsc{ii}.7 the only effect of color discussed is the effect in terms of which the transparent is defined---the transparent is not visible in itself, but \emph{owning its visibility to the color of another thing}. Is this change, the rendering visible of the transparent, sufficient to understand Aristotle's definition? If it is, this would explain Aristotle's apparent silence about the nature of the change induced in the transparent by color---he merely says nothing \emph{further}, having \emph{already} specified the nature of the change in his definition of the transparent. Given the paucity of textual evidence, a conservative strategy would begin with this hypothesis and only abandon it in favor of speculation should it prove to be an insufficient basis for understanding Aristotle's definition.

Third, a doubt may be registered about the occurrence of transparency in Aristotle's definition. Color is a proper object of sight, and, as such, partly defines the nature of sight. It might reasonably be thought that color could only play a role in defining sight if it had a nature independent of sight. But defining color in terms of the power to move what is actually transparent potentially threatens this order of explanation given the definitional connection between transparency and visibility. It is on these grounds that \citet{Zabarella:1605kx} rejects Aristotle's definition \citep[see][for discussion]{Broackes:1999uq}.


% chapter color (end)
