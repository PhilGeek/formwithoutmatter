%!TEX TS-program = xelatex 

%!TEX encoding = UTF-8 Unicode
%
%  proposal
%
%  Created by Mark Eli Kalderon on 2013-03-29.
%  Copyright (c) 2013. All rights reserved.
%

\documentclass[12pt]{article} 

% Definitions
\newcommand\mykeywords{Aristotle, perception, color}
\newcommand\myauthor{Mark Eli Kalderon}

% Packages
\usepackage{geometry} \geometry{a4paper}

% XeTeX
\usepackage[cm-default]{fontspec}
\usepackage{xltxtra,xunicode}
\defaultfontfeatures{Scale=MatchLowercase,Mapping=tex-text}
\setmainfont{Hoefler Text}

% PDF Stuff
\usepackage[plainpages=false, pdfpagelabels, bookmarksnumbered, backref, pdftitle={Form Without Matter}, pagebackref, pdfauthor={\myauthor}, pdfkeywords={\mykeywords}, xetex, colorlinks=true, citecolor=gray, linkcolor=gray, urlcolor=gray]{hyperref}

%%% BEGIN DOCUMENT
\begin{document}

% % Title Page
\author{\myauthor}
\title{Book Proposal\\
Form without Matter\\
Empedocles and Aristotle on Color Perception}
\date{}

\maketitle

% Layout Settings
\setlength{\parindent}{1em}

% Main Content

\section{Overview} % (fold)
\label{sec:overview}

Aristotle's definition in \emph{De Anima} of perception as the assimilation of sensible form without the matter of the perceived object is notoriously difficult to interpret. The present essay provides a novel interpretation of Aristotle's definition by reading it in light of a puzzle about sensory presentation to be found in the work of Empedocles. Empedocles held a general conception of sensory awareness for which ingestion provides the model. In order for something to be perceived it must be taken within so that it may be in contact with the sense organ. This raises a puzzle about color vision since color vision presents itself as the perception of the colors of distant particulars. Empedocles resolves this puzzle with his theory of effluences. If the colors of distant particulars are the effluences that they emit, then the colors may be assimilated by the organ of sight and so be seen. While Aristotle rejects the theory of effluences and the claim that to be perceptible is to be palpable to sense, he retains a conception of sensory awareness as a mode of assimilation. Thus it is natural to think of perception as a mode of taking in. But how can we take in what remains external? And if we can, what does taking in here mean such that we could? A generalized form of Empedoclean puzzlement consists in the persistence of this latter question. This puzzlement persists to this day. Thus Broad remarks that  ``It is a natural, if paradoxical, way of speaking to say that seeing seems to ‘bring us into \emph{contact} with \emph{remote} objects’ and to reveal their shapes and colors.'' What is novel in the present essay is the attempt to understand Aristotle's definition of perception as a response to such puzzlement. The assimilation of sensible form is meant to be the sense in which we take in the scene before us.

The present essay is an essay in the philosophy of perception written in the medium of historiography. It aims to both articulate a philosophical problem about the nature of sensory presentation that contemporary philosophers of perception are scarcely conscious of as well as present Aristotle's solution. In so doing, I make important claims about the metaphysics of sensory presentation.

% section overview (end)

\section{The Audience} % (fold)
\label{sec:the_audience}

The present essay has a divided audience. It is addressed both to philosophers of perception as well as historians (as well as graduate students working in this field). I use Aristotle's texts as a means of attending to the phenomena under investigation and so make philosophical claims about color and color perception. But at the same time I mean to be making important claims of interpretation. Hence the divided audience.

% section the_audience (end)

\section{The Competition} % (fold)
\label{sec:the_competition}

There are a number of recent books about perception in Aristotle. Let me confine myself to monographs (thus ignoring anthologies), and only to those monographs that have significant overlap of concern with the present essay:

\begin{enumerate}
	\item Deborah Modrak, 1987, \emph{Aristotle, The Power of Perception}, University of Chicago Press.
	\item T.K. Johansen, 1997, \emph{Aristotle on the Sense Organs}, Cambridge University Press.
	\item Stephen Everson, 1997, \emph{Aristotle on Perception}, Oxford University Press.
	\item Pavel Gregoric, 2007, \emph{Aristotle on Common Sense}, Oxford University Press.
	\item Ronald Polansky, 2007, \emph{Aristotle's} De Anima, Cambridge University Press.
\end{enumerate}

Modrak's book is now sadly out of print. Johansen's has a focus on the nature of the sense organs. It is thus both broader and narrower than the present work. It is broader since I only attend in detail to the organ of sight. It is narrower since it's focus is on why we need sense organ's if we are to perceive. Everson defends a ``literalist'' interpretation familiar from Slakey and Sorabji according to which sense organs literally take on the sensible form of the perceived object. (In contrast to Johansen who defends Burnyeat's ``spiritualist'' interpretation.) It is broader in scope than the present essay. As is Polansky's book. Polansky aims to provide a systematic overview of \emph{De Anima}. Gregoric's book is primarily focused on the common sense, a topic not addressed in the present essay, but I include it since it has, as well, important discussions about Aristotle on perceptual capacities.

The present essay differs from these in its aim and scope. While many of these discuss Aristotle's Presocratic sources, none discuss the Empedoclean puzzlement let alone use that as a guide to reading Aristotle. Since the Empedoclean puzzlement, in its original form, arises with respect to color vision, the scope of the present essay is somewhat narrower than these since color and color perception is the primary focus.

% section the_competition (end)

\section{Chapters} % (fold)
\label{sec:chapters}

The present essay consists of 9 chapters:

\begin{enumerate}
	\item \emph{Empedocles} Discusses the original form of Empedoclean puzzlement and advances an interpretation of Empedocles' theory of vision.
	\item \emph{Perception at a Distance} Discusses Aristotle's commitments that lead to Empedoclean puzzlement as well as his anit-Empedoclean arguments.
	\item \emph{Transparency} Discusses Aristotle on transparency both in \emph{De Anima} and \emph{De Sensu}.
	\item \emph{Color} Discusses Aristotle's definition of color.
	\item \emph{Light and Dark} Discusses the Presocratic antecedents of Aristotle's account of the hues, specifically in Parmenides and Empedocles.
	\item \emph{The Generation of the Hues} Discusses Aristotle's three models of how the hues may be generated from a proportion of light and dark.
	\item \emph{The Eye} Discusses the Aristotelian anatomy of the eye.
	\item \emph{Two Transitions to Actuality} Discusses the nature of our capacity for sight as well as the nature of its exercise.
	\item \emph{Form without Matter} Discusses Aristotle's definition of perception as a resolution of the generalized form of Empedoclean puzzlement
\end{enumerate}

% section chapters (end)

\section{Illustrations} % (fold)
\label{sec:book_length_illustrations}

I have, in an experimental spirit, included illustrations. These include diagrams and reproduction of artworks. Allow me to explain the rationale. Aristotle is notoriously brief in his psychological works. His visual examples thus require careful elaboration. A lot of effort has been devoted to this, and I have cast many in narrative form in order to evoke the reader's use of their visual imagination. The illustrations are meant to be an aid to this in getting the reader to actually look at things. There is an additional motive since the interpretation of some Empedoclean fragments in \emph{Light and Dark} requires an extensive discussion of the actual practices of 5th century \textsc{bc} painting. I have produced all of the diagrams (save one, but will reproduce it myself) and these can be printed in black and white. Many of the reproductions of artworks can be done in black and white but several would lose their point if not in color. In principle all illustrations could be dispensed with, but I feel that they add real value (beyond the decorative). I don't know enough about the costs to say whether they would be prohibitive. I will defer to your editorial decision asking only that my request be considered.

% section book_length_illustrations (end)

\section{Marketing} % (fold)
\label{sec:marketing}

The present essay's divided audience presents a potential marketing challenge. I would not want the book to be in a series devoted to ancient philosophy lest it go unnoticed by the philosophers of perception. But again, I do take myself to be doing history of philosophy.

% section marketing (end)

\section{About the Author} % (fold)
\label{sec:about_the_author}

I am a professor of philosophy at UCL and I have written extensively about color and color perception.

% section about_the_author (end)

\end{document}
