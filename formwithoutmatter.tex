%!TEX TS-program = xelatex 
%!TEX TS-options = -synctex=1 -output-driver="xdvipdfmx -q -E"
%!TEX encoding = UTF-8 Unicode
%
%  formwithoutmatter
%
%  Created by Mark Eli Kalderon on 2011-03-05.
%  Copyright (c) year. All rights reserved.
%

\documentclass[12pt]{article} 

% Definitions
\newcommand\mykeywords{Aristotle, perception, color}
\newcommand\myauthor{Mark Eli Kalderon} 
\newcommand\mytitle{Form Without Matter}

% Packages
\usepackage{geometry} \geometry{a4paper} 
\usepackage{url}
\usepackage{txfonts}
\usepackage{color}
\definecolor{gray}{rgb}{0.459,0.438,0.471}
% \usepackage{setspace}
% \doublespace % Uncomment for doublespacing if necessary
% \usepackage{epigraph} % optional

% XeTeX
\usepackage[cm-default]{fontspec}
\usepackage{xltxtra,xunicode}
\defaultfontfeatures{Scale=MatchLowercase,Mapping=tex-text}
\setmainfont{Hoefler Text}
\setsansfont{Gill Sans}
\setmonofont{Inconsolata}

% Section Formatting
\usepackage[]{titlesec}
\titleformat{\section}[hang]{\fontsize{14}{14}\scshape}{\S{\thesection}}{.5em}{}{}
\titleformat{\subsection}[hang]{\fontsize{12}{12}\scshape}{\S{\thesubsection}}{.5em}{}{}
\titleformat{\subsubsection}[hang]{\fontsize{12}{12}\scshape}{\S{\thesubsubsection}}{.5em}{}{}

% Bibliography
\usepackage[round]{natbib} 

% Title Information
\title{\mytitle} % For thanks comment this line and uncomment the line below
%\title{\mytitle\thanks{}}% 
\author{\myauthor} 
% \date{} % Leave blank for no date, comment out for most recent date

% PDF Stuff
\usepackage[plainpages=false, pdfpagelabels, bookmarksnumbered, backref, pdftitle={\mytitle}, pagebackref, pdfauthor={\myauthor}, pdfkeywords={\mykeywords}, xetex, dvipdfmx, colorlinks=true, citecolor=gray, linkcolor=gray, urlcolor=gray]{hyperref} 



%%% BEGIN DOCUMENT
\begin{document}

% Title Page
\maketitle
% \begin{abstract} % optional
% \noindent
% \end{abstract} 
\vskip 2em \hrule height 0.4pt \vskip 2em
% \epigraph{text of epigraph}{\textsc{author of epigraph}} % optional; make sure to uncomment \usepackage{epigraph}

% Layout Settings
\setlength{\parindent}{1em}

% Main Content

\section{Empedocles} % (fold)
\label{sec:empedocles}
In the \emph{Meno} Socrates attributes to Empedocles a conception of perception as a mode of assimilation of material effluences:
\begin{quotation}
    \textsc{meno}: And how do you define color?
    
    \ldots
    
    \textsc{socrates}: Would you like an answer in the style Gorgias, such as you most readily follow?
    
    \textsc{meno}: Of course I should.
    
    \textsc{socrates}: You and he believe in Empedocles' theory of effluences, do you not?
    
    \textsc{meno}: Wholeheartedly.
    
    \textsc{socrates}: And passages in which and through which the effluences make their way?
    
    \textsc{meno}: Yes.
    
    \textsc{socrates}: Some of the effluences fit into some of the passages whereas others are too great or too small.
    
    \textsc{meno}: That is right.
    
    \textsc{socrates}: Now you recognize the term `sight'?
    
    \textsc{meno}: Yes.
    
    \textsc{socrates}: From these notions, then, `grasp what I would tell,' as Pindar says. Color is an effluence from shapes commensurate with sight and perceptible to it. (\emph{Meno} 76\( ^{a-d} \))
\end{quotation}

The main elements of the account are relatively clear. Objects emit material effluences. Effluences are fine bodies that are kind differentiated in terms of magnitude. There are passages in which and through which material effluences may flow. Whether a material effluence may enter a passage depends on its magnitude. The magnitudes of some kinds of material effluences are too great or too small for them to flow through a given passage. Such passages exist in the membrane of the eye, thus allowing the eye to assimilate only a certain kind of material effluence, that is, the kind whose magnitude permits entry in ocular passages.

Thus we arrive at the answer in the style of Gorgias. That answer has three components. It specifies a kind of thing and two conditions that must be satisfied for a thing of that kind to be color. Color is (1) a kind of material effluence that is (2) commensurate with sight and (3) perceptible. First, color is a kind of material effluence, a chromatic effluence, say. Since material effluences are kind differentiated by magnitude, chromatic effluences have a distinctive magnitude. Second, chromatic effluences are commensurate with sight insofar as their distinctive magnitude permits entry in the passages of the membrane of the the eye, the organ of sight. Notice, however, the assimilation of chromatic effluences by the organ of sight is not, by itself, the sensing of colors. Otherwise the final condition would be redundant. The assimilation of chromatic effluence is at best a material precondition for their sensing. The thought seems to be this: In order for the chromatic effluences to be the object of sense, they first must be assimilated by the organ of sensation. It is only by assimilating chromatic effluences that they are presented to sight and are thereby seen. Socrates claims that the answer in the style of Gorgias may be generalized to the other sensory objects such as sound and smell (\emph{Meno} 76\( ^{d} \)), a claim echoed by Theophrastus' account of Empedocles (\emph{De Sensibus} \textsc{vii}). If that is right, then Empedocles, at least as presented by Socrates, is in the grip of a general conception of sensory awareness for which ingestion provides the model. Compare---in eating an olive, the matter of the olive is taken in and presented to the organ of taste and thereby tasted. On the ingestion model, to be perceptible is to be palpable to sense. 

The underlying thought is that in order for something to be the object of sense, it must be presented to the sense organ. On the ingestion model, it is a general feature that taste shares with paradigmatic cases of touch that is operative---the object of sensation must be in contact with the sense organ for it to be sensed. I say \emph{paradigmatic} cases of touch since it is arguable, at least, that one can feel something that one is not in direct contact with. Thus one might feel the wooden frame of a Victorian rocking horse through the padding. Consider Aristotle's embarrassment at the thought that one cannot touch what is wet (\emph{De Anima} \textsc{ii}.11 423\( ^{a} \)21--423\( ^{b} \)16). Theophrastus' commentary supports the suggestion that an object must be in contact with the sense organ for it to be sensed. Following Aristotle (\emph{De Anima} \textsc{i}.2 404\( ^{b} \)12--15; cf \emph{Metaphysics} \textsc{iii}.4 1000\( ^{b} \)5--8), Theophrastus counts Empedocles as a likeness theorist---as explaining perception in terms of the similarity of the elements that compose the object of sense and the sense organ. However, in \emph{De Sensibus} \textsc{xv}, Theophrastus concedes that Empedocles remains silent on the compositional similarity of the material effluence and the sense organ that assimilates it, emphasizing instead the role of \emph{contact} in perception \citep{Kamtekar:2009fk,Sedley:1992uq}.


To be perceptible is to be palpable to sense. If one began with that thought, a puzzle would naturally arise about vision, for vision seems to present the colors of distant objects. Color perception seems to involve the presentation of color qualities inhering in bounded particulars located at a distance from the perceiver. But how can one assimilate what remains inherent in a bounded particular remote from one? 

I conjecture that, whatever independent reasons Empedocles may have had for believing in material effluences, it is precisely this puzzlement that effluences are meant to address in his theory of vision. The basic idea is simple enough, at least in broad outline. Distant objects may be sensed by sensing the material effluences they emit. If the color of an object is the material effluence that it emits, then the color of a remote object can be assimilated and so be palpable to sight. In this way, we can see the color of a bounded particular remote from us  consistent with the constraints imposed by the ingestion model. One may wonder whether the theory of effluences is wholly adequate to this task, at least without supplementation. Thus a Berkelean worry naturally arises about the immediate objects of sensation, the assimilated effluences, screening off the external objects that emit them. Moreover, it is not just colored objects that appear at a distance, but the colors themselves seem confined to the remote bounded region in which they inhere. Fortunately, it is the puzzlement that arises from Empedocles' conception of sensory presentation, and not his resolution of it, that is our focus here. 

I believe that Empedocles' puzzlement about the sensory presentation of the remote objects of sight is a natural one. The puzzlement survives abandoning what surely is the immediate culprit in the ingestion model, the principle that to be perceptible is to be palpable to sense (\emph{De Sensu} 440\( ^{a} \)7--19). Indeed, in its most general form the puzzlement persists to this day. Thus Broad remarks that:
\begin{quote}
    It is a natural, if paradoxical, way of speaking to say that seeing seems to `bring us into \emph{contact} with \emph{remote} objects' and to reveal their shapes and colors. \citep[33]{Broad:1952kx}
\end{quote}
In its most general form, the puzzlement is one aspect of the problem discussed by contemporary philosophers under the rubric of presence in absence. The puzzlement consists in an inability to understand how to coherently combine the distal character of the objects of sight with a conception of sensory awareness as a mode of assimilation. It would be premature to dismiss that conception, even in its original Empedoclean form, as primitive physiology of vision. Indeed, Aristotle retains the conception of perception as a mode of assimilation even as he transforms it in rejecting Empedocles' theory of effluences. Aristotle retains that conception presumably because he felt that there was an insight that should be preserved in Empedocles' opinion. And Aristotle is not alone in thinking that there is an insight to be preserved in conceiving of sensory awareness as a mode of assimilation. Thus \citet{Broad:1952kx} speaks of the presentation of the objects of sensory awareness as a mode of prehension.  ``Prehension'' belongs to a primordial family of broadly tactile metaphors for sensory awareness that includes ``grasping'', and ``apprehending''. What unites these metaphors is that they are all a mode of assimilation, and ``ingestion'' is a natural variant \citep[see][7]{Johnston:2006uq,Price:1932fk}. For what it is worth, assimilation as a metaphor for perception is inscribed in the history of the English language; the word ``perception'' derives from the Latin \emph{perceptio} meaning to take in, or assimilate---evidence, at least, for the persistence of an inclination. It is natural, then, to think of seeing as taking in the external scene before one. But then the question arises: How can one take in what remains external? And if one can, what could taking in mean, here, such that one could? Empedoclean puzzlement, in its most general form, consists in the persistence of this latter question.

It is worth wondering about the prevalence of tactile metaphors for visual awareness. There is an Aristotelian explanation, I think. The explanation is Aristotelian, not in the sense that Aristotle gives the explanation or even entertains it; rather, it is Aristotelian in that it draws on resources available in Aristotle's thought. According to Aristotle, taste and touch are primitive forms of sensation common to all animals (though animals can and do differ in their possession of other sensory capacities). Suppose then, at least in human beings, the primitive character of touch is manifest in our emotional responses to things. Often when we see something we are drawn to touch it, even though there is no doubt about its presence or solidity. It is as if a thing's presence is most keenly felt when grasped. Thus we must endeavor to teach children to keep their hands to themselves, and even in maturity, polite notices are required to remind adults to not touch the display cabinet. If the presence of things is most keenly felt when grasped, if in the resistance to touch their presence is manifest in a primitively compelling manner, then it would be no surprise that we reach for tactile metaphors in characterizing sensory presentation, even as it figures in nontactile modes of sensory awareness such as vision. If the Aristotelian explanation is correct, then the tactile metaphors are emphasizing the \emph{presentation} of the objects of sensory awareness. It is because objects grasped are presented to us in a primitively compelling manner, that grasping can serve as a paradigm of sensory presentation. 

If the Aristotelian explanation is correct, then Empedoclean puzzlement, in its original form,  is the result of overgeneralizing from a paradigm case. To be sure, vision involves the presentation of colors inhering in bounded particulars remote from the perceiver. But if the presentation of color in sensory consciousness is too closely modeled on a capacity to grasp or take in something material, then a puzzle arises that only the theory of effluences may resolve. If indeed it does. I have already mentioned two reservations. As we will see, Aristotle has his own criticisms to make of Empedocles' theory of color vision. If Aristotle's criticisms prove cogent, then Empedoclean puzzlement, in its original form, not only involves an overgeneralization from a paradigm case but a misconception of it as well. But even if we resist the temptation to which Empedocles apparently succumbed, there remains the question: What could the sensory presentation of remote qualities be, if not simply their being palpable to sense?

In \emph{De Anima} \textsc{ii}.12 424\( ^{a} \); \textsc{ii}.5 417\( ^{b} \) Aristotle defines perception as a mode of assimilation of the sensible form without the matter of an external particular. This is an instance of Aristotle's dialectical refinement of the \emph{endoxa} \citep[on Aristotle's dialectic in \emph{De Anima} see][]{Witt:1995kx}. While denying that sight involves the assimilation of material effluences, Aristotle retains Empedocles' conception of sensory awareness as a mode of assimilation, it is just that we assimilate form without matter. Indeed, this pattern of dialectical refinement continues in the very next line where Aristotle uses Plato's metaphor of  wax receiving an impression, not to characterize judgment as Plato does in the \emph{Theaetetus} 194\( ^{c} \)--195\( ^{a} \), but to characterize the assimilation of sensible form in perception. Given this pattern of dialectical refinement, we can be confident that Aristotle was engaging with Empedocles' thought in his definition of perception. And while it remains controversial how to understand the assimilation of sensible form, I believe progress can be made by interpreting Aristotle's definition of perception as addressing Empedocles' puzzlement about how remote objects can be present in sensory consciousness. Recall Empedoclean puzzlement begins with the natural thought that in seeing one takes in the external scene. The question then arises: How can we take in what remains external? And if one can, what could taking in mean such that one could? The proposal is to interpret Aristotle's definition of perception as an answer to this latter question---a remote object can be present in sensory consciousness by assimilating its sensible form while leaving its matter in place. Understanding how Aristotle's definition of perception so much as could be a resolution of Empedoclean puzzlement imposes a substantive constraint on interpreting that definition. For so interpreted, it is making an important claim about the metaphysics of sensory presentation.

Aristotle's definition of perception is a dialectical refinement of the \emph{endoxa} insofar as it seeks to preserve an Empedoclean insight while resolving a puzzle about how remote objects can be present to sensory consciousness. Empedocles' puzzlement about the nature of sensory presentation persists to this day, though now in the guise of discussions of presence in absence. Perhaps there are insights of Aristotle's own that ought to be preserved when confronting Empedoclean puzzlement as it arises in its modern guise. If there are, then a conception of sensory presentation that preserved Aristotle's insight into the proper resolution of Empedoclean puzzlement would itself be a dialectical refinement of the respected opinion of Aristotle. What these insights might be and whether any conception of sensory presentation answers this description remains to be determined.

% section empedocles (end)

\section{Perception at a distance} % (fold)
\label{sec:perception_at_a_distance}

\begin{quote}
	Again, actual sensation corresponds to the exercise of knowledge; with this difference, that the objects of sight and hearing (and similarly for those of the other senses), which produce the actuality of sensation are external. This is because actual sensation is of particulars, whereas knowledge is of universals; these in a sense exist in the soul itself. So it lies in man's power to use his mind whenever he chooses, but it is not in his power to experience sensation; for the presence of the sensible object is essential. (\emph{De Anima} \textsc{ii}.5 417\( ^{b} \)?--?)
\end{quote}
This passage is part of an extended comparison of perception and knowledge. It makes two related points. Not only does perception and knowledge differ in object, but they are the exercise of different kinds of capacities as well. Moreover, this latter difference is partly explained in terms of the former. The overall lesson will be: Perceptual capacities would not be the kind of capacities that they are---they would not be a mode of sensitivity---unless perception takes as its object an external particular.

Let us begin with the difference in object. Whereas as the objects of perception are particulars, the objects of knowledge are universal. So whereas one may see the sun as it is at the moment of perception, burning white say, what one sees is a particular. But particulars, according to Aristotle, are not known. The objects of knowledge are universal in a way that precludes their being particulars. It is not just knowledge whose objects are universals, this is a general feature of our cognitive capacities. When one thinks that the sun is burning white, on thinks that thought not with the whiteness with which the sun actually burns but with a whiteness that the sun may share with the son of Diares, at least when viewed from a distance. 

The claim that the objects of perception are particulars and their qualities is tolerably clear. The claim that the objects of knowledge and thought are universals is less so. He does not mean something wholly present wherever instantiated like redness; though one can know things about redness, redness is not an object of knowledge. I suspect that ``particular'' wears the trousers in the distinction here. If so, then by ``universal'' Aristotle merely means non-particular. If that is right, then he is committed to no substantive positive characterization of universals. Moreover, perhaps that is a good thing. Any substantive positive characterization would be controversial. Suppose, for example, that by ``universal'', Aristotle meant a generality that holds universally. Universal validity is a hard standard. Moreover, it is plausibly more than is required to distinguish the objects of knowledge from particulars. Perhaps what's known involves a kind of generality that precludes it from being a particular. However, having a generality that precludes being a particular may fall short of strict universality.


Aristotle's claim that perception and knowledge can be distinguished in this way by the nature of their objects is echoed by Prichard:
\begin{quote}
	There seems to be no way of distinguishing perception and conception as the apprehension of different realities except as the apprehension of the individual and the universal respectively. Distinguished in this way, the faculty of perception is that in virtue of which we apprehend the individual, and the faculty of conception is that power of reflection in virtue of which a universal is made the explicit object of thought. \citep[]{Prichard:1909yg}
\end{quote}


Not only does perception and knowledge differ in object, they are the exercise of distinct kinds of capacities. Perception may be the exercise of the perceiver's sensory capacities, but sensory capacities are capacities of a distinctive kind. In Nietzsche's \citeyearpar{Nietzsche1887On-the-Genealog} terminology, they are \emph{reactive} capacities. Sensory capacities only act by reacting to the presence of the sensible particular. Reactive capacities are only exercised by reacting to the presence of something. Aristotle made this point earlier by means of an analogy with combustion:
\begin{quote}
	The question arises as to why we have no sensation of the senses themselves; that is, why they give no sensation apart from external objects, although they contain fire and earth and the other elements which (either in themselves, or by their attributes) excite sensation. It is clear from this that the faculty of sensation has no actual but only potential existence. So it is like the case of fuel, which does not burn by itself without something to set fire to it; for otherwise it would burn itself, and would not need any fire actually at work. (\emph{De Anima} \textsc{ii}.5 417\( ^{a} \)3--10)
\end{quote}
The presence of the sensible particular ignites sensory consciousness. Perception is essentially a reactive capacity, otherwise it would not be a mode of sensitivity to external particulars and their qualities.

Perception and knowledge are the exercise of different kinds of capacities. Our epistemic capacities, and cognitive capacities more generally, are not reactive capacities like our sensory capacities. Their exercise does not require the presence of any particular. One can think of the sun burning white even when the sun is absent and night has fallen. Our epistemic and cognitive capacities do not act by reacting. They are \emph{active}, not reactive. Kantians would describe the difference between perception and knowledge as a difference between the exercise of receptivity and spontaneity.

In its original form, Empedoclean puzzlement about the sensory presentation of remote objects is generated by a general conception of sensory awareness---the ingestion model. Specifically, given the ingestion model, a question arises about how to coherently combine the distal character of the objects of sight with a key feature of that model, the principle that to be perceivable is to be palpable to sense. A cogent argument against that principle would undermine whatever puzzlement that it generates. Aristotle believes that a simple empirical observation constitutes such an argument:
\begin{quote}
	If one puts that which has color right up to the eye, it will not be visible. (\emph{De Anima} \textsc{ii}.7 419\( ^{a} \)13--14)
\end{quote}
If a colored particular's being in contact with the eye blinds the perceiver to its color, then the colored particular must be at a distance from the perceiver if its color is to be seen. And if the colored particular is remote from the perceiver, an intervening medium is necessary in order for the the organ of sight to be acted upon, as it must be if it is to be a mode of sensitivity:
\begin{quote}
	Color moves the transparent medium, such as the air, and this, being continuous, acts upon the sense organ. Democritus is mistaken in thinking that if the intervening space were empty, even an ant in the sky would be clearly visible; for this is impossible. For vision occurs when the sensitive faculty is acted upon; and it cannot be acted upon by the actual color which is seen, there only remains the medium to act on it, so that some medium must exist. In fact, if the intervening space were void, not merely would accurate vision be impossible, but nothing would be seen at all. (\emph{De Anima} \textsc{ii}.7 418\( ^{b} \)13--22)
\end{quote}
While Empedocles is not mentioned in this passage, Democritus instead being singled out for criticism, when this issue is raised again in \emph{De Sensu}, the connection with Empedocles is made explicit:
\begin{quote}
	But to say, as the old philosophers did, that colors are effluences from objects and visible on this account, is unreasonable; for in any case they would have to explain sensation by contact, so that it would be better to say at once that sensation is caused because the sensible object sets in motion the medium of sensation, that is by contact not effluence. (\emph{De Sensu} \textsc{iii} 440\( ^{a} \)16--21)
\end{quote}

We can distinguish a specific claim about color and a more general claim about the objects of sense, one true of sound and smell as well, say:
\begin{enumerate}
	\item A colored particular is imperceptible if it is in contact with the organ of sight.
	\item A sensible object is imperceptible if it is in contact with the relevant sense organ.
\end{enumerate}

Let us begin with the specific claim about color. Here the thought is that in order to have a colored particular in view the perceiver must have a view on that colored particular. A colored particular's contact with the eye, the organ of sight, would preclude a point of view on that particular and its color. It is a necessary condition for a perceiver to have a point of view on a particular and its color that the particular be at a distance from the perceiver. To have a point of view on something is for that thing to be remote from one. 

The specific claim about color is echoed in Aristotle's criticism of the likeness theory. According to the likeness theory, perception is to be explained in terms of the similarity of the elements with which the sense organ and the object of sense are composed. The likeness theory is subject to a range of criticisms especially in the first book of \emph{De Anima}. However, in \emph{De Sensu}, Aristotle writes:
\begin{quote}
	For it is not true that the one sees, and the other is seen, just because the two are in a certain relation, \emph{e.g.}, that of equality; for in that case there would be no need for each of them to be in some particular place; for when things are equal it makes no difference whether they are near to or far from something. (\emph{De Sensu} \textsc{iii} 446\( ^{b} \)?--?)
\end{quote} 
Aristotle's complaint here is that equality, understood as complete compositional similarity of the sense organ and the object of sense, does not afford the perceiver with a point of view. The perceiver's point of view on a particular depends on the particular being at some distance from the perceiver. Moreover, that point of view varies as the object of sense is near or far. However, compositional similarity does not determine that the object of sense is any particular distance from the perceiver and hence fails to determine a point of view on that particular.

At least with respect to color vision, then, Aristotle's rejection of the Empedoclean principle, to be perceivable is to be palpable to sense, is unequivocal. Far from even being a necessary condition on sight, contact with a colored particular blinds us to that particular and its color. Consistent with that denial, the Empedoclean principle may nevertheless be true of other objects of sense, such as taste and touch. A more thoroughgoing rejection of the principle, then, would regard the specific claim about color as an instance of the more general claim about the objects of sense. An object being in contact with the relevant sense organ, far from being a necessary condition for sensing that object, precludes it from being the object of sensation. The claim here is general, applicable to all objects of sense---contact precludes sensation, to be palpable is to be imperceptible.

While Aristotle at least makes the specific claim about color, his complete case against the Empedoclean principle, to be perceivable is to be palpable to sense, may involve the more general claim. The distinction between the specific and more general claim is relevant not only to the depth of Aristotle's case against the ingestion model---if the general claim is true, then not only is the ingestion model false of color vision, it is false of every mode of sensory awareness---but the distinction is relevant as well as to the relative plausibility of these claims. Even if the more general claim should prove to be false, the specific claim about color may yet be true. It could turn out that vision is distinctive in being a sensory mode of presentation of the qualities of remote objects. \citet[]{Broad:1952kx} claims that a comparative phenomenology of our sensory capacities supports this view (even if he thinks that our phenomenology is misleading in this regard, and that the distinctive phenomenological character of vision is ultimately undermined by the common causal mechanisms underlying all of our sensory capacities).

As Aristotle's discussion of the special senses continues, however, it becomes clear that he endorses the more general claim that contact precludes perception, that to be palpable is to be imperceptible:
\begin{quote}
	The same theory applies also to sound and smell; no sound or smell provokes sensation because it touches the sense organ, but movement is produced in the medium by smell and sound, and in the appropriate sense organ by the medium; but when one puts the sounding or smelling object in contact with the sense organ, no sensation is produced. The same thing is true of touch and taste, although it is not apparent \ldots\ (\emph{De Anima} \textsc{ii}.7 419\( ^{a} \)26--34)
\end{quote}
And later, in a discussion of why humans can only smell when they inhale, the general denial of the Empedoclean principle is invoked as a constraint on an adequate explanation:
\begin{quote}
	That what is placed on the sense organ should be imperceptible is common to all senses. (\emph{De Anima} \textsc{ii}.9 421\( ^{b} \)16--18)
\end{quote}


% section perception_at_a_distance (end)

\section{Transparency} % (fold)
\label{sec:transparency}

Let us turn now to the transparent. In so doing we are jumping into the middle of things---both in the order of Aristotle's exposition, but also in that transparency is a common nature or power of the external medium that separates the perceiver and the remote object of vision.

There are philosophical reasons for considering the nature of the transparent.

First, Aristotle defines color in terms of the transparent. Specifically, in \emph{De Anima} \textsc{ii}.7 418\( ^{a-b} \) Aristotle defines color as the power to move what is actually transparent. Our understanding of color is incomplete if we do not understand the state of the external medium, such as air or water, which is a precondition for the activity of color. The effect of this incomplete understanding ramifies given Aristotle's avowed strategy of explaining perceptual capacities in terms of perceptual activities that are their exercise and to explain perceptual activities in terms of the objects of those activities (\emph{De Anima} \textsc{ii}.4 415\( ^{a} \)14--22).

Second, given that color is the power to move what is actually transparent, there is an alteration that the external medium undergoes when in a transparent state as a result of the activity of color. Suppose the alteration that the external medium undergoes when in a transparent state is imparted to the internal medium---the transparent medium that constitutes the interior of the sense organ, in the case of the eye, the \emph{vitreous humour}. Then we would have in place an important piece of the puzzle involved in interpreting the assimilation of sensible forms. Moreover, progress can be made here while forestalling the controversies, which must eventually be faced, surrounding the metaphysics of \emph{De Anima} \textsc{ii}.5.

Third, external particulars remote from the perceiver are arrayed in an external medium through which, when transparent, they appear. Empedoclean puzzlement highlights the way in which this is a remarkable fact. Attending to the details of Aristotle's discussion of transparency we may gain insight into his reaction to Empedoclean puzzlement about the sensory presentation of remote objects.

It is worth enumerating the reasons for considering Aristotle's discussion of transparency given its reception. Some commentators suggest that Aristotle's discussion of transparency is of antiquarian interest only. Others have expressed incredulity at the way Aristotle's account conflicts with the manifest facts of experience. While leaving open the possibility of errors and omissions, we should try to understand Aristotle's account by attending to the phenomena as we understand it to be and by asking how that phenomena might have appeared as Aristotle describes it. As a methodology, this is little more than a minimal exercise of charity. However, it is worth stating explicitly if only because it is routinely flouted. Adhering to this precept yields and understanding of transparency that is sensible, phenomenologically adequate, and a reasonable approximation of the truth. It is also a philosophically revealing exercise since to interpret Aristotle in this way is to use Aristotle's text as a means of attending to the phenomena under investigation.

\subsection{Transparency in \emph{De Anima}} % (fold)
\label{sub:transparency_in_de_anima}

In \emph{De Anima}, Aristotle defines the transparent as follows:
\begin{quote}
	By transparent I mean that which is visible, though not visible in itself, but owing its visibility to the color of another thing. (\emph{De Anima} \textsc{ii}.7 418\( ^{b} \)4--6)
\end{quote}

First, one might be surprised that Aristotle defines transparency in terms the manner of its visibility, or the way in which it appears in perceptual experience. Isn't it more natural to think of transparency in terms of that through which remote objects appear---that is to say, not in terms of the manner of its visibility, but in terms of its being a condition on the visibility of other things? We will return to this issue.

The second thing to remark about Aristotle's definition is the nature of the intended contrast between something being visible in itself and something owing it's visibility to another thing. Aristotle does not have in mind here what he elsewhere calls incidental perception (\emph{De Anima} \textsc{ii}.6 418\( ^{a} \) 20--23). Seeing the transparent medium by seeing colors arrayed in it is not like seeing the son of Diares (a distant ancestor of Ortcutt, \citealt{Quine:1956qp}) by seeing a white speck. Being the son of Diares is not sensible the way being transparent is (\emph{De Anima} \textsc{ii}.6 418\( ^{a} \) 23--26).

Third, once we recover from our initial surprise at Aristotle defining transparency in terms of the manner of its visibility, and we consider the plausibility of that claim, quite apart from its status as a definition, we discover a potential insight. I, at least, have the corresponding intuition about illumination. I believe we see the character of the illumination by seeing the way objects are illuminated.The former is a state of the external medium whereas the latter is a property of a particular arrayed in that medium (though, of course, a property that the particular could only have given the state of the medium). So when viewing a brightly lit pantry, one sees the the brightness of the pantry by seeing the brightly lit objects arranged in it.  Hilbert makes a similar phenomenological observation:
\begin{quote}
	Do we see how an object is illuminated or do we see the illumination itself? On phenomenological grounds the first option seems better to me. What we see as changing with the illumination is an aspect of the object itself, not the light source or the space surrounding the object. \citep[150--151]{Hilbert:2007qy}
\end{quote}
At the very least, then, the claim enshrined in Aristotle's definition of transparency receives indirect support from the plausibility of the corresponding claim about illumination.

Transparency is a nature or power common to different substances. It is shared by liquids, like air and water, and certain solids and is incidental to the nature of each (\emph{De Anima} \textsc{ii}.7 418\( ^{b} \)7--9). A medium is actually transparent not due to its nature but due rather to the contingent presence of the fiery substance (\emph{De Anima} \textsc{ii}.7 418\( ^{b} \)11--13). The continual presence of the fiery substance is required for the transparency of the medium to persist. Suppose I light a fuse with a cigar. Prudence councils that I should remove myself from the scene. Should I be enjoying the cigar I might take it with me. But while the fuse would remain lit even when the cigar is removed, the air would not remain transparent when the fiery substance is removed. When the fiery substance is removed, darkness supervenes (\emph{De Anima} \textsc{ii}.7 418\( ^{b} \)18--21; \emph{De Sensu} \textsc{iii} 439\( ^{a} \)18--21). Not only does the persistence of transparency depend upon the continual presence of the fiery substance, but, arguably at least, it depends as well on its continual activity \citep[\emph{pace}][424]{Burnyeat:1995fk}. This may be taken to be implied by his claim that light is the activity of the transparent \emph{qua} transparent (\emph{De Anima} \textsc{ii}.7 418\( ^{b} \) 9--10). Since transparency just is the presence of the fiery substance, the activity of the transparent \emph{qua} transparent just is the activity of the present fiery substance. That some states require continual activity to sustain them should be no surprise. Consider Ryle's \citeyearpar[149]{Ryle:1949qr} example of keeping the enemy at bay, or the connection between heat and molecular motion.

Light is a state that the medium is in when it is actually transparent. Aristotle denies that light is fire, or a body, or an effluence (\emph{De Anima} \textsc{ii}.7 418\( ^{b} \) 13--18). He denies as well that light moves, otherwise its motion would be visible as it travels from East to West (\emph{De Anima} \textsc{ii}.7 418\( ^{b} \) 21-27). These claims are puzzling if by light Aristotle means, at least approximately, what we mean by light. But why assume that? 

Begin by focusing on Aristotle's claim that light is a \emph{state} (\emph{hexis}, \emph{De Anima} \textsc{iii}.5 430\( ^{a} \)15) that a medium is in when it is actually transparent. Light could not be a body since the medium is a body and two bodies cannot occupy the same space. As \citet{Burnyeat:1995fk} has emphasized, state is really the wrong category for light as we presently understand it to be. But now, in line with our avowed methodology, let us ask whether there could be a state that we can recognize on our present understanding that could reasonably be what Aristotle had in mind when he speaks of light? With the question so framed the resolution of our difficulties should be obvious. What state is a medium in when it is actually transparent, and where the persistence of this state depends on the continual presence and activity of a fiery substance? When it is illuminated, of course. By light, Aristotle means a state of illumination (see \citealt[122]{Thorp:1982fk}, for a similar interpretation). And that a medium when it is actually transparent is in a state of illumination sustained by the presence and activity of a fiery substance strikes me as a not unreasonable approximation of the truth. Moreover, it coheres well with the phenomenology of illumination. Consider what must have been the familiar experience of lighting an oil lamp to illuminate a room.

There remain anachronistic elements in Aristotle's account. Consider his empirical argument that light does not move (\emph{De Anima} \textsc{ii}.7 418\( ^{b} \) 21-27). If light is a state of a medium sustained by the presence and activity of a fiery substance, then this claim has the consequence that the fiery substance cannot propagate through a potentially transparent medium. For as the fiery substance propagated through the medium, the illuminated region of that medium would change over time. It would be consistent with Aristotle's claim that light does not move that the fiery substance instantaneously propagates through a potentially transparent medium, but that would be a Pickwickian sense of propagation. Propagation is a \emph{process} that unfolds through time and so could not, strictly speaking, be instantaneous.

The reasoning assumes that the illuminated region of the medium changing over time consists in a change in the position of the state of illumination. Without some such assumption, there is no valid inference from the propagation of the fiery substance through a medium to the motion of light as Aristotle understands it to be, as a state of illumination.

% subsection transparency_in_de_anima (end)

\subsection{Transparency in \emph{De Sensu}} % (fold)
\label{sub:transparency_in_de_sensu}

In \emph{De Anima}, Aristotle defines the transparent as that which is visible, though not visible in itself, but owing its visibility to the color of another thing (\emph{De Anima} \textsc{ii}.7 418\( ^{b} \) 4--6). I have remarked that it might seem more natural to characterize transparency, not in terms of the manner of its visibility, but in terms of its being that through which remote objects appear---as a condition on the visibility of other things. However, this latter conception is not entirely absent in Aristotle. It is at least implicit in the corresponding discussion of color and transparency in \emph{De Sensu}.

In \emph{De Sensu} Aristotle sets out to explain what each of the sense objects ``must be to produce the sensation in full actuality'' (\emph{De Sensu} \textsc{iii} 439\( ^{a} \)10). This is a further inquiry, not directly addressed by \emph{De Anima}. Unsurprisingly, then, there are novel elements to the \emph{De Sensu} discussion. Thus, novel claims that emerge include, for example, that color resides in the proportion of transparent that exists in all bodies, and an account of the generation of the hues in terms of the ratio of black and white in a mixture. Given these novel elements, the question arises whether \emph{De Sensu} represents an extension of the doctrines of \emph{De Anima}, or a change of mind. While there is some evidence that Aristotle has not completely harmonized new ideas with old, I believe that Aristotle meant to be offering an extension of the \emph{De Anima} account, and not a substantive revision of it. Or at any rate, this will be my working hypothesis (see \citealt{Kahn:1966zr} for discussion; see also \citealt[291]{Caston:2005cr} \citealt[37]{Nussbaum:1995ly}).

One novel element is the characterization of color as ``the limit of the transparent in a determinately bounded body'' (\emph{De Sensu} \textsc{iii} 439\( ^{b} \)11). This prompted the Renaissance commentator Jacopo \citet{Zabarella:1605kx} to complain that Aristotle has defined color twice over \citep{Broackes:1999uq}. However, there is no evidence in the text that Aristotle regarded this claim as a definition. Rather, it appears as the conclusion of an argument \citep[65]{Broackes:1999uq}. In that argument, Aristotle explains that color inheres not only in unbounded things, such as air and water, but in bounded things as well. What's the distinction between the bounded and the unbounded? The examples of the transparent are restricted in \emph{De Sensu} to air and water. On this basis, it might be thought, naturally enough, that that the distinction is between transparent liquids, like air and water, and opaque solid objects \citep[59]{Broackes:1999uq}. To describe liquids as unbounded is to highlight their lack of fixed boundaries. However, I doubt that is what Aristotle had in mind. In \emph{De Anima}, Aristotle claims that not only are liquids such as air and water transparent, but so are certain solid objects. He does not himself give examples of transparent solids. But glass, ice, crystals, tortoise shells, and certain animal horns would do, and we can be confident that Aristotle had first hand experience with at least some of these. The problem, then, is that any such example would possess fixed boundaries and yet would remain transparent, but the transparent is meant to be undounded. 

What could the unbounded be if it is not simply the lack of fixed boundaries? I believe that good sense can be made of Aristotle's distinction if we understand it in perceptual terms. Nontransparent bodies, such as opaque solids, are perceptually impenetrable. Unlike transparent bodies you cannot see in them or through them. Their surface is the site of visual resistance; perceptual impenetrability determines a visual boundary through which nothing further can appear. Transparent bodies, in contrast, are perceptually penetrable. One can see in them and through them. The particulars arrayed in a transparent medium appear through that medium. The transparent is unbounded since it offers insufficient visual resistance to determine a perceptually impenetrable boundary. And this is true of transparent solids such as crystals and tortoise shells as well as transparent liquids such as air and water.

The transparent is unbounded since it offers insufficient visual resistance to determine a perceptually impenetrable boundary. Which is not, of course, to say that the transparent can offer no visual resistance. In \emph{De Sensu}, Aristotle emphasizes that transparency comes in \emph{degrees}. When Aristotle speaks of color as the limit of the transparent in bounded bodies, he has in mind surface color. But he also speaks of the color of transparent media:
\begin{quote}
    Air and water obviously have color; for their brightness is of the nature of color. But in their case because the color resides in something unbounded, air and sea do not show the same color near at hand and to those who approach them as they have at a distance. (\emph{De Sensu} \textsc{iii} 439\( ^{b} \)1--3)
\end{quote}
Air and water, when transparent, are bright. And brightness, Aristotle claims, is of the nature of color. The attribution of brightness, however, requires attributing no particular hue to the medium. If the medium is perfectly transparent, then the only visible hues will be the colors of bounded particulars arrayed in that medium. But the next line contains the suggestion that imperfectly transparent media, while remaining perceptually penetrable to some degree, may themselves have a particular hue---in modern parlance, not a surface color but a \emph{volume} color. From a cliff overhanging the sea, the sea may appear a clear blue even as one sees rocks lying below its surface. But, if enticed by the sea, one were to descend to the beach and examine a handful of sea water, it would not be blue at all, but transparent. Similarly, looking up at the sky on a clear autumn afternoon, one sees an expanse of blue. But if one were to travel to that region of the sky, by helicopter, say, nothing blue would be found. The implicit thought is that the visual resistance of an imperfectly transparent medium increases with an increase in volume. The further one sees into a transparent medium, the more resistance that medium offers to sight. And volume color is the effect of this resistance.

The color of an imperfectly transparent medium does not occlude the bound\-ed particulars arrayed in it. But the color of the transparent medium may affect their color appearance. Thus the sun, which in itself appears white, takes on a crimson hue when seen through a fog or cloud of smoke (\emph{De Sensu} 3 440\( ^{a} \)10--11). This might be what Aristotle has in mind when he claims that bounded particulars have a fixed color unless affected by atmospheric conditions (\emph{De Sensu} 3 439\( ^{b} \)5--7). The color of a bounded particular will affect the medium differently depending on its degree of perceptual penetrability and resulting volume color. Notice, considered in and of itself, this claim implies at most that the color of the sun appears differently when obscured by a fog or cloud of smoke; there need be not commitment to the sun changing color from white to red when so obscured, nor its appearing to so change. Aristotle's position allows for the possibility of a variation in color appearance without a variation in presented color. Notice the thought that the state of a medium can alter the appearance of a sensible object without a variation in the object of sense is what animates Austin's \citeyearpar{Austin:1962lr} example of a straight stick looking bent in water (on Austin see Kalderon and Travis \citeyear{Kalderon:2010fk} and Martin \citeyear{Martin:2000nx}; on Austin and the argument from conflicting appearances see Burnyeat \citeyear{Burnyeat:1979mv}).

This is potential evidence about Aristotle's attitude towards the argument from conflicting appearances. While discussion of conflicting appearances is largely absent from \emph{De Anima} and \emph{De Sensu}, it is not entirely absent, and I believe we have an important point of contact here. Looking up from a battlefield one sees the sun burning white. As smoke from the battle obscures the sun, it takes on a crimson hue. Red and white are contraries. Nothing can be red and white all over at the same time. Supposing, as is plausible, that the smoke from the battle did not alter the sun's color so that the color of the sun remains constant through the variation in its appearance, it might seem as if at least one of these appearances were illusory. However, if there can be a variation in color appearance without a variation in presented color, then the white and red appearances do not conflict. The color of the sun does not appear to change from white to red; red is simply the way radiant white things appear when viewed through smoke filled media (just as bent is the way that straight things look when viewed through refracting media, \citealt{Austin:1962lr}).

Against the present interpretation it might be objected that Aristotle makes a claim about the color of the transparent that conflicts with it. Thus Aristotle claims that the transparent lacks color and so is receptive to color (\emph{De Anima} \textsc{ii}.7 418\( ^{b} \)26--29). The force of this objection is mitigated somewhat by the recognition that Aristotle seems to make inconsistent claims about the color of the transparent. Thus he claims that:
\begin{enumerate}
	\item Light, or brightness, is the color of the transparent. (\emph{De Anima} \textsc{ii}.7 418\( ^{b} \)11-12; \emph{De Sensu} \textsc{iii} 439\( ^{b} \)1--2)
	\item The transparent is seen to have different colors when near and far. (\emph{De Sensu} \textsc{iii} 439\( ^{b} \)2--3)
	\item The transparent lacks color and so is receptive to color. (\emph{De Anima} \textsc{ii}.7 418\( ^{b} \)26--29)
\end{enumerate}
How might 1.--3. be interpreted so as to be consistent? We have already observed that the attribution of brightness requires attributing no particular hue to the transparent medium. Moreover, since the medium is transparent, the color of the remote particular appears through that medium. This may even be so in an imperfectly transparent medium, one such that owing to the resistance it offers to vision itself appears a certain volume color. The color of a remote particular may appear differently when viewed through perfectly and imperfectly transparent media, but the volume color, if any, of the transparent medium does not occlude the surface color of the remote bounded particular. But so long as the surface color of the remote bounded particular is not occluded by varying the color of the medium as it volume varies, the transparent medium remains receptive of that color. If, however, the medium were to become perceptually impenetrable and so take on a surface color, the color of the remote bounded particular would be occluded and the medium would no longer be receptive to color. The denial in 3. is the denial of surface color to transparent media, but that is consistent with imperfectly transparent media, such as the sea and the sky, having volume color. Properly interpreted, 1.--3. are consistent.

There is thus a progression of qualitative states from the perfectly transparent to the colored and opaque. The qualitative states in the progression are ordered by their decreasing degree of perceptual penetrability culminating in the perceptual impenetrable. It is thus a progression to a limit. We can envision the progression from perfect transparency in the following manner. Consider a tank of clear water into which is poured a blue dye. Suppose the absorption rate of the dye is too quick to be visible. So we do not see clouds of blue dye propagating through the clear liquid; rather, we see the volume taking on the blue and become increasingly opaque. At the end of this progression, the tank is surface blue---no thing can appear in it or through it. Color, that is surface color, is in this sense the limit of the transparent---it is the terminal qualitative state of a progression of qualitative states ordered by decreasing degree of perceptual penetrability.

One may be forgiven for thinking that Aristotle has fallen into a category mistake in speaking of color as the limit of the transparent \citep[65]{Broackes:1999uq}. He seems, on the surface, to be making an identification, but color is a \emph{quality} in the way that a limit could not be. However, on the interpretation that I have been urging, Aristotle is not identifying color qualities with limits; rather, in the progression of qualitative states from the perceptually penetrable to the perceptually impenetrable, color (that is, surface color) is the terminal qualitative state. This is \emph{one} way of understanding Aquinas, in his commentary on \emph{De Sensu}, when he writes:
\begin{quote}
	Thus color is not in the category of quantity---like surface, which is the limit of a body---but in the category of quality. The transparent is also in the category of quality, because \emph{a limit and that of which it is the limit belong to one category}. [my emphasis] (\emph{Sententia De Sensu Et Sensato} \textsc{v}, commentary on \emph{De Sensu} \textsc{iii} 439\( ^{b} \)11 in \citealt{White:2005vn})
\end{quote}

In \emph{De Sensu}, Aristotle not only speaks of the limit of the transparent but also of the limit of a body:
\begin{quote}
	Color lies at the limit of the body, but this limit is not a real thing; we must suppose that the same nature which exhibits color outside, also exists within. (\emph{De Sensu} \textsc{iii} 439\( ^{a} \)32--439\( ^{b} \)35)
\end{quote}
The limit of a body is its external surface, a bulgy two-dimensional particular, in \citet[]{Sellars:1956xp} apt phrase. Color lies at the limit of the body, but it is not the limit of the body. [Pythagoras, Sellars, Sherwin-Willaims paint]

Does the consideration that tells against color being the limit of the body tell equally against color being the limit of the transparent? Not obviously. Opaque solids are perceptually impenetrable, and their perceptual impenetrability determines a visual boundary through which nothing further can appear. This visual boundary coincides with the limit of the body. This could only seem inconsistent with the claim that the same nature which exhibits color outside also exists within if one ignored Aristotle's reminder at the opening of \emph{De Sensu} \textsc{iii} 439\( ^{a} \)13--14 that ``each of these terms is used in two senses: as actual or potential.'' Aquinas insightfully heeds this reminder. In his commentary on \emph{De Sensu} he writes that ``bodies have surface in their interior in potentiality but not actuality'' (\emph{Sententia De Sensu Et Sensato} \textsc{v}, commentary on \emph{De Sensu} \textsc{iii} 439\( ^{b} \)11 in \citealt{White:2005vn}). When the perceptually impenetrable is actually resisting sight a visual boundary is determined at the limit of the opaque body. But that a portion of the interior of such a body offers no such visual resistance in being occluded from view is consistent with its being perceptually impenetrable, with its potentially determining such a visual boundary. 

Aristotle's discussion of transparency and the unbounded is evidence that, despite his defining transparency in terms of the manner of its visibility, he retains a conception of the transparent as that in which and through which remote objects may appear, as a condition on the visibility of other things. That conception, in the guise of perceptual penetrability, is central to Aristotle's understanding of the unbounded. Two observations are relevant. First, given our working hypothesis that \emph{De Sensu} is to be read as an extension of the \emph{De Anima} account and not a substantive revision of it, we can assume that this conception is meant to be at least consistent with the \emph{De Anima} definition.  Second, Empedoclean puzzlement about the sensory presentation of remote objects highlights the way in which perceptual penetrability of transparent media is a remarkable fact. It is a remarkable fact. Moreover, in \emph{not} defining transparency as that in which and through which remote objects may appear, Aristotle arguably acknowledges that it is. That the colors of remote objects are seen through transparent media is a fact to be explained. And if the nature of the transparent is to play a role in that explanation, the transparent must be defined in some way other than as being a condition on the visibility of remote objects. The explanation is given in \emph{De Anima}---in terms of the way in which color alters the transparent and the role that alteration plays in the exercise of our perceptual capacities. The first step in understanding that explanation is to understand the manner in which color acts upon the transparent medium. That is the task of the next section. The next step is to understand the role this alteration plays in the realization of our perceptual capacities. That is the task of the subsequent section.

% subsection transparency_in_de_sensu (end)

% section transparency (end)

\section{Color} % (fold)
\label{sec:color}

Aristotle defines color as the power to move what is actually transparent (\emph{De Anima} \textsc{ii}.7 418\( ^{a} \)31--418\( ^{b} \)33; \textsc{ii}.7 419\( ^{a} \)10--12).

First, as \citet[367]{Hicks:1907uq} observes, by motion Aristotle does not mean locomotion or change in position. Rather, in \emph{De Anima}, \emph{kinēsis} is Aristotle's general term for change of any kind. Thus \emph{kinētikon} in Aristotle's definition means productive of change rather than productive of spatial movement, more narrowly.

Second, and frustratingly, Aristotle does not directly specify the nature of the change color induces in the transparent medium it acts upon. Indeed, in \emph{De Anima} \textsc{ii}.7 the only effect of color discussed is the effect in terms of which the transparent is defined---the transparent is not visible in itself, but \emph{owning its visibility to the color of another thing}. Is this change, the rendering visible of the transparent, sufficient to understand Aristotle's definition? If it is, this would explain Aristotle's apparent silence about the nature of the change induced in the transparent by color---he merely says nothing \emph{further}, having \emph{already} specified the nature of the change in his definition of the transparent. Given the paucity of textual evidence, a conservative strategy would begin with this hypothesis and only abandon it in favor of speculation should it prove to be an insufficient basis for understanding Aristotle's definition.

Third, a doubt may be registered about the occurrence of transparency in Aristotle's definition. Color is a proper object of sight, and, as such, partly defines the nature of sight. It might reasonably be thought that color could only play a role in defining sight if it had a nature independent of sight. But defining color in terms of the power to move what is actually transparent potentially threatens this order of explanation given the definitional connection between transparency and visibility. It is on these grounds that \citet{Zabarella:1605kx} rejects Aristotle's definition \citep[see][for discussion]{Broackes:1999uq}.


% section color (end)

\section{The Eye} % (fold)
\label{sec:the_eye}

% section the_eye (end)

\section{Two Transitions to Actuality} % (fold)
\label{sec:two_kinds_of_potentiality}

% section two_kinds_of_potentiality (end)

\section{Form Without Matter} % (fold)
\label{sec:form_without_matter}

% section form_without_matter (end)

\nocite{}

% Bibligography
\bibliographystyle{plainnat} 
\bibliography{Philosophy} 

\end{document}
