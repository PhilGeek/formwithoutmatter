%!TEX TS-program = xelatex 
%!TEX TS-options = -synctex=1 -output-driver="xdvipdfmx -q -E"
%!TEX encoding = UTF-8 Unicode
%
%  formwithoutmatter
%
%  Created by Mark Eli Kalderon on 2011-03-05.
%  Copyright (c) year. All rights reserved.
%

\documentclass[12pt]{article} 

% Definitions
\newcommand\mykeywords{Aristotle, perception, color}
\newcommand\myauthor{Mark Eli Kalderon} 
\newcommand\mytitle{Form Without Matter}

% Packages
\usepackage{geometry} \geometry{a4paper} 
\usepackage{url}
\usepackage{txfonts}
\usepackage{color}
\definecolor{gray}{rgb}{0.459,0.438,0.471}
% \usepackage{setspace}
% \doublespace % Uncomment for doublespacing if necessary
% \usepackage{epigraph} % optional

% XeTeX
\usepackage[cm-default]{fontspec}
\usepackage{xltxtra,xunicode}
\defaultfontfeatures{Scale=MatchLowercase,Mapping=tex-text}
\setmainfont{Hoefler Text}
\setsansfont{Gill Sans}
\setmonofont{Inconsolata}

% Section Formatting
\usepackage[]{titlesec}
\titleformat{\section}[hang]{\fontsize{14}{14}\scshape}{\S{\thesection}}{.5em}{}{}
\titleformat{\subsection}[hang]{\fontsize{12}{12}\scshape}{\S{\thesubsection}}{.5em}{}{}
\titleformat{\subsubsection}[hang]{\fontsize{12}{12}\scshape}{\S{\thesubsubsection}}{.5em}{}{}

% Bibliography
\usepackage[round]{natbib} 

% Title Information
\title{\mytitle} % For thanks comment this line and uncomment the line below
%\title{\mytitle\thanks{}}% 
\author{\myauthor} 
% \date{} % Leave blank for no date, comment out for most recent date

% PDF Stuff
\usepackage[plainpages=false, pdfpagelabels, bookmarksnumbered, backref, pdftitle={\mytitle}, pagebackref, pdfauthor={\myauthor}, pdfkeywords={\mykeywords}, xetex, dvipdfmx, colorlinks=true, citecolor=gray, linkcolor=gray, urlcolor=gray]{hyperref} 



%%% BEGIN DOCUMENT
\begin{document}

% Title Page
\maketitle
% \begin{abstract} % optional
% \noindent
% \end{abstract} 
\vskip 2em \hrule height 0.4pt \vskip 2em
% \epigraph{text of epigraph}{\textsc{author of epigraph}} % optional; make sure to uncomment \usepackage{epigraph}

% Layout Settings
\setlength{\parindent}{1em}

% Main Content

\section{Empedocles} % (fold)
\label{sec:empedocles}
In the \emph{Meno} Socrates attributes to Empedocles a conception of perception as a mode of assimilation of material effluences:
\begin{quotation}
    \textsc{meno}: And how do you define color?
    
    \ldots
    
    \textsc{socrates}: Would you like an answer à la Gorgias, such as you most readily follow?
    
    \textsc{meno}: Of course I should.
    
    \textsc{socrates}: You and he believe in Empedocles' theory of effluences, do you not?
    
    \textsc{meno}: Wholeheartedly.
    
    \textsc{socrates}: And passages to which and through which the effluences make their way?
    
    \textsc{meno}: Yes.
    
    \textsc{socrates}: Some of the effluences fit into some of the passages whereas others are too coarse or too fine.
    
    \textsc{meno}: That is right.
    
    \textsc{socrates}: Now you recognize the term `sight'?
    
    \textsc{meno}: Yes.
    
    \textsc{socrates}: From these notions, then, `grasp what I would tell,' as Pindar says. Color is an effluence from shapes commensurate with sight and perceptible to it. \emph{Meno} 76a--d
\end{quotation}

Objects emit material effluences. Effluences are fine bodies that are kind differentiated in terms of magnitude. There are passages in which and through which material effluences may flow. Whether a material effluence may enter a passage depends on its magnitude. The magnitudes of some kinds of material effluences are too great or too small for them to flow through a given passage. Such passages exist in the membrane of the eye, thus allowing the eye to assimilate only a certain kind of material effluence, that is, the kind whose magnitude permits entry in occular passages.

Thus we arrive at the answer in the style of Gorgias. That answer has three components. It specifies a kind of thing and two conditions that must be satisfied for a thing of that kind to be color. Color is (1) a kind of material effluence that is (2) commensurate with sight and (3) perceptible. First, color is a kind of material effluence, a chromatic effluence let's say. Since material effluences are kind differentiated by magnitude, chromatic effluences have a distinctive magnitude. Second, chromatic effluences are commensurate with sight insofar as their magnitude admits entry in the passages of the membrane of the the eye, the organ of sight. Notice, however, the assimilation of chromatic effluences by the organ of sight is not, by itself, the sensing of colors. Otherwise the final condition would be redundant. The assimilation of chromatic effluence is at best a material precondition for their sensing. The thought seems to be this: In order for the chromatic effluences to be the object of sense, they first must be assimilated by the organ of sensation. It is only by assimilating chromatic effluences that they are presented to sight and are thereby seen. Socrates claims that the answer in the style of Gorgias may be generalized to the other sensory objects such as sound and smell (\emph{Meno} 76d). If that's right, then Empedocles, at least as presented by Socrates, is in the grip of a general conception of sensory awareness for which ingestion provides the model. Compare---in eating an olive, the matter of the olive is taken in and presented to the organ of taste and thereby tasted. On the ingestion model, to be perceptible is to be palpable to sense. Olfaction, on one conception of it at any rate, provides a complementary model---an odor is the object of smell only if the particulate matter that constitutes that odor is in contact with the nasal membrane. 

The underlying thought is that in order for something to be the object of sensation, it must be present to the sense organ. On the ingestion model it is a general feature that taste shares with paradigmatic cases of touch that is operative---the object of sensation must be in contact with the sense organ for it to be sensed. Theophrastus' commentary supports this suggestion. Following Aristotle's account of the \emph{endoxa} in \emph{De Anima} \textsc{i}, Theophrastus counts Empedocles as a likeness theorist---as attempting to explain perception in terms of the compositional similarity of the object of sense and the sense organ---indeed on the same textual basis as Aristotle:
\begin{quote}
    We see earth by earth, water by water\\ 
    Bright aether by aether, and obliterating fire by fire\\ 
    Love by love, and strife by baneful strife. (\emph{De Anima} \textsc{i}.2 404\( ^{b} \)12--15; cf. \emph{Metaphysics} \textsc{iii}.4 1000\( ^{b} \)5--8)
\end{quote}
However, in \emph{De Sensibus}, Theoprastus concedes that Empedocles remains silent of the compositional likeness of the material effluence and the sense organ that assimilates it, emphasizing instead the importance of contact in perception:
\begin{quote}
    He attributes recognition to two things, likeness and contact, on account of which he says ‘fit’. The result is that if the smaller should touch the larger, there would be perception. And indeed, on his view as a whole, likeness too is taken away, but commensurateness by itself is sufficient. On account of this he says that things do not perceive one another because they have incommensurate pores, but whether the effluences are like or unlike, he has not yet defined. (\emph{De Sensibus} 15)
\end{quote}

To be perceptible is to be palpable to sense. If one began with that thought, a puzzle would naturally arise about vision, for vision seems to present the colors of distant objects. Color perception seems to involve the presentation of color qualities inhering in bounded particulars located at a distance from the perceiver. But how can one assimilate what remains inherent in a bounded particular remote from one? It is precisely this puzzlement that the theory of effluences is meant to address. Distant objects may be sensed by sensing the material effluences they emit. If the color of an object is the material effluence that it emits, then the color of a remote object can be assimilated and so be palpable to sight. One may wonder whether the theory of effluences is wholly adequate to this task, at least without supplementation. Thus a Berkelean worry naturally arises about the immediate objects of sensation, the assimilated effluences, screening off the external objects that emit them. Fortunately, it is the puzzlement that arises from Empedocles' conception of sensory presentation, and not his resolution of it, that is our focus here. 

I believe that Empedocles' puzzlement about the nature of sensory presentation involved in vision is a natural one. The puzzlement persists to this day. Thus Broad remarks that:
\begin{quote}
    It is a natural, if paradoxical, way of speaking to say that seeing seems to `bring us into \emph{contact} with \emph{remote} objects' and to reveal their shapes and colors. \citep[33]{Broad:1952kx}
\end{quote}
Indeed it is one aspect of the problem discussed by contemporary philosophers under the rubric of presence in absence. The puzzlement consists in an inability to understand how to coherently combine the distal character of the objects of sight with a conception of sensory awareness as a mode of assimilation. It would be premature to dismiss that conception as primitive form of psychophysics. Indeed Aristotle retains the conception of perception as a mode of assimilation even as he transforms it in rejecting Empedocles' theory of effluences. Aristotle retains that conception presumably because he felt that there was an insight that should be preserved in Empedocles' opinion. And Aristotle is not alone in thinking that there is an insight to be preserved in conceiving of sensory awareness as a mode of assimilation. Thus \citet{Broad:1952kx} speaks of the presentation of the objects of sensory awareness as a mode of prehension.  ``Prehension'' belongs to a primordial family of broadly tactile metaphors for sensory awareness that includes ``grasping'', and ``apprehending''. What unites these metaphors is that they are all a mode of assimilation, and ``ingestion'' is a natural variant \citep[see][7]{Johnston:2006uq,Price:1932fk}. It is natural, then, to think of seeing as taking in the external scene before one.  But then the question naturally arises: How can one take in what remains external? And if one can, what could taking in mean such that one could?

In \emph{De Anima} Aristotle defines perception as a mode of assimilation of the sensible form without the matter of an external particular (\emph{De Anima} II, 12, 424a; II, 5, 417b). This is an instance of Aristotle's dialectical refinement of the \emph{endoxa}. While denying that sight involves the assimilation of material effluences, Aristotle retains Empedocles' conception of sensory awareness as a mode of assimilation, it is just that we assimilate form without matter. Indeed, this pattern of dialectical refinement continues in the very next sentence where Aristotle uses Plato's metaphor of  wax receiving the impression of a signet-ring, not to characterize judgment as Plato does in the \emph{Theaetetus}, but to characterize the assimilation of sensible form in perception. Given this pattern of dialectical refinement, we can be confident that Aristotle was engaging with Empedocles' thought in his definition of perception. And while it remains controversial how to understand the assimilation of sensible form, I believe progress can be made by interpreting Aristotle's definition of perception as addressing Empedocles' puzzlement about how remote objects can be present in sensory consciousness. Recall Empedoclean puzzlement begins with the natural thought that in seeing one takes in the external scene. The question then arises: How can we take in what remains external? Assimilation is a process of internalization. But if wholly internalized, nothing external remains. The proposal is to take Aristotle's definition of perception to be an answer to this question---a remote object can be present in sensory consciousness by assimilating its sensible form while leaving its matter in place. Understanding how Aristotle's definition of perception could be a resolution of Empedoclean puzzlement imposes a substantive constraint on interpreting that definition. For so interpreted, it is making an important claim about the metaphysics of sensory presentation.

Aristotle's definition of perception is a dialectical refinement of the \emph{endoxa} insofar as it seeks to preserve an Empedoclean insight while resolving a puzzle about how remote objects can be present to sensory consciousness. Empedocles' puzzlement about the nature of sensory presentation persists to this day, though now in the guise of discussions of presence in absence. Perhaps there are insights of Aristotle's that ought to be preserved when confronting Empedoclean puzzlement as it arises in its modern guise. If there are, then a conception of sensory presentation that preserved Aristotle's insight into the proper resolution of Empedoclean puzzlement would itself be a dialectical refinement of the respected opinion of Aristotle. What these insights might be and whether any conception of sensory presentation answers this description remains to be determined.

% section empedocles (end)

\section{Perception at a distance} % (fold)
\label{sec:perception_at_a_distance}



% section perception_at_a_distance (end)

\section{Transparency} % (fold)
\label{sec:transparency}

Let's turn now to the transparent. In so doing we are jumping into the middle of things---both in the order of Aristotles's exposition, but also in that transparency is a common nature or power of the external medium that separates the perceiver and the remote object of vision.

There are philosophical reasons for considering the nature of the transparent.

First, Aristotle defines color in terms of the transparent. Specifically, in \emph{De Anima} Aristotle defines color as the power to move what is actually transparent. Our understanding of color is incomplete if we do not understand the state of the external medium, such as air or water, which is a pre-condition for the activity of color. The effect of this incomplete understanding ramifies given Aristotle's avowed strategy of explaining perceptual capacities in terms of perceptual activities that are their exercise and to explain perceptual activities in terms of the objects of those activities.

Second, given that color is the power to move what is actually transparent, there is an alteration that the external medium undergoes when in a transparent state as a result of the activity of color. Suppose the alteration that the external medium undergoes when in a transparent state is imparted to the internal medium---the transparent medium that constitutes the interior of the sense organ, in the case of the eye, the \emph{vitreous humour}. Then we would have in place an important piece of the puzzle involved in interpreting the assimilation of sensible forms. Moreover, progress can be made here while forestalling the controversies, which must eventually be faced, surrounding the metaphysics of \emph{De Anima} II 5.

Third, external particulars remote from the perceiver are arrayed in an external medium through which, when transparent, they appear. Empedoclean puzzlement highlights the way in which this is a remarkable fact. Attending to the details of Aristotle's discussion of transparency we may gain insight into his reaction to Empedoclean puzzlement about the presentation of remote objects in sensory consciousness.

It is worth enumerating the reasons for considering Aristotle's discussion of transparency in light of its reception. Some commentators suggest that Aristotle's discussion of transparency is of antiquarian interest only. Others have expressed incredulity at the way Aristotle's account conflicts with the manifest facts of experience. While leaving open the possibility of errors and omissions, we should try to understand Aristotle's account by attending to the phenomena as we understand it to be and by asking how that phenomena might have appeared as Aristotle describes it. As a methodology, this is little more than the minimum exercise of charity. However, it is worth stating explicitly if only because it is routinely flouted. Adhering to this precept, yields and understanding of transparency that is sensible, phenomenologically adequate, and a reasonable approximation of the truth. It is also a philosophically revealing exercise since to interpret Aristotle's text in this way involves attending to the phenomena under discussion.

Aristotle defines the transparent as that which is visible, though not visible in itself, but owing its visibility to the color of another thing. 

First, modern thinkers may register surprise that Aristotle defines transparency in terms the manner of its visibility, or the way in which it appears in perceptual experience. Isn't it more revealing of the nature of transparency to think of it in terms of that through which remote objects appear---that is to say, not in terms of the manner of its visibility, but in terms of its being a precondition on the visibility of things? We will return to this issue.

The second thing to remark about Aristotle's definition is the nature of the intended contrast between being visible in itself and owing one's visibility to another thing. Aristotle does not have in mind here what he elsewhere calls incidental perception. Seeing the transparent medium by seeing colors arrayed in it is not like seeing the son of Diares by seeing a white speck. Being the son of Diares is not a sensible quality the way being transparent is.

Third, once we recover from our initial surprise at Aristotle defining transparency in terms of the manner of its visibility, and we consider the plausibility of that claim, quite apart from its status as a definition, we discover a potential insight. I at least have the corresponding intuition about the character of illumination. I believe we see the character of the illumination by seeing the way objects are illuminated. The former is the nature of a state of the external medium whereas the latter is a property of a particular arrayed in that medium (though, of course, a property that the particular could only have given the nature of the state of the medium). Hilbert reports the intuition as follows:
\begin{quote}
	Do we see how an object is illuminated or do we see the illumination itself? On phenomenological grounds the first option seems better to me. What we see as changing with the illumination is an aspect of the object itself, not the light source or the space surrounding the object. \citep[150--151]{Hilbert:2007qy}
\end{quote}

Transparency is a nature or power common to different substances. It is shared by liquids like air and water and certain solids like glass beads and is incidental to the nature of each. A medium composed of air or water is actually transparent not due to their respective natures but due rather to the contingent presence of the fiery substance. The continual presence of the fiery substance is required for the transparency of the medium to persist. Consider the following. I might light a fuse with a cigar. Prudence councils that I should remove myself from the scene. Should I be enjoying the cigar I may take it with me. While the fuse remains lit even when the cigar is removed, the air does not remain transparent when the fiery substance is removed. When the fiery substance is removed, darkness supervenes. Not only does the persistence of transparency depend upon the continual presence of the fiery substance, but, arguably at least, it depends as well on its continual activity. This may be taken to be implied by his claim that light is the activity of the actual transparent.

Light is a state that the medium is in when it is actually transparent. Aristotle denies that light is fire, or a body, or a material effluence. He denies as well that light moves, otherwise its motion would be visible as it travels from East to West. These claims are puzzling if by light Aristotle means something that could be a reasonable approximation of what we mean. But why assume that? Begin by focusing on Aristotle's claim that light is a \emph{state} that a medium is in when it is actually transparent. State is really the wrong category for light as we presently understand it to be. But now, in line with our avowed methodology, let's ask whether there could be a state that we can recognize on our present understanding that could reasonably be what Aristotle had in mind when he speaks of light? With the question so framed the resolution of our difficulties should be obvious. What state is a medium in when it is actually transparent, where the persistence of this state depends on the continual presence and activity of a fiery substance? When it is illuminated, of course. By light Aristotle means a state of illumination. And that a medium when it is actually transparent is in a state of illumination sustained by the presence and activity of a fiery substance strikes me as a not unreasonable approximation of the truth. Moreover, it coheres well with the phenomenology of illumination. Consider what must have been the familiar experience of lighting an oil lamp to illuminate a room.

There remain anachronistic elements in Aristotle's account. Consider his empirical argument that light does not move. If light is a state that a medium is in due to the presence and activity of a fiery substance, then this claim has the consequence that the fiery substance cannot propagate through a potentially transparent medium. For as the fiery substance propagated through the medium, the illuminated region would change over time. It would be consistent with Aristotle's claim that light does not move that the fiery substance instantaneously propagates through a potentially transparent medium, but that would be a degenerate form of propagation.



% section transparency (end)

% Bibligography
\bibliographystyle{plainnat} 
\bibliography{Philosophy} 

\end{document}
