%!TEX root = /Users/markelikalderon/Documents/Git/formwithoutmatter/aristotle.tex
\chapter{The Generation of the Hues} % (fold)
\label{cha:the_generation_of_the_hues}

\section{Aristotle's Three Models} % (fold)
\label{sec:aristotle_s_three_models}

That white\index{white} and black\index{black}, or, better yet, light\index{black} and dark\index{dark}, are the primary colors\index{color!primary}, the colors in terms of which all other colors can be explained, is an ancient doctrine arguably of Homeric\index{Homer!color scheme} roots. As presented by Parmenides\index{Parmenides}, in the Way of Mortal Opinion\index{Parmenides!The Way of Mortal Opinion}, Fire\index{Parmenides!Fire} and Night\index{Parmenides!Night} are cosmic principles standing in opposition\index{opposites} whose attributes consists of sensible qualities\index{sensible qualities} arrayed in pairs of contraries\index{contraries}. Brightness\index{bright} and darkness\index{dark} as they appear in sensory experience are one such pair of attributes. Brightness\index{bright} is an attribute of Fire\index{Parmenides!Fire} just as darkness\index{dark} is an attribute of Night\index{Parmenides!Night}, and the opposition\index{opposites} of these cosmic principles is partly manifest in this pair of sensible qualities\index{sensible qualities} being contraries\index{contraries}. Empedocles\index{Empedocles} shares Parmenides'\index{Parmenides} conception of light\index{light} and dark\index{dark} as contrary\index{contraries} sensible qualities\index{sensible qualities}. According to Parmenides\index{Parmenides}, brightness\index{bright} is an attribute of Fire\index{Parmenides!Fire}. It has others. Fire must then be independent of brightness\index{bright}, in some appropriate sense. In seeing the sun\index{sun} burning bright\index{bright}, what one sees is a manifestation of the operation of the cosmic principle of Fire\index{Parmenides!Fire}. It is the activity of the fiery principle that explains the brightness\index{bright} of distal objects. Empedocles\index{Empedocles} also takes over from Parmenides\index{Parmenides} this explanatory priority. White\index{white} or light\index{light} is explained in terms of the element\index{elements} of fire\index{fire} composing the effluences\index{Empedocles!effluence!chromatic} emitted from distal objects themselves composed of a preponderance of fire\index{fire}. In contrast, black\index{black} or dark\index{dark} is explained in terms of the element of water\index{water} composing the effluences\index{Empedocles!effluence!chromatic} emitted from distal objects themselves composed of a preponderance of water. 

Empedocles\index{Empedocles}, however, makes two important contributions (at least on my partial and selective account of this ancient tradition). First, not only are the sensible qualities\index{sensible qualities}, light\index{light} and dark\index{dark}, conceived as contraries\index{contraries}, but, like hot\index{hot} and cold\index{cold}, as endpoints of an ordered range of sensible qualities. The chromatic hues\index{hue} are the sensible qualities\index{sensible qualities} intermediate between the extremes of light\index{light} and dark\index{dark}. Second, on Parmenides'\index{Parmenides!color} account, the chromatic hues that objects appear to have, as well as ``the whole world of sense'', are the result of ``blending'' Fire\index{Parmenides!Fire} and Night\index{Parmenides!Night} (Plutarch,\index{Plutarch} \emph{Adversus Colotem} 1114\( ^{b-c} \)).\index{Adversus Colotem@\emph{Adversus Colotem}} However, the Way of Mortal Opinion,\index{Parmenides!The Way of Mortal Opinion} at least in the fragmentary state in which it has come down to us, does not elaborate how the blending of Fire\index{Parmenides!Fire} and Night\index{Parmenides!Night} results in the appearance of chromatic hues\index{hue} in our sensory experience. Empedocles\index{Empedocles} second contribution is that it is the \emph{proportion}\index{proportion} or \emph{ratio}\index{ratio} of Fire\index{Parmenides!Fire} and Night\index{Parmenides!Night}, or in terms of his own cosmology, the ``roots''\index{Empedocles!roots} or elements\index{elements} fire\index{fire} and water\index{water}, that gives rise to the appearance of chromatic hues\index{hue} in our sensory experience of the natural environment. It is plausible that neither contribution is original to Empedocles\index{Empedocles}. Thus, for example, \citet{Gladstone:1858fk}\index{Gladstone, W.E.} discerns the former in the Homeric color scheme.\index{Homer!color scheme} Moreover, the emphasis on the proportion\index{proportion} or ratio\index{ratio} of Fire\index{Parmenides!Fire} and Night\index{Parmenides!Night} is arguably implicit in the Parmenidean fragments: Justice\index{Justice} governs the sensible world presumably by governing the changing proportions of Fire and Night in the cosmic mixture.

Much of Empedocles\index{Empedocles} work is an attempt to reconcile putative Parmenidean\index{Parmenides} insights with the way things appear in our sensory experience. Empedocles accepts the central lesson of the Way of Mortal Opinion\index{Parmenides!The Way of Mortal Opinion} that one must posit a plurality of principles in opposition if one is to accommodate the plurality and opposition encountered in sensory experience and so abandon's Parmenides' monism.\index{Parmenides!monism} Aristotle too wishes to save the phenomena\index{manifest image} while preserving the insights of his predecessors, Parmenides and Empedocles prominent among them. Indeed he is the great defender of the manifest image in the classical world. Moreover, Aristotle takes over from Empedocles the general idea that the chromatic hues result from the proportion or ratio of light and dark. Aristotle provides an extended discussion of how these ratios might be implemented. First, he offers three accounts, in terms of (1) juxtaposition,\index{juxtaposition model} (2) overlap,\index{overlap model} and (3) mixture,\index{mixture model} opting for the third. Second, he provides an account of what the chromatic proportions\index{proportion} or ratios\index{ratio} are, and makes some important related claims about the ordering of sensible qualities\index{sensible qualities} between the extremes of light and dark.

\subsection{Juxtaposition} % (fold)
\label{sub:juxtaposition}

\index{juxtaposition model|(}
Aristotle presents the first account as follows:
\begin{quote}
	We must now treat of the other colours, reviewing the several ways in which they can come about. It is conceivable that the white\index{white} and the black\index{black} should be juxtaposed in quantities so minute that either separately would be invisible, though the joint product would be visible; and that they should thus have the other colours for resultants. Their product could, at all events, appear neither white nor black; and, as it must have some colour, and can have neither of these, this colour must be of a mixed character---in fact, a species of colour different from either. Such, then, is a possible way of conceiving the existence of a plurality of colours besides the white and black; (Aristotle, \emph{De Sensu} \textsc{iii} 439\( ^{b} \)18--28; Beare in \citealt[8]{Barnes:1984uq})
\end{quote}
Aristotle asks us to imagine a visible compound composed of white and black parts, themselves too small to be visible. Since the compound is visible it must have some color. Since the white and black parts are too small to be visible, the color of the compound could not be either of these. So the compound must have some other kind of color. And it is the proportion of white and black components that determines the given chromatic hue. The remainder of the passage develops this suggestion. 

Familiarity with pointillism\index{pointillism} and color halftone printing\index{color halftone printing} can obscure for us the real achievement in Aristotle's entertaining the possibility that the color of a compound can differ from the color of its parts. Pointillist paintings and color halftone prints have minute parts that differ in color from the painting or print as a whole at least when viewed from a suitable distance. 

Michel Eugène Chevreul\index{Chevreul, Michel Eugène}, a French chemist appointed by Louis \textsc{xvii}\index{Louis \textsc{xvii}} as the director of the dye department of Manufacture Royale des Gobelins\index{Manufacture Royale des Gobelins}, upon receiving complaints that the black\index{black} dyes they produced looked different when used alongside blue\index{blue} dye, investigated the matter and produced a systematic study of the phenomena of simultaneous color contrast\index{color!contrast effects}---that the appearance of a color can vary as the color of the surrounding scene varies. The phenomenon was not unknown to the ancients. Indeed, Aristotle makes just the same observation that prompted Chevreul's investigation: 
\begin{quote}
	Bright dyes too show the effect of contrast. In woven and embroidered stuffs the appearance of colours is profoundly affected by their juxtaposition with one another (purple, for instance, appears different on white and on black wool), and also by differences of illumination. Thus embroiderers say that they often make mistakes in their colours when they work by lamplight, and use the wrong ones. (Aristotle, \emph{Meteorologica} \textsc{iii} 375\( ^{a} \)23--28; Webster in \citealt[61]{Barnes:1984uq})\index{Meteorologica@\emph{Meteorologica}}
\end{quote}
Not only does Aristotle discuss color contrast effects\index{color!contrast effects}, but the last observation of this passage is a clear description of the problems posed by the phenomenon of metamerism\index{color!metamerism}. Two different colors are metameric pairs if, in a given circumstance of perception, they present the same color appearance. So by lamplight\index{lantern} different colored dyes presenting the same color appearance are metameric pairs. And it was this metameric pairing, in the given circumstances of perception, that was an obstacle for the embroiderers' visually discriminating different colored dyes. Thus not only did Aristotle recognize color contrast effects, but he recognized the phenomenon of metamerism as well.

Chevreul\index{Chevreul, Michel Eugène} reported his findings about simultaneous color contrast in his book, \emph{De la Loi du Contraste Simultan\'{e} des Couleurs}\index{De la Loi du Contraste Simultan\'{e} des Couleurs@\emph{De la Loi du Contraste Simultan\'{e} des Couleurs}} (1839; translated into English by Charles Martel as \emph{The Principles of Harmony and Contrast of Colours and Their Application to the Arts} \citeyear{Chevreul:1855kx}), a book that influenced the work of French painters such as Eug\'{e}ne Delacroix\index{Delacroix, Eug\'{e}ne}, Georges-Pierre Seurat\index{Seurat, Georges-Pierre}, and Paul Signac\index{Signac, Paul}. Fascinated by the appearance of a color being influenced by adjacent colors, Seuratindex{Seurat, Georges-Pierre} eventually paints the pointillist masterpiece, “Un Dimanche Apr\`{e}s-midi \`{a} l'\^{I}le de la Grande Jatte”\index{Un Dimanche Apr\`{e}s-midi \`{a} l'\^{I}le de la Grande Jatte} in 1884--6. Using only primary unblended pigments, including the newly available zinc yellow\index{zinc yellow}, these were distributed in small dots across the surface of the canvas giving rise to the appearance, at an appropriate distance, of a differently colored scene of Parisian suburbanites relaxing by the river Seine. The analogy with pointillism\index{pointillism} and color halftone printing\index{color halftone printing} is imperfect, however, in that the minute parts of the painting and print are merely too small to be seen from a suitable distance, where as the white and black parts of Aristotle's compound are too small to be seen at any distance.

% (see figure~\ref{fig:jatte})
% \begin{figure}[htbp]
%     \centering
%         \includegraphics[scale=0.70]{graphics/jatte.jpg}
%     \caption{Georges-Pierre Seurat, “Un dimanche après-midi à l'Île de la Grande Jatte”}
%     \label{fig:jatte}
% \end{figure}


Notice that on the proposed account color is not dissective\index{dissective} in something like Goodman's \citeyearpar[53]{Goodman:1951ww}\index{Goodman, Nelson} sense of the term. A property \( p \) is \emph{dissective} just in case if \( p \) is instantiated by a whole, \( p \) is instantiated by each of its parts. (Dissectivity is a broader notion than being homeomerous\index{homeomerous} since the latter is restricted to substance\index{substance} kinds, \emph{De Generatione et Corruptione} \textsc{i} 10 328\( ^{a} \)6ff.)\index{De Generatione et Corruptione@\emph{De Generatione et Corruptione}} So if color were dissective\index{dissective}, then the color of the whole would be the color of its parts. The color of a whole may be a function of the color of its parts, a point on which Aristotle and Goodman agree, but that does not mean that the function will determine color to be dissective:
\begin{quote}
	Different perceptible parts of any object may be differently colored even if the object itself is uniform and unvarying in color. This is no more paradoxical than the fact that a single object contains spatiotemporally different parts. As the self-identical object is a function of its parts, so the single unchanging color of the object is a function of the colors of its parts. The nature and interrelation of the lesser elements that make up the whole determine what kind of thing the whole is: the kind and arrangement of the colors exhibited by these various parts determine what color the whole is said to have. \citep[130]{Goodman:1951ww}
\end{quote}

On the present account, we can see the blue\index{blue} of the sea\index{sea!color} even though we fail to see its white\index{white} and black\index{black} parts since they are too are too small to be seen. A consequence of the juxtaposition model is that it is possible to see the color of a whole without seeing the distinct colors of the parts that compose it. That thought, however, is only intelligible set against a background conception of perception as providing a \emph{partial} perspective\index{perspective} on the natural environment. The partiality of perception\index{partiality of perception} has recently been defended by \citet{Hilbert:1987jq}\index{Hilbert, David R.}, but it has ancient roots as well---arguably, Heraclitus\index{Heraclitus!perception} is an advocate \citep[see][]{Burnyeat:1979mv,Kalderon:2006tg}. Not only is perception partial in the sense that there are properties of an object not perceptually available (objects may have unobservable aspects), not only is perception partial in the sense that some sensible qualities of an object may be occluded from view (the backs of objects are colored as well), but perception is also partial in the sense that there are sensible qualities of an object that are not determined by a given perception. If one can see the color of a whole while failing to see the distinct colors of its parts, then one can see some if not all of an object's chromatic features. One sees the blue of the sea but not that it is partly white and partly black. This is only possible if perception is partial in something like the sense described above.

I am uncertain whether Aristotle genuinely subscribes to some version of this doctrine. While it is a commitment of the juxtaposition model, this is a model that he rejects. However, while a commitment of the juxtaposition model, the partiality of perception\index{partiality of perception} is not itself committed to that model. Doubts about the juxtaposition model need not undermine the partiality of perception. Earlier we registered a disanalogy between the juxtaposition model, on the one hand, and pointillism\index{pointillism} and color halftone printing\index{color halftone printing}, on the other. While the latter involves parts too small to be seen from a certain distance\index{distance}, the former involves parts too small to be seen at any distance. The partiality of perception\index{partiality of perception} is manifest in viewing the color of a pointillist painting---one sees the color of the whole without seeing the colors of its parts. Moreover, this is consistent with the Aristotelian denial of invisible magnitudes. So even if there are no parts too small to be seen from any distance, this would not, by itself, cast doubt on the partiality of perception. If Aristotle does indeed retain some, perhaps attenuated, version of this doctrine, this would go some way towards explaining his sanguine attitude towards putative cases of conflicting appearances\index{conflicting appearances} \citep[on how the partiality of perception can help dissipate some appearances of conflict see][]{Kalderon:2006tg}.

Aristotle rejects the juxtaposition model partly on the grounds that it posits colored objects too small to be seen (\emph{De Sensu} \textsc{iii} 440\( ^{a} \)21--25).\index{De Sensu@\emph{De Sensu}}\index{magnitude!must be seen} Such parts would have magnitude and yet would be invisible. But, according to Aristotle, there are no invisible magnitudes. Every magnitude is visible from some distance. And while the color of some wholes dissolve upon closer inspection, such as Seurat's masterpiece or the color halftone printing of the Marvel Comics\index{Marvel Comics} of my childhood, not all do. There are some surfaces that retain their color no matter how closely we look (compare \emph{De Sensu} \textsc{iii} 440\( ^{b} \)16--18 discussed further in the next section~\ref{sub:overlap}). So the juxtaposition model is implausibly revised to claim instead that the colors of compounds are determined by the juxtaposition of minute white and black parts that are normally not visible. It is open to ready empirical disconfirmation when we fail to discover these black and white parts despite our best efforts.

% (figure~\ref{fig:jatte})
% (figure~\ref{fig:wolverine})
% \begin{figure}[htbp]
%     \centering
%         \includegraphics[scale=5]{graphics/wolverine.jpg}
%     \caption{Detail of Wolverine from \emph{X-Men}, 1963}
%     \label{fig:wolverine}
% \end{figure}

In his initial presentation of the juxtaposition model, Aristotle considers a case involving the spatial juxtaposition of white\index{white} and black\index{black} parts.\index{juxtaposition model!spatial variant} He also considers a variant of this model, where the white and black things are not spatially juxtaposed but are instead temporally juxtaposed.\index{juxtaposition model!temporal variant} Set in the context of the theory of effluences,\index{Empedocles!effluence} the idea is that the temporal juxtaposition of the white\index{white} and black\index{black} effluences\index{Empedocles!effluence!chromatic} assimilated by the organ of sight\index{sight} gives rise to a chromatic appearance. Just as spatial inhomogeneities of the compound body composed of white and black parts determines a proportion\index{proportion} or ratio\index{ratio} of light\index{light} and dark\index{dark} characteristic of say, blue\index{blue}, it is the temporal inhomogeneities of the assimilated effluences---now white, now black---that determines a proportion of light and dark characteristic of blue. In this regard, the temporal variation of the juxtaposition model is analogous to the way in which Benham's spinning disk\index{Benham disk} can give rise to chromatic appearances.

Bergson's \citeyearpar[75--77]{Bergson:1912pi}\index{Bergson, Henri} otherwise idiosyncratic account of how qualitative heterogeneity\index{qualitative heterogeneity} emerges from quantitative homogeneity\index{quantitative homogeneity} directly echoes the temporal variant of the juxtaposition model:
\begin{quote}
	The qualitative heterogeneity of our successive perceptions of the universe results from the fact that each, in itself, extends over a certain depth of duration, and that memory condenses in each an enormous multiplicity of vibrations which appear to us all at once, although they are successive. \citep[76--77]{Bergson:1912pi}
\end{quote}
Individual vibrations, of a quantifiably specifiable amplitude and duration, succeed one another too quickly to be individually perceptible. A perception, no matter how momentary, always has duration. And it is the condensation of the multiplicity of vibrations in what appears all at once that results in the perception of qualitative heterogeneity. In a remarkable passage, Bergson\index{Bergson, Henri} writes:
\begin{quote}
	If we could stretch out this duration, that is to say, live it at a slower rhythm, should we not, as the rhythm slowed down, see these colours pale and lengthen into successive impressions, still coloured, no doubt, but nearer to coincidence with pure vibration? \citep[268--269]{Bergson:1912pi}
\end{quote}
To get a sense of this, consider Benham's spinning disk.\index{Benham disk} Recall the bottom half of the disk is black\index{black} and the top half has twelve arcs of concentric circles drawn upon it. If rotated clockwise, the innermost arcs form reddish\index{reddish} rings, the next greenish\index{greenish} rings, the next light blue\index{blue} rings, and the outermost arcs form violet\index{violet} rings. As the disk's spinning slows, the chromatic appearances fade over time until one can discern alternating patterns of black\index{black} and white\index{white}. Bergson's\index{Bergson, Henri} variant of the temporal juxtaposition model\index{juxtaposition model!temporal variant} is what you would get if you took the Fechner-Benham colors to be the model for how all colors are realized.

The temporal variant of the juxtaposition model\index{juxtaposition model!temporal variant} faces a parallel problem as the spatial variant. Just as the spatial variant of the juxtaposition model was committed to imperceptible spatial magnitudes,\index{magnitude!must be perceptible} the temporal variant is committed to imperceptible temporal magnitudes and for much the same reason:
\begin{quote}
	If we accept the hypothesis of juxtaposition, we must assume not only invisible magnitude, but also imperceptible time, in order that the arrival of the movements may be unperceived, and that the colour may appear to be one because they seem to be simultaneous. (Aristotle, \emph{De Sensu} \textsc{iii} 440\( ^{a} \)20--25; Beare in \citealt[9]{Barnes:1984uq})\index{De Sensu@\emph{De Sensu}}
\end{quote}
Consider alternating assimilations\index{assimilation!material} of white\index{white} and black\index{black} effluences\index{Empedocles!effluence!chromatic} by the organ of sight\index{sight} occurring in a certain temporal ratio. The pattern of alternating assimilations is perceptible. The pattern at least determines the experience of a chromatic hue. But our experience of a color of a particular is not the experience of a succession of light and dark. So the assimilation of white and black effluences must occur too quickly to be individually perceptible. However, if temporally juxtaposed in the right proportion, the temporal compound, the pattern of alternating assimilations, would be perceptible.  Indeed, it would be the perception of the chromatic hue. Unfortunately, just as Aristotle rejects imperceptible spatial magnitudes, he also rejects imperceptible temporal magnitudes and with it the temporal variant of the juxtaposition model.
\index{juxtaposition model|)}

% subsection juxtaposition (end)

\subsection{Overlap} % (fold)
\label{sub:overlap}
\index{overlap model|(}
On the first model, chromatic hues\index{hue} are determined by a proportion\index{proportion} of light\index{light} and dark\index{dark} that arises from light and dark objects being temporally\index{juxtaposition model!temporal variant} or spatially juxtaposed\index{juxtaposition model!spatial variant}. The second model that Aristotle considers works not by means of juxtaposition, but by means of overlap:
\begin{quote}
	Another is that the black\index{black} and white\index{white} appear the one through the medium\index{medium} of the other, giving an effect like that sometimes produced by painters overlaying a less vivid upon a more vivid colour, as when they desire to represent an object appearing under water\index{water} or enveloped in a haze, and like that produced by the sun\index{sun}, which in itself appears white\index{white}, but takes a crimson\index{crimson} hue\index{hue} when beheld through a fog\index{fog} or a cloud\index{cloud} of smoke\index{smoke}. On this hypothesis, too, a variety of colours may be conceived to arise in the same way as that already described\index{color!generation of the hues}; for between those at the surface\index{surface} and those underneath a definite ratio\index{ratio} might sometimes exist; in other cases they might stand in no determinate ratio. (Aristotle, \emph{De Sensu} \textsc{iii} 440\( ^{a} \)7--15; Beare in \citealt[8--9]{Barnes:1984uq})\index{De Sensu@\emph{De Sensu}}
\end{quote}
Perhaps colors are not so much juxtaposed as they are overlapping. The overlapping colors, however, are importantly perceptually penetrable at least to some degree---they appear through one another. Suppose one color overlays another color. If the overlaying color is perceptually impenetrable\index{perceptual impenetrability}, if it determines a visual boundary\index{bounded} through which nothing further could appear, the underlying color would be occluded\index{occlusion}, and this would not be a method of color combination since only the overlaying color could be seen. If overlaying and underlying colors are genuinely combined by overlap, then at least the overlaying color must be perceptually penetrable\index{perceptual penetrability} at least to some degree. Moreover, it cannot be perfectly transparent. If the overlaying color were perfectly transparent, it would be wholly receptive of the underlying color, and, again, this would not be a method of color combination since only the underlying color could be seen. For the overlap model to work, at least the overlaying color must be imperfectly transparent\index{transparency!degrees of}. The overlaying color's contribution to the resulting chromatic appearance consists, in part, in the visual resistance it offers:
\begin{quote}
	\ldots\ the stimulatory process produced in the medium\index{medium} by the upper colour, when this is itself unaffected, will be different in kind from that produced by it when affected by the underlying colour\index{color}. Hence it presents itself as a different colour, i.e. as one which is neither white\index{white} nor black\index{black}. (Aristotle, \emph{De Sensu} \textsc{iii} 440\( ^{a} \)24--28; Beare in \citealt[9]{Barnes:1984uq})
\end{quote}
Moreover, the ratio\index{ratio} of the overlapping colors that results in the novel color is partly determined by the degree of visual resistance offered by the imperfectly transparent overlaying color.

The painting analogy is arguably a deliberate echo of Empedocles (\textsc{dk} 31\textsc{b}23)\index{Empedocles!painting analogy}. As in the Emepedoclean fragment, the method of color combination deployed by the painters is overlaying semi-transparent colored washes---the method that Plutarch\index{Plutarch} attributes to Apollodorus\index{Apollodorus} and is characteristic of Greek four-color\index{four-color palette} painting more generally. Aristotle's choice of depicted content further emphasizes this: He draws our attention to how a painter might depict something appearing through water\index{water} or mist\index{mist} by overlaying a wash of some appropriate color. Here perceptually penetrable washes of pigment\index{pigment} are the means of representing something that is itself perceptually penetrable\index{perceptual penetrability}---the water\index{water} or mist\index{mist} through which the object appears. He draws our attention to the imperfectly transparent subject matter as a way of emphasizing the imperfectly transparent means of representing that subject matter. The painting analogy thus further confirms that at least the overlaying color must be imperfectly transparent.

The sun\index{sun} seen through a fog\index{fog} or cloud\index{cloud} of smoke\index{smoke} is Aristotle's second analogy. The sun\index{sun} is white\index{white}, and the smoke is black\index{black}. And yet when the cloud of smoke is superimposed over the sun, it gives rise to a crimson\index{crimson} appearance. If the black of the smoke were perceptually impenetrable\index{perceptual impenetrability}, if it determined a visual boundary\index{bounded} through which nothing further could appear, then the white\index{white} of the sun\index{sun} would have been occluded\index{occlusion} by the black\index{black} of the smoke\index{smoke}, and a method of color combination could not be understood on this analogy since only the overlaying color could be seen. If on the other hand, the smoke\index{smoke} were perfectly transparent, it would be wholly receptive to the white\index{white} of the sun\index{sun} and, again, a method of color combination could not be understood on this analogy since only the underlying color could be seen. For the analogy to work, the smoke must be imperfectly transparent, the blackness\index{black} of the smoke contributes to the resulting chromatic appearance, in part, by the visual resistance it offers. Though it remains receptive of the white of the sun, otherwise it would be opaque, the darkness of the smoke resists perceptual penetration insofar as it can. The resulting proportion\index{proportion} of light\index{light} and dark\index{dark} presented to the organ of sight is determined in part by the degree of perceptual penetrability of the smoke. And it is the ratio of light and dark that determines the sun's crimson appearance when obscured by smoke from a battle. So for the analogy to hold, on the overlap model, it is the ratio of light and dark that results from overlap that determines the chromatic hues.\index{color!the generation of the hues}

The overlap model postulates neither invisible magnitudes\index{magnitude} nor imperceptible time, and so is not subject to the difficulties facing the juxtaposition model. Moreover, it retains what is by Aristotle's lights the salutary doctrine that chromatic hues are determined by a ratio\index{ratio} of light\index{light} and dark\index{dark}. However, Aristotle rejects the overlap and juxtaposition models in favor of a model that works by mixture. What's wrong with the overlap model? Aristotle writes:
\begin{quote}
	It is plain that when bodies\index{body} are mixed their colours\index{color} also are necessarily mixed\index{mixture} at the same time; and that this is the real cause determining the existence of a plurality of colours—not superposition or juxtaposition. For when bodies are thus mixed, their resultant colour presents itself as one and the same at all distances alike; not varying as it is seen nearer or farther away. (\emph{De Sensu} \textsc{iii} 440\( ^{b} \)16--18; Beare in \citealt[10]{Barnes:1984uq})
\end{quote}
This can initially strike one as an odd response. Indeed, the complaint seems best directed at an alternative to the juxtaposition model that does not posit invisible magnitudes, but rather magnitudes too small to be seen in normal circumstances. Think again of pointillist painting\index{pointillism} and color halftone printing\index{color halftone printing}. The color of the painting or the print only seems uniform at a suitable distance but dissolves into differently colored parts when near at hand. But not all visible particulars are like that. A laurel leaf\index{laurel leaf} will look green no matter how close you look at it and still count as looking. (Look too closely, by pressing the receptive part of the eye, the pupil, against the leaf, and you will see neither the leaf nor its color.) What is puzzling is how this objection could get a grip on the present model. How can the variability of color with distance arise by means of overlap?

Consider the sun\index{sun} seen through a cloud\index{cloud} of smoke\index{smoke}. The dark\index{dark} smoke overlays the sun burning bright\index{bright}. The reduction of the sun's brilliance results in its crimson\index{crimson} appearance. This is due to the black\index{black} particulate matter\index{matter} of the smoke\index{smoke} suspended in the transparent\index{transparency} medium\index{medium}, in the present case, the air\index{air}. Suppose the black particulate matter is uniformly distributed in the region of the cloud\index{cloud}. Then the degree to which the sun's\index{sun} brilliance is decreased will depend on the depth of the intervening region. Holding fixed the density of the particulate matter, understood as the number of particles per unit volume, then a greater region of smoke will result in a greater reduction in the sun's brilliance than would result had the sun been seen through a smaller region. A smaller region of smoke, with the same density, while dark, would not be as dark as the greater region. And the sun seen through the smaller region would be brighter than the sun seen through the darker region. Indeed, seen through the smaller region of smoke, the sun would not appear crimson\index{crimson}, but orange\index{orange}, say. But this is just the variability of color with distance that Aristotle objects to.

Aristotle's complaint is that ``colour presents itself as one and the same at all distances alike; not varying as it is seen nearer or farther away.'' There are two ways to understand this objection. On the first understanding, what is uniform is the color\index{color} appearance presented by the particular\index{particular} when viewed from all distances. On the second understanding, what is uniform is the color the particular appears to have at all distances from which its color can be seen.

On the first understanding, there are at least some particulars whose chromatic appearance is relatively uniform at any distance from which its color can be seen. On the first understanding of the objection, then, the overlap model is at best an overgeneralization of a special case. The chromatic appearance of at least some particulars are relatively uniform at any distance. The example of the laurel leaf\index{laurel leaf} looking green\index{green} no matter how close you look at it and still count as looking may encourage this thought. It is true that proximity to the laurel leaf does not reveal it to be partly white\index{white} and partly black\index{black}. But that is not to say that the green\index{green} of the leaf appears the same way at every distance from which it can be seen. Indeed, it is unobvious that there are such particulars. The son of Diares\index{son of Diares} looks like a white\index{white} speck when seen from a distance in the way that he does not closer up. Moreover, even particulars with a relatively stable chromatic appearance in a range of familiar circumstances can be affected by atmospheric conditions. Think of blue\index{blue} mountains\index{mountain}. The problem with the present understanding is not just that it seems false, but that it can be seen to be false by reflecting on Aristotle's own examples. 

On the second understanding, what remains uniform is the perception of the particular's color despite that color's appearance varying with the distance from which it is viewed. On this understanding, that the color of a particular seems uniform at all distances just is seeing the constant color of the particular at any distance at which it is visible despite its appearance varying with the circumstances of perception. On this second understanding of the objection, then, the overlap model is inconsistent with an aspect of color constancy\index{color!constancy}\index{perceptual constancy}. On the overlap model, color varies with distance. But one can at least sometimes see that a particular has an unchanging color despite its color appearance changing with the distance from which it is viewed. There can be variation in color appearance without a variation in presented color. If color varies with distance, then one cannot perceive a particular to have an unchanging color even as its appearance changes with viewing distance.

The fundamental problem with the present account is that it too closely models color combination in terms of the appearance of a color through an imperfectly transparent\index{transparency} medium\index{medium} with a given volume color\index{color!volume}. The surface color\index{color!surface} of a figure can be seen through water\index{water} or mist\index{mist}, just as the radiant color\index{color!radiant} of the sun\index{sun} can be seen through fog\index{fog} or a cloud\index{cloud} of smoke\index{smoke}. In seeing a colored particular through a colored medium, the resulting chromatic appearance is partly due to the surface or radiant color of the particular\index{particular} and partly due to the volume color of the medium. But this is at best an account of how colors jointly combine to determine a chromatic appearance, and not an account of color combination. The way in which the overlap model runs afoul of color constancy\index{color!constancy}\index{perceptual constancy} is a symptom of this. That color appearances vary with distance was mistaken for the colors themselves varying with distance. Once the mistake is made, there is no color that persists as the object of visual awareness throughout the flux\index{flux} of sensory appearances that arise through changing one's point of view.
\index{overlap model|)}

% subsection overlap (end)

\subsection{Mixture} % (fold)
\label{sub:mixture}

\index{mixture model|(}
The juxtaposition and overlap models may be subject to the difficulties described above, but larger philosophical concerns are at work in Aristotle's claim that it is the ratio\index{ratio} of light\index{light} and dark\index{dark} in a \emph{mixture} that determines chromatic hues\index{color!generation of the hues}. Specifically, Aristotle's views about elemental composition prompt this view of chromatic composition.

According to Empedocles\index{Empedocles}, the combination of the ``roots''\index{Empedocles!roots} or elements\index{elements} operates on the model of juxtaposition\index{juxtaposition model}. The divine glues of harmony\index{Empedocles!divine glues of harmony} bind the elements not by mixture\index{mixture}, but as small pieces standing next to each other touching (\textsc{dk} 31\textsc{b}96). It is in these terms that Empedocles\index{Empedocles} sought to explain the growth and decay of compound bodies. What mortals describe as ``growth'' and ``decay'' are really the result of the combination and separation of unalterable, ungenerated, and imperishable elements (\textsc{dk} 31\textsc{b}8). 
% Thus Aëtius reports:
% \begin{quote}
%     Empedocles says that there is growth of nothing, but rather a mixture and separation of the elements. For in book one of the physics he writes thus:
%     \begin{verse}
%         I shall tell you something else. There is no growth of any of all mortal things,\\
%         nor any end in destructive death,\\
%         but only mixture and interchange of what is mixed\\
%         exist, and growth is the name given by mortal men.\\
%     \end{verse}
% (Aëtius, \emph{Dox. Gr.} 326 10--21; \citealt[\textsc{ctxt}-16 94]{Inwood:2001ve})
% \end{quote}
While Empedocles resisted in this way the full thrust of Parminedean skepticism about generation and corruption, the concessions he makes to Parmenides\index{Parmenides} distinguishes his view from sixth century \textsc{bc} thinkers as yet untouched by Parmenidean doubts. Thus Kahn\index{Kahn, Charles H.} remarks:
\begin{quote}
	The Parmenidean attack on generation and corruption dominates the entire development of natural philosophy in the fifth century. At the same time, it signifies a radical break with the older point of view. \ldots\ That ``coming-to-be'' and ``perishing'' played an essential role in all previous doctrines is the natural conclusion to be drawn from a reading of his poem; and this view is fully confirmed by the fragments of Xenophanes\index{Xenophanes} and Heraclitus\index{Heraclitus}. In contrast to the denial of Parmenides\index{Parmenides}, Anaxagoras\index{Anaxagoras}, and Empedocles these earlier men speak unhesitatingly of ``generation,'' ``growth,'' and ``death.'' The fundamental difference between the sixth and fifth centuries lies not in the abandonment of monism for plurality, but in the passage from a world of birth and death to one of mixture and separation. \citep[154--155]{Kahn:1994qf}
\end{quote}
Aristotle's preferred account of the generation of the hues is modeled on his preferred account of elemental composition, itself a return to the sixth century \textsc{bc} view.

On Aristotle's view, the Emepdoclean tetrad\index{Empedocles!roots}---water\index{water}, earth\index{earth}, air\index{air}, and fire\index{fire}---are only elements\index{elements} so-called. The so-called elements, water, earth, air, and fire, are the result of the combination of the simple primary ingredients of a compound (\emph{Metaphysica} \( \Delta \) 3 1014\( ^{a} \)26ff)\index{Metaphysica@\emph{Metaphysica}}, the primary opposites\index{primary opposites}: Hot\index{Hot}, Cold\index{Cold}, Dry\index{Dry}, and Wet\index{Wet}. Thus water\index{water} is Cold and Wet, earth\index{earth} is Cold and Dry, air\index{air} is Hot and Wet, and fire is Hot and Dry. Since the Empedoclean tetrad are only elements so-called, they are subject to a cycle of transformation familiar from ancient times.\index{cycle of elemental transformation}

In a passage self-consciously recounting the older view\index{cycle of elemental transformation}, Plato\index{Plato} describes the cycle of elemental transformation thus:
\begin{quote}
	In the first place, we see that what we just now called water\index{water}, by condensation, I suppose, becomes stone\index{stone} and earth\index{earth}, and this same element, when melted and dispersed, passes into vapor and air\index{air}. Air, again, when inflamed, becomes fire\index{fire}, and, again, fire, when condensed and extinguished, passes once more into the form of air, and once more air, when collected and condensed, produces cloud\index{cloud} and mist\index{mist}---and from these, when still more compressed, comes flowing water, and from water comes earth and stones once more---and thus generation appears to be transmitted from one to the other in a circle. (Plato, \emph{Timaeus} 49\( ^{b-c} \); Jowett in \citealt[1176]{Hamilton:1989fk})\index{Timaeus@\emph{Timaeus}}
\end{quote}
While Plato\index{Plato} recounts the older view, he does not, in the end, himself endorse the cycle of elemental transformation\index{cycle of elemental transformation}, at least as it is described here. Earth\index{earth} does not transform into other elements and earth's apparent role in the cycle of elemental transformation is explained by its being dispersed as if by Strife\index{Empedocles!Strife} and being gathered together again as if by Love\index{Empedocles!Love} (\emph{Timaeus} 56\( ^{c-d} \)). When earth seems to transform into air\index{air}, it is really divided into small parts and dispersed. When compacted water\index{water} seems to transform into earth\index{earth} and stone, the dispersed particles of earth, that never went out of existence nor became anything other than earth, gather together and unite. Earth may be successively divided and united, but earth never goes out of existence nor changes its elemental nature. And all of this is explained, according to Plato\index{Plato}, by the shape of the underlying geometrical particles.

For the most part, the cycle of elemental transformation\index{cycle of elemental transformation} described by Plato\index{Plato} seems phenomenologically apt, at least with respect to the grosser forms of the so-called elements\index{elements} that we encounter in sensory experience. Moderns may struggle, however, to understand how earth\index{earth} and stones could be the result of water compacting (or, at least, those moderns unafflicted by London\index{London} limescale). If one thought that ice is the result of water\index{water} compacting with the increase in cold\index{cold}, this would at least leave you open to the idea that compacting water can result in solid bodies with fixed boundaries. However, the passage does not mention ice\index{ice}, and it can still seem mysterious how compacting water can result in solid bodies composed of earth and stone. What experience, available to the ancients, could be vivid enough to elicit conviction in this elemental transformation?\index{cycle of elemental transformation} Consider a river\index{river} destroyed by drought\index{drought}, a fearful and ruinous experience for agrarian societies. An ancient spectator to this tragedy would watch as the river contracted, day after day, leaving in the end, nothing but the sun\index{sun} parched stones and earth\index{earth} that once constituted the river's\index{river} bed. Such an experience, I conjecture, would be vivid and significant enough to produce cosmic conviction. The desalination\index{desalination} of brine\index{brine} in salt\index{salt} production, a procedure dating back over eight millennia, is an equally marvelous, if less tragic, experience that might elicit conviction in this elemental transformation as well.

Aristotle, unlike Plato\index{Plato}, regards the continuous transformation of the Empedoclean tetrad into one another as an established fact of observation\index{cycle of elemental transformation}. So conceived, they could not be the Parmenidean\index{Parmenides} beings that Empedocles\index{Empedocles} understands them to be. The combination of the so-called elements'index{elements}, is no longer understood in terms of the juxtaposition of unaltering, ungenerated, and imperishable beings. Water, earth, air, and fire transform into one another and in so doing interfuse. And it is complete interfusion that is mixture\index{mixture} properly so-called. In a compound body composed of different items from the Empedoclean tetrad, the so-called elements combine by interfusing, that is, by blending or mixture (\emph{De Sensu} \textsc{iii} 440\( ^{b} \)3--4\index{De Sensu@\emph{De Sensu}}; though Aristotle's developed account of mixture is more complex than this, \emph{De Generatione et Corruptione} \textsc{i} 10, see \citealt{Cooper:2004fk})\index{De Generatione et Corruptione@\emph{De Generatione et Corruptione}}. With respect to elemental composition, Aristotle's view thus represents a return to the sixth century \textsc{bc} world of birth and death.

Aristotle's preferred model of the generation of the hues\index{color!generation of the hues} should be understood set against this larger reaction to Parmendiean\index{Parmenides} skepticism about growth\index{generation} and decay\index{corruption}\index{change}. It is because he regards combination and separation (understood on the model of juxtaposition) as an imperfect surrogate for growth and decay (\emph{De Generatione et Corruptione} \textsc{i} 10)\index{De Generatione et Corruptione@\emph{De Generatione et Corruptione}}, that he understands elemental composition instead in terms of mixture\index{mixture}. And it is natural, if not inexorable, that he should have a parallel understanding of chromatic composition. Aristotle's conception of color combination as mixture substantiates the grounds of Plato's charge of impiety\index{impiety} (\emph{Timaeus} 68\( ^{d} \))\index{Timaeus@\emph{Timaeus}}. If colors are completely interfused when mixed, then no mortal possesses God's power to again resolve the one into many. But that, according to Plato\index{Plato}, is what would be required to verify by experiment the ratio\index{ratio} of primary colors\index{color} combined in the mixture.

That white and black, or light and dark, are the primary colors, the colors in terms of which all other colors are explained, is an ancient doctrine, arguably of Homeric roots, that Parmenides and Empedocles share. Aristotle follows them in this. Moreover, Aristotle takes over from Parmenides and Empedocles the idea that light and dark are contraries that constitute the extreme ends of an ordered range of sensible qualities. Moreover, he emphasizes Empedocles' contribution to this tradition in claiming that it is the ratio of light and dark when combined that determines an intermediary color. Aristotle, however, departs from Empedoclean doctrine precisely in the method of combination. Aristotle understands the combination of light and dark in terms of a conception of mixture at home in pre-Parmenidean natural philosophy, in the sixth century \textsc{bc} world of birth and death. 
\index{mixture model|)}

% subsection mixture (end)

\subsection{Two Puzzles about Mixture} % (fold)
\label{sub:two_puzzles_about_mixture}

While an intellectually satisfying narrative, I do not think that we can accept it without qualification. The claim that chromatic composition involves the mixture of light\index{light} and dark\index{dark} faces two distinct though potentially related puzzles. Each puzzle concerns the kinds of things that admit of mixture. Their lesson might very well be: Chromatic composition involves the mixture of light and dark only in a Pickwickian sense of mixture\index{mixture!Pickwickian sense}.

Qualties\index{quality} are inseparable from the substances\index{substance} in which they inhere, but the ingredients of a mixture\index{mixture} must admit of separation\index{separation}. Since qualities and states\index{state} do not admit of separate existence, neither do they admit to mixture:
\begin{quote}
	Now we do not speak of the wood as combined with the fire\index{fire}, nor of its burning as a combining either of its particles with one another or of itself with the fire: what we say is that the fire is coming-to-be, but the wood is passing-away. Similarly, we speak neither of the food as combining with the body, nor of the shape as combining with the wax and thus fashioning the lump. Nor can body combine with white\index{white}, nor (to generalize) properties and states with things; for we see them persisting unaltered. But again white and knowledge\index{knowledge} cannot be combined either, nor anything else which is not separable. (Indeed, this is a blemish in the theory of those who assert that once all things were together and combined. For not everything can combine with everything. On the contrary, both of the constituents that are combined must originally have existed in separation\index{separation}; but no property can have separate existence.) (Aristotle, \emph{De Generatione et Corruptione} \textsc{i} 327\( ^{b} \)12--22; Joachim in \citealt[30--31]{Barnes:1984kx})\index{De Generatione et Corruptione@\emph{De Generatione et Corruptione}}
\end{quote}
Qualities and states do not mix with the things whose qualities and states they are. Nor can qualities and states mix with other qualities and states. Only that which admits of separate existence can be combined in mixture but qualities and states are inseperable from the things whose qualities and states they are. The argument, here, is the basis for rejecting Empedocles'\index{Empedocles} conception of the cosmic cycle. The world, as we experience it, is in an intermediate stage of the cosmic cycle, where neither Love nor Strife dominate. At times in the cosmic cycle Strife\index{Empedocles!Strife} dominates, and everything is scattered. At others Love\index{Empedocles!Love} dominates, and everything is combined in perfect Parmenidean\index{Parmenides} sphere. But, according to Aristotle, the latter is not possible since not everything can be combined in the envisioned manner. Only what admits of separate existence may be combined, but not every category of being admits of separate existence. Even the divine glues of harmony\index{Empedocles!divine glues of harmony} could not bind what what does not admit of separate existence. There are limits, apparently, to even Aphrodite's\index{Aphrodite} Love\index{Empedocles!Love}. While Empedocles is the plausible target of criticism here, Aristotle's argument would also be the basis for rejecting the Way of Mortal Opinion\index{Parmenides!The Way of Mortal Opinion}. According to the Way of Mortal Opinion ``the whole world of sense'' in which appear ``earth\index{earth}, heaven\index{heaven}, sun\index{sun}, moon\index{moon}, and stars\index{stars}'' is the result of the ``blending'' or mixture\index{mixture} of light\index{light} and dark\index{dark}. But since light and dark do not admit of separate existence, there could be no such ``blending''. 

The first puzzle, then, is this. The colors are inseparable from the particulars in which they inhere. But only that which can exist separately may be combined in a mixture. So the setting sun's\index{sun} crimson\index{crimson} hue\index{hue} could not be a mixture of white\index{white} and black\index{black}, or light\index{light} and dark\index{dark}, at least not literally.

We have remarked how Aristotle's account departs from Empedocles\index{Empedocles} in the method of color combination. Aristotle contributes to this ancient tradition in a further way. Parmenides\index{Parmenides} and Empedocles\index{Empedocles} both explain brightness\index{bright} in terms of the presence of fire\index{fire}, an explanation that Aristotle himself echoes. However, whereas Parmenides and Empedocles posit positive determinants for darkness (Night\index{Parmenides!Night} and water\index{water}, respectively), Aristotle explains darkness\index{dark} in terms of the absence\index{absence} of the fiery substance\index{fiery substance}. There is no positive determinant, be it a cosmic principle or an element, for darkness. This is partly a manifestation of Aristotle's insight concerning the connection between illumination and visibility. The fiery substance illuminates the transparent\index{transparency} medium\index{medium} and only thus are the particulars arrayed in that medium visible. If the fiery substance\index{fiery substance} is removed, darkness supervenes. 

Aristotle's contribution has an additional source in the general metaphysics of change\index{change} presented in Book \textsc{i} of \emph{Physica}\index{Physica@\emph{Physica}}. All change involves opposition, but the general form of opposition\index{opposites} involved in all change involves form\index{form} and privation\index{privation}. One may wonder how these claims could be true together. Consider a kind of substance\index{substance}. The generation of a kind of substance involves form and privation, but as substances have no opposites\index{opposites}, there is no opposition. Form and privation seems to be a more general distinction than the distinction between opposites. There may be more to opposition than form and privation, but opposition itself involves form and privation. Thus Aristotle understands the traditional opposites, discussed by Parmenides\index{Parmenides} and Empedocles\index{Empedocles} among others, in terms of form\index{form} and privation\index{privation}. In the opposition of light\index{light} and dark\index{dark}, light is the form determined by the presence and activity of the fiery substance\index{fiery substance} and darkness is the privation of the light-giving fiery substance. 

That darkness\index{dark} is determined by the absence\index{absence} of the fiery substance\index{fiery substance} rather than the presence of Night\index{Parmenides!Night} or water\index{water}, raises the second puzzle about the sense of mixture involved in the generation of the hues\index{color!generation of the hues}. If darkness\index{dark} is the sensible aspect of the absence\index{absence} of fire\index{fire}, then in what sense can brightness\index{bright} be mixed with darkness? Perhaps by mixture\index{mixture} of light and dark Aristotle just means relative brightness\index{bright}, but that would be a Pickwickian sense of mixture\index{mixture!Pickwickian sense}. 

The lesson of the first puzzle is that qualities\index{quality} and states\index{state} cannot be combined in a mixture\index{mixture}. But suppose that a quality or state is determined by the presence and activity of something that admits of separate existence. While qualities such as light and dark could not mix, perhaps the determinants of these qualities mix, at least if they admit of separate existence.\index{separation}

The lesson of the second puzzle is that absences\index{absence} do not mix. But while you cannot mix an absence, you can mix things which are more or less resistant to the illuminating presence and activity of the fiery substance\index{fiery substance}. Earth\index{earth} resists the presence and activity of the fiery substance, just as air\index{air} is receptive to it. You cannot mix absences, but you can mix something which will preclude the presence of the fiery substance or at least retard its activity. That is to say you can mix things which differ in the degree of their transparency\index{transparency}, the degree to which they are receptive to the illuminating presence and activity of the fiery substance.

Read in light of the \emph{De Sensu}\index{De Sensu@\emph{De Sensu}} doctrine that color resides in the proportion of the transparent\index{color!residing in the proportion of the transparent} that exists in all bodies, the lessons of our two puzzles might be jointly satisfied in the following manner. Consider mixing two ingredients with different degrees of transparency that result from their different elemental compositions. Prior to the mixture\index{mixture}, the ingredient with a greater degree of transparency\index{transparency!degrees of} will be brighter\index{bright} in color than the ingredient with a lesser degree of transparency which will be darker\index{dark}. Combining these ingredients in a mixture results in a mixed body\index{body} with an intermediate degree of transparency and hence a color\index{color} intermediate between the lighter and the darker. But it is not light\index{light} and dark\index{dark} that are mixed but elemental compounds with different degrees of transparency. Thus the lesson of the first puzzle is satisfied. And it is not the presence and absence\index{absence} of the fiery substance\index{fiery substance} that is mixed but, again, elemental compounds with different degrees of transparency, different degrees to which they are receptive to the presence and activity of the fiery substance. Thus the lesson of the second puzzle is satisfied. Moreover, jointly satisfying the lessons of our two puzzles in this way also provides an interpretation for why mixing bodies involves mixing color (\emph{De Sensu} \textsc{iii} 15--16).\index{De Sensu@\emph{De Sensu}}

While I believe that this is the best way to understand (or perhaps develop) Aristotle's account of the generation of the hues\index{color!generation of the hues} in terms of mixture\index{mixture}, a residual doubt remains. 

On the present development of Aristotle's model, chromatic composition is understood in terms of mixing ingredients whose elemental compositions determine their different degrees of transparency. When ingredients combine in a mixture, they alter one another. But what makes for chromatic composition is that the ingredients in the mixture are altering one another's degree of transparency, the degree to which they are receptive to the illuminating presence and activity of the fiery substance. But not every mixture of bodies results in mutually altering their degree of transparency in proportion with the corporeal mixture. There are certain colors that artists cannot achieve by mixing pigments\index{pigment} (\emph{Meteorologica} \textsc{iii} 2 372\( ^{a} \)2--9)\index{Meteorologica@\emph{Meteorologica}}. When the proportion\index{proportion} or ratio\index{ratio} of bodies mixed does not correspond to the degree to which the mixed bodies altered one another's degree of transparency this is due to a kind of material recalcitrance\index{material recalcitrance}. The possibility of material recalcitrance establishes that not every combining of bodies\index{body} in a mixture\index{mixture} is a combining of their color\index{color} in that mixture. Only certain bodily mixtures are color combinations, properly so called. Only bodily mixtures where the proportion or ratio of bodies mixed correspond to the degree to which they alter one another's degree of transparency\index{transparency!degrees of} are genuine methods of color combination.

The juxtaposition\index{juxtaposition model} and overlap models\index{overlap model} each provided explanations (ultimately unsatisfactory, at least by Aristotle's lights) for the ratio\index{ratio} of light\index{light} and dark\index{dark} that determines the resulting color\index{color!generation of the hues}. But the claim that a mixture\index{mixture} is only a method of color combination if the ingredients in the mixture mutually alter their degrees of transparency in proportion with the corporeal mixture comes perilously close to assuming what the other models explain. After all, to say that in the case of color combination ingredients mutually alter their degree of transparency, the degree to which they are receptive to the illuminating presence and activity of the fiery substance, is just to say that they determine a ratio of light and dark. But that just is the Pickwickian sense of mixture\index{mixture!Pickwickian sense}. So understood, the mixture of light and dark just is relative brightness. The remaining residual doubt consists in this apparent explanatory deficit. 


% subsection Two Puzzles about Mixture (end)

% section aristotle_s_three_models (end)

\section{Chromatic Ratios of Light and Dark} % (fold)
\label{sec:chromatic_ratios_of_light_and_dark}

While Aristotle sometimes departs from Empedoclean\index{Empedocles} doctrine, at others, he elaborates it. Specifically, Aristotle provides the beginnings of an account of the proportions\index{proportion} or ratios\index{ratio} of light\index{light} and dark\index{dark} that determine the chromatic hues\index{color!generation of the hues}.

The ordering of the intermediate colors between the extremes of light\index{light} and dark\index{dark} is not structured as a continuum\index{continuum}. This is a general thesis about sensible qualities\index{sensible qualities}. All sensible qualities are understood to be ordered in a range of contrary qualities between opposites\index{opposites}. And Aristotle maintains that there are only ever finitely many species from a given range (\emph{De Sensu} \textsc{vi} 445\( ^{b} \)21--446\( ^{a} \)19).\index{De Sensu@\emph{De Sensu}}

Specifically, there are seven species of color:
\begin{quote}
	Savours and colours contain respectively about the same number of species\index{species}. For there are seven species of each, if, as is reasonable, we regard grey\index{gray!as a variety of black} as a variety of black\index{black} (for the alternative is that yellow\index{yellow} should be classed with white\index{white}, as rich\index{rich} with sweet\index{sweet}); while crimson\index{crimson}, violet\index{violet}, leek-green\index{leek-green}, and deep blue\index{blue}, come between white and black, and from these all others are derived by mixture. (Aristotle, \emph{De Sensu} \textsc{iv} 442\( ^{a} \)20--442\( ^{a} \)25; Beare in \citealt[12]{Barnes:1984uq})\index{De Sensu@\emph{De Sensu}}
\end{quote}
Three of these are the colors of the rainbow\index{rainbow}---crimson\index{crimson}, violet\index{violet}, and leek-green\index{leek-green}, though some report yellow\index{yellow} coming between crimson\index{crimson} and leek-green\index{leek-green} (\emph{Meteorologica} \textsc{iii} 2 372\( ^{a} \)8--10)\index{Meteorologica@\emph{Meteorologica}}. In the passage, Aristotle regards gray\index{gray} as an achromatic hue\index{hue}, indeed it is a kind of black\index{black}, presumably because it is less bright\index{bright} than white\index{white} and so reckoned a privation the way black is. In other works, its achromatic character is overlooked, gray\index{gray} counting as an intermediary color just as much as yellow\index{yellow} (\emph{Categoriae}, \textsc{x} 12\( ^{a} \)18\index{Categoriae@\emph{Categoriae}}; see also \emph{Topica} \textsc{i} 15 106\( ^{b} \)5\index{Topica@\emph{Topica}}, \emph{Metaphysica} \textsc{i} 1056\( ^{a} \)27)\index{Metaphysica@\emph{Metaphysica}}.

So there are seven species\index{species} of color\index{color!species}---white\index{white}, yellow\index{yellow}, crimson\index{crimson}, violet\index{violet}, leek-green\index{leek-green}, deep blue\index{blue}, and black\index{black}. Corresponding to the five intermediate species are simple ratios\index{ratio} of light\index{light} and dark\index{dark}. So far we have, in line with ancient tradition, two primary colors, white and black, or better yet, light and dark. The two primary colors\index{color!primary} are opposites\index{opposites} at the extreme ends of an ordered range of colors. In addition to the two primary colors there are five intermediate colors that are the result of mixing the primary colors light and dark in simple ratios. Following \citet[297]{Sorabji:2022qf}\index{Sorabji, Richard} call these the secondary colors\index{color!secondary}. In addition to primary and secondary colors, Aristotle seems to recognize a third group of colors. Moreover, these seem to be the result of mixing secondary colors: ``crimson\index{crimson}, violet\index{violet}, leek-green\index{leek-green}, and deep blue\index{blue}, come between white\index{white} and black\index{black}, and from these all others are derived by mixture'' (\emph{De Sensu} \textsc{iv} 442\( ^{a} \)25; Beare in \citealt[12]{Barnes:1984uq})\index{De Sensu@\emph{De Sensu}}. So the tertiary colors\index{color!tertiary} are all the colors other than the primary and secondary colors and are the result of a mixture of secondary colors. 

The Aristotelian color scheme raises a number of questions.

First, if the mixture\index{mixture} of light\index{light} and dark\index{dark} in simple ratios\index{ratio!simple} determines the secondary color\index{color!secondary} whose mixture in turn determines the tertiary colors\index{color!tertiary}, is it possible to determine the tertiary colors directly in terms of non-simple ratios\index{ratio!non-simple} of light and dark? There should be no mathematical obstacle to this. Does Aristotle's failure to describe the direct determination of the tertiary colors by non-simple ratios of light and dark read in light of the claim that he does make, that the tertiary colors are a mixture of secondary colors, suggest, instead, that he believed that there was no such direct determination? Given the mathematical possibility, that the tertiary colors\index{color!tertiary} are not actually determined in this way could be evidence for a kind of material recalcitrance\index{material recalcitrance} at work in chromatic mixture.

Second, a question arises concerning the number of tertiary colors\index{color!tertiary}. From the fact that they correspond to non-simple ratios\index{ratio!non-simple} of light\index{light} and dark\index{dark} it does not follow that for every non-simple ratio of light and dark there exists a tertiary color. So we cannot assume that there are infinitely many. Just as there are finitely many primary colors\index{color!primary} and finitely many secondary colors\index{color!secondary}, perhaps there are finitely many tertiary colors. Aristotle in the passage, however, declines to directly say. This issue must be settled by appeal to more general considerations.

Third, in speaking of the secondary colors\index{color!secondary} as species\index{species}, is Aristotle allowing for the possibility that different colors might each be members of the same chromatic species the way different individual animals---a pair of hogs\index{hogs}, say---may each be members of the same species. This bears on how we are to think of the ordering of the colors. The intermediate secondary colors\index{color!secondary} are ordered between the extremes\index{opposites} of light\index{light} and dark\index{dark} in part because of the similarities and differences they bear to these opposites. However, there are similarities and differences between members of a species\index{species}. This hog has greater magnitude than that hog, and yet both protest when hungry\index{hogs}. By analogy, there would be similarities and differences between colors that are members of the same chromatic species\index{color!species}. Perhaps there are discriminable shades of crimson\index{crimson}. Each is recognizable as belonging to the relevant chromatic species, crimson, but being discriminable, there are visible differences between them. Indeed, Aristotle gives just such an example. Shades of black\index{black} and shades of gray\index{gray} are members of the same chromatic species\index{color!species}, they are each black, or better yet, dark\index{dark}, but there are visible differences between them.  If that is right, then the ordering of the intermediate secondary colors\index{color!secondary} between the extremes of light\index{light} and dark\index{dark}, while a similarity ordering determined by a decreasing proportion\index{proportion} of light, is not a complete similarity ordering. There will be color similarities and differences---the similarities and differences between members of the chromatic species---not captured by that ordering. In modern parlance, the incomplete ordering would be determining color determinables and not determinate colors. (On the determinable--determinate distinction see \citealt{Johnson:1921fk}. For a contemporary defense of the idea that we perceive determinable qualities see \citealt{Allen:2010ak,Hilbert:1987jq,Stazicker:2011an}.)

Fourth, and finally, that there are discriminable shades of color that belong to the same chromatic species bears on the question of cardinality. Is the question whether there are finitely many species of color?\index{color!species} Or is it rather whether there are finitely many discriminable shades of color? There could be finitely many chromatic kinds and infinitely many discriminable shades (especially if we bear in mind that a discriminable shade is only potentially discriminated, in some potential circumstance of perception, not necessarily the actual one).

To get a better sense of the mathematical content of Aristotle's theory, let us begin with the simple ratios\index{ratio!simple} in terms of which the five intermediate secondary colors\index{color!secondary} are understood. Aristotle first introduces this idea in the context of explaining the juxtaposition model\index{juxtaposition model}, an account of the generation of the hues\index{color!generation of the hues} that he rejects. In fact, Aristotle introduces two distinct ideas, only one of which remerges in his discussion of chromatic mixture\index{mixture}:
\begin{quote}
	Such, then, is a possible way of conceiving the existence of a plurality of colours besides the white\index{white} and black\index{black}; and we may suppose that many are the result of a ratio\index{ratio}; for they may be juxtaposed in the ratio of 3 to 2, or of 3 to 4, or in ratios expressible by other numbers; while some may be juxtaposed according to no numerically expressible ratio, but according to some incommensurable relation of excess or defect; and, accordingly, we may regard all these colours as analogous to concords, and suppose that those involving numerical ratios, like the concords in music, may be those generally regarded as most agreeable; as, for example, purple\index{purple}, crimson\index{crimson}, and some few such colours, their fewness being due to the same causes which render the concords\index{concord} few. The other compound colours may be those which are not based on numbers. Or it may be that, while all colours whatever are based on numbers, some are regular in this respect, others irregular; and that the latter, whenever they are not pure, owe this character to a corresponding impurity\index{color!impure} in their numerical ratios. This then is one way to explain the genesis of intermediate colours. (Aristotle, \emph{De Sensu} \textsc{iii} 439\( ^{b} \)26--440\( ^{a} \)6; Beare in \citealt[8]{Barnes:1984uq})
\end{quote}

The first idea then involves the application of musical intervals to colors. On this idea, the secondary colors are like consonances\index{consonance}. Consonant notes blend when played simultaneously and are pleasant to listen to. Just like concords\index{concord}, the secondary colors\index{color!secondary} are more agreeable than the tertiary colors\index{color!tertiary}. Consonant notes stand in certain intervals that can be represented by simple numerical ratios\index{ratio}. Given the empirical success of ancient theories of acoustics\index{acoustical theory} in developing a mathematical account of these intervals, the suggestion is that these theories, or theories of their type, may be extended to other sensible objects, in the present case, color\index{color}. So some colors are the result of simple numerical ratios\index{ratio!simple} that correspond to consonant\index{consonance} musical intervals, whereas other colors cannot be expressed as rational numbers. The analogy Aristotle draws between color theory and acoustical theory\index{acoustical theory} proved influential. Even Newton\index{Newton, Issac} accepted the analogy. Newton\index{Newton, Issac}, like Aristotle, believed that colors have consonances\index{consonance} just as pitches\index{pitch} do, and he divided the spectrum into seven divisions (excluding white\index{white} and black\index{black}, \emph{pace} Aristotle) corresponding to the seven notes of the just diatonic scale\index{diatonic scale} (see \citealt[619]{Shapiro:1994uq}).

The second idea is like the first, except that all colors are represented by rational numbers, each are determined by a numerical ratio\index{ratio} of light\index{light} and dark\index{dark}, it is just that some are regular\index{ratio!regular} and some are irregular\index{ratio!irregular}. The colors associated with regular ratios are pure\index{color!pure}, with irregular ratios\index{ratio!irregular} impure\index{color!impure}. \citet[155-156]{Ross:1906fk} makes the plausible suggestion that the impurity of the tertiary colors\index{color!tertiary} determined by irregular ratios\index{ratio!irregular} is a chromatic desaturation\index{color!desaturation}. The secondary colors\index{color!secondary} are more saturated\index{color!saturation} than the tertiary colors\index{color!tertiary} which are unsaturated to varying degrees. No doubt it is, in part, the increase in saturation that makes the secondary colors\index{color!secondary} more pleasant to behold and so chromatic consonances\index{consonance}.

This second idea is only entertained in the context of the juxtaposition model\index{juxtaposition model}. Aristotle probably does not accept it (though he plausibly retains the idea that impure\index{color!impure} tertiary colors\index{color!tertiary} are unsaturated\index{color!desaturation}). At any rate, when we come to Aristotle's account of the generation of hues\index{color!generation of hues} in terms of the mixture\index{mixture} of light\index{light} and dark\index{dark}, he endorses only the first idea:
\begin{quote}
	Colours will thus, too be many in number on account of the fact that the ingredients may be combined with one another in a multitude of ratios; some will be based on determinate numerical ratios, while others again will have as their basis a relation of quantitative excess. And all else that was said in reference to the colours, considered as juxtaposed or superposed, may be said of them likewise when regarded as mixed. (Aristotle, \emph{De Sensu} \textsc{iii} 440\( ^{b} \)18--23; Beare in \citealt[10]{Barnes:1984uq})
\end{quote}

This is less an account than the beginnings of one. Aristotle never specifies which intermediate secondary color\index{color!secondary} goes with which simple ratio\index{ratio!simple}, such as 3 to 2, or 3 to 4. Nor does he explore to what extent ancient acoustical theory may be extended to the colors in the way that he envisions (much to his detriment, see \citealt{Sorabji:2022qf}). Nor are we ever told how, exactly, the intermediate secondary colors\index{color!secondary} are ordered. Given its candidacy for being included in the extreme chromatic species\index{color!species}, white\index{white} or light\index{light}, yellow\index{yellow}, understood as a distinct intermediary, is the brightest of the secondary colors\index{color!secondary}. But the ordering of the rest of the intermediate secondary colors is left unspecified. This is less a theory, than a research program. Given the empirical success of ancient acoustical theory\index{acoustical theory}, the role of ratio\index{ratio} or harmony\index{harmony} in the respected opinions of the wise, and his own experience gleaned from dialectical engagement\index{dialectical argument} with the \emph{endoxa}\index{endoxa@\emph{endoxa}}, Aristotle most likely felt that there were good reasons to believe that this research program could in fact be carried out. But the \emph{De Sensu}\index{De Sensu@\emph{De Sensu}} account is not the result of that program but merely its statement.


Let us return to the question of cardinality. Aristotle claims that there are finitely many sensible qualities\index{sensible qualities} in an ordered range between opposites\index{opposites}. But in what sense does he intend this claim? Is it that there are only finitely many sensible species in the range? Or if distinct sensible qualities may be members of the same sensible species\index{color!species} (such as gray\index{gray} and black\index{black} seem to be), is Aristotle claiming, in addition, that sensible species only have finite members? To resolve these issues, we turn to an important if difficult passage from \emph{De Sensu} \textsc{vi}:\index{De Sensu@\emph{De Sensu}}
\begin{quote}
	For in all classes of things lying between extremes the intermediates must be limited. But contraries are extremes, and every object of sense-perception involves contrariety; e.g. in colour, white\index{white} and black\index{black}; in savour\index{savour}, sweet\index{sweet} and bitter\index{bitter}, and in all the other sensibles also the contraries\index{contraries} are extremes\index{opposites}. Now, that which is continuous\index{continuum} is divisible into an infinite number of unequal parts, but into a finite number of equal parts, while that which is not per se continuous is divisible into species which are finite in number. Since then, the several sensible qualities of things are to be reckoned as species, while continuity always subsists in these, we must take account of the difference between the potential\index{potential} and the actual\index{actual}. It is owing to this difference that we do not see its ten-thousandth part in a grain of millet, although sight\index{sight} has embraced the whole grain within its scope; and it is owing to this, too, that the sound contained in a quarter-tone escapes notice, and yet one hears the whole strain, inasmuch as it is a continuum\index{continuum}; but the interval between the extreme sounds\index{sound} escapes the ear. So, in the case of other objects of sense, extremely small constituents are unnoticed; because they are only potentially not actually visible, unless when they have been parted from the wholes. So the foot-length too exists potentially in the two-foot length, but actually only when it has been separated from the whole. But increments so small might well, if separated from their totals, be dissolved in their environments, like a drop of sapid moisture poured out into the sea\index{sea}. But even if this were not so still, since the increment of sense-perception is not perceptible in itself, nor capable of separate existence (since it exists only potentially in the more distinctly perceivable whole of sense-perception), so neither will it be possible to perceive its correlatively small object when separated in actuality. But yet this is to be considered as perceptible: for it is both potentially so already, and destined to be actually so when it has become part of an aggregate. Thus, therefore, we have shown that some magnitudes and their sensible qualities escape notice, and the reason why they do so, as well as the manner in which they are still perceptible or not perceptible in such cases. Accordingly then, when these are so great as to be perceptible actually, and not merely because they are in the whole, but even apart from it, it follows necessarily that their sensible qualities, whether colours or tastes or sounds, are limited in number. (Aristotle, \emph{De Sensu} \textsc{vi} 445\( ^{b} \)22-446\( ^{a} \)19; Beare in \citealt[18]{Barnes:1984uq})\index{De Sensu@\emph{De Sensu}}
\end{quote}

Some commentators see in this passage a contrast between the motion involved in a spatial magnitude\index{magnitude} changing position over time\index{locomotion} and the motion\index{motion} involved in qualitative alteration\index{alteration}. The motion of a body\index{body} through space is continuous\index{continuum}. The path it travelled from the starting point to its resting place is divisible into an infinite number of unequal parts but a finite number of equal parts. In contrast the motion\index{motion} of a body\index{body} through the process of qualitative alteration\index{alteration} is not continuous. The path it traveled through the quality space is not divisible into an infinite number of unequal parts. Indeed, as \citet[395-396]{Sherry:1986uq} observes, it is no path at all. If a body were to alter its color from the extremes of light\index{light} to dark\index{dark}, throughout that process it would take on the finitely many intermediate colors\index{color!species}. \citet[128--129]{White:2002kx}, however, fails to discern the intended contrast between spatial magnitudes and qualities in this passage. Rather, Aristotle is stressing their analogy.

Like Aristotle's resolution of Zeno's\index{Zeno} paradox, distinguishing between the actual\index{actual} and the potential\index{potential} is meant to resolve an \emph{aporia}\index{aporia@\emph{aporia}}: The several sensible qualities\index{sensible qualities} are reckoned to be finitely many species\index{color!species} and yet ``continuity always subsists in these''. The distinction between the actual and the potential is supposed to explain how this may be so. 

Begin with the case of spatial magnitude\index{magnitude}. When we actually divide the path that the body travelled, we only ever do so finitely many times. We only ever actually mark the path into finitely many stages. Moreover, when we actually divide a homeomerous\index{homeomerous} body\index{body} this only ever results in finitely many parts (think of a butcher dividing a carcass\index{carcass}). Corresponding to marking the path into stages, or carving the body into parts, there are different stages that a body undergoes in the process of qualitative alteration. Thus between the extremes of light\index{light} and dark\index{dark} the body must pass through intermediary stages. During that process it comes to have a color\index{color} intermediate between the opposite extremes\index{opposites}. Whereas, in the case of spatial magnitudes, the path was marked by some action of ours, our dividing it, in the case of chromatic qualities, the path through the quality space was marked by some action of ours, our discriminating that intermediary color from the lighter color that preceded it. While the path that the body travelled is only actually divided finitely many times, it remains true that it is infinitely divisible. The path is infinitely divisible in that there are within it an infinite number of unequal potential divisions. Similarly while the path that the body travelled through color space as it changed from light to dark is only actually divided finitely many times, by our finite color discriminations in the given circumstance of perception, it remains infinitely divisible. There are an infinite number of potential perceptual discriminations, though perhaps not all possible relative to the same point of view. On this understanding, perceptual discrimination imposes a discontinuous chunking on a sensible continuity. In making finitely many actual discriminations we reckon finitely many sensible species, but a continuity always subsists in these since there are infinitely many potential discriminations to be made, perhaps in other, more fortuitous, potential circumstances of perception.

\citet[]{Sorabji:1976fk}\index{Sorabji, Richard} offers a different interpretation. While the ordering from light\index{light} to dark\index{dark} is discontinuous, comprised as it is of finite species of color\index{color!species}, it is, nonetheless, derivatively continuous\index{derivatively continuous}. \citet[80]{Sorabji:1976fk} elaborates this idea by modifying a musical example of Aristotle's: ``What Aristotle seems to have in mind is that a change to the next discriminable pitch\index{pitch}, in the discontinuous series of discriminable pitches, may be produced by a continuous movement of a stopper along a vibrating string.'' The discontinuous series of qualitative states passed through in the process of qualitative alteration\index{alteration} are determined by a continuous change in its underlying material cause.\index{derivatively continuous} Thus the diminution of the activity of the fiery substance\index{fiery substance} reflected\index{reflection} from an opaque\index{opacity} surface\index{surface} of a darkening body may be continuous\index{continuum}, but the body undergoes a discrete series of qualitative transitions in the process. The problem is that Aristotle does not talk about continuous change in an underlying material cause for the motion of qualitative alteration. So it is difficult to understand the notion of derivative continuity as offering a resolution of the \emph{aporia}\index{aporia@\emph{aporia}} with which we began.

The discussion of the two paragraphs that preceded this passage supports, instead, the present interpretation (\emph{De Sensu} \textsc{vi} 445\( ^{b} \)7--20)\index{De Sensu@\emph{De Sensu}}. There are no imperceptible magnitudes\index{magnitude!must be perceptible}. That was the general principle that ruled out the spatial and temporal variants of the juxtaposition model\index{juxtaposition model} (chapter~\ref{sub:juxtaposition}). And that every magnitude is perceptible, at least in some potential circumstance, is linked with an infinite number of potential perceptual discriminations (\emph{De Sensu} \textsc{vi} 445\( ^{b} \)7--10). Moreover, having established that link, Aristotle goes on to argue directly against the possibility of imperceptible magnitudes, as if to reinforce the claim that there an infinite number of perceptual discriminations, the source of continuity that subsists in sensibilia. The sensible is not derivatively continuous\index{derivatively continuous}, in the sense that motion\index{motion} of qualitative alteration\index{alteration} has an underlying continuous material cause. Rather the sensible is continuous in the sense that there are infinitely many potential discriminations to be made. And this is consistent with our only ever actually making finitely many perceptual discriminations that determine finitely many sensible species\index{color!species}.

Let us consider just one of Aristotle's arguments here, because it is interesting in its own right, and revealing about the metaphysics of color:
\begin{quote}
	Since if it were not so, we might conceive a body existing but having no colour, or weight, or any such quality; accordingly not perceptible at all. For these quantities are the objects of sense-perception. On this supposition, every perceptible object should be regarded as composed of non-perceptible parts. Yet it must be really composed of perceptible parts, since assuredly it does not consist of mathematical qualities. (Aristotle, \emph{De Sensu} \textsc{vi} 445\( ^{b} \)11--15; Beare in \citealt[18]{Barnes:1984uq})\index{De Sensu@\emph{De Sensu}}
\end{quote}
The most likely target here is Plato's\index{Plato} cosmology in the \emph{Timaeus}\index{Timaeus@\emph{Timaeus}}. Plato describes how geometrical particles, imperceptible magnitudes, give rise to perceptible elements which in turn constitutes the whole world of sense (\emph{Timaeus} 53c-61c). Plato thus claims to do what Parmenides\index{Parmenides} claimed could not be done. In what is perhaps the first instance of the explanatory gap\index{explanatory gap}, Parmenides\index{Parmenides} complains of his predecessors that they infer sensible body from extension. But there is no valid inference from intelligible form to sensible form. Becoming pertains to the sensible, and the one being of the Way of Truth has only unchanging intelligible form (see \citealt[49]{Guthrie:1965ys}). Aristotle is echoing this Parmenidean complaint. But instead of eliminating the sensible from the true account of the world as the unnamed goddess\index{unnamed goddess} recommends, Aristotle retains the sensible and eliminates explanatorily otiose imperceptible magnitudes. It is noteworthy, in this regard, that the primary opposites---Hot\index{Hot} and Cold\index{Cold}, Dry\index{Dry} and Wet\index{Wet}---are sensible qualities\index{sensible qualities}. Instead of positing fundamental explanatory properties that would pose an explanatory gap problem for the sensible world, Aristotle seeks within the manifest image of nature for the explanatory resources that he needs. 

This historical debate bears on how the metaphysics of color should be conceived and conducted. In the terms of the debate between Parmenides\index{Parmenides}, Plato\index{Plato}, and Aristotle, the metaphysics of color\index{color} is a special case of the problem of the manifest\index{manifest image}, as it was for Empedocles\index{Empedocles} as well. The task is to determine whether chromatic form may intelligibly be instantiated, as our experience of nature presents it to be, consistent with our best scientific understanding of nature. The \emph{De Sensu}\index{De Sensu@\emph{De Sensu}} doctrines that every body participates in the transparent to some degree\index{color!residing in the proportion of the transparent} and that this determines a ratio\index{ratio} of light\index{light} and dark\index{dark} that in turn determines that body's chromatic hue\index{hue}, is an attempt, on Aristotle's part, to resolve the problem of the manifest\index{manifest image} as it arises from our experience of the colors of remote external particulars (see chapter~\ref{sec:color_and_transparency}).


% section chromatic_ratios_of_light_and_dark (end)

\section{Assessment} % (fold)
\label{sec:assessment}

What are we to make of this ancient tradition, elaborated by Aristotle, that understands the chromatic hues in terms of a combination of light\index{light} and dark\index{dark}?\index{color!generation of the hues}

Any assessment does well to distinguish the separable strands of thought to be found in this tradition. Let me here focus exclusively on two:
\begin{enumerate}[(1)]
	\item that lightness and darkness constitute a dimension of similarity along which all chromatic hues are aligned; and\index{brightness!as dimension of color similarity}
	\item that lightness and darkness are explanatory determinants of the chromatic hues.\index{brightness!as amount of light}
\end{enumerate}
Whereas the first strand cannot survive given what we now know about colors and their perception, arguably at least, the second strand survives, \emph{inter alia}, in modern reflectance theories\index{color!reflectance theory}.

According to the ancient tradition, colors are subject to a unidimensional similarity ordering that includes white and black as the extreme endpoints.\index{color!similarity} The first problem arises from the fact that a complete color similarity ordering must be multidimensional. For a color similarity ordering to be complete is for every color to have a unique position in that ordering. For a color similarity ordering to be multidimensional is for there to be a plurality of dimensions of similarity in that ordering. Indeed, there are a plurality of dimensions along which the different colors are more or less similar. There are different models of the multidimensional color space that are responsive to different practical needs. Most philosophers are familiar with thinking of color similarity in terms of a three-dimensional space---a nineteenth century innovation---one dimension each for hue\index{hue}, saturation\index{saturation}, and brightness\index{brightness}. The incompleteness of the three-dimensional color space can be directly observed, however. Where on the three-dimensional color space is metallic green? More advanced models provided by colorimetrists typically posit more than three dimensions. It is an open empirical question just how many dimensions there are to color similarity (and one that may lack content independently of the practical need underlying the given metric). However, the incompleteness of the unidimensional model posited by the ancients is easily established: There are distinct hues of equal brightness\index{brightness}. An ordering of the colors by brightness is incomplete. It is not the case that colors can be identified by their place in the ordering from light to dark. Colors of distinct hues may be equally bright. A complete color similarity ordering is multidimensional. Recall, Aristotle is himself committed to denying that the similarity ordering from light\index{light} to dark\index{dark} is complete. Black\index{black} and gray\index{gray} are each dark\index{dark}, they are each members of the same chromatic species\index{color!species} determined by a simple ratio\index{ratio!simple} of light\index{light} and dark\index{dark}. But black\index{black} and gray\index{gray} chromatically differ. So the ordering in terms of which black and gray both count as species of dark is incomplete. Nevertheless, Aristotle never considers let alone acknowledges that a complete color similarity ordering would be multidimensional.

The second problem concerns how the endpoints of the ordering are conceived.\index{color!similarity} The ancient tradition conceives of these endpoints as included in the similarity ordering. However, the endpoints of the brightness\index{brightness} dimension are not included in the ordering so much as they are a limit to which colors in that ordering may approach. There is no black\index{black} darker than any other black, nor any white\index{white} lighter than any other white. There is rather an approach to a limit. By including the endpoints in the similarity ordering, the ancient tradition obscures this.

The third problem is one of omission. Aristotle and the ancient tradition fail to properly distinguish between chromatic brightness understood as a dimension of color similarity\index{brightness!as dimension of color similarity} and brightness understood as an increase in the amount of light\index{brightness!as amount of light} (for a contemporary discussion of the psychology of seeing black and white and light and dark see \citealt{Gilchrist:2006kx}\index{Gilchrist, Alan}). The distinction is difficult to directly observe without some means of measuring light. It is natural to think that an increase in the amount of light\index{light} reflected from a surface\index{surface} would produce a chromatically brighter\index{bright} appearance. Unfortunately these distinct senses of brightness can come apart. Consider viewing a page of black\index{black} print on white paper, first indoors under artificial illumination, then outside in natural daylight. The intensity of the light\index{light} reflected by the white\index{white} area of the page indoors is approximately the same as the intensity of the light reflected by the black print in sunlight \citep[199]{Peter-K:1996th}. Despite being equally bright\index{brightness!as amount of light}, in the sense that the reflected light is equally intense, the apparent colors differ in chromatic brightness\index{brightness!as dimension of color similarity}---the surrounding white\index{white} when viewed indoors seems brighter than the black\index{black} print when viewed in daylight. So not only are these distinctions conceptually distinct (one speaks of color the other speaks of light), but they are extensionally distinct as well (surfaces can reflect the same amount of light and yet differ in chromatic brightness). This omission is related to a problem raised earlier in chapter~\ref{sec:color_and_transparency}. Recall, an opaque\index{opacity} body\index{body} with a surface color\index{color!surface} and an imperfectly transparent\index{transparency} medium\index{medium} with a volume color may both share the same hue. But if the amount of light\index{light} both determines the hue\index{hue} and the degree of transparency how could this be? Both enjoy the same hue and so both must participate the same ratio\index{ratio} of light\index{light} and dark\index{dark}. But the threshold\index{threshold}, if there is one, that makes for transparency\index{transparency} is met in the case of the imperfectly transparent medium but is not met in the case of the opaque body. The problem arises from the theoretical uses to which the fiery substance is being put, as jointly determining the hue\index{hue} of a body and its degree of transparency\index{transparency!degrees of}. The problem thus arises by not recognizing that brightness understood as a dimension of color similarity\index{brightness!as dimension of color similarity} and brightness understood as the amount of light can come apart.\index{brightness!as amount of light}

A final problem concerns a commitment specific to Aristotle. The juxtaposition\index{juxtaposition model}, overlap\index{overlap model}, and mixture models\index{mixture model} are meant to be formally equivalent in the sense that the same proportions\index{proportion} of light\index{light} and dark\index{dark} are determined, at least in principle. The discovery by \citet{Helmholtz:1852aa}\index{Helmholz, Hermann von} of subtractive\index{subtractive color mixing process} and additive color mixing processes\index{additive color mixing processes} provides the means of establishing their nonequivalence. Thus color mixing that results from mixing pigments is a subtractive process. As is the color mixing that results from overlaying filters. However, color mixing that is the result of juxtaposition, such as the pointillist technique\index{pointillism} involved in Seurat's\index{Seurat, Georges-Pierre} masterpiece, is an additive process. And what Helmholtz\index{Helmholz, Hermann von} showed was that not every color determined by an additive process can be matched with a color determined by a subtractive process. So there are colors determined by juxtaposition that could not be determined by overlap or mixture. The three models are not formally equivalent the way that Aristotle supposed.

I suspect that skepticism about this tradition, when not due to bad ethnolinguistics, is due in large part to an intuitive recognition of its empirical inadequacy as an account of color similarity\index{color!similarity}. Given what we know about the colors, it strikes us as manifestly false. While its central claims about color similarity are known to be false, this is insufficient grounds for a dismissive attitude towards the ancients. After all, our knowledge about the colors and their perception is a significant historical achievement as yet unavailable to them. Thus \citet[291]{Broackes:2010uq} writes that ``It is a topic on which psychologists, physicists, biologists, and neurophysiologists--not to mention paint manufacturers, dyers, and makers of photographic equipment---have reason to be proud and glad of the convergence of interests and views.'' If the mistakes of the ancients seem obvious to us, it is not, \emph{pace} \citet[162]{Platnauer:1921bh}, because we are more attentive than they were to ``the qualitative differences of decomposed and partially absorbed light''.\index{Broackes, Justin} Rather, we are the beneficiaries of chromatic knowledge hard won by a variety of interested parties over the centuries. Moreover, this hard won chromatic knowledge includes knowledge of the phenomenological character of color perception. Phenomenology is something about which discoveries can be made. Thus opponent processing theory makes a number of important phenomenological predictions that have been verified by psychophysical experimentation. But the phenomenological commitments of opponent processing theory, if indeed true, are not obvious merely upon reflection on the character of color experience. Otherwise Ewald Herring's\index{Herring, Ewald} claim that there are four basic colors would have been obvious upon reflection, and Hermann von Helmholtz\index{Helmholz, Hermann von}, James Clerk Maxwell\index{Maxwell, James Clerk}, and Thomas Young\index{Young, Thomas} who maintained that there were three instead would have been culpably inattentive to their own color experience. As a matter of fact, Herring's basic phenomenological claims about color vision had to wait for Hurvich\index{Hurvich, Leo M.} and Jameson's\index{Jameson, D.} research for empirical support. Mistaken beliefs about phenomenology\index{phenomenology}, even about the phenomenologically vivid facts of color similarity as experience presents it to be, require neither that the subject be insensitive to color similarity nor that they be inattentive to that sensitivity manifest in their color experience. (On the importance of psychophysics to phenomenological investigation see \citealt{Hilbert:2007qy,Phillips:2012af})\index{phenomenology!and psychophysics}

Fortunately, the ancient tradition does not consist entirely of claims about color similarity. At the heart of the ancient view is a claim about the explanatory priority of Cosmic Fire\index{Cosmic Fire}. According to the cosmology of the way of Mortal Opinion, Fire\index{Parmenides!Fire} and Night\index{Parmenides!Night} are fundamental and irreducible principles standing in opposition. Fire is partly manifest in the sensible quality of brightness\index{bright}---it has other ``signs''\index{Parmenides!signs} or attributes as well. Fire is a cosmic principle that is independent and explanatory of instances of its attributes. Thus the brightness of a particular is due to the presence and activity of the cosmic principle of Fire. Importantly, this is an explanatory claim about the determinants of brightness and not a claim about color similarity. 

The doctrines of Empedocles\index{Empedocles} and Aristotle echo, in their own way, the explanatory priority of Cosmic Fire\index{Cosmic Fire}. According to Empedocles, the brightness\index{brightness} of bone\index{bone} is due to the preponderance of fire\index{fire} in its elemental\index{elements} composition (\textsc{dk} 31\textsc{b}96). And the blackness of the river's\index{river} depths is due to the presence of water (\textsc{dk} 31\textsc{b}94). It is because of the preponderance of fire in the bone's composition that it gives off fiery effluences\index{Empedocles!effluence!fire}, and it is because of the preponderance of water in the river's depths, that it gives off watery effluences\index{Empedocles!effluence!water}. Elemental composition, for Empedocles, is a metaphysical determinant of the colors.

The presence and activity of the fiery substance\index{fiery substance} makes a potentially transparent medium actually transparent. Insofar as actual transparency is materially sufficient for the visibility of colored particulars, this merely concerns the perceptual availability of the colors and not their constitution. However, fire\index{fire} plays a role in the determination of the colors as well. The presence and activity of the fiery substance\index{fiery substance} explains the brightness of a transparent medium and brightness is of the nature of color (\emph{De Sensu} \textsc{iii} 439\( ^{b} \)1)\index{De Sensu@\emph{De Sensu}}. Brightness may be a color, but it is not the only color that fire helps to determine. One important contribution of Empedocles\index{Empedocles} was to make explicit what was merely implicit in the Parmenidean\index{Parmenides} fragments, at least as they have come down to us---that the chromatic hues are due to the ratio of light and dark. This is manifest in Aristotle's account of the generation of the hues\index{color!generation of the hues}---that it is the ratio\index{ratio} of light\index{light} and dark\index{dark} in a mixture\index{mixture} that determines the chromatic hue\index{hue} of a particular\index{particular}. Since light or brightness is determined by fire, then the presence of fire partly determines the ratio of light and dark in a mixture that itself determines the chromatic hue of the particular.

Aristotle's definition of color\index{color!definition}\index{color!power to affect light} as the power to move what is actually transparent can be seen as an ancient prefiguration of modern reflectance theories\index{color!reflectance theory}. Thus \citet[23]{Ross:1906fk} writes ``It is noteworthy that if one were to define black and white in the modern way as the capacity of a surface to reflect none or all of the light cast upon it, one could still describe the chromatic tints as intermediate between these, as diverse aptitudes for reflecting one portion and absorbing the rest of the total light.''\index{Ross, G.R.T.} Consider the simplest form of the reflectance theory\index{color!reflectance theory} according to which a surface color\index{color!surface} is the disposition to reflect\index{reflection} a certain percentage of light\index{light} in each of the wavelengths of the visible spectrum \citep[see][]{Hilbert:1987jq}. This account can be extended to volume color\index{color!volume} and radiant color\index{color!radiant} in the obvious ways. Thus volume color\index{color!volume} would be the disposition of a volume to transmit a certain amount of light in each of the wavelengths of the visible spectrum, and a radiant color\index{color!radiant} would be the disposition of a light source to emit a certain amount of light in each of the wavelengths of the visible spectrum (though see \citealt{Byrne:2003we} for an argument that the transmission and emission of light should be understood as belonging to the unified class of productances\index{Byrne, Alex}\index{Hilbert, David R.}\index{productance}). Color, so understood, would be the power to affect light in a certain way. Only two claims separate the reflectance theory from Aristotle's definition. The first claim is a specification. In maintaining that surface color is a disposition to reflect, transmit, or emit a certain amount of light, modern reflectance theories specifies the way in which color affects light---by reflection\index{reflection}, transmission\index{transmission}, or emission\index{emission}. Moreover, the amount of light reflected, transmitted, or emitted just is the mixture of light and dark, in the Pickwickian sense of mixture\index{mixture!Pickwickian sense} to which Aristotle is committed. The second claim is that there are different kinds of light. \citet{Newton:1704qv} recognized the necessity of distinguishing different kinds of light in the way that Aristotle did not. Though \citet{Goethe:1810uq}\index{Goethe, Johan Wolfgang von} saw matters differently, the intrusion of the Newtonian\index{Newton, Issac} distinction between kinds of light, while anachronistic, is nevertheless a consistent extension of Aristotle's metaphysics. Nothing Aristotle says is inconsistent with their being different kinds of light. A latter day Aristotelian could very well accept that there are consistent with their Aristotelianism. They might even retain the claim that the different kinds of light are both visible and structured as a continuum, so long as they were careful to attend to the actual and the potential and their various senses. Thus whereas the first claim---that color reflects, transmits, or emits a certain amount of light---was a specification of Aristotle's more general claim that color is power to affect light, the second claim---that there are different kinds of light that may be reflected, transmitted, or emitted---is a consistent extension of Aristotle's account. Conjoining the specification of Aristotle's definition with the Newtonian extension just is a statement of the simple reflectance theory with which we began. That colors are ways of affecting light\index{color!power to affect light} is an important genus in the metaphysics of color, one that Aristotle's definition of color belongs to, and it is arguable that Aristotle inaugurates this metaphysical tradition.

Newton's\index{Newton, Issac} disagreement with Aristotle has less to do with there being different kinds of light, or even that these kinds of light constitute a continuum\index{continuum} subject to a finite subdivision into seven chromatic consonances\index{consonance}, but with a fundamental claim of the ancient tradition, that white\index{white} and black\index{black}, or rather, light\index{light} and dark\index{dark} are the primary colors\index{color!primary} in terms of which all other colors are ultimately to be explained. White\index{white} or light\index{light}, instead of being mixed in various proportions to produce the chromatic hues, is now the mixture of all the spectral colors. Newton\index{Newton, Issac!the most Paradoxicall of all my assertions} describes this as ``the most Paradoxicall of all my assertions, \& met with the most universall \& obstinate Prejudice'' (Newton to Oldenburg, 7 December 1675, in \citealt[385]{Turnbull:1959kx}). This paradoxical assertion is nothing less than the demolition of the explanatory framework of the ancient tradition, and it is this which most likely drew Goethe's\index{Goethe, Johan Wolfgang von} ire. Modern reflectance theories\index{color!reflectance theory} reject the framework within which Newton's paradoxical assertion was made, that determinate colors are associated with kinds of light in the spectral continuum \citep{Hilbert:1987jq}\index{Hilbert, David R.}. Indeed, reflectance theories were motivated, in part, by the criticism of Edwin Land\index{Land, Edwin}, the inventor of the Polaroid\index{Polaroid}, of Newton's theory (\citealt{Land:1971aa}, \citealt{Land:1977aa}).

Color is the power to affect light\index{color!power to affect light}, understood as a state of illumination sustained by the contingent presence and activity of the fiery substance\index{fiery substance}. Color affects light by affecting the extent and degree of the activity of the fiery substance, and in cases of reflection\index{reflection} and diffraction\index{diffraction}, its direction of influence as well. From a modern perspective, one problem for Aristotle's version of the explanatory priority of Cosmic Fire\index{Cosmic Fire} is the resulting conception of the fiery substance\index{fiery substance}. Being fire\index{fire}, its being consists in its burning. But since it pervades bodies of air\index{air} and water\index{water}, it must be incorporeal. But an \index{fiery substance!incorporeal activity}incorporeal activity that can instantaneously pervade a potentially transparent region insofar as it is a unity can strike us as odd or strange. Fortunately, it was but one small step towards a better conception. Philoponus'\index{Philoponus, John} elaboration of Aristotle's account, with the aid of concepts from his novel dynamics, is a step towards the wave conception of light (see \citealt{Wolff:1987vn}).

Aristotle's account of the generation of the hues\index{color!generation of the hues} is the culmination of an ancient tradition that understands the chromatic hues in terms of a combination of light\index{light} and dark\index{dark}, an ancient tradition that has few adherents among moderns, Goethe notwithstanding. The largest obstacle moderns face in appreciating this ancient tradition, when not blindsided by bad ethnolinguistics, is incredulity at associated claims about color similarity\index{color!similarity}. If this were all there were to this tradition, it would indeed be of antiquarian interest only. But this is not the case. Importantly, and more centrally, the ancient tradition makes claims about the explanatory determinants of the colors. At the heart of this ancient tradition is a claim about the explanatory priority of Cosmic Fire\index{Cosmic Fire}. And it is this which is of lasting significance. 

% section assessment (end)

% chapter the_generation_of_the_hues (end)