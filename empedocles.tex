%!TEX root = /Users/markelikalderon/Documents/Git/formwithoutmatter/formwithoutmatter.tex
\chapter{Empedocles} % (fold)
\label{cha:empedocles}

\section{Dialectic} % (fold)
\label{sec:dialectic}

% section dialectic (end)

\section{The Answer in the Style of Gorgias} % (fold)
\label{sec:the_answer_in_the_style_of_gorgias}

In the \emph{Meno} Socrates attributes to Empedocles a conception of perception as a mode of assimilation of material effluences:
\begin{quotation}
    \textsc{meno}: And how do you define color?
    
    \ldots
    
    \textsc{socrates}: Would you like an answer in the style Gorgias, such as you most readily follow?
    
    \textsc{meno}: Of course I should.
    
    \textsc{socrates}: You and he believe in Empedocles' theory of effluences, do you not?
    
    \textsc{meno}: Wholeheartedly.
    
    \textsc{socrates}: And passages in which and through which the effluences make their way?
    
    \textsc{meno}: Yes.
    
    \textsc{socrates}: Some of the effluences fit into some of the passages whereas others are too great or too small.
    
    \textsc{meno}: That is right.
    
    \textsc{socrates}: Now you recognize the term `sight'?
    
    \textsc{meno}: Yes.
    
    \textsc{socrates}: From these notions, then, `grasp what I would tell,' as Pindar says. Color is an effluence from shapes commensurate with sight and perceptible to it. (\emph{Meno} 76\( ^{a-d} \))
\end{quotation}

The main elements of the account are relatively clear. Objects emit material effluences. Effluences are fine bodies that are kind differentiated in terms of magnitude. There are passages in which and through which material effluences may flow. Whether a material effluence may enter a passage depends on its magnitude. The magnitudes of some kinds of material effluences are too great or too small for them to flow through a given passage. Such passages exist in the membrane of the eye, thus allowing the eye to assimilate only a certain kind of material effluence, that is, the kind whose magnitude permits entry in ocular passages.

Thus we arrive at the answer in the style of Gorgias. That answer has three components. It specifies a kind of thing and two conditions that must be satisfied for a thing of that kind to be color. Color is (1) a kind of material effluence that is (2) commensurate with sight and (3) perceptible. First, color is a kind of material effluence, a chromatic effluence, say. Since material effluences are kind differentiated by magnitude, chromatic effluences have a distinctive magnitude. Second, chromatic effluences are commensurate with sight insofar as their distinctive magnitude permits entry in the passages of the membrane of the eye, the organ of sight. Notice, however, the assimilation of chromatic effluences by the organ of sight is not, by itself, the sensing of colors, otherwise the final condition would be redundant. The assimilation of chromatic effluence is at best a material precondition for their sensing. The thought seems to be this: In order for the chromatic effluences to be the object of sense, they first must be assimilated by the organ of sensation. It is only by assimilating chromatic effluences that they are presented to sight and are thereby seen. Socrates claims that the answer in the style of Gorgias may be generalized to the other sensory objects such as sound and smell (\emph{Meno} 76\( ^{d} \)), a claim echoed by Theophrastus' account of Empedocles (\emph{De Sensibus} \textsc{vii}). If that is right, then Empedocles, at least as presented by Socrates, is in the grip of a general conception of sensory awareness for which ingestion provides the model. Compare---in eating an olive, the matter of the olive is taken in and presented to the organ of taste and thereby tasted. On the ingestion model, to be perceptible is to be palpable to sense. 

The underlying thought is that in order for something to be the object of sense, it must be presented to the sense organ. On the ingestion model, it is a general feature that taste shares with paradigmatic cases of touch that is operative---the object of sensation must be in contact with the sense organ for it to be sensed. I say \emph{paradigmatic} cases of touch since it is arguable, at least, that one can feel something that one is not in direct contact with. Thus one might feel the wooden frame of a Victorian rocking horse through the padding. Consider Aristotle's embarrassment at the thought that one cannot touch what is wet (\emph{De Anima} \textsc{ii}.11 423\( ^{a} \)21--423\( ^{b} \)16) or Dennett's \citeyearpar{Dennett:1993ce} example of feeling a surface by tapping a pen. Theophrastus' commentary supports the suggestion that, according to Empedocles, an object must be in contact with the sense organ for it to be sensed. Following Aristotle's doxographic taxonomy (\emph{De Anima} \textsc{i}.2 404\( ^{b} \)12--15; cf \emph{Metaphysics} \textsc{iii}.4 1000\( ^{b} \)5--8), Theophrastus counts Empedocles as a likeness theorist---as explaining perception in terms of the similarity of the elements that compose the object of sense and the sense organ. However, in \emph{De Sensibus} \textsc{xv}, Theophrastus concedes that Empedocles remains silent on the compositional similarity of the material effluence and the sense organ that assimilates it, emphasizing instead the role of \emph{contact} in perception \citep{Kamtekar:2009fk,Sedley:1992uq}:
\begin{quote}
	For he attributes our recognition of things to two factors---namely, to likeness and to contact; and so he uses the expression ``to fit''. Accordingly if the smaller touched the larger ones, there would be perception. And likeness also, speaking generally, is out of the question, at least according to him, and commensurateness alone suffices. For he says that substances fail to perceive one another because their passages are not commensurate. But whether the effluence is like or unlike he leaves quite undetermined. (\emph{De Sensibus} 15; \citealt{Stratton:1917vn})
\end{quote}

To be perceptible is to be palpable to sense. If one began with that thought, a puzzle would naturally arise about vision, for vision seems to present the colors of distant objects. Color perception seems to involve the presentation of color qualities inhering in bounded particulars located at a distance from the perceiver. But how can one assimilate what remains inherent in a bounded particular remote from one? The puzzlement arise from the apparent tension between two claims:
\begin{enumerate}[(1)]
    \item The objects of color perception are qualities of external particulars located at a distance from the perceiver.
    \item \emph{The Empedoclean principle}: To be perceptible is to be palpable to sense---in order for something to be the object of perception it must be in contact with the relevant sense organ.
\end{enumerate}

I conjecture that, whatever independent reasons Empedocles may have had for believing in material effluences, it is precisely this puzzlement that effluences are meant to address in his theory of vision. The basic idea is simple enough, at least in broad outline. Distant objects may be sensed by sensing the material effluences they emit. If the color of an object is the material effluence that it emits, then the color of a remote object can be assimilated and so be palpable to sight. In this way, we can see the color of a bounded particular remote from us  consistent with the constraints imposed by the ingestion model. One may wonder whether the theory of effluences is wholly adequate to this task, at least without supplementation. Thus a Berkelean worry naturally arises about the immediate objects of sensation, the assimilated effluences, screening off the external objects that emit them. Moreover, it is not just colored objects that appear at a distance, but the colors themselves seem confined to the remote bounded region in which they inhere. Fortunately, it is the puzzlement that arises from Empedocles' conception of sensory presentation, and not his resolution of it, that is our focus here. 

% section the_answer_in_the_style_of_gorgias (end)

\section{Empedocles' Theory of Vision} % (fold)
\label{sec:empedocles_theory_of_vision}

Empedocles' own theory of color vision is more elaborate than the answer in the style of Gorgias. Despite being more elaborate, the ingestion model remains at the core of that theory. One element missing from the account that Socrates offers to Meno is the eye's emission of fiery effluences. That effluences from the interior of the eye may pass through the eye's membrane is at least consistent with if not suggested by Socrates' general description of passages ``in which and through which'' effluences may flow. This is worth considering, since it can seem to offer an explanation that conflicts with the explanation of color vision in terms of the assimilation of material effluences. And while Aristotle notes the apparent tension, Theophrastus sees clearly the role the emission of fiery effluences plays in the assimilation of chromatic effluences.

Aristotle cites the following passage from Empedocles:
\begin{quote}
	As when a man, thinking to sally forth, furnishes him with a lamp, a glow of fire blazing through the stormy night, to protect it against all winds fits thereto screens, which scatter the breath of the winds as they blow; and leaping forth, because it is more tenuous, the light shines over his threshold with tireless rays, so did Love surround the web-enclosed primeval fire, even the round pupil, with fine membranes; and these shut out the depth of the surrounding water, but let the fire pass through because it was more tenuous. (\emph{De Sensu} \textsc{ii} 437\( ^{b} \)27--438\( ^{a} \)3; Diels 84)
\end{quote}
On this basis, Aristotle maintains that, like Plato in the \emph{Timaeus} (at least on his physicalist interpretation of that dialogue), Empedocles explains vision in terms of the emission of fire. But then he remarks ``At times, then, he explains vision in this way, but at other times he accounts for it by effluences from objects seen'' (\emph{Se Sensu} \textsc{ii} 438\( ^{a} \)4--5). So, at the very least, Aristotle thinks that Empedocles is potentially offering distinct explanations of color vision---one in terms of the eye's emission of a fiery effluence and the other in terms of the eye's assimilation of material effluences. But if Empedocles explains vision by the assimilation of material effluence, why does he also need the emission of a fiery effluence? Is Empedocles really offering conflicting explanations of vision? 

In the passage cited by Aristotle, Empedocles describes the anatomy of the eye by analogy with an artifact, a screened lamp, constructed from shaved horn, say. Just as there is fire in the interior of a screened lamp, there is a primeval fire in the interior of the eye, or perhaps the pupil. And just as the screen of shaved horn surrounds the lamp, there is a membrane that surrounds the eye. Moreover, the membrane plays a similar role to the screen. Just as the screen protects the interior fire from the wind which would extinguish it, the primeval fire is protected from the surrounding water by the membrane of the eye. What is the surrounding water? If the membrane surrounds the eye, then the surrounding water is the lachrymal fluid on the surface of the eye \citep{Sedley:1992uq}. And, finally, just as light passes through the screen, the primeval fire can pass through passages in the membrane of the eye.

Why must fiery effluences be emitted if the eye is to assimilate chromatic effluences? Theophrastus offers the basis of an explanation:
\begin{quote}
	The passages <of the eye> are arranged alternately of fire and of water: by the passages of fire we perceive white objects; by those of water, things black; for in each of these case <the objects> fit into the given <passage> Colors are brought to our sight by an effluence. (\emph{De Sensibus} 7; \citealt{Stratton:1917vn})
\end{quote}
The membrane prevents the surrounding water from getting into the eye and extinguishing the primeval fire. But while water cannot get into the eye, fire can get out. Thus the alternating arrangement of fire and water must take place on the surface of the eye. That is why the fire must first pass out of the eye to mix with water on its outer surface \citep[25]{Sedley:1992uq}.

Theophrastus' insight is that the emission of fiery effluences is not an alternative account of color vision as Aristotle supposed, but a necessary precondition for the assimilation of chromatic effluences. So even on Empedocles' more elaborate theory of color vision, a general conception of sensory awareness as a mode of assimilation remains at its core.

% section empedocles_theory_of_vision (end)

\section{Empedoclean Puzzlement} % (fold)
\label{sec:empedoclean_puzzlement}

I believe that Empedocles' puzzlement about the sensory presentation of the remote objects of sight is a natural one. The puzzlement survives abandoning what surely is the immediate culprit in the ingestion model, the principle that to be perceptible is to be palpable to sense (\emph{De Sensu} 440\( ^{a} \)7--19). Indeed, in its most general form the puzzlement persists to this day. Thus Broad remarks that:
\begin{quote}
    It is a natural, if paradoxical, way of speaking to say that seeing seems to `bring us into \emph{contact} with \emph{remote} objects' and to reveal their shapes and colors. \citep[33]{Broad:1952kx}
\end{quote}
In its most general form, the puzzlement is one aspect of the problem discussed by contemporary philosophers under the rubric of presence in absence. The puzzlement consists in an inability to understand how to coherently combine the distal character of the objects of sight with a conception of sensory awareness as a mode of assimilation. It would be premature to dismiss that conception, even in its original Empedoclean form, as primitive physiology of vision. Indeed, Aristotle retains the conception of perception as a mode of assimilation even as he transforms it in rejecting Empedocles' theory of effluences. Aristotle retains that conception presumably because he felt that there was an insight that should be preserved in Empedocles' opinion. And Aristotle is not alone in thinking that there is an insight to be preserved in conceiving of sensory awareness as a mode of assimilation. Thus \citet{Broad:1952kx} speaks of the presentation of the objects of sensory awareness as a mode of prehension.  ``Prehension'' belongs to a primordial family of broadly tactile metaphors for sensory awareness that includes ``grasping'', and ``apprehending''. What unites these metaphors is that they are all a mode of assimilation, and ``ingestion'' is a natural variant \citep[see][7]{Johnston:2006uq,Price:1932fk}. For what it is worth, assimilation as a metaphor for perception is inscribed in the history of the English language; the word ``perception'' derives from the Latin \emph{perceptio} meaning to take in, or assimilate---evidence, at least, for the persistence of an inclination. It is natural, then, to think of seeing as taking in the external scene before one. But then the question arises: How can one take in what remains external? And if one can, what could taking in mean, here, such that one could? Empedoclean puzzlement, in its most general form, consists in the persistence of this latter question.

It is worth wondering about the prevalence of tactile metaphors for visual awareness. There is an Aristotelian explanation, I think. The explanation is Aristotelian, not in the sense that Aristotle gives the explanation or even entertains it; rather, it is Aristotelian in that it draws on resources available in Aristotle's thought. According to Aristotle, taste and touch are primitive forms of sensation common to all animals (though animals can and do differ in their possession of other sensory capacities). Suppose then, at least in human beings, the primitive character of touch is manifest in our emotional responses to things. Often when we see something we are drawn to touch it, even though there is no doubt about its presence or solidity. It is as if a thing's presence is most keenly felt when grasped. Thus we must endeavor to teach children to keep their hands to themselves, and even in maturity, polite notices are required to remind adults to not touch the display cabinet. If the presence of things is most keenly felt when grasped, if in the resistance to touch their presence is manifest in a primitively compelling manner, then it would be no surprise that we reach for tactile metaphors in characterizing sensory presentation, even as it figures in nontactile modes of sensory awareness such as vision. If the Aristotelian explanation is correct, then the tactile metaphors are emphasizing the \emph{presentation} of the objects of sensory awareness. It is because objects grasped are presented to us in a primitively compelling manner, that grasping can serve as a paradigm of sensory presentation. 

If the Aristotelian explanation is correct, then Empedoclean puzzlement, in its original form,  is the result of overgeneralizing from a paradigm case. To be sure, vision involves the presentation of colors inhering in bounded particulars remote from the perceiver. But if the presentation of color in sensory consciousness is too closely modeled on a capacity to grasp or take in something material, then a puzzle arises that only the theory of effluences may resolve. If indeed it does. I have already mentioned two reservations. As we shall see, Aristotle has his own criticisms to make of Empedocles' theory of color vision. If Aristotle's criticisms prove cogent, then Empedoclean puzzlement, in its original form, not only involves an overgeneralization from a paradigm case but a misconception of it as well. But even if we resist the temptation to which Empedocles apparently succumbed, there remains the question: What could the sensory presentation of remote qualities be, if not simply their being palpable to sense?

% section empedoclean_puzzlement (end)

\section{Definition} % (fold)
\label{sec:definition}

In \emph{De Anima} \textsc{ii}.12 424\( ^{a} \); \textsc{ii}.5 417\( ^{b} \) Aristotle defines perception as a mode of assimilation of the sensible form without the matter of an external particular. This is an instance of Aristotle's dialectical refinement of the \emph{endoxa} \citep[on Aristotle's dialectic in \emph{De Anima} see][]{Witt:1995kx}. While denying that sight involves the assimilation of material effluences, Aristotle retains Empedocles' conception of sensory awareness as a mode of assimilation, it is just that we assimilate form without matter. Indeed, this pattern of dialectical refinement continues in the very next line where Aristotle uses Plato's metaphor of  wax receiving an impression, not to characterize judgment as Plato does in the \emph{Theaetetus} 194\( ^{c} \)--195\( ^{a} \), but to characterize the assimilation of sensible form in perception. Given this pattern of dialectical refinement, we can be confident that Aristotle was engaging with Empedocles' thought in his definition of perception. And while it remains controversial how to understand the assimilation of sensible form, I believe progress can be made by interpreting Aristotle's definition of perception as addressing Empedocles' puzzlement about how remote objects can be present in sensory consciousness. Recall Empedoclean puzzlement begins with the natural thought that in seeing one takes in the external scene. The question then arises: How can we take in what remains external? And if one can, what could taking in mean such that one could? The proposal is to interpret Aristotle's definition of perception as an answer to this latter question---a remote object can be present in sensory consciousness by assimilating its sensible form while leaving its matter in place. Understanding how Aristotle's definition of perception so much as could be a resolution of Empedoclean puzzlement imposes a substantive constraint on interpreting that definition; for so interpreted, it is making an important claim about the metaphysics of sensory presentation.

Aristotle's definition of perception is a dialectical refinement of the \emph{endoxa} insofar as it seeks to preserve an Empedoclean insight while resolving a puzzle about how remote objects can be present to sensory consciousness. Empedocles' puzzlement about the nature of sensory presentation persists to this day, though now in the guise of discussions of presence in absence. Perhaps there are insights of Aristotle's own that ought to be preserved when confronting Empedoclean puzzlement as it arises in its modern guise. If there are, then a conception of sensory presentation that preserved Aristotle's insight into the proper resolution of Empedoclean puzzlement would itself be a dialectical refinement of the respected opinion of Aristotle. What these insights might be and whether any conception of sensory presentation answers this description remains to be determined.

% section definition (end)

% chapter empedocles (end)
