%!TEX TS-program = xelatex 
%!TEX TS-options = -synctex=1 -output-driver="xdvipdfmx -q -E"
%!TEX encoding = UTF-8 Unicode
%
%  generationhandout
%
%  Created by Mark Eli Kalderon on 2011-10-03.
%  Copyright (c) 2011. All rights reserved.
%

\documentclass[10pt]{article} 

% Definitions
\newcommand\mykeywords{Aristotle, perception, color}
\newcommand\myauthor{Mark Eli Kalderon}

% Packages
\usepackage{geometry} \geometry{a4paper} 
\usepackage{url}
\usepackage{txfonts}
\usepackage{color}
\usepackage{enumerate}
\usepackage{tikz}
\definecolor{gray}{rgb}{0.459,0.438,0.471}

% Bibliography
\usepackage[round]{natbib}

% XeTeX
\usepackage[cm-default]{fontspec}
\usepackage{xltxtra,xunicode}
\defaultfontfeatures{Scale=MatchLowercase,Mapping=tex-text}
\setmainfont{Hoefler Text}

% Title Information
\title{The Generation of the Hues}
\author{\myauthor} 
\date{} % Leave blank for no date, comment out for most recent date

% PDF Stuff
\usepackage[plainpages=false, pdfpagelabels, bookmarksnumbered, pdftitle={The Generation of the Hues}, pdfauthor={\myauthor}, pdfkeywords={\mykeywords}, xetex]{hyperref} 

%%% BEGIN DOCUMENT
\begin{document}

% Title Page
\maketitle

% Layout Settings
\setlength{\parindent}{1em}

% Main Content

\begin{enumerate}
    \item There will be no difficulty in seeing how and by what mixtures the colors derived from these are made according to the rules of probability. He, however, who should attempt to verify all this by experiment would forget the difference between human and divine nature. For God only has the knowledge and also the power which are able to combine many things into one and again resolve the one into many. But no man either is or ever will be able to accomplish either the one or the other operation. (Plato, \emph{Timaeus} 68d; Jowett \citeyear[1192]{Hamilton:1961fk})
    \item But Parmenides \ldots\ has actually made a cosmic order, and by blending as elements the light and the dark produces out of them and by their operation the whole world of sense. Thus he has much to say about earth, heaven, sun, moon, and stars, and has recounted the genesis of man; and for an ancient natural philosopher---who has put together a book of his own, and is not pulling apart the book of another---he has left nothing of real importance unsaid. (Plutarch, \emph{Adversus Colotem} 1114 b--c; \citealt[231]{Einarson:1967zr})
    \item And if, concerning these things, your conviction is in any way wanting,\\
    as to how from the blending of water and earth and aither and sun\\
    the forms and colours of mortals came to be,\\
    which have now come to be, fitted together by Aphrodite.\\
    (Empedocles, DK \textsc{b}71; \citealt[74 249]{Inwood:2001ve})
    \item As when painters adorn votive offerings,\\
    men well-learned in their craft because of cunning,\\
    and so when they take in their hands many-coloured pigments,\\
    mixing them in harmony, some more, others less,\\
    from them they prepare forms resembling all things,
    making trees and men and women\\
    and beasts and birds and water-nourished fish\\
    and long-lived gods, first in their prerogatives.\\
    In this way let not deception overcome your thought organ\\
    that the source of mortal things, as many as have become obvious---countless---is anything else,\\
    but know these things clearly, having heard the story from a god.\\ 
    (Empedocles, DK \textsc{b}23; \citealt[27 231]{Inwood:2001ve})
    \item And pleasant earth in her well-built channels\\
    received two parts of gleaming Nestis out of the eight\\
    and four of Hephaistos; and they become white bones\\
    fitted together with the divine glues of harmony.\\
    (Empedocles DK \textsc{b}96; \citealt[62 245]{Inwood:2001ve})
    \item And in the depths of the river a black colour is produced by the shadow,\\
    and in the same way it is observed in cavernous grottoes.\\
    (Empedocles, DK \textsc{b}94; \citealt[105 261]{Inwood:2001ve})
    \item Is it because the depth is the mother of blackness inasmuch as it blunts and weakens the sun's rays before they can get to it? But since the surface is immediately affected by the sun, it is reasonable that it receives the gleam of light.  (Plutarch, \emph{Historia Naturalis} 39; \citealt[\textsc{CTXT}-87 137--138]{Inwood:2001ve})
	\item For he assigns their perception neither to the minute passages of fire nor to those of water nor to others composed of both these elements together. Yet we see the compound colours no whit less than we do the simple. (\emph{De Sensibus} 17; \citealt{Stratton:1917vn})
    \item Another theory is that they appear through one another, as sometimes painters produce them, when they lay a colour over another more vivid one, \emph{e.g.}, when they want to make a thing show through water or mist; just as the sun appears white when seen directly, but red when seen through fog or smoke. But on this view too the multiplicity of colours will be explained in the same way as before; for there will be some definite ratio between the superimposed colours and those below, and others again will not be in any expressible ratio. (Aristotle, \emph{De Sensu} \textsc{iii} 440\( ^{a} \)7--15; \citealt[235]{Hett:1936fk})
    \item But it must be clear that colours must be mixed when the bodies in which they occur are mixed, and that this is the real reason why there are many colours; it is not due either to overlaying or alternation; for it is not from afar only (but not from near at hand) that the color of mixed bodies seem uniform, but from all distances. (Aristotle, \emph{De Sensu} \textsc{iii} 440\( ^{b} \)16--18; \citealt[237]{Hett:1936fk})
    \item In the first place, take the thing we now call water. This, when it is compacted, we see (as we imagine) becoming earth and stones, and this same thing, when it is dissolved and dispersed, becoming wind and air; air becoming fire by being inflamed; and, by a reverse process, fire, when condensed and extinguished, returning once more to the form of air, and air coming together again and condensing as mist and cloud; and from these, as they are yet more closely compacted, flowing water; and from water once more earth and stones: and thus, as it appears, they transmit in a cycle the process of passing into one another. (Plato, \emph{Timaeus} 49\( ^{b-c} \); \citealt[179]{Cornford:1935fk})
\end{enumerate}

% % Bibligography
\bibliographystyle{plainnat} 
\bibliography{Philosophy}

\end{document}
