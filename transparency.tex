%!TEX root = /Users/markelikalderon/Documents/Git/formwithoutmatter/formwithoutmatter.tex
\chapter{Transparency} % (fold)
\label{cha:transparency}

\section{Motive and Method} % (fold)
\label{sec:motive_and_method}

Let us turn now to the transparent. In so doing we are jumping into the middle of things---both in the order of Aristotle's exposition, but also in that transparency is a common nature or power of the external medium that separates the perceiver and the remote object of vision.

There are philosophical reasons for considering the nature of the transparent.

First, Aristotle defines color in terms of the transparent. Specifically, in \emph{De Anima} \textsc{ii}.7 418\( ^{a-b} \) Aristotle defines color as the power to move what is actually transparent. Our understanding of color is incomplete if we do not understand the state of the external medium, such as air or water, which is a precondition for the activity of color. The effect of this incomplete understanding ramifies given Aristotle's avowed strategy of explaining perceptual capacities in terms of perceptual activities that are their exercise and to explain perceptual activities in terms of the objects of those activities (\emph{De Anima} \textsc{ii}.4 415\( ^{a} \)14--22).

Second, given that color is the power to move what is actually transparent, there is an alteration that the external medium undergoes when in a transparent state as a result of the activity of color. Suppose the alteration that the external medium undergoes when in a transparent state is imparted to the internal medium---the transparent medium that constitutes the interior of the sense organ, in the case of the eye, the \emph{vitreous humour}. Then we would have in place an important piece of the puzzle involved in interpreting the assimilation of sensible forms. Moreover, progress can be made here while forestalling the controversies, which must eventually be faced, surrounding the metaphysics of \emph{De Anima} \textsc{ii}.5.

Third, external particulars remote from the perceiver are arrayed in an external medium through which, when transparent, they appear. Empedoclean puzzlement highlights the way in which this is a remarkable fact. Attending to the details of Aristotle's discussion of transparency we may gain insight into his reaction to Empedoclean puzzlement about the sensory presentation of remote objects.

It is worth enumerating the reasons for considering Aristotle's discussion of transparency given its reception. Some commentators suggest that Aristotle's discussion of transparency is of antiquarian interest only. Others have expressed incredulity at the way Aristotle's account conflicts with the manifest facts of experience. While leaving open the possibility of errors and omissions, we should try to understand Aristotle's account by attending to the phenomena as we understand it to be and by asking how that phenomena might have appeared as Aristotle describes it. As a methodology, this is little more than a minimal exercise of charity. However, it is worth stating explicitly if only because it is routinely flouted. Adhering to this precept yields and understanding of transparency that is sensible, phenomenologically adequate, and a reasonable approximation of the truth. It is also a philosophically revealing exercise since to interpret Aristotle in this way is to use Aristotle's text as a means of attending to the phenomena under investigation.

% section motive_and_method (end)

\section{Transparency in \emph{De Anima}} % (fold)
\label{sec:transparency_in_de_anima}

In \emph{De Anima}, Aristotle defines the transparent as follows:
\begin{quote}
	By transparent I mean that which is visible, though not visible in itself, but owing its visibility to the color of another thing. (\emph{De Anima} \textsc{ii}.7 418\( ^{b} \)4--6)
\end{quote}

First, one might be surprised that Aristotle defines transparency in terms the manner of its visibility, or the way in which it appears in perceptual experience. Isn't it more natural to think of transparency in terms of that through which remote objects appear---that is to say, not in terms of the manner of its visibility, but in terms of its being a condition on the visibility of other things? We will return to this issue.

The second thing to remark about Aristotle's definition is the nature of the intended contrast between something being visible in itself and something owing it's visibility to another thing. Aristotle does not have in mind here what he elsewhere calls incidental perception (\emph{De Anima} \textsc{ii}.6 418\( ^{a} \) 20--23). Seeing the transparent medium by seeing colors arrayed in it is not like seeing the son of Diares (a distant ancestor of Ortcutt, \citealt{Quine:1956qp}) by seeing a white speck. Being the son of Diares is not sensible the way being transparent is (\emph{De Anima} \textsc{ii}.6 418\( ^{a} \) 23--26).

Third, once we recover from our initial surprise at Aristotle defining transparency in terms of the manner of its visibility, and we consider the plausibility of that claim, quite apart from its status as a definition, we discover a potential insight. I, at least, have the corresponding intuition about illumination. I believe we see the character of the illumination by seeing the way objects are illuminated.The former is a state of the external medium whereas the latter is a property of a particular arrayed in that medium (though, of course, a property that the particular could only have given the state of the medium). So when viewing a brightly lit pantry, one sees the the brightness of the pantry by seeing the brightly lit objects arranged in it.  Hilbert makes a similar phenomenological observation:
\begin{quote}
	Do we see how an object is illuminated or do we see the illumination itself? On phenomenological grounds the first option seems better to me. What we see as changing with the illumination is an aspect of the object itself, not the light source or the space surrounding the object. \citep[150--151]{Hilbert:2007qy}
\end{quote}
At the very least, then, the claim enshrined in Aristotle's definition of transparency receives indirect support from the plausibility of the corresponding claim about illumination.

Transparency is a nature or power common to different substances. It is shared by liquids, like air and water, and certain solids and is incidental to the nature of each (\emph{De Anima} \textsc{ii}.7 418\( ^{b} \)7--9). A medium is actually transparent not due to its nature but due rather to the contingent presence of the fiery substance (\emph{De Anima} \textsc{ii}.7 418\( ^{b} \)11--13). The continual presence of the fiery substance is required for the transparency of the medium to persist. Suppose I light a fuse with a cigar. Prudence councils that I should remove myself from the scene. Should I be enjoying the cigar I might take it with me. But while the fuse would remain lit even when the cigar is removed, the air would not remain transparent when the fiery substance is removed. When the fiery substance is removed, darkness supervenes (\emph{De Anima} \textsc{ii}.7 418\( ^{b} \)18--21; \emph{De Sensu} \textsc{iii} 439\( ^{a} \)18--21). Not only does the persistence of transparency depend upon the continual presence of the fiery substance, but, arguably at least, it depends as well on its continual activity \citep[\emph{pace}][424]{Burnyeat:1995fk}. (That some states require continual activity to sustain them should be no surprise; consider Ryle's, \citeyear[149]{Ryle:1949qr}, example of keeping the enemy at bay, or the connection between heat and molecular motion.) This may be taken to be implied by his claim that light is the activity of the transparent \emph{qua} transparent (\emph{De Anima} \textsc{ii}.7 418\( ^{b} \) 9--10). Since transparency just is the presence of the fiery substance, the activity of the transparent \emph{qua} transparent just is the activity of the present fiery substance.

Light is a state that the medium is in when it is actually transparent. Aristotle denies that light is fire, or a body, or an effluence (\emph{De Anima} \textsc{ii}.7 418\( ^{b} \) 13--18). He denies as well that light moves, otherwise its motion would be visible as it travels from East to West (\emph{De Anima} \textsc{ii}.7 418\( ^{b} \) 21-27). These claims are puzzling if by light Aristotle means, at least approximately, what we mean by light. But why assume that? 

Begin by focusing on Aristotle's claim that light is a \emph{state} (\emph{hexis}, \emph{De Anima} \textsc{iii}.5 430\( ^{a} \)15) that a medium is in when it is actually transparent. Light could not be a body since the medium is a body and two bodies cannot occupy the same space. As \citet{Burnyeat:1995fk} has emphasized, state is really the wrong category for light as we presently understand it to be. But now, in line with our avowed methodology, let us ask whether there could be a state that we can recognize on our present understanding that could reasonably be what Aristotle had in mind when he speaks of light? With the question so framed the resolution of our difficulties should be obvious. What state is a medium in when it is actually transparent, and where the persistence of this state depends on the continual presence and activity of a fiery substance? When it is illuminated, of course. By light, Aristotle means a state of illumination (see \citealt[122]{Thorp:1982fk}, for a similar interpretation). And that a medium when it is actually transparent is in a state of illumination sustained by the presence and activity of a fiery substance strikes me as a not unreasonable approximation of the truth. Moreover, it coheres well with the phenomenology of illumination. Consider what must have been the familiar experience of lighting an oil lamp to illuminate a room.

What about Aristotle's claim that light does not move? There are distinguishable aspects to Aristotle's overall case. 

Begin with light as conceived by Aristotle's opponents---as fire, body, or material effluence. With light so conceived, Aristotle's case is straightforwardly empirical. To conceive of light as fire, body, or material effluence is to conceive of light as being capable of locomotion---as potentially changing its location over time. However, we do not see light from a morning sunrise moving from East to West. And while movement across short distances may be too quick to be visible, Aristotle maintains that the corresponding claim is implausible given the magnitude of the distance involved. Though Aristotle's empirical argument fails (given his overconfidence in there being some magnitude over which motion would be perceptible, no doubt abetted by his conviction that every magnitude is perceptible at some distance), it remains an honorable failure. Aristotle's remarks, here, are best understood set against the Milesian tradition of preferring first-hand experience to the deliverances of authority (even where, as in the present case, the relevant authorities are not Hesiod and Homer, but Empedocles and Plato).

Not only is there an empirical argument that light---conceived as fire, body, or material effluence---does not move, but there is a distinct argument that light---conceived instead as a state (\emph{hexis})---cannot move. This latter argument is not empirical; rather, reflection on the nature of a state reveals that it precludes space-occupancy. Bodies occupy space; they have location and are there\-by susceptible to locomotion, at least potentially. But a state of a body does not occupy space the way a body does, not even the space occupied by the body whose state it is. Since states do not occupy space, they lack locations and thus are not susceptible to locomotion.

\citet[430 n29; appendix]{Burnyeat:1995fk} tries to make vivid the madness of these claims by quoting, at length, Prichard echoing them:
\begin{quote}
	I once made what I thought the unquestionable remark to a German mathematician who was also a physicist that only a body could move---so that, for example, the centre of gravity of a body or of a system of bodies, which is a geometrical point, could not move. He as I rather expected, thought I was just mad. In this case I should certainly have said I was certain that a centre of gravity could not move, and I think he would have said he was certain that it could. Here I personally should assert he could not possibly have been more that uncertain that it could not, and that, if he had thought a bit more, he would have been certain that it could not; you cannot make a man think, any more than you can make a horse drink. \citep[99; this is just the initial paragraph of the material that Burnyeat quotes]{Prichard:1950tg}
\end{quote}
At least in my case, however, Burnyeat's \citeyearpar[430 n29]{Burnyeat:1995fk} rhetorical strategy backfired. Far from recognizing ``an eccentricty from the home of lost causes'' that ``would meet with Aristotle's approval'', I instead heeded Prichard's advice. And having thought a bit more, I now regard Aristotle's claim about states and space-occupancy to be prima facie plausible, if controversial, in a way that his claim about the empirical significance of our failure to observe the motion of light no longer could be.

% There remain anachronistic elements in Aristotle's account. Consider his empirical argument that light does not move (\emph{De Anima} \textsc{ii}.7 418\( ^{b} \) 21-27). If light is a state of a medium sustained by the presence and activity of a fiery substance, then this claim has the consequence that the fiery substance cannot propagate through a potentially transparent medium. For as the fiery substance propagated through the medium, the illuminated region of that medium would change over time. It would be consistent with Aristotle's claim that light does not move that the fiery substance instantaneously propagates through a potentially transparent medium, but that would be a Pickwickian sense of propagation. Propagation is a \emph{process} that unfolds through time and so could not, strictly speaking, be instantaneous.
% 
% The reasoning assumes that the illuminated region of the medium changing over time consists in a change in the position of the state of illumination. Without some such assumption, there is no valid inference from the propagation of the fiery substance through a medium to the motion of light as Aristotle understands it to be, as a state of illumination.

% section transparency_in_de_anima (end)

\section{Transparency in \emph{De Sensu}} % (fold)
\label{sec:transparency_in_de_sensu}

In \emph{De Anima}, Aristotle defines the transparent as that which is visible, though not visible in itself, but owing its visibility to the color of another thing (\emph{De Anima} \textsc{ii}.7 418\( ^{b} \) 4--6). I have remarked that it might seem more natural to characterize transparency, not in terms of the manner of its visibility, but in terms of its being that through which remote objects appear---as a condition on the visibility of other things. However, this latter conception is not entirely absent in Aristotle. It is at least implicit in the corresponding discussion of color and transparency in \emph{De Sensu}.

In \emph{De Sensu} Aristotle sets out to explain what each of the sense objects ``must be to produce the sensation in full actuality'' (\emph{De Sensu} \textsc{iii} 439\( ^{a} \)10). This is a further inquiry, not directly addressed by \emph{De Anima}. Unsurprisingly, then, there are novel elements to the \emph{De Sensu} discussion. Thus, novel claims that emerge include, for example, that color resides in the proportion of transparent that exists in all bodies, and an account of the generation of the hues in terms of the ratio of black and white in a mixture. Given these novel elements, the question arises whether \emph{De Sensu} represents an extension of the doctrines of \emph{De Anima}, or a change of mind. While there is some evidence that Aristotle has not completely harmonized new ideas with old, I believe that Aristotle meant to be offering an extension of the \emph{De Anima} account, and not a substantive revision of it. Or at any rate, this will be my working hypothesis (see \citealt{Kahn:1966zr} for discussion; see also \citealt[291]{Caston:2005cr} \citealt[37]{Nussbaum:1995ly}).

One novel element is the characterization of color as ``the limit of the transparent in a determinately bounded body'' (\emph{De Sensu} \textsc{iii} 439\( ^{b} \)11). This prompted the Renaissance commentator Jacopo \citet{Zabarella:1605kx} to complain that Aristotle has defined color twice over \citep{Broackes:1999uq}. However, there is no evidence in the text that Aristotle regarded this claim as a definition. Rather, it appears as the conclusion of an argument \citep[65]{Broackes:1999uq}. In that argument, Aristotle explains that color inheres not only in unbounded things, such as air and water, but in bounded things as well. What's the distinction between the bounded and the unbounded? The examples of the transparent are restricted in \emph{De Sensu} to air and water. On this basis, it might be thought, naturally enough, that that the distinction is between transparent liquids, like air and water, and opaque solid objects \citep[59]{Broackes:1999uq}. To describe liquids as unbounded is to highlight their lack of fixed boundaries. However, I doubt that is what Aristotle had in mind. In \emph{De Anima}, Aristotle claims that not only are liquids such as air and water transparent, but so are certain solid objects. He does not himself give examples of transparent solids. But glass, ice, crystals, tortoise shells, and certain animal horns would do, and we can be confident that Aristotle had first hand experience with at least some of these. The problem, then, is that any such example would possess fixed boundaries and yet would remain transparent, but the transparent is meant to be undounded. 

What could the unbounded be if it is not simply the lack of fixed boundaries? I believe that good sense can be made of Aristotle's distinction if we understand it in perceptual terms. Nontransparent bodies, such as opaque solids, are perceptually impenetrable. Unlike transparent bodies you cannot see in them or through them. Their surface is the site of visual resistance; perceptual impenetrability determines a visual boundary through which nothing further can appear. Transparent bodies, in contrast, are perceptually penetrable. One can see in them and through them. The particulars arrayed in a transparent medium appear through that medium. The transparent is unbounded since it offers insufficient visual resistance to determine a perceptually impenetrable boundary. And this is true of transparent solids such as crystals and tortoise shells as well as transparent liquids such as air and water.

The transparent is unbounded since it offers insufficient visual resistance to determine a perceptually impenetrable boundary. Which is not, of course, to say that the transparent can offer no visual resistance. In \emph{De Sensu}, Aristotle emphasizes that transparency comes in \emph{degrees}. When Aristotle speaks of color as the limit of the transparent in bounded bodies, he has in mind surface color. But he also speaks of the color of transparent media:
\begin{quote}
    Air and water obviously have color; for their brightness is of the nature of color. But in their case because the color resides in something unbounded, air and sea do not show the same color near at hand and to those who approach them as they have at a distance. (\emph{De Sensu} \textsc{iii} 439\( ^{b} \)1--3)
\end{quote}
Air and water, when transparent, are bright. And brightness, Aristotle claims, is of the nature of color. The attribution of brightness, however, requires attributing no particular hue to the medium. If the medium is perfectly transparent, then the only visible hues will be the colors of bounded particulars arrayed in that medium. But the next line contains the suggestion that imperfectly transparent media, while remaining perceptually penetrable to some degree, may themselves have a particular hue---in modern parlance, not a surface color but a \emph{volume} color. From a cliff overhanging the sea, the sea may appear a clear blue even as one sees rocks lying below its surface. But, if enticed by the sea, one were to descend to the beach and examine a handful of sea water, it would not be blue at all, but transparent. Similarly, looking up at the sky on a clear autumn afternoon, one sees an expanse of blue. But if one were to travel to that region of the sky, by helicopter, say, nothing blue would be found. The implicit thought is that the visual resistance of an imperfectly transparent medium increases with an increase in volume. The further one sees into a transparent medium, the more resistance that medium offers to sight. And volume color is the effect of this resistance.

The color of an imperfectly transparent medium does not occlude the bound\-ed particulars arrayed in it. But the color of the transparent medium may affect their color appearance. Thus the sun, which in itself appears white, takes on a crimson hue when seen through a fog or cloud of smoke (\emph{De Sensu} 3 440\( ^{a} \)10--11). This might be what Aristotle has in mind when he claims that bounded particulars have a fixed color unless affected by atmospheric conditions (\emph{De Sensu} 3 439\( ^{b} \)5--7). The color of a bounded particular will affect the medium differently depending on its degree of perceptual penetrability and resulting volume color. Notice, considered in and of itself, this claim implies at most that the color of the sun appears differently when obscured by a fog or cloud of smoke; there need be not commitment to the sun changing color from white to red when so obscured, nor its appearing to so change. Aristotle's position allows for the possibility of a variation in color appearance without a variation in presented color. Notice the thought that the state of a medium can alter the appearance of a sensible object without a variation in the object of sense is what animates Austin's \citeyearpar{Austin:1962lr} example of a straight stick looking bent in water (on Austin see Kalderon and Travis \citeyear{Kalderon:2010fk} and Martin \citeyear{Martin:2000nx}; on Austin and the argument from conflicting appearances see Burnyeat \citeyear{Burnyeat:1979mv}).

This is potential evidence about Aristotle's attitude towards the argument from conflicting appearances. While the argument from conflicting appearances is discussed in \emph{Metaphysica} \( \Gamma \), discussion of it is largely absent in \emph{De Anima} and \emph{De Sensu}. While largely absent from \emph{De Anima} and \emph{De Sensu}, it is not entirely absent, and I believe we have an important point of contact here. Looking up from a battlefield one sees the sun burning white. As smoke from the battle obscures the sun, it takes on a crimson hue. Red and white are contraries. Nothing can be red and white all over at the same time. Supposing, as is plausible, that the smoke from the battle did not alter the sun's color so that the color of the sun remains constant through the variation in its appearance, it might seem as if at least one of these appearances were illusory. However, if there can be a variation in color appearance without a variation in presented color, then the white and red appearances do not conflict. The color of the sun does not appear to change from white to red; red is simply the way radiant white things appear when viewed through smoke filled media (just as bent is the way that straight things look when viewed through refracting media, \citealt{Austin:1962lr}). In \emph{Metaphysica} \( \Gamma \) Aristotle expresses a complementary attitude:

\begin{quote}
	Again, it is fair to express surprise at our opponent's raising the question whether magnitudes are as great, and colors are of such a nature, as they appear to people at a distance, or as they appear to those close at hand and whether they are such as they appear to the healthy or to the sick, and whether those things are heavy which appear so to the weak or those which appear so to the strong, and those things which appear to the sleeping or to the waking. For obviously, they do not think these to be open questions. (\emph{Metaphysica} \( \Gamma \) 5 1010\( ^{b} \)3--9)
\end{quote}

Against the present interpretation it might be objected that Aristotle makes a claim about the color of the transparent that conflicts with it. Thus Aristotle claims that the transparent lacks color and so is receptive to color (\emph{De Anima} \textsc{ii}.7 418\( ^{b} \)26--29). The force of this objection is mitigated somewhat by the recognition that Aristotle seems to make inconsistent claims about the color of the transparent. Thus he claims that:
\begin{enumerate}[(1)]
	\item Light, or brightness, is the color of the transparent. (\emph{De Anima} \textsc{ii}.7 418\( ^{b} \)11-12; \emph{De Sensu} \textsc{iii} 439\( ^{b} \)1--2)
	\item The transparent is seen to have different colors when near and far. (\emph{De Sensu} \textsc{iii} 439\( ^{b} \)2--3)
	\item The transparent lacks color and so is receptive to color. (\emph{De Anima} \textsc{ii}.7 418\( ^{b} \)26--29)
\end{enumerate}
How might (1)--(3) be interpreted so as to be consistent? We have already observed that the attribution of brightness requires attributing no particular hue to the transparent medium. Moreover, since the medium is transparent, the color of the remote particular appears through that medium. This may even be so in an imperfectly transparent medium, one such that owing to the resistance it offers to vision itself appears a certain volume color. The color of a remote particular may appear differently when viewed through perfectly and imperfectly transparent media, but the volume color, if any, of the transparent medium does not occlude the surface color of the remote bounded particular. But so long as the surface color of the remote bounded particular is not occluded by varying the color of the medium as it volume varies, the transparent medium remains receptive of that color. If, however, the medium were to become perceptually impenetrable and so take on a surface color, the color of the remote bounded particular would be occluded and the medium would no longer be receptive to color. The denial in (3) is the denial of surface color to transparent media, but that is consistent with imperfectly transparent media, such as the sea and the sky, having volume color. Properly interpreted, (1)--(3) are consistent.

There is thus a progression of qualitative states from the perfectly transparent to the colored and opaque. The qualitative states in the progression are ordered by their decreasing degree of perceptual penetrability culminating in the perceptual impenetrable. It is thus a progression to a limit. We can envision the progression from perfect transparency in the following manner. Consider a tank of clear water into which is poured a blue dye. Suppose the absorption rate of the dye is too quick to be visible. So we do not see clouds of blue dye propagating through the clear liquid; rather, we see the volume taking on the blue and become increasingly opaque. At the end of this progression, the tank is surface blue---no thing can appear in it or through it. Color, that is surface color, is in this sense the limit of the transparent---it is the terminal qualitative state of a progression of qualitative states ordered by decreasing degree of perceptual penetrability.

One may be forgiven for thinking that Aristotle has fallen into a category mistake in speaking of color as the limit of the transparent \citep[65]{Broackes:1999uq}. He seems, on the surface, to be making an identification, but color is a \emph{quality} in the way that a limit could not be. However, on the interpretation that I have been urging, Aristotle is not identifying color qualities with limits; rather, in the progression of qualitative states from the perceptually penetrable to the perceptually impenetrable, color (that is, surface color) is the terminal qualitative state. This is \emph{one} way of understanding Aquinas, in his commentary on \emph{De Sensu}, when he writes:
\begin{quote}
	Thus color is not in the category of quantity---like surface, which is the limit of a body---but in the category of quality. The transparent is also in the category of quality, because \emph{a limit and that of which it is the limit belong to one category}. [my emphasis] (\emph{Sententia De Sensu Et Sensato} \textsc{v}, commentary on \emph{De Sensu} \textsc{iii} 439\( ^{b} \)11 in \citealt{White:2005vn})
\end{quote}

In \emph{De Sensu}, Aristotle not only speaks of the limit of the transparent but also of the limit of a body: The limit of a body is its external surface, a bulgy two-dimensional particular, in Sellars' \citeyearpar[\textsc{iv} §23]{Sellars:1956xp} apt phrase. \citet[\textsc{iv} §23]{Sellars:1956xp} explains that it is two-dimensional in the sense that ``though it may be \emph{bulgy}, and in \emph{this} sense three-dimensional, it has no \emph{thickness}''. Color lies at the limit of the body, and this, Aristotle claims, encouraged the Pythagoreans to call the surface of a body its color. In so doing, however, the Pythagoreans undertook a further commitment: Color not only lies at the limit of a body, but color is itself the limit. In calling the surface of a body its color, the Pythagoreans identify color with the limit of the body. However, while color may lie at the limit of the body, color is not itself the limit:
\begin{quote}
	Color lies at the limit of the body, but this limit is not a real thing; we must suppose that the same nature which exhibits color outside, also exists within. (\emph{De Sensu} \textsc{iii} 439\( ^{a} \)32--439\( ^{b} \)35)
\end{quote}
Aristotle's opposition to the Pythagorean conception of color is elaborated by Sellars two millennia hence:
\begin{quote}
	Certainly, when we say of an object that it is red, we commit ourselves to no more than that it is red ``at the surface''. And sometimes it is red at the surface by having what we would not hesitate to call a ``part'' which is red through and through---thus, a red table which is red by virtue of a layer of red paint. But the red paint is not itself red by virtue of a component---a `surface' or `expanse'; a particular with no thickness---which is red. \citep[\textsc{iv} §23]{Sellars:1956xp}
\end{quote}
It is thus misleading, I believe, for \citet{Silverman:1989ve} to liken colors, as conceived by Aristotle, to Sherwin-Williams paint.

Does the consideration that tells against color being the limit of the body tell equally against color being the limit of the transparent? Not obviously. Opaque solids are perceptually impenetrable, and their perceptual impenetrability determines a visual boundary through which nothing further can appear. That is what their opacity consists in. This visual boundary coincides with the limit of the body. This could only seem inconsistent with the claim that the same nature which exhibits color outside also exists within if one ignored Aristotle's reminder at the opening of \emph{De Sensu} \textsc{iii} 439\( ^{a} \)13--14 that ``each of these terms is used in two senses: as actual or potential.'' Aquinas insightfully heeds this reminder. In his commentary on \emph{De Sensu} he writes that ``bodies have surface in their interior in potentiality but not actuality'' (\emph{Sententia De Sensu Et Sensato} \textsc{v}, commentary on \emph{De Sensu} \textsc{iii} 439\( ^{b} \)11 in \citealt{White:2005vn}). When the perceptually impenetrable is actually resisting sight a visual boundary is determined at the limit of the opaque body. But that a portion of the interior of such a body offers no such visual resistance in being occluded from view is consistent with its being perceptually impenetrable, with its potentially determining such a visual boundary.

Another consideration is relevant here. The limit of the transparent is a qualitative state. However, as Aquinas observed, the limit of a body is not a qualitative state; the limit of a body belongs, rather, to the category of quantity (compare \emph{Metaphysica} \( \Delta \) 13, 17). An argument to the conclusion that color is not a species of quantity---in the present instance, the limit of a body---does not by itself constitute an argument against the claim that color is a qualitative state distinguished by its place in an ordering of qualitative states.

In his discussion of the unbounded, then, there are thus two notions of limit in play. Aristotle distinguishes:
\begin{enumerate}[(1)]
	\item the limit of the transparent
	\item the limit of a body
\end{enumerate}
These are distinct limits. Whereas the former is qualitative, the latter is quantitative. However, importantly they coincide. A bounded body, in being perceptually impenetrable, determines a visual boundary that coincides with the limit of the body. Moreover, Aristotle's claim, that Zabarella mistakes for a definition, that color is the limit of the transparent in a determinately bounded body gives expression to just this coincidence. Color, that is, surface color, is the limit of the transparent in being the terminal qualitative state in a progression of qualitative states ordered by decreasing perceptual penetrability. A determinately bounded body is one such that, being perceptual impenetrable, determines a visual boundary through which nothing further may appear. This visual boundary is spatially coincident with the limit of the body <and></and> is where the body's surface color is seen to inhere.

Aristotle's discussion of transparency and the unbounded is evidence that, despite his defining transparency in terms of the manner of its visibility, he retains a conception of the transparent as that in which and through which remote objects may appear, as a condition on the visibility of other things. That conception, in the guise of perceptual penetrability, is central to Aristotle's understanding of the unbounded. Two observations are relevant. First, given our working hypothesis that \emph{De Sensu} is to be read as an extension of the \emph{De Anima} account and not a substantive revision of it, we can assume that this conception is meant to be at least consistent with the \emph{De Anima} definition.  Second, Empedoclean puzzlement about the sensory presentation of remote objects highlights the way in which perceptual penetrability of transparent media is a remarkable fact. It is a remarkable fact. Moreover, in \emph{not} defining transparency as that in which and through which remote objects may appear, Aristotle arguably acknowledges that it is. That the colors of remote objects are seen through transparent media is a fact to be explained. And if the nature of the transparent is to play a role in that explanation, the transparent must be defined in some way other than as being a condition on the visibility of remote objects. The explanation is given in \emph{De Anima}---in terms of the way in which color alters the transparent and the role that alteration plays in the exercise of our perceptual capacities. % The first step in understanding that explanation is to understand the manner in which color acts upon the transparent medium. That is the task of the next section. The next step is to understand the role this alteration plays in the realization of our perceptual capacities. That is the task of the subsequent section.

% section transparency_in_de_sensu (end)

% chapter transparency (end)
