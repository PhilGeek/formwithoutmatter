%!TEX root = /Users/markelikalderon/Documents/Git/formwithoutmatter/aristotle.tex
\chapter{Transparency} % (fold)
\label{cha:transparency}

\section{Motive and Method} % (fold)
\label{sec:motive_and_method}

Let us turn now to the transparent. In so doing we are jumping into the middle of things---both in the order of Aristotle's exposition, but also in that transparency is a common nature or power of the external medium that separates the perceiver and the remote object of vision.

There are philosophical reasons for considering the nature of the transparent.

First, Aristotle defines color in terms of the transparent. Specifically, in \emph{De Anima} \textsc{ii} 7 418\( ^{a-b} \), Aristotle defines color as the power to move what is actually transparent. A medium must be actually transparent in order for a particular's color to act upon it. Our understanding of color is incomplete if we do not understand the state of the external medium, such as air or water, which is a precondition for the activity of color. The effect of this incomplete understanding ramifies given Aristotle's avowed strategy of explaining perceptual capacities in terms of perceptual activities that are their exercise and to explain perceptual activities in terms of the objects of those activities (\emph{De Anima} \textsc{ii} 4 415\( ^{a} \)14--22).

Second, given that color is the power to move what is actually transparent, there is an alteration that the external medium undergoes when in a transparent state as a result of the activity of color. Suppose the alteration that the external medium undergoes when in a transparent state is imparted to the internal medium---the transparent medium that constitutes the interior of the sense organ, in the case of the eye, the vitreous humor. Then we would have in place an important piece of the puzzle involved in interpreting the assimilation of sensible forms. Moreover, progress can be made here while forestalling the controversies, which must eventually be faced, surrounding the metaphysics of \emph{De Anima} \textsc{ii} 5.

Third, external particulars remote from the perceiver are arrayed in an external medium through which, when transparent, they appear. Empedoclean puzzlement highlights the way in which this is a remarkable fact. Attending to the details of Aristotle's discussion of transparency we may gain insight into his reaction to Empedoclean puzzlement about the sensory presentation of remote objects.

It is worth enumerating the reasons for considering Aristotle's discussion of transparency given its reception. Some commentators suggest that Aristotle's discussion of transparency is of antiquarian interest only. Others have expressed incredulity at the way Aristotle's account conflicts with the manifest facts of experience. While leaving open the possibility of errors and omissions, we should try to understand Aristotle's account by attending to the phenomena as we understand it to be and by asking how that phenomena might have appeared as Aristotle describes it. As a methodology, this is little more than a minimal exercise of charity. However, it is worth stating explicitly if only because it has been routinely flouted. Adhering to this precept yields and understanding of transparency that is sensible, phenomenologically adequate, and a reasonable approximation of the truth. It is also a philosophically revealing exercise since to interpret Aristotle in this way is to use Aristotle's text as a means of attending to the phenomena under investigation.

% section motive_and_method (end)

\section{Transparency in \emph{De Anima}} % (fold)
\label{sec:transparency_in_de_anima}

In \emph{De Anima}, Aristotle defines the transparent as follows:
\begin{quote}
	\ldots\ by `transparent' I mean what is visible, and yet not visible in itself, but rather owing its visibility to the colour of \emph{something else}. (Aristotle, \emph{De Anima} \textsc{ii} 7 418\( ^{b} \)4--6; Smith in \citealt[32]{Barnes:1984uq})
\end{quote}

First, one might be surprised that Aristotle defines transparency in terms the manner of its visibility, or the way in which it appears in perceptual experience. Is it not more natural to think of transparency in terms of that through which remote objects appear---that is to say, not in terms of the manner of its visibility, but in terms of its being a condition on the visibility of other things? We will return to this issue.

The second thing to remark about Aristotle's definition is the nature of the intended contrast between something being visible in itself and something owing it's visibility to another thing. Aristotle does not have in mind here what he elsewhere calls incidental perception (\emph{De Anima} \textsc{ii} 6 418\( ^{a} \) 20--23). Seeing the transparent medium by seeing colors arrayed in it is not like seeing the son of Diares (a distant ancestor of Ortcutt, \citealt{Quine:1956qp}) by seeing a white speck. Being the son of Diares is not sensible the way being transparent is (\emph{De Anima} \textsc{ii} 6 418\( ^{a} \) 23--26).

Third, once we recover from our initial surprise at Aristotle defining transparency in terms of the manner of its visibility, and we consider the plausibility of that claim, quite apart from its status as a definition, we discover a potential insight. I, at least, have the corresponding intuition about illumination. I believe we see the character of the illumination by seeing the way objects are illuminated. The former is a state of the external medium whereas the latter is a property of a particular arrayed in that medium (though, of course, a property that the particular could only have given the state of the medium). So when viewing a brightly lit pantry, one sees the the brightness of the pantry by seeing the brightly lit objects arranged in it.  Hilbert makes a similar phenomenological observation:
\begin{quote}
	Do we see how an object is illuminated or do we see the illumination itself? On phenomenological grounds the first option seems better to me. What we see as changing with the illumination is an aspect of the object itself, not the light source or the space surrounding the object. \citep[150--151]{Hilbert:2007qy}
\end{quote}
At the very least, then, the phenomenological claim enshrined in Aristotle's definition of transparency receives indirect support from the plausibility of the corresponding claim about illumination.

Finally, a question may be raised about the adequacy of Aristotle's definition. While the phenomenological claim enshrined in Aristotle's definition may be plausible, it may be an inadequate definition of transparency if things other than the transparent are visible, though not visible in themselves, but owing their visibility to the colors of other things. Consider Parmenides' striking description of the moon's reflectance:
\begin{quote}
	Night-shining foreign light wandering round earth. (Parmenides, \textsc{dk} 28\textsc{b}14; \citealt[156]{McKirahan:1994ve})
\end{quote}
The moon is visible, but not by virtue of its own color, but by virtue of the color of the foreign light with which it shines. So the moon, as described by Parmenides, satisfies Aristotle's purported definition of transparency, but the moon is not transparent. 

To make vivid Parmenides' claim that the moon appears the color of the foreign light with which it shines, consider how the moon appears unilluminated, during a lunar eclipse. Casati provides a compelling description:
\begin{quotation}
	\noindent For the first time, I \emph{saw} the moon for what it really was, and I wanted to put it down in words. The moon is a fairly large, shadowy rock hanging a certain distance over my head; oddly, it doesn't fall down and hit me. Naturally, I know about the laws that kept it safely in orbit; but my eyes, unaccustomed to seeing stones hanging in the sky, didn't want to believe it. Likewise, it had escaped my notice that the moon was---as I knew perfectly well---a huge dark rock; usually the diaphanous glow of the lunar surface tricks the eye with the illusion that it's a delicate, airy lantern.
	
	During the eclipse the moon loses its character as a demigoddess: it splits itself off from the royal court of the other visible, shining celestial objects. Even those planets that are dark like the moon and glow with reflected light aren't seen as planets; our not very selective vision lumps them in together with the stars. Light offers the moon a weightlessness that renders it more acceptable---makes it seem almost normal that the moon should sail in the night like a paper lantern hung from the black ceiling of the sky. \citep[3--4]{Casati:2003aa}
\end{quotation}
During a lunar eclipse the moon no longer shines with a foreign light. When reflecting the sun's light, the moon appears bright. But during the eclipse, when the moon no longer reflects the sun's light, the moon is dark. During the eclipse, the moon is visible with its own color, gray, and not with the color of the sun's light it reflects, a diaphanous bright color akin to a paper lantern. Thus the shadow of the Earth reveals the moon's true color.

The problem is general and does not depend on the veracity of Parmenides' description of the moon's reflectance (on the astronomical significance of this fragment see \citealt{Popper:1998aa}). Consider any highly reflective surface, a mirror, say (on mirrors in antiquity see \citealt{Schweig:1941yu}). A mirror is visible, though not in itself, but owing its visibility to the colors of other things, the things whose colors are reflected therein. But mirrors are not transparent. Aristotle's claim about the manner in which the transparent is visible---that it owes its visibility to the colors of other things---may be true, and yet fail as a definition of transparency because it fails to provide a sufficient condition for something to be transparent. The phenomenological claim is plausibly true not only of the transparent but of the reflective as well.

Transparency is a nature or power common to different substances. It is shared by liquids, like air and water, and certain solids and is incidental to the nature of each (\emph{De Anima} \textsc{ii} 7 418\( ^{b} \)7--9). A medium is actually transparent not due to its nature but due rather to the contingent presence of the fiery substance (\emph{De Anima} \textsc{ii} 7 418\( ^{b} \)11--13). The fiery substance is a substance, but insofar as it pervades a material medium, such as a body of air or water, it could not itself be a body, since two bodies cannot occupy the same space (\emph{De Anima} \textsc{ii} 7 418\( ^{b} \)19). The continual presence of the fiery substance is required for the transparency of the medium to persist. Suppose I light a fuse with a cigar. Prudence councils that I should remove myself from the scene. Should I be enjoying the cigar I might take it with me. But while the fuse would remain lit even when the cigar is removed, the air would not remain transparent when the fiery substance is removed. When the fiery substance is removed, darkness supervenes (\emph{De Anima} \textsc{ii} 7 418\( ^{b} \)18--21; \emph{De Sensu} \textsc{iii} 439\( ^{a} \)18--21). 

Not only does the persistence of transparency depend upon the continual presence of the fiery substance, but, arguably at least, it depends as well upon its continual activity (\emph{pace} \citealt[]{Sambursky:1958aa}, \citealt[424]{Burnyeat:1995fk}). That some states require continual activity to sustain them should be no surprise. Consider Ryle's, \citeyearpar[149]{Ryle:1949qr}, example of keeping the enemy at bay, or the connection between heat and molecular motion. That the persistence of transparency depends upon the continual activity of the fiery substance may be taken to be implied by Aristotle's claim that light is the activity of the transparent \emph{qua} transparent (\emph{De Anima} \textsc{ii} 7 418\( ^{b} \) 9--10). Since transparency just is the presence of the fiery substance, the activity of the transparent \emph{qua} transparent just is the activity of the present fiery substance.

A more speculative reason concerns the nature of fire. In the \emph{Theaetetus}, in elaborating the Secret Doctrine of Protagoras, Plato writes:
\begin{quote}
	\ldots\ being (what passes for such) and becoming are a product of motion, while not-being and passing-away result from a state of rest. There is evidence for it in the fact that heat or fire, which presumably controls everything else, is itself generated out of movement and friction---these being motions. \ldots\ Moreover, the growth of living creatures depends upon these same sources. (Plato, \emph{Theaetetus} 153\( ^{a-b} \); Levett and Burnyeat in \citealt[70]{Cooper:1997fk})
\end{quote}
Here, Socrates is summarizing Heraclitus' view (on Cosmic Fire in Heraclitus see \citealt{Wiggins:1982kx}). The first line is a general thesis of Heraclitean metaphysics. Fire and living creatures are cited as examples meant to illustrate and motivate the more general thesis. The being and continued existence of fire depends upon its activity. Fire burns. Burning, here, is not restricted to the burning of grosser forms of sublunary fire familiar from sensory experience. Rather, burning is the most general activity of fire, even in its rarer forms. So understood, should fire cease to burn, it would cease to be. Similarly, to be a living creature is to act or, at the very least, to have a capacity to act. Should a living creature lose its capacity to act altogether, it would cease to be. While Aristotle denies the general claim that what passes for being is the product of motion (as well the claim that fire controls all), he can nevertheless accept the description of the examples that illustrate and motivate the more general claim. Thus, it is arguable that the passage contains the germ of his conception of a living being as refined and elaborated throughout \emph{De Anima}. If that is right, then it is at least open, in principle, for Aristotle to accept the corresponding claim about fire. That the being of fire depends upon its distinctive activity, that a fire would cease to be should it cease to burn, is anyway, a \emph{prima facie} plausible and phenomenologically compelling claim. So conceived, however, the presence of the fiery substance will depend upon its continued activity. Given the nature of fire, the continued presence of the fiery substance just is its continued activity. Thus if the transparency of a medium depends upon the continued presence of the fiery substance, it will depend, as well, upon its continued activity. 

Suppose that Heraclitean metaphysics is right to the extent that for fire, at least, to be is to burn. Putting this together with Aristotle's denial that the fiery substance is a body, we arrive at a conception of the fiery substance as an incorporeal activity. The presence of the fiery substance in a potentially transparent medium, be it air or water, just is the occurrence of this incorporeal activity, a kind of rarefied burning that instantaneously pervades the medium insofar as it is a unity.

That the presence of the fiery substance depends upon its activity provides Aristotle with the resources to explain the directionality of light. That light has direction is vividly manifest in elementary facts about occlusion, reflection, and shadow and is fundamental to geometrical optics. If the presence of the fiery substance did not depend upon its activity, then Aristotle would be hard pressed to explain the directionality of light in terms of the mere presence of the fiery substance, as Sorabji, echoing Philoponus, observes: 
\begin{quote}
	Mere presence is not enough to explain the directionality of light. Why, for example, are there any shadows at all, including the shadows that constitute night, and lunar eclipse? For the sun and other fire-like stuff is present in the universe surrounding the earth, a surrounding all of which is transparent. The requirement of presence does not explain why there is not light round the corners. \citep[132]{Sorabji:2004fk}
\end{quote}
So understood, Aristotle's conception of light would be an inadequate foundation for a geometrical optics. Given the empirical fecundity of that discipline, Aristotle's theory, in contrast, would seem to be of antiquarian interest only. And given the vividness of elementary facts about occlusion, reflection, and shadow, Aristotle's conception, in leaving the direction of light out of account, would seem to be poorly observed as well. However, if the presence of the fiery substance were constituted by its activity, if the being of fire consisted precisely in its burning, then, since activity can have a direction of influence, the direction of light could be explained in terms of the direction of the activity of the fiery substance that constitutes its being and continued existence. Light has a direction even though, being a state, it could not travel. Light has a direction that it inherits from the direction of influence of the fiery substance whose activity constitutes its being and continued existence.

The present understanding of the fiery substance echoes important aspects of Philoponus' discussion of light and vision. In an extended \emph{theôria} (\emph{On} De Anima 325 1---341 9) following a comment on \emph{De Anima} \textsc{ii} 7 418\( ^{b} \)9--10, Philoponus argues for lines of influence directed from the colored object to the perceiver. He presents this as an interpretation of Aristotle. And while he sometimes presents his own views as interpretations of Aristotle, and while the exposition of the \emph{theôria} extends the \emph{De Anima} discussion, I believe that Philoponus was both sincere and insightful in claiming to make explicit what was merely implicit in Aristotle. \citet{Sambursky:1958aa} does not understand the presence of the fiery substance in terms of its activity. He thus attributes to Aristotle a purely static conception of light, a conception he takes Philoponus to be criticizing, offering instead a kinetic conception of light. \citet[26--30]{Sorabji:1987vn} takes a more accommodating view. While light does not move, Aristotle has a conception of the direction of light. Thus he gives the correct explanation of the lunar eclipse as the Earth's shadow cast upon the moon as opposed to the moon's occlusion by a third opaque body (\emph{Analytica Posteriora} \textsc{ii} 8 93\( ^{a} \)29---93\( ^{b} \)3). But how could there be cast shadows unless light had direction? And Aristotle describes reflection not as the reflection of visual rays (the usual model of perception underlying ancient geometrical optics such as Euclid's or Galen's), but as the reflection of the colored object's movement of a unified medium (\emph{De Anima} \textsc{iii} 12 435\( ^{a} \)5--10). But such movement must have direction if it is to be thus reflected. While Sorabji acknowledges that Aristotle recognizes the directionality of light, he claims that Aristotle lacks the means to explain it and sees Philoponus' \emph{theôria} as providing the wanted explanation. The present interpretation is more accommodating still. Light is a state of a potentially transparent medium and does not move but is constituted by the presence of the fiery substance. The being of the fiery substance consists in its incorporeal activity. This incorporeal activity has a direction of influence, a direction that the state of illumination determined by it inherits. Philoponus is perhaps more concerned than Aristotle to provide an alternative foundation for geometrical optics (see especially \emph{On} De Anima 331 1ff). But even here the roots of his strategy can be found in \emph{De Anima}. Whereas in Euclid's geometrical optics, the lines that are subject to geometrical reasoning are determined by visual rays emitted by the eye, Philoponus understands them instead as lines of influence proceeding from the perceived colored object. But that was implicit in Aristotle's remark about reflection (\emph{De Anima} \textsc{iii} 12 435\( ^{a} \)5--10). While there are novel elements of Philoponus' \emph{theôria} (such as the thermal effects of light, that light propagates in stages, and the application to incorporeal activity of concepts from his genuinely paradigm-shifting dynamics, on this last see \citealt{Kuhn:1962ss,Wolff:1987vn}), he is elaborating upon core ideas genuinely to be found in \emph{De Anima} (on the differences between Philoponus and Aristotle see \citealt{Groot:1983fa}).

Light is a state that the medium is in when it is actually transparent. Aristotle denies that light is fire, or a body, or an effluence (\emph{De Anima} \textsc{ii} 7 418\( ^{b} \)13--18). He denies as well that light moves, otherwise its motion would be visible as it travels from East to West (\emph{De Anima} \textsc{ii} 7 418\( ^{b} \)21--27). These claims are puzzling if by light Aristotle means, at least approximately, what we mean by light. But why assume that? 

Begin by focusing on Aristotle's claim that light is a state (\emph{hexis}, \emph{De Anima} \textsc{iii} 5 430\( ^{a} \)15) that a medium is in when it is actually transparent. Light could not be a body since the medium is a body and two bodies cannot occupy the same space (\emph{De Anima} \textsc{ii} 7 418\( ^{b} \)19). As \citet{Burnyeat:1995fk} has emphasized, state is really the wrong ontological category for light as we presently understand it to be. But now, in line with our avowed methodology, let us ask whether there could be a state that we can recognize on our present understanding that could reasonably be what Aristotle had in mind when he speaks of light? With the question so framed the resolution of our difficulties should be obvious. What state is a medium in when it is actually transparent, and where the persistence of this state depends on the continual presence and activity of a fiery substance? When it is illuminated, of course. By light, Aristotle means a state of illumination (see \citealt[122]{Thorp:1982fk}, for a similar interpretation). And that a medium when it is actually transparent is in a state of illumination sustained by the presence and activity of a fiery substance strikes me as a not unreasonable approximation of the truth. Moreover, it coheres well with the phenomenology of illumination. Consider what must have been the familiar experience of lighting an oil lamp to illuminate a room.

What about Aristotle's claim that light does not move? There are distinguishable aspects to Aristotle's overall case. That light moves is contrary to both reason and the observed facts (\emph{De Anima} \textsc{ii} 7 418\( ^{b} \)23).

Begin with light as conceived by Aristotle's opponents---as fire, body, or material effluence. With light so conceived, Aristotle's case is straightforwardly empirical. To conceive of light as fire, body, or material effluence is to conceive of light as being capable of locomotion---as potentially changing its location over time. However, we do not see light from a morning sunrise moving from East to West. And while movement across short distances may be too quick to be visible, Aristotle maintains that the corresponding claim is implausible given the magnitude of the distance involved. In this way is the hypothesis that light moves contrary to the observed facts. Though Aristotle's empirical argument fails (given his overconfidence in there being some magnitude over which motion would be perceptible, no doubt abetted by his conviction that every magnitude is perceptible at some distance), it remains an honorable failure. Aristotle's remarks, here, are best understood set against the Milesian tradition of preferring first-hand experience to the deliverances of authority (even where, as in the present case, the relevant authorities are not Hesiod and Homer, but Empedocles and Plato).

Not only is there an empirical argument that light---conceived as fire, body, or material effluence---does not move, but there is a distinct metaphysical argument that light---\-con\-ceived instead as a state (\emph{hexis})---cannot move. That light moves is contrary to reason as well as the observed facts. This latter argument is not empirical. Rather, reflection on the nature of a state reveals that it precludes space-occupancy. And if states do not occupy space, then they cannot change their locations over time and so are not susceptible to motion and indeed change more generally.  In this way it is contrary to reason to suppose that light, conceived as a state of illumination, moves. 

Aristotle distinguishes two ways in which something may move: 
\begin{quote}
	There are two senses in which anything may be moved either indirectly, owing to something other than itself, or directly, owing to itself. Things are indirectly moved which are moved as being contained in something which is moved, e.g. sailors, for they are moved in a different sense from that in which the ship is moved; the ship is directly moved, they are indirectly moved, because they are in a moving vessel. (Aristotle, \emph{De Anima} \textsc{i} 3 406\( ^{a} \)3--8; Smith in \citealt[9]{Barnes:1984uq})
\end{quote}
According to Aristotle's metaphysical argument, then, states, such as being illuminated or possessing the attribute of whiteness, are not directly susceptible to motion.

States and the bodies whose states they are differ in being:
\begin{quote}
	Whiteness will be different from what has whiteness. Nor does this mean that there is anything that can exist separately, over and above what is white. For whiteness and that which is white differ in definition, not in the sense that they are things which can exist apart from each other. (Aristotle, \emph{Physics} \textsc{i} 3 186\( ^{a} \)27--31; Hardie and Gaye in \citealt[6]{Barnes:1984uq})
\end{quote}
Being white differs in being from that in which the whiteness inheres, an opaque surface of a solid material body, say. Whiteness could not exist apart from something in which it inheres. But being white and that in which the whiteness inheres have different modes of being. This difference is spatially manifest. States differ in being from the bodies whose states they are in that states do not occupy space the way that bodies do. 

States do not occupy space. They lack location and are there\-by not directly susceptible to motion understood as locomotion or change in position. Indeed, states are not directly susceptible to motion even when understood more generally. In \emph{De Anima}, motion, \emph{kinēsis}, is Aristotle's general term for change of any kind. Each of the four varieties of change that Aristotle acknowledges (locomotion, alteration, growth, and decay) requires space occupancy for something to be subject to them (\emph{De Anima} \textsc{i} 3 406\( ^{a} \)12). Since states do not occupy space, they are directly susceptible to neither locomotion nor motion more generally:
\begin{quote}
	But if the essence of soul be to move itself, its being moved cannot be incidental to it, as it is to what is white or three cubits long; they too can be moved, but only incidentally---what is moved is that of which white and three cubits long are the attributes, the body in which they inhere; hence they have no place: but if the soul naturally partakes in movement, it follows that it must have a place. (Aristotle, \emph{De Anima} \textsc{i} 3 406\( ^{a} \)14--21; Smith in \citealt[9]{Barnes:1984uq})
\end{quote}
Bodies occupy space. They have location and are there\-by directly susceptible to locomotion and motion more generally. But a state of a body, such as possessing the attribute of whiteness, does not occupy space the way a body does, not even the space occupied by the body whose state it is. Since states do not occupy space, they lack locations and thus are not directly susceptible to locomotion and motion more generally. They are susceptible to locomotion or motion, at best, indirectly. States inhere in things capable of motion. Moreover, states, though incapable of direct motion, can, perhaps, inherit motion from the bodies whose states they are and so indirectly move. So understood, states would be passengers, and the bodies whose states they are the ships that carry them (though, as \citealt[174]{Witt:1995kx}, observes: ``Here the relationship is not one of parts to wholes, or contents to containers, but rather of inherent to subject''; \emph{Categoriae} \textsc{i}\( ^{a} \)24--5).

States differ in mode of being from that in which they inhere. The distinctive mode of being of states precludes space occupancy and hence being directly susceptible to locomotion and motion more generally. Since light is a state of illumination, light, so conceived, is precluded by its very nature, by being the kind of thing that it is, a state of a medium, from space occupancy, and, hence, locomotion. Light, conceived as a state of illumination, could not move, at least not directly. \citet[430 n29; appendix]{Burnyeat:1995fk} tries to make vivid the madness of these claims by quoting, at length, Prichard echoing them:
\begin{quote}
	I once made what I thought the unquestionable remark to a German mathematician who was also a physicist that only a body could move---so that, for example, the centre of gravity of a body or of a system of bodies, which is a geometrical point, could not move. He as I rather expected, thought I was just mad. In this case I should certainly have said I was certain that a centre of gravity could not move, and I think he would have said he was certain that it could. Here I personally should assert he could not possibly have been more uncertain that it could not, and that, if he had thought a bit more, he would have been certain that it could not; you cannot make a man think, any more than you can make a horse drink. \citep[99; this is just the initial paragraph of the material that Burnyeat quotes]{Prichard:1950tg}
\end{quote}
At least in my case, however, Burnyeat's \citeyearpar[430 n29]{Burnyeat:1995fk} rhetorical strategy backfired. Far from recognizing ``an eccentricity from the home of lost causes'' that ``would meet with Aristotle's approval'', I instead heeded Prichard's advice. And having thought a bit more, I now regard Aristotle's claim about states and space-occupancy to be \emph{prima facie} plausible, if controversial, in a way that his claim about the empirical significance of our failure to observe the motion of light no longer could be.

To fix ideas consider the following simple example. Consider walking down a corridor where the sole source of illumination is an oil lamp that you are carrying. Suppose the oil lamp is sufficiently bright to illuminate only a portion of the transparent medium that pervades the corridor, one third, say. At the beginning of your journey the first third of the corridor is illuminated, at the middle the second third, and the end only the final third of the corridor remains illuminated. As you travel over time different regions of the corridor are illuminated. But the illuminated region of the medium changing over time does not consist in a change in the position of the state of illumination. Rather, things with different positions, different regions of the transparent medium that pervades the corridor, are illuminated at different times. (\citealt[99]{Prichard:1950tg}, makes parallel remarks about wave movement.) A change of state and travel are different (\emph{De Sensu} \textsc{vi} 446\( ^{b} \)28).

One potential problem for the claim that states do not occupy space concerns the colors themselves. This would be ironic since one of Aristotle's own examples, possessing the attribute of whiteness, may itself be a counterexample to the claim that states lack location and hence are not directly susceptible to locomotion. Being white, possessing the attribute of whiteness, is a state potentially had by at least some opaque surfaces (as presented by the son of Diares, at least when viewed from a certain distance) and some radiant objects (such as the sun). But, it may be objected, being white is located, indeed, located in the opaque surface or radiant light source in which it inheres. Moreover, colors seem essentially extended. Only spatial magnitudes are colored. If something possesses the attribute of whiteness, then that state necessarily extends across some region---that part of the surface in which the whiteness inheres, say. However, it is possible to capture the intuitions that motivate these claims consistently with the denial that states occupy space. Thus colors are located only indirectly, in the sense that the particulars in which the colors inhere are located. When we ascribe location to a color we are merely representing the location of the particular, or at least that part of the particular that instantiates the color. It is the particulars that instantiate the colors and not the colors that are located. And colors are essentially extended only indirectly, in the sense that the particulars in which the colors inhere are extended. Colors are only instantiated by spatial magnitudes. It is the particulars that instantiate the colors and not the colors themselves that are essentially extended.

Prichard mentions, without directly addressing, another potential counterexample. A center of gravity is a state of at least some particulars. But the center of gravity of a body, or a system of bodies, has a definite location. Prichard's silence is not an expression of embarrassment in the face of recalcitrant evidence. It has another, rhetorical function. But consider how this potential counterexample might be explained away. Identifying the center of gravity of a body, or system of bodies, with some definite point within the interior is both vivid and informative. But it is in one way misleading. The center of gravity is a state of the entire body, or system of bodies, and not a proper part of it, and is explanatorily relevant to the entire body's. or system's, capacity for locomotion. Identifying the center of gravity with a point in the body's, or system's, interior is merely a representation of global state of the body, or system of bodies. Identifying the center of gravity of a body, or system of bodies, as a point in its interior may be vivid and informative, but that is consistent with the represented global state, a state enjoyed by the entire body, or system, not being the kind of thing that so much as could occupy space.

Light is a state that a potentially transparent medium is in due to the contingent presence and activity of the fiery substance. Light, as conceived by Aristotle, is a state of illumination. Being a state, light is not directly located. Light is, however, indirectly located in the transparent medium whose state it is, understood as a particular body of water or air. Light, lacking location, does not move. Similarly, the determinant of light, the form-giving fiery substance, does not propagate through the potentially transparent medium. Though the medium is extended, it is illuminated all at once. Aristotle will make this commitment explicit in \emph{De Sensu}:
	\begin{quote}	
		But with regard to light the case is different. For light is due to the presence of something, but it is not a movement. And in general, even in qualitative change the case is different from what it is in local movement. Local movements, of course, arrive first at a point midway before reaching their goal (and sound, it is currently believed, is a movement of something locally moved), but we cannot go on to assert this in like manner of things which undergo qualitative change. For this kind of change may possibly take place in a thing all at once, without one half of it being changed before the other; e.g. it is possible that water should be frozen simultaneously in every part. (Aristotle, \emph{De Sensu} \textsc{vi} 446\( ^{b} \)28-447\( ^{a} \)3; Smith in \citealt[63]{Barnes:1984uq})
	\end{quote}
Change of state and travel are different. Light from a radiant light source can pervade the whole of a transparent volume without first reaching a midway point. This would be problematic if light were due to the presence of the fiery substance and the fiery substance were a body. Such a body would have to travel at infinite speed to instantaneously traverse an extended region. Once the fiery substance is understood as an incorporeal activity, there no longer is any obstacle to thinking of this activity as occurring all at once throughout an extended region. Indeed, Philoponus takes this behavior as proof of the incorporeal nature of the fiery substance:
	\begin{quote}	
		If it were a body, again, how would it be possible for the movement of a body to occur thus all at once? For the sun comes above the horizon, and at that moment suddenly in no time at all the whole hemisphere above the earth has been lit up. And if I cover a lamp and bring it into the house and then take it out of the covering, the whole house is lit all at once. How could a body move thus in no time at all? (Philoponus, \emph{On} De Anima, 327 3--7; \citealt[11]{Charlton:2005fk})
	\end{quote}
The fiery substance, being incorporeal, can instantaneously illuminate entire transparent regions insofar as they are a unity. Being incorporeal, it does not travel, and so does not have to travel at infinite speeds to illuminate the unified whole all at once. 

The incorporeal activity of the fiery substance, a kind of rarefied burning, can, nevertheless, have a limited sphere of influence. Aristotle and Philoponus speak of the sun illuminating the sky, and Philoponus speaks of a lamp illuminating a house. But if that same lamp were taken out into a large enough field at night, while it would illuminate a region immediately surrounding the lamp, most of the field would remain in darkness. That observation is available to Aristotle, maintaining, as he does, that we can see distant fires in the dark (\emph{De Anima} \textsc{ii} 7 419\( ^{a} \)23--24). If light were a body that travelled, light's limited sphere of influence might be explained in terms of the resistance the medium offered to the propagation of light. But the supposition that light is a body is unnecessary for such an explanation. Aristotle acknowledges that a dense medium (\emph{Meterologica} \textsc{i} 5 342\( ^{b} \)5--8), such as a fog or cloud of smoke (\emph{De Sensu} \textsc{iii} 440\( ^{a} \)10--11), can result in a reduction of brilliance. A dense medium can result in a reduction of brilliance because it is not wholly receptive to activity of the fiery substance, when particles of earth are suspended in it, say. The fiery substance's limited sphere of influence need not be thought on the model of an increasing impediment to its propagation, light slowing and becoming weaker as it penetrates the dense darkness until it can no more. The fiery substance illuminates the entire region to the extent that it does all at once. Given the direction of influence of the incorporeal activity and decreased receptivity to its activity, the limited and variably illuminated sphere is instantaneously determined with need of neither propagation nor travel.

The metaphysical argument that states, such as being illuminated or being white, do not occupy space and so are not directly susceptible to motion and change more generally is philosophically significant when read in light of Empedoclean puzzlement about how the colors of remote external particulars could be present in visual consciousness. If colors are conceived as material effluences with distinctive magnitudes, then they occupy space and so may be in contact with the organ of sight. Recall one problem with Empedocles' theory of color vision was that colors, conceived as material effluences, are bodies, and body is really the wrong ontological category for chromatic attributes. If the color of a particular is not a body but instead a state, then since states lack location, it could not be in contact with the organ of sight. The colors of things are at best indirectly located, inheriting their location from the particulars in which they inhere. So the colors of things could at best be indirectly in contact with the sense organ, by the colored particular being in contact. But contact with a colored particular blinds the perceiver to the particular and its color. To be palpable is to be imperceptible.

The colors of things, being states of particulars, preclude, by their very nature, contact with the organ of sight. The colors of things, beings states, also preclude, by their very nature, not only the necessity of travel, but its possibility. On Empedocles's conception of color perception, the color of a distant particular travels to the perceiver so that it may be assimilated and so be made palpable to the organ of sense. That the colors of remote external particulars travel to the perceiver was the resolution of Empedoclean puzzlement, a resolution that is not a genuine metaphysical option if the colors of things are states. 

From the perspective of the metaphysical argument that states do not occupy space and so are not directly susceptible to locomotion and change more generally, effluences are not genuine candidates for being the colors since they belong to the wrong ontological category. Moreover, as we have seen, the problem that the identification of colors with effluences is meant to resolve is not genuine, since the color of a particular, being a state, could be in contact with nothing, at least not directly.

Progress is made with Empedoclean puzzlement when we recognize that an object's being white is not located, not even where the object's white parts are. For if the color of an object is not located, it need not, indeed could not, travel to act upon the organ of sight so that it may be seen. In \emph{Leviathan}, in a chapter approvingly cited by \citet[26 n7]{Burnyeat:1992fk}, \citet{Hobbes:1651fk} writes:
\begin{quote}
	But the Philosophy-schooles, through all the Universities of Christendome, grounded upon certain Texts of \emph{Aristotle}, teach another doctrine; and say, For the cause of \emph{Vision}, that the thing seen, sendeth forth on every side a \emph{visible species} (in English) a \emph{visible shew}, \emph{apparition}, or \emph{aspect}, or \emph{a being seen}; the receiving whereof into the Eye, is \emph{Seeing}. \ldots\ I say not this, as disapproving the use of Universities: but because I am to speak hereafter of their office in a Common-wealth, I must let you see on all occasions by the way, what things would be amended in them; amongst which the frequency of insignificant Speech is one. (Hobbes, \emph{Leviathan} \textsc{i}.1)
\end{quote}
That doctrine may have been taught in Philosophy schools in Universities throughout all of Christendom, and it may have been grounded in certain texts of Aristotle's (at least on a reading of them), but it is not Aristotle's doctrine. The color of an external particular, like the illumination of a transparent medium, is a state of that particular. It thus enjoys a mode of being that precludes space-occupancy and so could not be sent forth on every side.

% section transparency_in_de_anima (end)

\section{Transparency in \emph{De Sensu}} % (fold)
\label{sec:transparency_in_de_sensu}

In \emph{De Anima}, Aristotle defines the transparent as that which is visible, though not visible in itself, but owing its visibility to the color of another thing (\emph{De Anima} \textsc{ii} 7 418\( ^{b} \) 4--6). I have remarked that it might seem more natural to characterize transparency, not in terms of the manner of its visibility, but in terms of its being that through which remote objects appear---as a condition on the visibility of other things. However, this latter conception is not entirely absent in Aristotle. It is at least implicit in the corresponding discussion of color and transparency in \emph{De Sensu}.

In \emph{De Sensu} Aristotle sets out to explain what each of the sense objects ``must be in itself, in order to produce actual sensation'' (\emph{De Sensu} \textsc{iii} 439\( ^{a} \)11; Beare in \citealt[7]{Barnes:1984uq}). This is a further inquiry, not directly addressed by \emph{De Anima}. Unsurprisingly, then, there are novel elements to the \emph{De Sensu} discussion. Thus, novel claims that emerge include, for example, that color resides in the proportion of transparent that exists in all bodies, and an account of the generation of the hues in terms of the ratio of black and white in a mixture. Given these novel elements, the question arises whether \emph{De Sensu} represents an extension of the doctrines of \emph{De Anima}, or a change of mind. While there is some evidence that Aristotle has not completely harmonized new ideas with old, I believe that Aristotle meant to be offering an extension of the \emph{De Anima} account, and not a substantive revision of it. Or at any rate, this will be my working hypothesis (see \citealt{Kahn:1966zr} for discussion; see also \citealt[291]{Caston:2005cr} \citealt[37]{Nussbaum:1995ly}).

One novel element is the characterization of color as ``the limit of the transparent in determinately bounded body'' (\emph{De Sensu} \textsc{iii} 439\( ^{b} \)11; Smith in \citealt[8]{Barnes:1984uq}). This prompted the Renaissance commentator Jacopo \citet{Zabarella:1605kx} to complain that Aristotle has defined color twice over \citep{Broackes:1999uq}. However, there is no evidence in the text that Aristotle regarded this claim as a definition. Rather, it appears as the conclusion of an argument \citep[65]{Broackes:1999uq}. In that argument, Aristotle explains that color inheres not only in unbounded things, such as air and water, but in bounded things as well. What is the distinction between the bounded and the unbounded? The examples of the transparent are restricted in \emph{De Sensu} to air and water. On this basis, it might be thought, naturally enough, that that the distinction is between transparent liquids, like air and water, and opaque solid objects (\citealt[59]{Broackes:1999uq}, \citealt[131]{Sorabji:2004fk}). To describe liquids as unbounded is to highlight their lack of fixed boundaries. However, I doubt that is what Aristotle had in mind. In \emph{De Anima}, Aristotle claims that not only are liquids such as air and water transparent, but so are certain solid objects. He does not himself give examples of transparent solids. But glass, ice, crystals, tortoise shells, and certain animal horns would do, and we can be confident that Aristotle had first hand experience with at least some of these. It would do no good to object, as \citet[131]{Sorabji:2004fk} does, that the glass, say, that Aristotle would have encountered would not have been perfectly transparent. In \emph{De Sensu}, Aristotle emphasizes that transparency comes in degrees. The problem, then, is that any such example would possess fixed boundaries and yet would remain transparent, but the transparent is meant to be unbounded. 

What could the unbounded be if it is not simply the lack of fixed boundaries? I believe that good sense can be made of Aristotle's distinction if we understand it in phenomenological terms. Nontransparent bodies, such as opaque solids, are perceptually impenetrable. Unlike transparent bodies you cannot see in them or through them. Their surface is the site of visual resistance. Perceptual impenetrability determines a visual boundary through which nothing further can appear. Transparent bodies, in contrast, are perceptually penetrable. One can see in them and through them. The particulars arrayed in a transparent medium appear through that medium. The transparent is unbounded since it offers insufficient visual resistance to determine a perceptually impenetrable boundary. And this is true of transparent solids such as crystals and tortoise shells as well as transparent liquids such as air and water.

The transparent is unbounded since it offers insufficient visual resistance to determine a perceptually impenetrable boundary. Which is not, of course, to say that the transparent can offer no visual resistance. In \emph{De Sensu}, Aristotle emphasizes that transparency comes in degrees. When Aristotle speaks of color as the limit of the transparent in bounded bodies, he has in mind surface color. But he also speaks of the color of transparent media:
\begin{quote}
    Air and water, too are evidently coloured; for their brightness is of the nature of colour. But the colour which air or sea presents, since the body in which it resides is not determinately bounded, is not the same when one approaches and views it close by as it is when one regards it from a distance. (Aristotle, \emph{De Sensu} \textsc{iii} 439\( ^{b} \)1--3; Beare in \citealt[7]{Barnes:1984uq})
\end{quote}
Air and water, when transparent, are bright. And brightness, Aristotle claims, is of the nature of color. The attribution of brightness, however, requires attributing no particular hue to the medium. If the medium is perfectly transparent, then the only visible hues will be the colors of bounded particulars arrayed in that medium. But the next line contains the suggestion that imperfectly transparent media, while remaining perceptually penetrable to some degree, may themselves have a particular hue---in modern parlance, not a surface color but a volume color. From a cliff overhanging the sea, the sea may appear a clear blue even as one sees rocks lying below its surface. But, if enticed by the sea, one were to descend to the beach and examine a handful of sea water, it would not be blue at all, but transparent. Similarly, looking up at the sky on a clear autumn afternoon, one sees an expanse of blue. But if one were to travel to that region of the sky, by helicopter, say, nothing blue would be found. The implicit thought is that the visual resistance of an imperfectly transparent medium increases with an increase in volume. The further one sees into a transparent medium, the more resistance that medium offers to sight. And volume color is the effect of this resistance. Aristotle is explicit about the effects of such resistance in \emph{Meterologica}: ``For a weak light shining through a dense medium \ldots\ will cause all kinds of colours to appear, but especially crimson and purple'' (\emph{Meterologica} \textsc{i} 5 342\( ^{b} \)5--8; Webster in \citealt[8--9]{Barnes:1984uq}).

Aristotle's insight, here, reveals one respect in which Broad's \citeyearpar{Broad:1952kx} description of vision as ``saltatory'' is inapt. According to Broad, vision is saltatory in that it seems to leap the spatial gap between the perceiver and so reveal shapes and colors confined to a spatial region remote from the perceiver. Broad is emphasizing just the feature of color vision that generates Empedoclean puzzlement, that vision seems to present us with the colors of remote external particulars. However, sight does not leap the spatial gap between the perceiver and the color's instantiation, so much as the perceiver sees through the spatial gap. The colored particular's distance from the perceiver and the density of the intervening medium could only make a difference to visual appearance if the perceiver were seeing through the medium to the distal particular. We not only see the colors of distant particulars, but we do so by seeing through intervening illuminated media. Broad is right to emphasize the distal character of the objects of vision, but his description of vision as saltatory is inapt since it fails to heed the perceptual penetrability of the intervening medium. Vision would leap the gap between the perceiver and the distal color if the object of visual awareness were confined to the remote spatial region where the color is instantiated. However, vision does not leap the gap between the perceiver and distal color. Rather, by means of it, the perceiver may peer through the intervening medium if it is transparent at least to some degree.

\citet[130--131]{Sorabji:2004fk} offers a different interpretation of the color of the sea. The color of the sea is a \emph{borrowed} color, due to reflection. The color of the sea is borrowed in the sense that the source of its color lies not within itself but in another thing, the sky whose color is reflected therein. Sorabji's suggestion, whether or not it is of genuine Aristotelian provenance, is at least endorsed by Al-Kind\={i} in a work overtly influenced by Aristotle: 
\begin{quote}
	We say: we find that water is free from impurities takes on every colour adjacent to it. Since it is transparent, it has no colour. If the colours sensed along with it [sc. the water] belonged to it, then it would not change its colour to the colour of what is adjacent to it, whenever something is adjacent to it. Therefore it [sc. the water] shows us whatever is adjacent to it, since the body of [the water] is neither acting as a screen, nor does it have colour. (Al-Kind\={i}, \emph{On the Body that by Nature Brings Colour and is One of the Four Elements, and which is the Cause of the Colour in Things other than Itself}, 11; \citealt[138]{Adamson:2012fk})
\end{quote}
This interpretation has the virtue of cohering with what we know about the color of large bodies of water, that their chromatic appearance is affected by the color of the sky reflected in them. However, if generalized to Aristotle's other examples of the transparent, it does less well---the color of the sky is not itself explained in terms of reflection, unless reflection is understood liberally enough to include diffraction (and, indeed, Al-Kind\={i} provides an alternative explanation of the blue of the sky in terms of particles of earth suspended in the air, \emph{On the Cause of the Blue Colour that is Seen in the Air in the Direction of the Sky, and is Thought to be the Colour of the Sky}, \citealt[139--143]{Adamson:2012fk}; On color and Al-Kind\=i see \citealt{Adamson:2006ys}). In support of this interpretation, \citet[130]{Sorabji:2004fk} refers us to \emph{Meterologica} \textsc{i} 5 and \textsc{iii}.2--6 where ``Aristotle cites reflection as the cause of various colours in the clouds as well as of such other optical effects as rainbows, haloes, mock suns, and rods.'' However, in \emph{Meterologica} \textsc{i} 5 Aristotle distinguishes two causes of color: (1) the visual resistance offered by an imperfectly transparent medium and (2) reflection. It is both weak light shining through a dense medium and air when it acts as a mirror that causes all kinds of colors to appear. And while the colors of clouds as well as other optical effects such as rainbows, haloes, mock suns, and rods may, by Aristotle's lights, be explicable in terms of reflection, this goes nowhere towards showing that Aristotle thought that the color of the sea is due to reflection. He never explicitly says that it is, but he does explicitly link the color of the sea to the visual resistance it offers both in \emph{De Sensu} and in \emph{Meterologica}. Lying behind this disagreement is a disagreement about how to understand the unbounded. \citet{Sorabji:2004fk}, like \citet{Broackes:1999uq}, understands the unbounded as the lack of fixed boundaries rather than being perceptually penetrable. This is manifest in the difficulty \citet[131]{Sorabji:2004fk} finds in reconciling his interpretation with Aristotle's claim that sea color varies with distance because it is unbounded, a difficulty that is avoided if the unbounded is instead understood in perceptual terms as I recommend.

The color of an imperfectly transparent medium does not occlude the bound\-ed particulars arrayed in it. But the color of the transparent medium may affect their color appearance. Thus the sun, which in itself appears white, takes on a crimson hue when seen through a fog or cloud of smoke (\emph{De Sensu} \textsc{iii} 440\( ^{a} \)10--11; \emph{Meterologica} \textsc{i} 5 342\( ^{b} \)18-21). This might be what Aristotle has in mind when he claims that bounded particulars have a fixed color unless affected by atmospheric conditions (\emph{De Sensu} \textsc{iii} 439\( ^{b} \)5--7). The color of a bounded particular will affect the medium differently depending on its degree of perceptual penetrability and resulting volume color. Notice, considered in and of itself, this claim implies at most that the color of the sun appears differently when obscured by a fog or cloud of smoke. There need be not commitment to the sun changing color from white to red when so obscured, nor its appearing to so change. Aristotle's position allows for the possibility of a variation in color appearance without a variation in presented color. Notice the thought that the state of a medium can alter the appearance of a sensible object without a variation in the object of sense is what animates Austin's \citeyearpar{Austin:1962lr} use of the Platonic example of a straight stick looking bent in water (Plato, \emph{Republic} \textsc{x} 602\( ^{c} \)--603\( ^{a} \); on Austin see Kalderon and Travis \citeyear{Kalderon:2010fk} and Martin \citeyear{Martin:2000nx}; on Austin and the argument from conflicting appearances see Burnyeat \citeyear{Burnyeat:1979mv}).

This is potential evidence about Aristotle's attitude towards the argument from conflicting appearances. While the argument from conflicting appearances is discussed in \emph{Metaphysica} \( \Gamma \), discussion of it is largely absent in \emph{De Anima} and \emph{De Sensu}. While largely absent from \emph{De Anima} and \emph{De Sensu}, it is not entirely absent, and I believe we have an important point of contact here. Looking up from a battlefield one sees the sun burning white. As smoke from the battle obscures the sun, it takes on a crimson hue. Nothing can be red and white all over at the same time. Supposing, as is plausible, that the smoke from the battle did not alter the sun's color so that the color of the sun remains constant through the variation in its appearance, it might seem as if at least one of these appearances were illusory. However, if there can be a variation in color appearance without a variation in presented color, then the white and red appearances do not conflict. The color of the sun does not appear to change from white to red. Red is simply the way radiant white things appear when viewed through smoke filled media (just as bent is the way that straight things look when viewed through refracting media---see Plato, \emph{Republic} \textsc{x} 602\( ^{c} \)--603\( ^{a} \); \citealt{Austin:1962lr}). 

In \emph{Metaphysica} \( \Gamma \) Aristotle expresses a complementary attitude:
\begin{quote}
	Again, it is fair to express surprise at our opponent's raising the question whether magnitudes are as great, and colors are of such a nature, as they appear to people at a distance, or as they appear to those close at hand and whether they are such as they appear to the healthy or to the sick, and whether those things are heavy which appear so to the weak or those which appear so to the strong, and those things which appear to the sleeping or to the waking. For obviously, they do not think these to be open questions. (Aristotle, \emph{Metaphysica} \( \Gamma \) 5 1010\( ^{b} \)3--9; Ross in \citealt[55]{Barnes:1984kx})
\end{quote}
Color appearance can vary with viewing distance. But the variable color appearances evidently do not conflict. If the variable color appearances were in conflict then it would make sense to ask which, if any, of these conflicting appearances are veridical. Aristotle denies, however, that this is an open question. And Aristotle's denial, here, is the expression of his conviction that the variable appearances do not genuinely conflict. Suppose color can appear differently when seen near and when seen far. These variable color appearances would not conflict if the difference in appearance were not a matter of what is presented in sensory experience. Suppose, instead, the difference in appearance were just the same color appearing differently. In seeing the color near and far, the subject perceives the color, it is present in their visual experience. It is just that the color is presented differently in the different circumstances of perception. Seen near, it is presented one way, seen far, it is presented another. If the difference were a matter of the presentation of incompatible colors, there would be a conflict between appearances. So the difference must be understood in some other way, not a difference in the object of sensory experience so much as a difference in the way that object is presented in sensory experience. Aristotle's example in \emph{Metaphysics} \( \Gamma \) is a case of color constancy, just as Plato's example of the straight stick looking bent in water is a case of shape constancy. Aristotle's insight, echoed by Austin in \emph{Sense and Sensibilia}, is that the variable appearances in cases of perceptual constancy are incapable of genuine conflict. (Further evidence about Aristotle's views on perceptual constancy will be discussed in chapter~\ref{sub:overlap}.)

I have argued that the color of the transparent medium may not occlude the colors of the particulars arrayed in it though it may affect their color appearance. Against the present interpretation it might be objected that Aristotle makes a claim about the color of the transparent that conflicts with it. Thus Aristotle claims that the transparent lacks color and so is receptive to color (\emph{De Anima} \textsc{ii} 7 418\( ^{b} \)26--29). The force of this objection is mitigated somewhat by the recognition that Aristotle seems to make inconsistent claims about the color of the transparent. Thus he claims that:
\begin{enumerate}[(1)]
	\item Light, or brightness, is the color of the transparent. (\emph{De Anima} \textsc{ii} 7 418\( ^{b} \)11-12; \emph{De Sensu} \textsc{iii} 439\( ^{b} \)1--2)
	\item The transparent is seen to have different colors when near and far. (\emph{De Sensu} \textsc{iii} 439\( ^{b} \)2--3)
	\item The transparent lacks color and so is receptive to color. (\emph{De Anima} \textsc{ii} 7 418\( ^{b} \)26--29)
\end{enumerate}
How might (1)--(3) be interpreted so as to be consistent? We have already observed that the attribution of brightness requires attributing no particular hue to the transparent medium. Moreover, since the medium is transparent, the color of the remote particular appears through that medium. This may even be so in an imperfectly transparent medium, one such that owing to the resistance it offers to vision itself appears a certain volume color. The color of a remote particular may appear differently when viewed through perfectly and imperfectly transparent media, but the volume color, if any, of the transparent medium does not occlude the surface color of the remote bounded particular. But so long as the surface color of the remote bounded particular is not occluded by varying the color of the medium as it volume varies, the transparent medium remains receptive of that color. If, however, the medium were to become perceptually impenetrable and so take on a surface color, the color of the remote bounded particular would be occluded and the medium would no longer be receptive to color. The denial in (3) is the denial of surface color to transparent media, but that is consistent with imperfectly transparent media, such as the sea and the sky, having volume color. Properly interpreted, (1)--(3) are consistent.

There is thus a progression of qualitative states from the perfectly transparent to the colored and opaque. The qualitative states in the progression are ordered by their decreasing degree of perceptual penetrability culminating in the perceptual impenetrable. It is thus a progression to a limit. We can envision the progression from perfect transparency in the following manner. Consider a tank of clear water into which is poured a blue dye. Suppose the absorption rate of the dye is too quick to be visible. So we do not see clouds of blue dye propagating through the clear liquid; rather, we see the volume taking on the blue and become increasingly opaque. At the end of this progression, the tank is surface blue---no thing can appear in it or through it. Color, that is surface color, is in this sense the limit of the transparent---it is the terminal qualitative state of a progression of qualitative states ordered by decreasing degree of perceptual penetrability.

One may be forgiven for thinking that Aristotle has fallen into a category mistake in speaking of color as the limit of the transparent \citep[65]{Broackes:1999uq}. He seems, on the surface, to be making an identification, but color is a quality in the way that a limit could not be. However, on the interpretation that I have been urging, Aristotle is not identifying color qualities with limits. Rather, in the progression of qualitative states from the perceptually penetrable to the perceptually impenetrable, color (that is, surface color) is the terminal qualitative state. This is one way of understanding Aquinas, in his commentary on \emph{De Sensu}, when he writes:
\begin{quote}
	Thus color is not in the category of quantity---like surface, which is the limit of a body---but in the category of quality. The transparent is also in the category of quality, because \emph{a limit and that of which it is the limit belong to one category}. [my emphasis] (\emph{Sententia De Sensu Et Sensato} \textsc{v}, commentary on \emph{De Sensu} \textsc{iii} 439\( ^{b} \)11 in \citealt{White:2005vn})
\end{quote}

In \emph{De Sensu}, Aristotle not only speaks of the limit of the transparent but also of the limit of a body: The limit of a body is its external surface, a bulgy two-dimensional particular, in Sellars' \citeyearpar[\textsc{iv} 23]{Sellars:1956xp} apt phrase. \citet[\textsc{iv} 23]{Sellars:1956xp} explains that it is two-dimensional in the sense that ``though it may be \emph{bulgy}, and in \emph{this} sense three-dimensional, it has no \emph{thickness}''. Color lies at the limit of the body, and this, Aristotle claims, encouraged the Pythagoreans to call the surface of a body its color. In so doing, however, the Pythagoreans undertook a further commitment: Color not only lies at the limit of a body, but color is itself the limit. In calling the surface of a body its color, the Pythagoreans identify color with the limit of the body. However, while color may lie at the limit of the body, color is not itself the limit:
\begin{quote}
	For [colour] is at the limit of the body, but it is not the limit of the body; but the same natural substance which is coloured outside must be thought to be so inside too. (Aristotle, \emph{De Sensu} \textsc{iii} 439\( ^{a} \)32--439\( ^{b} \)35; Beare in \citealt[7]{Barnes:1984uq})
\end{quote}
Aristotle's opposition to the Pythagorean conception of color is elaborated by Sellars two millennia hence:
\begin{quote}
	Certainly, when we say of an object that it is red, we commit ourselves to no more than that it is red ``at the surface''. And sometimes it is red at the surface by having what we would not hesitate to call a ``part'' which is red through and through---thus, a red table which is red by virtue of a layer of red paint. But the red paint is not itself red by virtue of a component---a `surface' or `expanse'; a particular with no thickness---which is red. \citep[\textsc{iv} 23]{Sellars:1956xp}
\end{quote}
It is thus misleading, I believe, for \citet{Silverman:1989ve} to liken colors, as conceived by Aristotle, to Sherwin-Williams paint.

\label{actual_potential} Does the consideration that tells against color being the limit of the body tell equally against color being the limit of the transparent? Not obviously. Opaque solids are perceptually impenetrable, and their perceptual impenetrability determines a visual boundary through which nothing further can appear. That is what their opacity consists in. This visual boundary coincides with the limit of the body. This could only seem inconsistent with the claim that the same nature which exhibits color outside also exists within if one ignored Aristotle's reminder at the opening of \emph{De Sensu} that ``each of them may be spoken of from two points of view, i.e., either as actual or as potential'' (\emph{De Sensu} \textsc{iii} 439\( ^{a} \)12--13; Beare in \citealt[7]{Barnes:1984uq}). Aquinas insightfully heeds this reminder. In his commentary on \emph{De Sensu} he writes that ``bodies have surface in their interior in potentiality but not actuality'' (\emph{Sententia De Sensu Et Sensato} \textsc{v}, commentary on \emph{De Sensu} \textsc{iii} 439\( ^{b} \)11 in \citealt{White:2005vn}). When the perceptually impenetrable is actually resisting sight a visual boundary is determined at the limit of the opaque body. But that a portion of the interior of such a body offers no such visual resistance in being occluded from view is consistent with its being perceptually impenetrable, with its potentially determining such a visual boundary.

Another consideration is relevant here. The limit of the transparent is a qualitative state. However, as Aquinas observed, the limit of a body is not a qualitative state; the limit of a body belongs, rather, to the category of quantity (compare \emph{Metaphysica} \( \Delta \) 13, 17). An argument to the conclusion that color is not a species of quantity---in the present instance, the limit of a body---does not by itself constitute an argument against the claim that color is a qualitative state distinguished by its place in an ordering of qualitative states.

In his discussion of the unbounded, then, there are thus two notions of limit in play. Aristotle distinguishes:
\begin{enumerate}[(1)]
	\item the limit of the transparent
	\item the limit of a body
\end{enumerate}
These are distinct limits. Whereas the former is qualitative, the latter is quantitative. However, importantly they coincide. A bounded body, in being perceptually impenetrable, determines a visual boundary that coincides with the limit of the body. Moreover, Aristotle's claim, that Zabarella mistakes for a definition, that color is the limit of the transparent in a determinately bounded body gives expression to just this coincidence. Color, that is, surface color, is the limit of the transparent in being the terminal qualitative state in a progression of qualitative states ordered by decreasing perceptual penetrability. A determinately bounded body is one such that, being perceptual impenetrable, determines a visual boundary through which nothing further may appear. This visual boundary is spatially coincident with the limit of the body and is where the body's surface color is seen to inhere.

Aristotle's discussion of transparency and the unbounded is evidence that, despite his defining transparency in terms of the manner of its visibility, he retains a conception of the transparent as that in which and through which remote objects may appear, as a condition on the visibility of other things. That conception, in the guise of perceptual penetrability, is central to Aristotle's understanding of the unbounded. Two observations are relevant. First, given our working hypothesis that \emph{De Sensu} is to be read as an extension of the \emph{De Anima} account and not a substantive revision of it, we can assume that this conception is meant to be at least consistent with the \emph{De Anima} definition.  Second, Empedoclean puzzlement about the sensory presentation of remote objects highlights the way in which perceptual penetrability of transparent media is a remarkable fact. It is a remarkable fact. Moreover, in not defining transparency as that in which and through which remote objects may appear, Aristotle arguably acknowledges that it is. That the colors of remote objects are seen through transparent media is a fact to be explained. And if the nature of the transparent is to play a role in that explanation, the transparent must be defined in some way other than as being a condition on the visibility of remote objects. The explanation is given in \emph{De Anima}---in terms of the way in which color alters the transparent and the role that alteration plays in the exercise of our perceptual capacities. 

% section transparency_in_de_sensu (end)

% chapter transparency (end)
