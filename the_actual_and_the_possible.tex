%!TEX root = /Users/markelikalderon/Documents/Git/formwithoutmatter/aristotle.tex
\chapter{Two Transitions to Actuality} % (fold)
\label{cha:two_kinds_of_potentiality}

\section{The Problem} % (fold)
\label{sec:the_problem}

The eyes, in seeing, are altered. They are filled with light. The existence and character of the interior illumination depends upon and derives from the existence and character of exterior illumination, itself affected by the colors of remote external particulars. Moreover, the eyes in seeing must be altered. The eyes are the organ of sight, and seeing is a reactive capacity. Sight only acts by reacting to the colors of remote external particulars. So the organ of sight must be acted upon in order to exercise the capacity for sight. Notoriously, however, Aristotle denies that the exercise of our capacity for sight, an episode of seeing, is itself an alteration. So the eyes, in seeing, must be altered, but seeing is not itself an alteration or, at least, it is not an alteration of the usual sort. What kind of change does a perceiver undergo when seeing? 

The avowed task of \emph{De Anima} \textsc{ii} 5 is to say, in a preliminary fashion, what the senses are in general. A discussion of the special senses follows in subsequent chapters, and it is only in \emph{De Anima} \textsc{ii} 12 that Aristotle give the official definition of perception as the assimilation of sensible form without the matter of the perceived particular. Despite the avowed task of the chapter, most of the discussion is bound up with a number of careful distinctions, and we only get the general characterization of sensation at the end of the chapter. 

% section the_problem (end)

\section{The \emph{Endoxa}} % (fold)
\label{sec:the_endoxa}

The chapter begins with a discussion of the \emph{endoxa}:
\begin{quote}
	Sensation depends, as we have said, on a process of movement or affection from without, for it is held to be some sort of change of quality. Now some thinkers assert that like is affected only by like. In what sense this is possible and in what sense impossible, we have explained in our general discussion of acting and being acted upon. (Aristotle, \emph{De Anima} \textsc{ii} 5 416\( ^{b} \)33--417\( ^{a} \)2; Smith in \citealt[29]{Barnes:1984uq})
\end{quote}
Aristotle is picking up on his previous discussion of the \emph{endoxa} from \emph{De Anima} \textsc{i} 5 410\( ^{a} \)24--410\( ^{b} \)14. That perception depends upon being moved and acted upon was attributed to previous thinkers in \emph{De Anima} \textsc{i} 5 410\( ^{a} \)25--26. That claim, at least in the present context, is importantly qualified. Perception is not identified with being moved and acted upon. It is only said to depend upon being moved and acted upon. Though perhaps a widely held opinion, not all previous thinkers assented to it. Notoriously, in the Secret Doctrine that Socrates attributes to Protagoras (\emph{Theaetetus} 156\( ^{a-c} \)), perception is represented as the outcome of active and passive forces in conflict and so no mere passive effect. And this is perhaps echoed in Plato's own account of perception as the outcome of emanations from the perceiver and the object perceived coalescing (\emph{Timaeus} 45\( ^{b} \)--46\( ^{a} \)). In an earlier chapter, Aristotle suggested that perception was a qualitative alteration (\emph{De Anima} \textsc{ii} 4 415\( ^{b} \)24). That claim reappears though again in qualified form. Smith translates the qualification as ``some sort of change of quality'', but as \citet[36--37]{Burnyeat:2002an} importantly points out, the qualification can also be understood as ``a change of quality of a sort'', that is, as an alteration only in an etiolated sense. In the first line of the passage the claim that perception is a sort of alteration, or alteration of a sort, is offered as a justification for thinking that perception is a way of being moved and acted upon. That justification draws, implicitly, upon Aristotle's physics. In \emph{Physica} \textsc{vii} 1 and \textsc{viii} 4, Aristotle argues that every case of motion involves being acted upon, and in \emph{De Generatione et Corruptione} \textsc{i} 6 323\( ^{a} \)13--24 he argues that alteration always involves being acted upon. The attribution of the claim that perception is some sort of alteration, or alteration of a sort, to previous thinkers is doubtful, however, as it makes essential use of the classificatory scheme of the \emph{Categoriae}. Here, as is not uncommon, Aristotle is endeavoring to understand the views of his predecessors in his own terms. Aristotle may be over-reading, or having dialectical rather than strictly historical concerns, but, to be fair, the connection with qualitative alteration is not far to seek insofar as previous thinkers maintained, as well, that like is affected only by like.

The like-by-like principle, that like is affected only by like, is more genuinely of pre-Aristotelian provenance. Empedocles, among others, is said to have held it (\emph{De Anima} \textsc{i} 2 404\( ^{b} \)7--15; though whether he actually held the like-by-like principle is controversial, see \citealt{Kamtekar:2009fk}). Concerning ``in what sense this is possible and in what sense impossible'', Aristotle refers us to \emph{De Generation et Corruptione} \textsc{i} 7 (on Aristotle's use of cross-referencing in this chapter see \citealt{Burnyeat:2002an}). There, Aristotle explains that there is a sense in which those who maintain that like is affected by like and those who maintain, instead, that like is unaffected by like are both right. Each position, though seemingly contradictory, capture part of a more complex truth (\emph{De Generatione et Corruptione} \textsc{1} 7 323\( ^{b} \)15--324\( ^{a} \)9). Consider the case of one thing acting upon another thus inducing a change of quality, say, fire heating a pot of cool water. (My choice of example, though drawn from \emph{De Generatione et Corruptione}, is far from innocent---in \emph{De Anima} \textsc{ii} 12 Aristotle will contrast a plant being warmed with an animal's sensation of heat.) At the start of the process, the agent and patient of the change, the fire and the water, have contrary qualities from the same range of qualities---the fire is hot and the water is cool and each is a quality of temperature. They are thus generically alike. Though they are generically alike prior to the alteration they are also unlike. The distinct qualities they possess from the same range are contraries. They need not be opposites, qualities at the extreme ends of an ordered range of qualities, they may be intermediate qualities from the same range, so long as they are contraries. Thus the fire and the water, while generically alike in being the kinds of thing that have temperature, are specifically unlike in possessing contrary qualities of temperature. When the fire comes into contact with the pot of water they interact in such a way that the water becomes hot and so like the way the fire was prior to the qualitative alteration. So the water was initially unlike the fire that heated it in being cool but became like that fire in becoming hot. 

In \emph{De Generatione et Corruptione} Aristotle argues against the like-by-like principle as follows:
\begin{quote}
	Moreover, if like can be affected by like, a thing can also be affected by itself; and yet if that were so---if like tended in fact to act \emph{qua} like---there would be nothing indestructible or immovable, for everything would move itself. (Aristotle, \emph{De Generatione et Corruptione} \textsc{i} 7 323\( ^{b} \)21--23; Joachim in \citealt[23]{Barnes:1984uq})
\end{quote}
A variant of this argument occurs at this point in \emph{De Anima} \textsc{ii} 5 (discussed earlier in chapter~\ref{sub:particular}): 
\begin{quote}
	Here arises a problem: why do we not perceive the senses themselves, or why without the stimulation of external objects do they not produce sensation, seeing that they contain in themselves fire, earth, and all the other elements, of which---either in themselves or in respect of their incidental attributes---there is perception? It is clear that what is sensitive is so only potentially, not actually. The power of sense is parallel to what is combustible, for that never ignites itself spontaneously, but requires an agent which has the power of starting ignition; otherwise it could have set itself on fire, and would not have needed actual fire to set it ablaze. (Aristotle, \emph{De Anima} \textsc{ii} 5 417\( ^{a} \)3--9; Smith in \citealt[29]{Barnes:1984uq})
\end{quote}
If the like-by-like principle were true, then the sense organ, being like itself---as, indeed, is everything else---would act upon itself. The exercise of the capacity for sight, which the eye endows its possessor with, would not require an external particular to activate it. However, as \citet[226--227]{Polansky:2007ly} observes, the argument, when restricted in this way to perception, accrues new significance. If the sense organ acts upon itself, then the exercise of the capacity it endows its possessor with would at best be a mode of self-consciousness. It would not be a mode of sensitivity to external particulars and their sensible qualities and thus could not be a mode of perception. The puzzle, as it arises in the context of \emph{De Anima} \textsc{ii} 5, thus concerns the very possibility of perception.

The puzzle about the like-by-like principle in the perceptual case is epistemically significant since it concerns the very possibility of perception. Perception must be a reactive capacity, if it is to be perception at all, otherwise it would not be a mode of sensitivity to remote external particulars arrayed in the natural environment. However, being a reactive capacity is merely a necessary condition for perception. The objectivity of perceptual content depends not only upon perception only ever acting by reacting to the presence of external particulars but also on the perceiver, or perhaps their perceptual experience, becoming like those external particulars actually are. Sheer receptivity is insufficient for sensory presentation. It requires, as well, the assimilation of the sensory object.

Not only does the puzzle about the like-by-like principle accrue new significance in \emph{De Anima} \textsc{ii} 5, but it is susceptible to a novel solution as well. Nowhere in \emph{De Generatione et Corruptione} \textsc{i} 7 does Aristotle explicitly appeal to the distinction between the actual and the potential. But that distinction is essential to the present resolution of the puzzle. The lesson of the puzzle about perception raised by the like-by-like principle is that the capacity for sight, the form and substance of the eye, is a kind of potentiality, one that requires an external particular, the object of perception, to act upon the eye in order for that capacity to be exercised in seeing that object. It is like fuel which requires an external fire for it to ignite. Perception, as we have seen, is essentially a reactive capacity, one which requires an external object to ignite sensory consciousness.

Perception is a power or potentiality that requires an external particular for it to be realized. The exercise of perceptual capacities is, at least in this regard, like the case of motion, studied in \emph{Physica}, since all motion requires a mover:
\begin{quote}
	To begin with let us speak as if there were no difference between being moved or affected, and being active, for movement is a kind of activity---an imperfect kind, as has elsewhere been explained. Everything that is acted upon or moved is acted upon by an agent which is actually at work. Hence it is that in one sense, as has already been stated, what acts and what is acted upon are like, in another unlike; for the unlike is affected, and when it has been affected it is like. (Aristotle, \emph{De Anima} \textsc{ii} 5 417\( ^{a} \)15--21; Smith in \citealt[30]{Barnes:1984uq})
\end{quote}
Aristotle invites us to speak as if being moved and acted upon, on the one hand, and perceptual activity such as seeing and hearing, on the other, were the same. He is not identifying perception with being moved or acted upon, but merely emphasizing the fact that, like the motion of natural bodies discussed in \emph{Physica}, the activity of the senses depends upon an external cause---the external particular mediately acting upon the organ of sensation.

In \emph{Physica} \textsc{iii} 1 201\( ^{a} \)10--11, Aristotle defines motion as the actuality of the potential as such (see also \emph{Metaphysica} \textsc{k} 9 1065\( ^{b} \)16). Against Parmenidean skepticism, Aristotle insists upon the reality of change. Becoming pertains to actuality, but, in a concession to Parmenidean skepticism, it does so in a qualified manner. Consider again our example of qualitative alteration, fire heating a pot of cool water. At the beginning of this process, the water is cool but potentially hot. At the end of this process, the water is no longer potentially but actually hot. When the process has finished and the water is actually hot, the water is no longer becoming hot. It is this transition from merely being potentially hot, possessed by the water prior to its contact with the heat of the flame, to being actually hot which is the motion of qualitative alteration. This is something that actually happens, hence the use of actuality in Aristotle's definition of motion. But throughout the process, until its termination, the water remains potentially hot. That is why motion is the actuality of the potential as such. When the water is actually hot it is no longer potentially hot. The alteration that the water undergoes is the actualization of its potentially being hot, a potentiality that is both realized and exhausted in the process. (That the relevant sense of potentiality is potential for being rather than changing see \citealt{Kosman:1969aa}; though see \citealt{Heinaman:1994aa} for criticism.)

We are now in a position to understand the sense in which the motion involved in qualitative alteration is imperfect or incomplete. It is incomplete in the sense that when the terminal state has been reached, the process of alteration is no longer happening, and so long as it is happening, the terminal state has not been reached. The incompleteness of motion is linguistically manifest in our inability to simultaneously use present and perfect tenses to describe that motion. Having built a house, the builder is no longer building. And so long as the builder is building the house has yet to have been built (\emph{Metaphysica} \( \Theta \) 6 1048\( ^{b} \)18--34). This would contrast with an activity whose end lies not without but is immanent in the activity itself. Aristotle marks such a contrast at the opening of the \emph{Ethica Nicomachea} \textsc{i} 1 1094\( ^{a} \)3--5, and there is an extended discussion in \emph{Metaphysica} \( \Theta \). Aristotle does not explicitly describe these activities as complete. However, if we follow the post-Aristotelian vocabulary, then complete activity, activity whose end is immanent in the activity itself, is linguistically manifest in the simultaneous availability of the present and perfect tenses in describing that activity. Having seen the white of the sun, the perceiver may yet be seeing it still. And in seeing the white of the sun, the perceiver has seen it. One is enjoying and has enjoyed. The intelligibility of conjoining the present and perfect tenses indicates that the described activity is an end in itself, in contrast to the case of motion which is always progress towards an end that lies without.

At the end of the passage, Aristotle offers a resolution of the apparent conflict between those who maintain that like affects like and those who maintain that like is unaffected by like that echoes the resolution offered in \emph{De Generatione et Corruptione} \textsc{i} 7. There is a sense in which both opinions are correct. The water, at the beginning of the process, is cool and so unlike the heat of the flame. But when the process is completed and the water's potential for being hot is realized, the water is like the way the fire was prior to the alteration. 

We have seen how perceptual activity is like qualitative alteration in that it requires an external object to act upon the sense organ to initiate that activity. The present resolution of the apparent conflict in the \emph{endoxa} hints at a further analogy. Applying it to the case of perception yields the following result. Prior to seeing the brilliant white of the sun, the perceiver, or perhaps their perceptual experience, is unlike the sun. But in coming to see the sun's brilliant whiteness, the perceiver, or perhaps their perceptual experience, becomes like the way the sun was prior to seeing it. The perceiver, or their perceptual experience, is potentially like the sensible object, and when they undergo the perceptual experience of that sensible object their experience becomes actually like the way the sensible object was prior to perception. We have here an anticipation of the general characterization of perception that occurs at the end of \emph{De Anima} \textsc{ii} 5 and the definition of perception that occurs in \emph{De Anima} \textsc{ii} 12.

Aristotle's discussion of the \emph{endoxa} provides insight into how he is understanding assimilation as it figures in his definition of perception. In cases of qualitative alteration, at the end of the process, the patient assimilates the qualitative character of the agent of that change. The assimilation, here, is no material assimilation. Contrast the Empedoclean story of the movement of effluences through passages. That is straightforwardly a case of material assimilation---a material body, an effluence, is received within another body through a fine passage. In the case of qualitative alteration, as Aristotle understands it, nothing material is assimilated. What is assimilated is not a body, but a quality or state. And as I have observed in chapter~\ref{sec:transparency_in_de_anima}, whereas bodies have locations, qualities and states do not. We have here an important first step towards the resolution of Empedoclean puzzlement. Perceptual activity may not be an instance of qualitative alteration. Nevertheless, as in the case of alteration, the perceiver, or perhaps their experience, becomes like the way the perceived object was prior to perception. The perceiver, or their experience, assimilates the sensible object in a manner necessarily distinct from the material assimilation of that object, since to be palpable is to be imperceptible.

% section the_endoxa (end)

\section{Distinctions and Refinements} % (fold)
\label{sec:distinctions_and_refinements}

But we are getting ahead of ourselves. Before Aristotle offers the general characterization of perception, he undertakes to investigate the sense of potentiality involved in being a perceiver. Is the realization of this potential incomplete, in the way that alteration is? Or is the end of sight immanent in seeing? If the latter, then this would explain why perception is alteration of a sort, in an etiolated sense only. For while there would be significant analogies between qualitative alteration and perception---each involves an external cause and the assimilation of sensible form---if perceptual activity were complete, then it would not be a mode of alteration, even if it requires that the organ of sensation be acted upon to elicit that activity.

% section distinctions_and_refinements (end)

\subsection{The Triple Scheme} % (fold)
\label{sub:the_triple_scheme}

As we have had occasion to remark, according to Aristotle, the actual and the potential are said of in many ways. In \emph{De Anima} \textsc{ii} 5 Aristotle distinguishes two senses of potentiality, and correspondingly, two senses of actuality. The distinction occurs earlier in \emph{De Anima} \textsc{ii} 1 412\( ^{a} \)24--26 and occurs as well in \emph{Physica} \textsc{viii} 4 255\( ^{a} \)30ff. However, in \emph{De Anima} \textsc{ii} 1 the emphasis is on the two corresponding senses of actuality whereas in \emph{De Anima} \textsc{ii} 5 the emphasis is instead on possibility. The reason for this shift of emphasis is that in \emph{De Anima} \textsc{ii} 5 Aristotle's real concern is with the nature of the transition to actuality in each case, that is, with the nature of the change involved. Given that Aristotle understands motion in terms of potentiality in \emph{Physica}, it is natural for him to distinguish the relevant senses of potentiality and then distinguish the motions that correspond to these.

In \emph{De Anima} \textsc{ii} 5 the distinction is explained in terms of knowledge:
\begin{quote}
	But we must now distinguish different senses in which things can be said to be potential or actual; at the moment we are speaking as if each of these phrases had only one sense. We can speak of something as a knower either as when we say that man is a knower, meaning that man falls within the class of beings that know or have knowledge, or as when we are speaking of a man who possesses a knowledge of grammar; each of these has a potentiality, but not in the same way: the one because his kind or matter is such and such, the other because he can reflect when he wants to, if nothing external prevents him. And there is the man who is already reflecting---he is a knower in actuality and in the most proper sense is knowing, e.g. this A. Both the former are potential knowers, who realize their respective potentialities, the one by change of quality, i.e. repeated transitions from one state to its opposite under instruction, the other in another way by the transition from the inactive possession of sense or grammar to their active exercise. (Aristotle, \emph{De Anima} \textsc{ii} 5 417\( ^{a} \)22--417\( ^{b} \)1; Smith in \citealt[30]{Barnes:1984uq})
\end{quote}

In the passage, Aristotle contrasts someone ignorant of some subject matter but educable with a learned person knowledgeable about that subject matter. The ignorant but educable person is a knower in the sense that they potentially know something. They do not, for example, know some particular point of grammar, but it is not beyond their ken, and they can be brought to know it through education. Being composed of the right matter, they have the capacity for knowledge. The learned person is a knower as well. They have mastered the relevant point of grammar and, having mastered it, can be said to know it in the way that the ignorant but educable cannot. And Aristotle maintains that this sense of being a knower is also a kind of potentiality. Specifically, to know something is to have the capacity to apply that knowledge in a reasonable manner given the practical circumstances. Thus to apply the grammatical knowledge is to deploy it in one's speech or writing or to recognize its significance in the speech or writing of others, by recognizing a letter as an alpha, say. In exercising their knowledge, they can be said to know as well but in a different sense from one who possesses this knowledge without exercising it. To possess grammatical knowledge, in the relevant sense, just is to have the potential to exercise that knowledge in a reasonable manner given the practical circumstances. But this is a distinct sense of potentiality than is involved in the ignorant but educable's potentially knowing that point of grammar.  In the traditional, post-Aristotelian vocabulary, Aristotle is distinguishing between first and second potentiality and, correspondingly, between first and second actuality (schematically represented in table~\ref{tab:triple}). The vocabulary is post-Aristotelian since while Aristotle does speak of first actuality he does not generalize the vocabulary in the obvious way.

\begin{table}[htbp]
	\footnotesize
	\centering
		\begin{tabular}{ccc}
			\hline
			\emph{First Potentiality} & the capacity for knowledge & ``is a knower''\\
			\hline
			\emph{First Actuality/Second Potentiality} & knowing something & ``is a knower'', ``knows''\\
			\hline
			\emph{Second Actuality} & exercising that knowledge & ``knows''\\
			\hline
		\end{tabular}
	\caption{The triple scheme}
	\label{tab:triple}
\end{table}

The ignorant but educable is a knower in that they potentially know something. This is the first potentiality. In coming to know something this potentiality is realized. In coming to know something they too are a knower, though in a different sense. This is the first actuality. The first actuality of knowledge is itself a kind of potentiality since in knowing something the knower has the power or potentiality to exercise that knowledge in a reasonable manner given the practical circumstances. So, in the case at hand, the first actuality is also a second potentiality. In exercising that knowledge, in applying it in a reasonable manner given the practical circumstances, this potentiality is realized. This is the second actuality. 

At the end of the passage, Aristotle draws our attention to a corresponding difference in the two transitions to actuality. The transition from first potentiality to first actuality is a form of qualitative alteration in the way that the transition from second potentiality to second actuality is not. Instead of being a change of quality, the transition to second actuality is described as a transition from an inactive potentiality to its active exercise.

Describing learning as qualitative alteration can sound odd to modern ears. In assessing the credibility of this claim we should bear two things in mind. First, for Aristotle, the fundamental explanatory properties---the primary opposites, Hot and Cold, Dry and Wet---are sensible qualities. It should be no surprise, then, that coming to know should be understood by him in terms of qualitative alteration. That may help us, to some degree, understand why Aristotle found this claim more natural than we perhaps do, but moderns may still protest incredulity. However, knowledge, even on a modern conception of it, is a state. In coming to know about some subject matter, a person while initially in a state of ignorance about that subject is no longer ignorant but knowledgeable. But the replacement of one state by another inconsistent with it might fairly be described as a case of alteration.

How might these distinctions apply to the case of perception? Seeing is the exercise of an animal's capacity for sight. An animal's capacity for sight is a power or potentiality. It need not be exercised---say, when the animal is asleep or in the dark (at least in the case of seeing color if not the fiery or shining, see chapter~\ref{sec:the_objects_of_perception}). It is natural, then, to understand second potentiality as the animal's possession of a perceptual capacity and second actuality as its exercise. Indeed, at the end of the passage, Aristotle makes this connection explicitly. It is both the inactive potential for knowledge and sense to their active exercise which is the transition to second actuality (\emph{De Anima} \textsc{ii} 5 417\( ^{b} \)1; see also \emph{De Anima} \textsc{ii} 5 417\( ^{b} \)19). In which case, the second potentiality of perception, the possession of a perceptual capacity, is also the first actuality (\emph{De Anima} \textsc{ii} 5 417\( ^{b} \)17--19). 
% This is schematically represented in 
% 
% \begin{table}[htbp]
% 	\centering
% 		\begin{tabular}{ccc}
% 			\hline
% 			\emph{First Potentiality} & the matter of an animal & ``is a perceiver''\\
% 			\hline
% 			\emph{Second Potentiality/First Actuality} & the capacity for perception & ``is a perceiver'', ``perceives''\\
% 			\hline
% 			\emph{Second Actuality} & exercising that perceptual capacity & ``perceives''\\
% 			\hline
% 		\end{tabular}
% 	\caption{The triple scheme}
% 	\label{tab:triple}
% \end{table}

Here, however, we encounter a certain awkwardness. Coming to possess a perceptual capacity just does not seem analogous to learning a point of grammar. Indeed, insofar as animals are animate natural beings with perception---that is what distinguishes them from plants---as soon as an animal comes to be, it is endowed with perceptual capacities. So endowing a natural being with the capacity for perception seems more like generation than alteration. This is substantiated later in the chapter when Aristotle writes:
\begin{quote}
	In the case of what is to possess sense, the first transition is due to the action of the male parent and takes place before birth so that at birth the living thing is, in respect of sensation, at the stage which corresponds to the possession of knowledge. (Aristotle, \emph{De Anima} \textsc{ii} 5 417\( ^{b} \)17--19; Smith in \citealt[31]{Barnes:1984uq})
\end{quote}
There remains, however, an important point of analogy. Recall, Aristotle's curious remark that the ignorant but educable has the capacity for knowledge because of their matter. On the hylomorphic theory, matter is a kind of potentiality, and form is a kind of actuality. As we have seen (chapter~\ref{sec:the_soul_of_the_eye}), the form of the eye, its actuality, is its capacity to see. And the matter of the eye---the internal water, the membrane---is the potential to have that capacity. In the epistemic and perceptual cases, then, first potentiality is associated with the potentiality of matter. 

In the epistemic and perceptual cases, the transition to first actuality involves the actualization of the potentiality of matter. Even if a certain awkwardness remains, there is a sense in which the strength of the analogy does not matter. In the perceptual case, Aristotle's real focus is on the transition to second actuality. It is this which is relevant to the general characterization of perception described at the end of the chapter and the definition of perception given in \emph{De Anima} \textsc{ii} 12.

% subsection the_triple_scheme (end)

\subsection{Destructive and Preservative Change} % (fold)
\label{sub:destructive_and_preservative_change}

The acquisition of knowledge is an alteration, whereas the application of knowledge is a different sort of transition, from the inactive possession of knowledge to its active exercise. Just as the actual and the potential are said of in many ways, so too is being moved or acted upon:
\begin{quote}
	Also the expression `to be acted upon' has more than one meaning; it may mean either the extinction of one of two contraries by the other, or the maintenance of what is potential by the agency of what is actual and already like what is acted upon, as actual to potential. For what possesses knowledge becomes an actual knower by a transition which is either not an alteration of it at all (being in reality a development into its true self or actuality) or at least an alteration in a quite different sense. (Aristotle, \emph{De Anima} \textsc{ii} 5 417\( ^{b} \)2--6; Smith in \citealt[30]{Barnes:1984uq})
\end{quote}
Aristotle distinguishes two ways of being moved or acted upon. The first way is a destruction of something by its contrary. This contrasts with the preservation of that which is potential by something actual that is like it. This later is also described as something's development into its true self, a kind of perfection or realization of its true nature. The distinction between the destructive and preservative ways of being moved and acted upon immediately follows Aristotle's contrast between the transition to first actuality and the transition to second actuality. Plausibly, it is an elaboration of how these transitions differ. Indeed, given that Aristotle understands motion in terms of potentiality in \emph{Physica}, it is natural for him to first distinguish the relevant senses of potentiality and then distinguish the motions that correspond to these.

Recall the transition to first actuality, at least in the epistemic case in which it is introduced, was a kind of qualitative alteration. The motion involved in alteration involves a process whereby a thing's quality is replaced by another from the same range which is its contrary. Thus when the water is heated, it becomes hot and so is no longer cool. When a thing is altered, it becomes other than what it was, at least in the relevant qualitative respect. The motion involved in qualitative alteration, as Aristotle understands it, is thus aptly described as a destruction of something by its contrary. The cool of the water is destroyed and replaced by its contrary, heat. Ignorance is destroyed and replaced by knowledge.

It would be inapt, however, to describe the transition to second actuality as a destruction of something by its contrary. When a learned person applies their grammatical knowledge---by recognizing a letter as an alpha, say---they do not lose that knowledge. The learned person does not thereby become ignorant in applying their knowledge in a reasonable manner given the practical circumstances. The application of knowledge does not involve the destruction of that knowledge and its replacement by its contrary, ignorance. Rather, to possess such knowledge is to have the potential to exercise that knowledge in a reasonable manner given the practical circumstances. In exercising that knowledge, this potentiality is realized. There is also a difference in the potentialities involved in these transitions. In the destructive motion of qualitative alteration, the water's potential for being hot is exhausted when it becomes actually hot. However, in exercising knowledge in a given circumstance, the learned person's potential to apply that knowledge is not exhausted in this way, but is rather preserved. There is a further salient difference. In the case of alteration, something becomes unlike the way it was. The water is now hot and so unlike the way it was when it was cool. But in applying their knowledge, the learned person does not become unlike, in epistemic respects, the way they were before they applied that knowledge. The retain their grammatical knowledge in applying it. Finally, the application of knowledge is a kind of perfection. In applying their knowledge in a reasonable manner given the practical circumstances, the learned person realizes their nature as a knower. However, in the case of alteration, it is not water's true nature to be hot and so heating it is not a development into its true nature, it is not a process of perfection. Given these differences, Aristotle concludes that the transition to second actuality is either not an alteration, or an alteration in a different sense. This can be read as echoing the earlier qualification of alteration as applied to sensation, which Smith translates as ``a sort of change of quality''. If it is, then Burnyeat is right in suggesting that it is better understood as alteration of a sort, as alteration in an etiolated sense only.

Earlier, I mentioned modern reservations about thinking of the acquisition of knowledge as a kind of alteration. Another source of doubt is of more Aristotelian provenance. In straightforward cases of qualitative alteration of the kind discussed in \emph{Physica}, the potential of the patient is exhausted in its realization. Recall, this is the sense in which the motion of alteration is incomplete. Such motion is the actuality of this potentiality, a potentiality that is exhausted upon its realization at which point the patient is no longer in motion. But is it really the case that a person's potential to be knowledgeable about some subject matter is exhausted when they actually know about that subject matter? As I observed earlier, in learning we have a transition from one state to another that is inconsistent with it, and this may justify speaking of alteration here, at least in a loose sense. But there are differences. Like the acquisition of a habit or virtue, the acquisition of the excellence of knowledge requires repeated trials---not so with the heating of water. Moreover, in the case of the cool water becoming hot, we have one quality being replaced by another from the same range. But ignorance and knowledge, while inconsistent conditions, are not contraries drawn from a common genus. The present worry is perhaps an anticipation of further qualifications that Aristotle makes in \emph{De Anima} \textsc{ii} 5 417\( ^{b} \)10--16 (discussed in the next section).

The transition to second actuality is either not an alteration or an alteration in a different sense. Aristotle immediately emphasizes this denial:
\begin{quote}
	Hence it is wrong to speak of a wise man as being `altered' when he uses his wisdom, just as it would be absurd to speak of a builder as being altered when he is using his skill in building a house. (Aristotle, \emph{De Anima} \textsc{ii} 5 417\( ^{b} \)7--9; Smith in \citealt[30]{Barnes:1984uq})
\end{quote}
A thinker does not cease to be a thinker when they think, just as a builder does not cease to be a builder when they build. Rather, they realize their nature as thinker and builder, respectively. The exercise of the builder's capacity to build is not an alteration. And this remains true even if in order to exercise that capacity the object of the activity is altered. Thus, bricks are arranged. Moreover, the exercise of the capacity to build is not an alteration even if in order to exercise that capacity the builder themself must undergo various alterations. Thus the location of their limbs alters over time, and building induces fatigue in the builder. Even in the more rarefied case of thinking, a thinker in thinking thoughts may become fatigued.

The point is usefully elaborated in terms of an example from \emph{De Generatione Animalium}:
\begin{quote}
	This is what we find in the products of art; heat and cold may make the iron soft and hard, but what makes a sword is the movement of the tools employed, this movement containing the principle of the art. For the art is the starting-point and form of the product; only it exists in something else, whereas the movement of nature exists in the product itself, issuing from another nature which has the form in actuality. (Aristotle, \emph{De Generatione Animalium} \textsc{ii} 1 734\( ^{b} \)37--735\( ^{a} \)4; Platt in \citealt[38]{Barnes:1984uq})
\end{quote}
A swordsmith is one who possesses the art of making swords. The swordsmith imposes the form of the sword on the heated iron through motions of their hammer that embody that form. The form of the sword exists prior to the sword in the art or perhaps the soul of the person who possesses that art. This form guides the motions of the swordsmith. The motions of their hammer occur and occur in the way they do because of that form. Through this process, the iron which potentially has the form of a sword takes on that form in actuality and so becomes a sword. As with the thinker and builder, the swordsmith, in exercising their art, realizes their nature as an artisan of that kind. But this requires that they move the object of their activity by undergoing motions themself, by swinging a hammer, say. 

Suppose these reflections carry over to the case of perception. The exercise of an animal's perceptual capacities, their undergoing a perceptual experience, is not an alteration, or at best an alteration in a different sense. This would remain true even if in order to exercise that capacity, the perceiver must be materially altered. Indeed, as we have seen, they must. Perception is a reactive capacity. The organ of sensation must be acted upon in order to exercise that capacity. In seeing, the perceiver's eyes are filled with light. The eye's interior illumination is necessary to exercise the reactive capacity they endow the perceiver with. While \citet{Burnyeat:1992fk} is right in claiming that coming to perceive is a psychological rather than a material change, he goes too far in insisting that this precludes the perceiver being, at the same time, materially altered.

% subsection destructive_and_preservative_change (end)

\subsection{Privative Change and Change to a Thing's Disposition and Nature} % (fold)
\label{sub:privative_and_nonprivative_change}

Once someone genuinely possess knowledge no further teaching or learning is required for them to apply that knowledge in a reasonable manner given the practical circumstances. The transition to second actuality requires neither teaching nor learning. It thus does not seem to involve being moved and acted upon at least not in the way that fire moves and acts upon water in heating it. That much is clear from what has so far been said. But what about the transition to first actuality? This was earlier described as qualitative alteration. However, here, Aristotle seems to be claiming that it is not just the transition to second actuality that is dubiously described as alteration, but the transition to first actuality as well:
\begin{quote}
	What in the case of thinking or understanding leads from potentiality to actuality ought not to be called teaching but something else. That which starting with the power to know learns or acquires knowledge through the agency of one who actually knows and has the power of teaching either ought not to be said `to be acted upon' at all---or else we must recognize two senses of alteration, viz. the change to conditions of privation, and the change to a thing's dispositions and to its nature. (Aristotle, \emph{De Anima} \textsc{ii} 5 417\( ^{b} \)10--16; Smith in \citealt[30--31]{Barnes:1984uq})
\end{quote}
We have already encountered a reason for this denial. The transition from ignorance to knowledge through teaching and learning is not the replacement of one quality by another from the same range of qualities. Though ignorance and knowledge are inconsistent states of a rational animal they are not contrary qualities from the same genus. Thus the ignorant but educable and the learned person are not generically alike the way they would be if the transition were a genuine case of qualitative alteration. 

At the end of the passage Aristotle distinguishes two sense of alteration. A change to a thing's privation and a change to a thing's dispositions and nature. The transition from ignorance to knowledge, insofar as these are inconsistent states, can be described as movement to a condition of privation. In learning about some subject matter, a person is deprived of their ignorance. Privative change seems to echo Aristotle's earlier notion of destructive change. The destruction of something by its contrary is a change to a condition of privation, on at least one reasonable understanding of that notion, perhaps not the only one. So when the cool of the water is replaced by heat, the process can be described as a movement to a condition of privation insofar as the water is deprived of its coolness. As the example makes clear, however, the destruction of something by its contrary may mean something more specific than privation. While a change to a condition of privation, it is also the replacement of one quality by another from a common genus. 

The alternative is more revealing about the sense of Aristotle's denial that learning is an alteration. A change to a thing's disposition and nature can sound like an aspect of preservative change, being a development into its true self. However, the other aspects of preservative change---the preservation as opposed to the exhaustion of the relevant potentiality, and the way the potentiality is like its realization---are not mentioned in this passage. Moreover, if a change to a thing's disposition and nature were identified with preservative change, Aristotle would lose the distinction between the transition to first and second actuality. A change to a thing's disposition and nature must be understood in some other way.

How might a change to a thing's disposition and nature be understood in such a way that (1) Aristotle retains the distinction between the transitions to first and second actuality and (2) and explains his denial that learning is an alteration? Earlier, we acknowledged a certain awkwardness in describing the transition to first actuality as a kind of alteration since a natural body acquiring perceptual capacities seems more like a case of generation than alteration. Here, perhaps, Aristotle is acknowledging that the difference is less extreme than it initially appeared. Part of the effect of the male parent in the generation of the animal, and one that Aristotle draws our attention to, is the development of that animal's natural capacities, specifically, their natural capacity for perception. Perhaps a change to a thing's disposition and nature can be understood as the development of that thing's natural capacities. Knowledge is a natural capacity, at least for rational animals. It is also a dispositional state, and like the acquisition of habit and virtue, the acquisition of knowledge through teaching and learning requires repeated application. So understood, the parallel between the transition to first actuality in the epistemic and perceptual cases is reinstated. Each involves the development of a thing's natural capacities. Moreover, this seems distinct both from preservative change characteristic of the transition to second actuality and from qualitative alteration. Whereas a change to a thing's disposition and nature is the development of the thing's natural capacities, preservative change is the exercise of a thing's natural capacities. In this way, Aristotle retains the distinction between the transitions to first and second actuality. Moreover, the development of a thing's natural capacities is distinct from qualitative alteration. As we observed, at least in the perceptual case, the development of an animal's perceptual capacities is better described as generation rather than alteration. And in the epistemic case, whereas the transition to a state of knowledge about some subject matter may be a kind of privation---one state of a knower is replaced by another state inconsistent with it, this is not, however, the destruction of something by its contrary, understood as the replacement of one quality by a contrary quality from a common genus.

In the epistemic and perceptual cases, the transition to first actuality involves the development of the animal's natural capacities, and the transition to second actuality involves the exercise of these capacities. Neither are qualitative alterations, strictly speaking, though analogies remain which perhaps justifies talk of alteration in a more expansive sense. Thus the transition to first actuality in the epistemic case was initially described as an alteration no doubt in part because ignorance is replaced with knowledge. But, as I have emphasized, this is not qualitative alteration since ignorance and knowledge are not contrary qualities from a common genus. And in the perceptual case, the transition to second actuality is like qualitative alteration in that it requires an external cause. Perception is essentially a reactive capacity. It only acts by reacting. The sense organ must be acted upon if the perceptual capacity that it endows its possessor with is to be exercised. Only in this way is perception a mode of sensitivity or receptivity. But again, despite the testimony of the \emph{endoxa}, the exercise of this capacity is not an alteration. The transition to second actuality is not incomplete the way that the motion involved in qualitative alteration is. In perceiving the brilliant white of the sun, the perceiver is seeing and has seen. Perceptual activity is complete at every instant, and the transition to second actuality is an instance of preservative change rather than qualitative alteration.

% subsection privative_and_nonprivative_change (end)

\section{The General Characterization of Perception} % (fold)
\label{sec:the_general_characterization_of_perception}

At the end of the chapter, Aristotle offers a general characterization of perception:
\begin{quote}
	As we have said, what has the power of sensation is potentially like what the perceived object is actually; that is, while at the beginning of the process of its being acted upon the two interacting factors are dissimilar, at the end the one acted upon is assimilated to the other and is identical in quality with it. (Aristotle, \emph{De Anima} \textsc{ii} 5 418\( ^{a} \)3-6; Smith in \citealt[31]{Barnes:1984uq})
\end{quote}
Earlier, Aristotle distinguishes two sense of perception:
\begin{quote}
	We use the word `perceive' in two ways, for we say that what has the power to hear or see, `sees' or `hears', even though it is at the moment asleep, and also that what is actually seeing or hearing, `sees' or `hears'. Hence `sense' too must have two meanings, sense potential, and sense actual. Similarly `to be a sentient' means either to have a certain power or to manifest a certain activity. (Aristotle, \emph{De Anima} \textsc{ii} 5 417\( ^{a} \)10--13; Smith in \citealt[19--30]{Barnes:1984uq})
\end{quote}
Perception is thus either the capacity to perceive or its exercise in perceptual experience. What is being characterized, here, is not the exercise of the capacity but the capacity itself. What has ``the power of sensation'' possesses that power even when asleep when that power remains dormant and so unexercised. So Aristotle is offering a general characterization of perception, understood not as perceptual experience but as the capacity to undergo such experiences. 

Perception, here, is understood as a power or potentiality. Specifically, perception is the potential to become like the perceived object actually is as the result of that object acting upon the perceiver. As should by now be clear, had this characterization been given at the beginning of the chapter, it would be easy to misunderstand the perceived object acting upon the perceiver as a case of qualitative alteration. After all, cool water in a pot has the potential to be like the fire that heats it. Part of the point of the refinements and qualifications that proceed this characterization is to forestall such misunderstandings. 

Perception is the potential to become like the perceived object. The emphasis on likeness in Aristotle's general characterization plays an important epistemic role (see \citealt[58]{Burnyeat:2002an}). Assimilation, along with the reactive character of our perceptual capacities, is what underwrites his perceptual realism. If perception involves becoming like the perceived object actually is, then it is a genuine mode of awareness. One can only perceptually assimilate what is there to be assimilated. If perceptual experience is a mode of assimilation, then one could not undergo such an experience consistent with a Cartesian demon eliminating the object of that experience. If there is no external object, then there is nothing which the perceiver, or perhaps their experience, has become like. In perceptual experience we simply confront the primary object of the given modality. We cannot be confronted truly or falsely, correctly or incorrectly. We simply confront what is presented to us in sensory conscious and so become like the way it is in actuality. As we shall see in chapter~\ref{cha:form_without_matter}, the assimilation of sensible form underwrites a strong form of perceptual realism rejected by the early moderns. Thinkers as diverse as Galileo, Descartes, Hobbes, Locke, and Boyle were united in rejecting this premodern form of realism.

Aristotle does not elaborate the sense in which the perceiver, or perhaps their perceptual experience, becomes like the perceived object is in actuality. However, an important clue emerged earlier in the chapter. It involves the way likeness must be understood in preservative change. One way, not the only way, to motivate a literalist interpretation involves a particular understanding of what is required in saying that one thing is like another. On the literalist interpretation, in seeing a colored particular the perceiver's eye takes on that color (\citealt{Slakey:1961ss,Sorabji:1974fk,Sorabji:2003fk,Everson:1997ep}). In seeing the brilliant white of the sun, the transparent medium within the eye itself becomes brilliant white. This would follow on a natural understanding of becoming like. On that understanding, if one thing is \emph{F} and another thing becomes like it in the relevant respect, then it too is actually \emph{F}. However, given that the potentiality involved in the transition to second actuality is like the actuality that realizes it, Aristotle is working with a broader understanding of likeness here. If the potentiality involved in possessing grammatical knowledge is already like its realization, there is no general requirement that if two things are alike in a certain respect they must actually have the relevant feature. The likeness involved in the general characterization of perception, understood as a sensory capacity, is not the likeness between that potentiality and its actualization, but the way in which the perceiver becomes like the object perceived. Nevertheless, the point is general. In order to sustain his judgment that the second potentiality of grammatical knowledge is like the second actuality that is its realization, Aristotle simply could not adhere to the principle that two things are only alike if they actually share the same feature. It is worth bearing in mind that Aristotle is working with a broader understanding of likeness when interpreting the doctrine of the assimilation of sensible form.

The point may be elaborated with Aristotle's example of the swordsmith from \emph{De Generatione Animalium}. In producing the sword, the iron becomes like the form of the sword contained in the soul of the person possessing that art. In the perceptual case, the perceiver assimilates the form of the perceived object. In the production of artifacts, the direction of assimilation is reversed. It is not a person assimilating the form of an external object, but an external object assimilating a form that is in some sense in the person. In the swordsmith's case, the iron assimilates the form contained in soul of the person possessing the relevant art. But there is no actual sword in the smith's soul. So, in perceiving the brilliant white of the sun, the perceiver may become like the sun actually is, but not by becoming brilliant white.

% section the_general_characterization_of_perception (end)

\section{On the Complexity of \emph{De Anima} \textsc{ii} 5} % (fold)
\label{sec:on_the_complexity_of_de anima_ii_5}

Even by the standards of \emph{De Anima}, \emph{De Anima} \textsc{ii} 5 is an unusually tortured chapter. Aristotle's evident hesitancy and incessant qualification makes the chapter difficult to read, and there is, quite rightly, substantive disagreement about how exactly all the qualifications hang together (see, \emph{inter alia}, \citealt{Burnyeat:2002an}, \citealt{Heinaman:2007ys}, and \citealt{Bowin:2011uq}). What is the significance of Aristotle's hesitancy and incessant qualification? Let me speculate about three potential sources. 

First, there is not in Greek the vocabulary to mark the distinctions he needs to mark (nor in Persian, nor Egyptian, nor Aramaic, nor in any of the other ancient languages). Aristotle is engaged in conceptual innovation and so must explain, as best he can, his intended meaning in terms that are liable to be misunderstood. Hesitancy and qualification is what one would naturally expect from one engaged in such a project of conceptual innovation. 

Second, there is a sense in which the hesitation and incessant qualification flows from the puzzling nature of the subject matter. According to the Heraclitean account of perception in the Secret Doctrine of the \emph{Theaetetus}, perception is the outcome of active and passive forces in conflict. As the Secret Doctrine makes clear, there are active and passive elements in perception that must be carefully disentangled lest we lapse into a kind of Protagorean relativism. If realism about the manifest image of nature is to be sustained, then the interplay of active and passive forces in perception must be carefully understood. But it is easy to be puzzled about how, exactly the active and passive combine in perception. Consider being moved and acted upon in cases of qualitative alteration. Fire acts upon the pot of cool water and thus the water becomes hot. The power to become hot is a passive capacity of the water. Heating is something that the fire does and that the water merely undergoes. Similarly, in the perceptual case, the perceived object must act upon the perceiver, at least if perception is a mode of sensitivity as it must be if perception is so much as possible. But if perceiving were merely a passive capacity, then seeing would not be something that a perceiver does, but rather something done to the perceiver by the object of perception. But seeing is not something done to the perceiver, even if the perceptual experience by which they see is something they undergo. Seeing is, rather, the exercise of the perceiver's capacity for sight. The chief virtue of the Nietzschean vocabulary of ``reactive'' capacities is that it makes clear that the capacity in question is not, or not merely, passive. It acts by reacting. The activity that is the exercise of a reactive capacity must be triggered by an external cause. A second potential source of Aristotle's hesitancy and incessant qualification, then, is as a reasonable response to a natural puzzlement about how to combine the active and passive elements in perception consistent with a realism about the manifest image of nature.

There is another potential source, this more controversial than the first two. Perhaps Aristotle, in this chapter, is straining against the limits of his physics. Physics, as Aristotle understands it, concerns the motion of natural bodies. Living beings are an important class of natural bodies, so it is plausible that the study of living beings takes place within the framework set out in \emph{Physica} and \emph{De Generatione et Corruptione}. However, there are elements of Aristotle's account of the soul that clearly fall outside of the purview of his physics. Thus the intellect is incorporeal and may be shared with divine beings that lack bodies. But the physics is the study of the motion of natural bodies. Perhaps the limitations of Aristotle's physics is met here too in \emph{De Anima} \textsc{ii} 5. Whereas the motion that is the subject matter of Aristotelian physics is incomplete, perceptual activity is complete at every instant. Aristotle might, nevertheless, be motivated to continue to work within the framework of his physics insofar as he can and eschew saying anything that may explicitly conflict with it, especially since his resolution of the \emph{aporia} and thus the establishment of the very possibility of perception, depends upon that framework. Perhaps, then, this is the source of his hesitancy and incessant qualification: He is attempting to downplay the fact that he is moving beyond the framework of his physics. 

This latter speculation is, as I have said, controversial. A commentator who sees Aristotle as rigidly adhering to the framework of his physics will differently interpret the various qualifications on offer than a commentator who sees Aristotle as straining against the limits of that framework. Moreover, I have given nothing like a full dress defense of the present interpretation against its competitors. However, for present purposes this is unnecessary. For if we are to understand Aristotle's definition of perception as the assimilation of sensible form our focus should be less on coming to perceive as an alteration of a sort than on the manner in which in perceiving, the perceiver, or perhaps their perceptual experience, becomes like the perceived object actually is.

% section on_the_complexity_of_de anima_ii_5 (end)

% chapter two_kinds_of_potentiality (end)
