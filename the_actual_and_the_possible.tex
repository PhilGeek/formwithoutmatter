%!TEX root = /Users/markelikalderon/Documents/Git/formwithoutmatter/aristotle.tex
\chapter{Two Transitions to Actuality} % (fold)
\label{cha:two_kinds_of_potentiality}

\section{The Problem} % (fold)
\label{sec:the_problem}

The eyes, in seeing, are altered. They are filled with light. The existence and character of the interior illumination depends upon and derives from the existence and character of exterior illumination, itself affected by the colors of remote external particulars. Moreover, the eyes in seeing must be altered. The eyes are the organ of sight, and seeing is a reactive capacity. Sight only acts by reacting to the colors of remote external particulars. So the organ of sight must be acted upon in order to exercise the capacity for sight. Notoriously, however, Aristotle denies that the exercise of our capacity for sight, an episode of seeing, is itself an alteration. So the eyes, in seeing must be altered, but seeing is not itself an alteration or, at least, it is not an alteration of the usual sort. What kind of change does a perceiver undergo when seeing? 

The avowed task of \emph{De Anima} \textsc{ii}.5 is to say, in a preliminary fashion, what the senses are in general. A discussion of the special senses follows in subsequent chapters, and it is only in \emph{De Anima} \textsc{ii}.12 that Aristotle give the official definition of perception as the assimilation of sensible form without the matter of the perceived particular. Despite the avowed task of the chapter, most of the discussion is bound up with a number of careful distinctions, and we only get the general characterization of sensation at the end of the chapter. 

% section the_problem (end)

\section{The \emph{Endoxa}} % (fold)
\label{sec:the_endoxa}

The chapter begins with a discussion of the \emph{endoxa}:
\begin{quote}
	Sensation depends, as we have said, on a process of movement or affection from without, for it is held to be some sort of change of quality. Now some thinkers assert that like is affected only by like. In what sense this is possible and in what sense impossible, we have explained in our general discussion of acting and being acted upon. (Aristotle, \emph{De Anima} \textsc{ii}.5 416\( ^{b} \)33--417\( ^{a} \)2; Smith in \citealt[29]{Barnes:1984uq})
\end{quote}
Aristotle is picking up on his previous discussion of the \emph{endoxa} from \emph{De Anima} \textsc{i}.5 410\( ^{a} \)24--410\( ^{b} \)14. That perception depends upon being moved and acted upon was attributed to previous thinkers in \emph{De Anima} \textsc{i}.5 410\( ^{a} \)25--26. That claim, at least in the present context, is importantly qualified. Perception is not identified with being moved and acted upon. It is only said to depend upon being moved and acted upon. Though perhaps a widely held opinion, not all previous thinkers assented to it. Notoriously, in the Secret Doctrine that Socrates attributes to Protagoras (\emph{Theaetetus} 156\( ^{a-c} \)), perception is represented as the outcome of active and passive forces in conflict and so no mere passive effect. And this is perhaps echoed in Plato's own account of perception as the outcome of emanations from the perceiver and the object perceived coalescing (\emph{Timaeus} 45\( ^{b} \)--46\( ^{a} \)). In an earlier chapter, Aristotle suggested that perception was a qualitative alteration (\emph{De Anima} \textsc{ii}.4 415\( ^{b} \)24). That claim reappears though again in qualified form. Smith translates the qualification as ``some sort of change of quality'', but as \citet[36--37]{Burnyeat:2002an} importantly points out, the qualification can also be understood as ``a change of quality of a sort'', that is, as an alteration only in an etiolated sense. In the first line of the passage the claim that perception is a sort of alteration, or alteration of a sort, is offered as a justification for thinking that perception is a way of being moved and acted upon. That justification draws, implicitly, upon Aristotle's physics. In \emph{Physica} \textsc{vii}.1 and \textsc{viii}.4, Aristotle argues that every case of motion involves being acted upon, and in \emph{De Generatione et Corruptione} \textsc{i}.6 323\( ^{a} \)13--24 he argues that alteration always involves being acted upon. The attribution of the claim that perception is some sort of alteration, or alteration of a sort, to previous thinkers is doubtful, however, as it makes essential use of the classificatory scheme of the \emph{Categoriae}. Here, as is not uncommon, Aristotle is endeavoring to understand the views of his predecessors in his own terms. Aristotle may be over-reading, or having dialectical rather than strictly historical concerns, but, to be fair, the connection with qualitative alteration is not far to seek insofar as previous thinkers maintained, as well, that like is affected only by like.

The like-by-like principle, that like is affected only by like, is more genuinely of pre-Aristotelian provenance. Empedocles, among others, is said to have held it (\emph{De Anima} \textsc{i}.2 404\( ^{b} \)7--15; though whether he actually held the like-by-like principle is controversial, see \citealt{Kamtekar:2009fk}). Concerning ``in what sense this is possible and in what sense impossible'', Aristotle refers us to \emph{De Generation et Corruptione} \textsc{i}.7 (on Aristotle's use of cross-referencing in this chapter see \citealt{Burnyeat:2002an}). There, Aristotle explains that there is a sense in which those who maintain that like is affected by like and those who deny it maintaining, instead, that like is unaffected by like are both right. Each position, though seemingly diametrically opposed, capture part of a more complex truth (\emph{De Generatione et Corruptione} \textsc{1}.7 323\( ^{b} \)15--324\( ^{a} \)9). Consider the case of one thing acting upon another thus inducing a change of quality, say, fire heating a pot of cool water. (My choice of example, though drawn from \emph{De Generatione et Corruptione}, is far from innocent---in \emph{De Anima} \textsc{ii}.12 Aristotle will contrast a plant being warmed with an animal's sensation of heat.) At the start of the process, the agent and patient of the change, the fire and the water, have contrary qualities from the same range of qualities---the fire is hot and the water is cool and each is a quality of temperature. They are thus generically alike. Though they are generically alike prior to the alteration they are also unlike. The distinct qualities they possess from the same range of qualities are contraries. Thus the fire and the water, while generically alike in being the kind of things that have temperature, are specifically unlike in possessing contrary qualities of temperature. When the fire comes into contact with the pot of water they interact in such a way that the water becomes hot and so like the way the fire was prior to the qualitative alteration.

In \emph{De Generatione et Corruptione} Aristotle argues against the like-by-like principle as follows:
\begin{quote}
	Moreover, if like can be affected by like, a thing can also be affected by itself; and yet if that were so---if like tended in fact to act \emph{qua} like---there would be nothing indestructible or immovable, for everything would move itself. (Aristotle, \emph{De Generatione et Corruptione} \textsc{i}.7 323\( ^{b} \)xx-xx; Joachim in \citealt[23]{Barnes:1984uq})
\end{quote}
A variant of this argument occurs at this point in \emph{De Anima} \textsc{ii}.5 (discussed earlier in chapter~\ref{sub:particular}): 
\begin{quote}
	Here arises a problem: why do we not perceive the senses themselves, or why without the stimulation of external objects do they not produce sensation, seeing that they contain in themselves fire, earth, and all the other elements, of which---either in themselves or in respect of their incidental attributes---there is perception? It is clear that what is sensitive is so only potentially, not actually. The power of sense is parallel to what is combustible, for that never ignites itself spontaneously, but requires an agent which has the power of starting ignition; otherwise it could have set itself on fire, and would not have needed actual fire to set it ablaze. (Aristotle, \emph{De Anima} \textsc{ii}.5 417\( ^{a} \)3--9; Smith in \citealt[29]{Barnes:1984uq})
\end{quote}
If the like-by-like principle were true, then the sense organ, being like itself---as, indeed, is everything else---would act upon itself. The exercise of the capacity for sight, which the eye endows its possessor with, would not require an external particular to activate it. However, as \citet[226--227]{Polansky:2007ly} observes, the argument, when restricted in this way to perception, accrues new significance. If the sense organ acts upon itself, then the exercise of the capacity it endows its possessor with would at best be a mode of self-consciousness. It would not be a mode of sensitivity to external particulars and their sensible qualities and thus could not be a mode of perception. The puzzle, as it arises in the context of \emph{De Anima} \textsc{ii}.5, thus concerns the very possibility of perception.

Not only does the puzzle about the like-by-like principle accrue new significance in \emph{De Anima} \textsc{ii}.5, but it is susceptible to a novel solution as well. Nowhere in \emph{De Generatione et Corruptione} \textsc{i}.7 does Aristotle explicitly appeal to the distinction between the actual and the potential. But that distinction is essential to the resolution of the present puzzle. The lesson of the puzzle about perception raised by the like-by-like principle is that the capacity for sight, the form and substance of the eye, is a kind of potentiality, one that requires an external particular, the object of perception, to act upon the eye in order for that capacity to be exercised in seeing that object. It is like fuel which requires an external fire for it to ignite. Perception, as we have seen, is essentially a reactive capacity, one which requires an external object to ignite sensory consciousness.

Perception is a power or potentiality that requires an external particular for it to be realized. The exercise of perceptual capacities is, at least in this regard, like the case of motion, studied in \emph{Physica}, since all motion requires a mover:
\begin{quote}
	To begin with let us speak as if there were no difference between being moved or affected, and being active, for movement is a kind of activity---an imperfect kind, as has elsewhere been explained. Everything that is acted upon or moved is acted upon by an agent which is actually at work. Hence it is that in one sense, as has already been stated, what acts and what is acted upon are like, in another unlike; for the unlike is affected, and when it has been affected it is like. (Aristotle, \emph{De Anima} \textsc{ii}.5 417\( ^{a} \)15--21; Smith in \citealt[30]{Barnes:1984uq})
\end{quote}
Aristotle invites us to speak as if being moved and acted upon, on the one hand, and perceptual activity such as seeing and hearing, on the other, were the same. He is not identifying perception with being moved or acted upon, but merely emphasizing the fact that, like the motion of natural bodies discussed in \emph{Physica}, the activity of the senses depends upon an external particular acting upon the organ of sensation.

In \emph{Physica} \textsc{iii}.1 201\( ^{a} \)10--11, Aristotle defines motion as the actuality of the potential as such (see also \emph{Metaphysica} 1065\( ^{b} \)16). Against Parmenidean skepticism, Aristotle insists upon the reality of change. Becoming pertains to actuality, but in a qualified manner. Consider again our example of qualitative alteration, fire heating a pot of cool water. At the beginning of this process, the water is cool but potentially hot. At the end of this process, the water is no longer potentially but actually hot. When the process has finished and the water is actually hot, the water is no longer becoming hot. It is this transition from merely being potentially hot, possessed by the water prior to its contact with the heat of the flame, to being actually hot which is the motion of qualitative alteration. This is something that actually happens, hence the use of actuality in Aristotle's definition of motion. But throughout the process, until its termination, the water remains potentially hot. That is why motion is the actuality of the potential as such. When the water is actually hot it is no longer potentially hot. The alteration that the water undergoes is the actualization of its potentially being hot, a potentiality that is both realized and exhausted in the process. (That the relevant sense of potentiality is potential for being rather than changing see \citealt{Kosman:1969aa}; though see \citealt{Heinaman:1994aa} for criticism.)

We are now in a position to understand the sense in which the motion involved in qualitative alteration is imperfect or incomplete. It is incomplete in the sense that when the terminal state has been reached, the process of alteration is no longer happening, and so long as it is happening, the terminal state has not been reached. This would contrast with an activity whose end lies not without but is immanent in the activity itself. Aristotle marks such a contrast at the opening of the \emph{Ethica Nicomachea} \textsc{i}.1 1094\( ^{a} \)3--5 and there is an extended discussion in \emph{Metaphysica} \( \Theta \) 8 though Aristotle does not explicitly describe these activities as complete.


At the end of the passage, Aristotle offers a resolution of the apparent conflict between those who maintain that like affects like and those who maintain that like is unaffected by like that echoes the resolution offered in \emph{De Generatione et Corruptione} \textsc{i}.7. There is a sense in which both opinions are correct. The water, at the beginning of the process, is cool and so unlike the heat of the flame. But when the process is completed and the water's potentially being hot is realized, the water is like the way the fire was prior to the alteration. Perceptual activity is like qualitative alteration in that it requires an external object to act upon the sense organ to initiate that activity. The present resolution of the apparent conflict in the \emph{endoxa} hints at a further analogy. Applying it to the case of perception yields the following result. Prior to seeing the brilliant white of the sun, the perceiver, or perhaps their perceptual experience, is unlike the sun. But in coming to see the sun's brilliant whiteness, the perceiver, or perhaps their perceptual experience, becomes like the way the sun was prior to seeing it. The perceiver, or their perceptual experience, is potentially like the sensible object, and when they undergo the perceptual experience of that sensible object their experience becomes actually like the way the sensible object was prior to perception. We have here an anticipation of the general characterization of perception that occurs at the end of \emph{De Anima} \textsc{ii}.5 and the definition of perception that occurs in \emph{De Anima} \textsc{ii}.12.

But before Aristotle reaches that point, he undertakes to investigate the sense of potentiality involved in being a perceiver. Is the realization of this potential incomplete, in the way that alteration is? Or is the end of sight immanent in seeing? If the latter, then this would explain why perception is alteration of a sort, in an etiolated sense only. For while there would be significant analogies between qualitative alteration and perception---each involves an external cause and the assimilation of sensible form---if perceptual activity were complete, then it would not be a mode of alteration, even if it requires that the organ of sensation be acted upon to elicit that activity.

% section the_endoxa (end)

\section{The Triple Scheme} % (fold)
\label{sec:the_triple_scheme}

As we have had occasion to remark, according to Aristotle, the actual and the potential are said of in many ways. In \emph{De Anima} \textsc{ii}.5 Aristotle distinguishes two senses of potentiality, and correspondingly, two senses of actuality. The distinction occurs earlier in \emph{De Anima} \textsc{ii}.1 (and occurs as well in \emph{Physica} \textsc{viii}.4 255\( ^{a} \)30ff). However, in \emph{De Anima} \textsc{ii}.1 the emphasis is on the two corresponding senses of actuality whereas in \emph{De Anima} \textsc{ii}.5 the emphasis is instead on possibility. The reason for this shift of emphasis is that in \emph{De Anima} \textsc{ii}.5 Aristotle's real concern is with the nature of the transition to actuality in each case, that is, with the nature of the change involved.  

In \emph{De Anima} \textsc{ii}.5 the distinction is explained in terms of knowledge:
\begin{quote}
	But we must now distinguish different senses in which things can be said to be potential or actual; at the moment we are speaking as if each of these phrases had only one sense. We can speak of something as a knower either as when we say that man is a knower, meaning that man falls within the class of beings that know or have knowledge, or as when we are speaking of a man who possesses a knowledge of grammar; each of these has a potentiality, but not in the same way: the one because his kind or matter is such and such, the other because he can reflect when he wants to, if nothing external prevents him. And there is the man who is already reflecting---he is a knower in actuality and in the most proper sense is knowing, e.g. this A. Both the former are potential knowers, who realize their respective potentialities, the one by change of quality, i.e. repeated transitions from one state to its opposite under instruction, the other in another way by the transition from the inactive possession of sense or grammar to their active exercise. (Aristotle, \emph{De Anima} \textsc{ii}.5 417\( ^{a} \)22--417\( ^{b} \)1; Smith in \citealt[30]{Barnes:1984uq})
\end{quote}

In the passage, Aristotle contrasts someone ignorant of some subject matter but educable with a learned person knowledgeable about that subject matter. The ignorant but educable person is a knower in the sense that they potentially know something. They do not, for example, know some particular point of grammar, but it is not beyond their ken, and they can be brought to know it through education. Being composed of the right matter, they have the capacity for knowledge. The learned person is a knower as well. They have mastered the relevant point of grammar and, having mastered it, can be said to know it in the way that the ignorant but educable cannot. And Aristotle maintains that this sense of being a knower is also a kind of potentiality. Specifically, to know something is to have the capacity to apply that knowledge in a reasonable manner given the practical circumstances. Thus to apply the grammatical knowledge is to deploy it in one's speech or writing or to recognize its significance in the speech or writing of others. In exercising their knowledge, they can be said to know as well but in a different sense from one who possesses this knowledge without exercising it. To possess grammatical knowledge, in the relevant sense, just is to have the potential to exercise that knowledge in a reasonable manner given the practical circumstances. But this is a distinct sense of potentiality than is involved in the ignorant but educable's potentially knowing that point of grammar.  In the traditional, post-Aristotelian vocabulary, Aristotle is distinguishing between first and second potentiality (and, correspondingly, between first and second actuality), schematically represented in in table~\ref{tab:triple}. (The vocabulary is post-Aristotelian since while Aristotle does speak of first actuality he does not generalize the vocabulary in the obvious way.)

\begin{table}[htbp]
	\centering
		\begin{tabular}{ccc}
			\hline
			\emph{First Potentiality} & the capacity for knowledge & ``is a knower''\\
			\hline
			\emph{Second Potentiality/First Actuality} & knowing something & ``is a knower'', ``knows''\\
			\hline
			\emph{Second Actuality} & exercising that knowledge & ``knows''\\
			\hline
		\end{tabular}
	\caption{The triple scheme}
	\label{tab:triple}
\end{table}

The ignorant but educable is a knower in that they potentially know something. This is the first potentiality. In coming to know something this potentiality is realized. In coming to know something they too are a knower, though in a different sense. This is the first actuality. The first actuality of knowledge is itself a kind of potentiality since in knowing something the knower has the power or potentiality to exercise that knowledge in a reasonable manner given the practical circumstances. So, in the case at hand, the first actuality is also a second potentiality. In exercising that knowledge, in applying it in a reasonable manner given the practical circumstances, this potentiality is realized. This is the second actuality. 

At the end of the passage, Aristotle draws our attention to a corresponding difference in the two transitions to actuality. The transition from first potentiality to first actuality is a form of qualitative alteration in the way that the transition from second potentiality to second actuality is not.

How would these distinctions apply to the case of perception? Seeing is the exercise of an animal's capacity for sight. An animal's capacity for sight is a power or potentiality. It need not be exercised, say, when the animal is asleep or in the dark. It is natural, then to understand second potentiality as the animal's possession of a perceptual capacity and second actuality as its exercise (\emph{De Anima} \textsc{ii}.5 417\( ^{b} \)19). In which case, the second potentiality of perception, the possession of a perceptual capacity, is also the first actuality (\emph{De Anima} \textsc{ii}.5 417\( ^{b} \)17--19). 
% This is schematically represented in 
% 
% \begin{table}[htbp]
% 	\centering
% 		\begin{tabular}{ccc}
% 			\hline
% 			\emph{First Potentiality} & the capacity for knowledge & ``is a perceiver''\\
% 			\hline
% 			\emph{Second Potentiality/First Actuality} & knowing something & ``is a perceiver'', ``perceives''\\
% 			\hline
% 			\emph{Second Actuality} & exercising that knowledge & ``perceives''\\
% 			\hline
% 		\end{tabular}
% 	\caption{The triple scheme}
% 	\label{tab:triple}
% \end{table}

Here we encounter, however, a certain awkwardness. Coming to possess a perceptual capacity just does not seem analogous to learning a point of grammar. Indeed, insofar as animals are animate natural beings with perception---that is what distinguishes them from plants---as soon as an animal comes to be, it is endowed with perceptual capacities. So endowing a natural being with the capacity for perception seems more like generation than alteration. This is substantiated later in the chapter when Aristotle writes:
\begin{quote}
	In the case of what is to possess sense, the first transition is due to the action of the male parent and takes place before birth so that at birth the living thing is, in respect of sensation, at the stage which corresponds to the possession of knowledge. (Aristotle, \emph{De Anima} \textsc{ii}.5 417\( ^{b} \)17--19; Smith in \citealt[31]{Barnes:1984uq})
\end{quote}
There remains, however, an important point of analogy. Recall, Aristotle's curious remark that the ignorant but educable has the capacity for knowledge because of their matter. On the hylomorphic theory, matter is a kind of potentiality, and form is a kind of actuality. As we have seen (chapter~\ref{sec:the_soul_of_the_eye}), the form of the eye, its actuality, is its capacity to see. And the matter of the eye---the internal water, the membrane---is the potential to have that capacity. In the epistemic and perceptual cases, then, first potentiality is associated with the potentiality of matter.

% section the_triple_scheme (end)

\section{Destructive and Preservative Change} % (fold)
\label{sec:destructive_and_preservative_change}

The acquisition of knowledge is an alteration, whereas the application of knowledge is a different sort of transition, from the inactive possession of knowledge to its active exercise. Just as the actual and the potential are said of in many ways, so too is being moved or acted upon:
\begin{quote}
	Also the expression `to be acted upon' has more than one meaning; it may mean either the extinction of one of two contraries by the other, or the maintenance of what is potential by the agency of what is actual and already like what is acted upon, as actual to potential. For what possesses knowledge becomes an actual knower by a transition which is either not an alteration of it at all (being in reality a development into its true self or actuality) or at least an alteration in a quite different sense. (Aristotle, \emph{De Anima} \textsc{ii}.5 417\( ^{b} \)2--6; Smith in \citealt[30]{Barnes:1984uq})
\end{quote}
Aristotle distinguishes two ways of being moved or acted upon. The first way is a destruction of something by its contrary. This contrasts with the preservation of that which is potential by something actual that is like it. This later is also described as something's development into its true self, a realization of its nature, as it were. The distinction between the destructive and preservative ways of being moved and acted upon immediately follows Aristotle's contrast between the transition to first actuality and the transition to second actuality. Plausibly, it is an elaboration of how these transitions differ. 

Recall the transition to first actuality, at least in the epistemic case in which it is introduced, was a kind of qualitative alteration. The motion involved in alteration involves a process whereby a thing's quality is replaced by another from the same range which is its contrary. Thus when the water is heated, it becomes hot and so is no longer cool. When a thing is altered, it becomes other than what it was, at least in the relevant qualitative respect. The motion involved in qualitative alteration, as Aristotle understands it, is thus aptly described as a destruction of something by its contrary. The cool of the water is destroyed and replaced by its contrary, heat.

It would be inapt, however, to describe the transition to second actuality as a destruction of something by its contrary. When a learned person applies their grammatical knowledge---by recognizing a letter as an alpha, say---they do not lose that knowledge. The learned person does not thereby become ignorant in applying their knowledge in a reasonable manner given the practical circumstances. The application of knowledge does not involve the destruction of that knowledge and its replacement by its contrary, ignorance. Rather, to possess such knowledge is to have the potential to exercise that knowledge in a reasonable manner given the practical circumstances. In exercising that knowledge, this potentiality is realized. There is also a difference in the potentialities involved in these transitions. In the destructive motion of qualitative alteration, the water's potential for being hot is exhausted when it becomes actually hot. However, in exercising knowledge in a given circumstance, the learned person's potential to apply that knowledge is not exhausted in this way, but is rather preserved. There is a further salient difference. In the case of alteration, something becomes unlike the way it was. The water is now hot and so unlike the way it was when it was cool. But in applying their knowledge, the learned person does not become unlike, in epistemic respects, the way they were before they applied that knowledge. The retain their grammatical knowledge in applying it. Given these differences, Aristotle concludes that the transition to second actuality is either not an alteration, or an alteration in a different sense. This can be read as echoing the earlier qualification of alteration as applied to sensation, which Smith translates as ``a sort of change of quality''. If it is, the Burnyeat is right in suggesting that it is better understood as alteration of a sort, as alteration in an etiolated sense only.

Allow me a brief digression on a point that will become important in understanding the definition of perception as the assimilation of sensible form. It involves the way likeness must be understood in preservative change. One way, not the only way, to motivate a literalist interpretation involves a particular understanding of what is required in saying that one thing is like another. On the literalist interpretation, in seeing a colored particular the perceiver's eye takes on that color. In seeing the brilliant white of the sun, the transparent medium within they eye itself becomes brilliant white. This would follow on a natural understanding of becoming like. On that understanding, if one thing is \emph{F} and another thing becomes like it in the relevant respect, then it too is actually \emph{F}. However, given that a potentiality involved in the transition to second actuality is like the actuality that realizes it, Aristotle is working with a broader understanding of likeness here. If the potentiality involved in possessing grammatical knowledge is already like its realization, there is no general requirement that if two things are alike in a certain respect they must actually have the relevant feature.

% section destructive_and_preservative_change (end)

% chapter two_kinds_of_potentiality (end)
