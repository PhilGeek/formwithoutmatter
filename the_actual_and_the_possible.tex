%!TEX root = /Users/markelikalderon/Documents/Git/formwithoutmatter/aristotle.tex
\chapter{Two Transitions to Actuality} % (fold)
\label{cha:two_kinds_of_potentiality}

\section{The Problem} % (fold)
\label{sec:the_problem}

The eyes, in seeing, are altered. They are filled with light. The existence and character of the interior illumination depends on and derives from the existence and character of exterior illumination, itself affected by the colors of remote external particulars. Moreover, the eyes in seeing must be altered. The eyes are the organ of sight, and seeing is a reactive capacity. Sight only acts by reacting to the colors of remote external particulars. So the organ of sight must be acted upon in order to exercise the capacity for sight. Notoriously, however, Aristotle denies that the exercise of our capacity for sight, an episode of seeing, is itself an alteration. So the eyes, in seeing must be altered, but seeing is not itself an alteration. What kind of change does a perceiver undergo when seeing?

% section the_problem (end)

\section{The Actual and the Potential} % (fold)
\label{sec:the_actual_and_the_potential}

% section the_actual_and_the_potential (end)

% chapter two_kinds_of_potentiality (end)
