%!TEX root = /Users/markelikalderon/Documents/Git/formwithoutmatter/aristotle.tex
\chapter{Two Transitions to Actuality} % (fold)
\label{cha:two_kinds_of_potentiality}

\section{The Problem} % (fold)
\label{sec:the_problem}

The eyes, in seeing, are altered. They are filled with light. The existence and character of the interior illumination depends on and derives from the existence and character of exterior illumination, itself affected by the colors of remote external particulars. Moreover, the eyes in seeing must be altered. The eyes are the organ of sight, and seeing is a reactive capacity. Sight only acts by reacting to the colors of remote external particulars. So the organ of sight must be acted upon in order to exercise the capacity for sight. Notoriously, however, Aristotle denies that the exercise of our capacity for sight, an episode of seeing, is itself an alteration. So the eyes, in seeing must be altered, but seeing is not itself an alteration. What kind of change does a perceiver undergo when seeing?

% section the_problem (end)

\section{The Triple Scheme} % (fold)
\label{sec:the_triple_scheme}

As we have had occasion to remark, according to Aristotle, the actual and the potential are said of in many ways. In \emph{De Anima} \textsc{ii}.5 Aristotle distinguishes two senses of potentiality, and correspondingly, two senses of actuality. The distinction occurs earlier in \emph{De Anima} \textsc{ii}.1 (and occurs as well in Physica \textsc{viii}.4 255\( ^{a} \)30ff). In \emph{De Anima} \textsc{ii}.5 the distinction is explained in terms of knowledge:
\begin{quote}
	But we must now distinguish different senses in which things can be said to be potential or actual; at the moment we are speaking as if each of these phrases had only one sense. We can speak of something as a knower either as when we say that man is a knower, meaning that man falls within the class of beings that know or have knowledge, or as when we are speaking of a man who possesses a knowledge of grammar; each of these has a potentiality, but not in the same way: the one because his kind or matter is such and such, the other because he can reflect when he wants to, if nothing external prevents him. And there is the man who is already reflecting---he is a knower in actuality and in the most proper sense is knowing, e.g. this A. Both the former are potential knowers, who realize their respective potentialities, the one by change of quality, i.e. repeated transitions from one state to its opposite under instruction, the other in another way by the transition from the inactive possession of sense or grammar to their active exercise. (Aristotle, \emph{De Anima} \textsc{ii}.5 417\( ^{a} \)22--417\( ^{b} \)1; Smith in \citealt[30]{Barnes:1984uq})
\end{quote}

In the passage, Aristotle contrasts someone ignorant of some subject matter but educable with a learned person knowledgeable about that subject matter. The ignorant but educable person is a knower in the sense that they potentially know something. They do not, for example, know some particular point of grammar, but it is not beyond their ken, and they can be brought to know it through education. Being composed of the right matter, they have the capacity for knowledge. The learned person is a knower as well. They have mastered the relevant point of grammar and, having mastered it, can be said to know it in the way that the ignorant but educable cannot. And Aristotle maintains that this sense of being a knower is also a kind of potentiality. Specifically, to know something is to have the capacity to apply that knowledge in a reasonable manner given the practical circumstances. Thus to apply the grammatical knowledge is to deploy it in one's speech or writing or to recognize its significance in the speech or writing of others. In exercising their knowledge, they can be said to know as well but in a different sense from one who possesses this knowledge without exercising it. To possess grammatical knowledge, in the relevant sense, just is to have the potential to exercise that knowledge in a reasonable manner given the practical circumstances. But this is a distinct sense of potentiality than is involved in the ignorant but educable's potentially knowing that point of grammar.  In the traditional, post-Aristotelian vocabulary, Aristotle is distinguishing between first and second potentiality (and, correspondingly, between first and second actuality), schematically represented in in table~\ref{tab:triple}. (The vocabulary is post-Aristotelian since while Aristotle does speak of first actuality he does not generalize the vocabulary in the obvious way.)

\begin{table}[htbp]
	\centering
		\begin{tabular}{ccc}
			\hline
			\emph{First Potentiality} & the capacity for knowledge & ``is a knower''\\
			\hline
			\emph{Second Potentiality/First Actuality} & knowing something & ``is a knower'', ``knows''\\
			\hline
			\emph{Second Actuality} & exercising that knowledge & ``knows''\\
			\hline
		\end{tabular}
	\caption{Two senses of the actual/potential distinction}
	\label{tab:triple}
\end{table}

The ignorant but educable is a knower in that they potentially know something. This is the first potentiality. In coming to know something this potentiality is realized. In coming to know something they too are a knower, though in a different sense. This is the first actuality. The first actuality of knowledge is itself a kind of potentiality since in knowing something the knower has the power or potentiality to exercise that knowledge in a reasonable manner given the practical circumstances. So, in the case at hand, the first actuality is also a second potentiality. In exercising that knowledge this potentiality is realized. This is the second actuality.

At the end of the passage, Aristotle draws our attention to a corresponding difference in the two transitions to actuality. The transition from first potentiality to first actuality is a form of qualitative alteration in the way that the transition from second potentiality to second actuality is not.

% section the_triple_scheme (end)

% chapter two_kinds_of_potentiality (end)
